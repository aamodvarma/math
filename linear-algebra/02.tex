\setcounter{chapter}{2}
\chapter{Linear Maps}
\section*{3A - Vector space of Linear Maps}

\begin{definition}[Linear Maps]
    A linear map is a function from $V$ to $W$ with the following properties,

        \qquad 1. $T(u + v) = T(u) + T(v)$

        \qquad 2. $T(kv) = k T(v)$

\end{definition}


\begin{lemma}
    $T(0) = 0$
\end{lemma}

\begin{proof}
    $$T(0v) = 0 T(v) = 0$$
\end{proof}


$L(V,W) $ is all the linear maps from  $V$ to $W$. $L(V) $ is the notatino for  $L(V,V)$



\begin{theorem}
    $L(V,W)$ is a linear space
\end{theorem}
\begin{proof}
     For $(S+T)(v) = S(v) + T(v)$
\end{proof}



Some properties of linear maps are, 


\begin{property}
    \[
        (T_1T_2)T_3 = T_1(T_2T_3)
    .\] 
\end{property}
\begin{property}
    \[
    TI = IT = T
    .\]  
\end{property}          
\begin{property}
    $$(S_1+S_2)T = S_1T + S_2T $$
\end{property}

\begin{lemma}
    Suppose $v_1,\dots,v_n$ is a basis of $v$, then for any $w_1,\dots,w_n \in W$. There exists a unique linear map $T: V \rightarrow W$ s.t. \[
        Tv_k = w_k \text { for $k = 1,\dots,n$}
    .\] 
\end{lemma}
\begin{proof}
        
\end{proof}

\subsection*{3B - Null Spaces and Ranges}

\begin{definition}[Null space]
    The null space of $T$ is defined as the vectors in $V$ such that $Tv = 0$
\end{definition}

\begin{lemma}
    null $T$ is a subspace of domain of $T$
\end{lemma}
\begin{definition}[Range]
    range$(T) = \{Tv : v \in V\}$
\end{definition}


\begin{lemma}
    range $(T)$ is a subspace of $W$
\end{lemma}
\begin{proof}
    1. $0 \in \text{ range}(T)$ because  $T(0) = 0$

    2. For $a,b \in$ range $(T)$ we can say, \[
    a + b = T(v_1) + T(v_2) = T(v_1 + v_2)
    .\] 
    So, $a + b \in $ range$(T)$

    3. For $a \in $ range  $(T)$ we can say,  \[
    a = T(v_1)
    .\] 
    $ka = k T(v_1) = T(kv_1)$
    so $ka \in $ range$(T)$
\end{proof}


\begin{definition}[Injective]
    A functino $T: V \rightarrow W$  is called injective if \[
    Tu = Tv \implies u = v
    .\] 
    or, \[
    u \ne v \implies Tu \ne Tv
    .\] 
\end{definition}

\begin{definition}[Surjective]  
    A function $T: V \rightarrow W$  is called surjective if, \[
        \forall w \in W, \exists \text{ unique } v \in V \text{ such that } Tv = w
    .\] 

    Or, range$(T) = W$
\end{definition}


\begin{lemma}
    $T$ is injective $\iff$ null space of  $T$ is $\{0\}$
\end{lemma}
