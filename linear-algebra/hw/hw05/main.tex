\documentclass[a4paper]{report}
\usepackage[utf8]{inputenc}
\usepackage[T1]{fontenc}
\usepackage{textcomp}

\usepackage{url}

% \usepackage{hyperref}
% \hypersetup{
%     colorlinks,
%     linkcolor={black},
%     citecolor={black},
%     urlcolor={blue!80!black}
% }

\usepackage{graphicx}
\usepackage{float}
\usepackage[usenames,dvipsnames]{xcolor}

% \usepackage{cmbright}

\usepackage{amsmath, amsfonts, mathtools, amsthm, amssymb}
\usepackage{mathrsfs}
\usepackage{cancel}

\newcommand\N{\ensuremath{\mathbb{N}}}
\newcommand\R{\ensuremath{\mathbb{R}}}
\newcommand\F{\ensuremath{\mathscr{F}}}
\newcommand\Z{\ensuremath{\mathbb{Z}}}
\renewcommand\O{\ensuremath{\emptyset}}
\newcommand\Q{\ensuremath{\mathbb{Q}}}
\newcommand\C{\ensuremath{\mathbb{C}}}
\let\implies\Rightarrow
\let\impliedby\Leftarrow
\let\iff\Leftrightarrow
\let\epsilon\varepsilon

% horizontal rule
\newcommand\hr{
    \noindent\rule[0.5ex]{\linewidth}{0.5pt}
}

\usepackage{tikz}
\usepackage{tikz-cd}

% theorems
\usepackage{thmtools}
\usepackage[framemethod=TikZ]{mdframed}
\mdfsetup{skipabove=1em,skipbelow=0em, innertopmargin=5pt, innerbottommargin=6pt}

\theoremstyle{definition}

\makeatletter

\declaretheoremstyle[headfont=\bfseries\sffamily, bodyfont=\normalfont, mdframed={ nobreak } ]{thmgreenbox}
\declaretheoremstyle[headfont=\bfseries\sffamily, bodyfont=\normalfont, mdframed={ nobreak } ]{thmredbox}
\declaretheoremstyle[headfont=\bfseries\sffamily, bodyfont=\normalfont]{thmbluebox}
\declaretheoremstyle[headfont=\bfseries\sffamily, bodyfont=\normalfont]{thmblueline}
\declaretheoremstyle[headfont=\bfseries\sffamily, bodyfont=\normalfont, numbered=no, mdframed={ rightline=false, topline=false, bottomline=false, }, qed=\qedsymbol ]{thmproofbox}
\declaretheoremstyle[headfont=\bfseries\sffamily, bodyfont=\normalfont, numbered=no, mdframed={ nobreak, rightline=false, topline=false, bottomline=false } ]{thmexplanationbox}


\declaretheorem[numberwithin=chapter, style=thmgreenbox, name=Definition]{definition}
\declaretheorem[sibling=definition, style=thmredbox, name=Corollary]{corollary}
\declaretheorem[sibling=definition, style=thmredbox, name=Proposition]{prop}
\declaretheorem[sibling=definition, style=thmredbox, name=Theorem]{theorem}
\declaretheorem[sibling=definition, style=thmredbox, name=Lemma]{lemma}



\declaretheorem[numbered=no, style=thmexplanationbox, name=Proof]{explanation}
\declaretheorem[numbered=no, style=thmproofbox, name=Proof]{replacementproof}
\declaretheorem[style=thmbluebox,  numbered=no, name=Exercise]{ex}
\declaretheorem[style=thmbluebox,  numbered=no, name=Example]{eg}
\declaretheorem[style=thmblueline, numbered=no, name=Remark]{remark}
\declaretheorem[style=thmblueline, numbered=no, name=Note]{note}

\renewenvironment{proof}[1][\proofname]{\begin{replacementproof}}{\end{replacementproof}}

\AtEndEnvironment{eg}{\null\hfill$\diamond$}%

\newtheorem*{uovt}{UOVT}
\newtheorem*{notation}{Notation}
\newtheorem*{previouslyseen}{As previously seen}
\newtheorem*{problem}{Problem}
\newtheorem*{observe}{Observe}
\newtheorem*{property}{Property}
\newtheorem*{intuition}{Intuition}


\usepackage{etoolbox}
\AtEndEnvironment{vb}{\null\hfill$\diamond$}%
\AtEndEnvironment{intermezzo}{\null\hfill$\diamond$}%




% http://tex.stackexchange.com/questions/22119/how-can-i-change-the-spacing-before-theorems-with-amsthm
% \def\thm@space@setup{%
%   \thm@preskip=\parskip \thm@postskip=0pt
% }

\usepackage{xifthen}

\def\testdateparts#1{\dateparts#1\relax}
\def\dateparts#1 #2 #3 #4 #5\relax{
    \marginpar{\small\textsf{\mbox{#1 #2 #3 #5}}}
}

\def\@lesson{}%
\newcommand{\lesson}[3]{
    \ifthenelse{\isempty{#3}}{%
        \def\@lesson{Lecture #1}%
    }{%
        \def\@lesson{Lecture #1: #3}%
    }%
    \subsection*{\@lesson}
    \testdateparts{#2}
}

% fancy headers
\usepackage{fancyhdr}
\pagestyle{fancy}

% \fancyhead[LE,RO]{Gilles Castel}
\fancyhead[RO,LE]{\@lesson}
\fancyhead[RE,LO]{}
\fancyfoot[LE,RO]{\thepage}
\fancyfoot[C]{\leftmark}
\renewcommand{\headrulewidth}{0pt}

\makeatother

% figure support (https://castel.dev/post/lecture-notes-2)
\usepackage{import}
\usepackage{xifthen}
\pdfminorversion=7
\usepackage{pdfpages}
\usepackage{transparent}
\newcommand{\incfig}[1]{%
    \def\svgwidth{\columnwidth}
    \import{./figures/}{#1.pdf_tex}
}

% %http://tex.stackexchange.com/questions/76273/multiple-pdfs-with-page-group-included-in-a-single-page-warning
\pdfsuppresswarningpagegroup=1

\author{Aamod Varma}
\setlength{\parindent}{0pt}


\title{Linear Alebgra HW05}
\author{Aamod Varma}
\begin{document}
\maketitle
\date{}


\section*{3A}

\subsection*{1}

\begin{proof}
    
We know for a linear map, $T(u + v) = T(u) + T(v)$ and $T(\lambda v ) = \lambda T(v)$

First we look at additivity, 

Consider an arbitrary  $u = (x_1,y_1,z_1)$ and $v = (x_2,y_2,z_2)$. So we have, 
$$ T(u + v) = T((x_1+x_2),(y_1+y_2),(z_1+z_2))$$ $$ = (2(x_1+x_2) - 4(y_1+y_2) + 3(z_1+z_2) + b, 6(x_1+x_2) + c(x_1+x_2)(y_1+y_2)(z_1+z_2)) $$ 


We need the above to be equal to, 
$$ T(u) + T(v) = (2x_1-4y_1+3z_1 + b, 6x_1 + cx_1y_1z_1) + (2x_2-4y_2+3z_2 + b, 6x_2 + cx_2y_2z_2) $$ 
$$ = (2(x_1+x_2) - 4(y_1+y_2) + 3(z_1+z_2) + 2b, 6(x_1+x_2) + c(x_1y_1z_1+x_2y_2z_2)$$ 

Comparing each of the terms we have, 
$$ 2(x_1+x_2) - 4(y_1+y_2) + 3(z_1+z_2)  + 2b  =  2(x_1+x_2) - 4(y_1+y_2) + 3(z_1+z_2) + b$$ 
$$ 2b = b $$ 
$$ b = 0 $$ 

Similarly comparing the second term we have, 
$$ 6(x_1+x_2) + c(x_1+x_2)(y_1+y_2)(z_1+z_2) =  6(x_1+x_2) + c(x_1y_1z_1+x_2y_2z_2) $$ 
$$ c(x_1+x_2)(y_1+y_2)(z_1+z_2) = c(x_1y_1z_1+x_2y_2z_2) $$ 
$$ c((x_1+x_2)(y_1+y_2)(z_1+z_2) - (x_1y_1z_1+x_2y_2z_2)) = 0 $$ 

For this to be true for any $x,y,z$ we need  $c = 0$. Hence for additvity we need $b = c = 0$

Now we check if $T(k v) = k T(v)$. Consider $v = (x,y,z)$. Then we have 
$$ T(k v) = T(k x,k y,k z) = (2kx - k4y + 3kz + b, 6kx + k^{3}cxyz) $$ 

We need this to be equal to $$kT(v) = k(2x - 4y + 3z + b, 6x + cxyz) = (2kx- 4ky + 3kz + bk, 6kx + kcxyz) $$

Comparing the terms we have, 
$$ 2kx - 4ky + 3kz + bk  = 2kx - 4ky + 3kz + b$$ 
$$ bk = b $$ 
$$b = 0$$

$6kx + kcxyz = 6kx+ k^{3}cxyz$ 
$$ c = k^2 c$$ 
$$ c = 0 $$ 

So we have $b = c = 0$

\end{proof}
\subsection*{6}
\begin{proof}
     
1. Associativity. We have $(T_1T_2)T_3 = T_1(T_2T_3)$

Consider the operation on a vector $v$ so we have,  $(T_1T_2)T_3v$ which is, 
$$ ((T_1T_2)(T_3(v)) = T_1(T_2(T_3(v)))$$ 

Now looking at the right side we have, $T_1(T_2T_3) = T_1(T_2(T_3(v)))$. So we showed tha tthe LHS is equal to the RHS.

2. Identity. Consider a vector $v$ we have,  
$$ TIv = T(I(v)) = T(v)$$ 

Now, 
$$ ITv = I(T(v))  = T(v) \text{ because $Iv = v, \forall v$}  $$ 

3. Distributive Property

To show that, 
$$ (S_1+S_2)T  = S_1T + S_2T$$ 

Consider an abitrary vector $v$ in the domain of $T$. We have,  
$$ (S_1+S_2)Tv = (S_1+S_2)(T(v)) $$ 
By definitino of addition of linera maps we have, 
$$= (S_1(T(v))) + (S_2(T(v)))$$ 

Simliary we have, 
$$ (S_1T + S_2T)v = S_1T(v) + S_2T(v) = S_1(T(v)) + S_2(T(v))$$ 

We see that the distributive property holds.


Now To show that $S(T_1+T_2) = ST_1 + ST_2$. Consider $v$ we have, 
$$ S(T_1+T_2)v = S(T_1(v) + T_2(v)) = S(T_1(v)) + S(T_2(v)) $$ 

And we have, 
$$ (ST_1 + ST_2)v = ST_1(v) + ST_2(v) = S(T_1(v)) + S(T_2(v))$$ 

We see that the property holds again.



\end{proof}

\subsection*{7}
\begin{proof}
    Let $V$ be a one dimentional vector space. This means that the basis of $V$ contains a single vector, let the basis be $\{v\}$. Now we are considering a linear map from  $V$ to itself. 

    So assume that the linear map $T$ maps some $v_0$ in $V$ to $w_0$. We need to show that $w_0 = \lambda v_0$ for some $\lambda \in F$. Because $T$ maps $V$ to itself we known that that $w_0 \in V$ for any $w_0$. If $w_0 \in V$ then wek now that it can be written as a linear compbination of its basis. As the basis only has one vector we can write $w_0 = \lambda_1 v$. Similarly as $v_0 \in V$ we can write $v_0 = \lambda_2 v$. 

    So we have, 
    $$ \frac{v_0}{\lambda_2} = v $$ 
    $$ w_0 = \lambda_1 \frac{v_0}{\lambda_2}  = \lambda v_0$$ 


    
\end{proof}

\subsection*{8}
\begin{proof}
    Consider the function that maps any vecotor $(x,y)$ to the $max(|x|,|y|)$. We can see that this satisfies homogeneity. For instance consider  $(2,6)$. Our function maps this to  $6$. Now consider  $(2 \times 3, 6 \times 3)$  which is mapped to $18$ which is  $3 \times  6$ as we saw above. 

    Now consider two vector $(1,0)$ and  $(0,4)$. Our function maps both these vectors to  $1$ and $4$ respectively. However it maps its sum  $(1,4)$ to  $4 \ne 4 + 1$. Hence it does not follow additvity. Hence not a linear space.
\end{proof}




\end{document}

