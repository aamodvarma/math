\documentclass[a4paper]{report}
\input{preamble.tex}
\title{Linear Alebgra HW06}
\author{Aamod Varma}
\begin{document}
\maketitle
\date{}


\section*{3C}


\subsection*{6}

\begin{proof}
    Consider a case where  range T =  W. In this case let $w_1,\dots,w_m$ be a basis for W which also would be a basis for range  $T$, we can find $v_1,\dots,v_n$ such that, 
    $$ Tv_1 = w_1,\dots,Tv_m=w_m,\dots,Tv_n = 0 $$ 

    So with respect to the basis $w_1,\dots,w_m$ in the first column we can have all zeroes except for the first row first element.

    Now if dim range T < dim W. We can find a $v_1$ such that $T(v_1) = 0$. Now the matrix for this will have all in the first column as zero (as none of them dependent on $w_1$)
\end{proof}


\subsection*{6}
We know that $T(v_1) = A_{1,1}w_1 + \dots + A_{n,1}w_m$. We need to show there exists a basis of $W$ such that all values except for possibly  $A_{1,1}$ is zero.

Consider the case when $T(v_1) = 0$ then we have $A_{1,1} = \dots = A_{n,1} = 0$ for any arbitrary basis of $W$.

If $T(v_1) \ne 0$ then consider all of $A_{2,1},\dots,A{n,1} = 0$ except for $A_{1,1}$ (the element in the first column and row). In this case consdier, 
$$ T(v_1) = w_1 $$ where $w_1$ is an arbitrary vector in $W$. Now we can extend $w_1$ to a basis of $W$.

So in both cases we have a basis of $W$, $w_1,\dots,w_n$ such that only possibly the first row first column element is zero.


\subsection*{10}
Let $A = \begin{bmatrix}
1 & 0\\
0 & 0 
\end{bmatrix}$
and let 
$B = \begin{bmatrix}
0 & 1\\
0 & 0 
\end{bmatrix}$


We get, 
$$ AB = \begin{bmatrix}
    0&1\\
0&0\end{bmatrix}$$
and 
$$ BA = \begin{bmatrix}
    0&0\\
0&0\end{bmatrix}$$ 


So $AB \ne BA$
\subsection*{8}

\begin{proof}
    We need to show $(AB)_{j,.} = A_{j,.}B$

    We have, 
    \begin{align*}
        (AB)_{j,.}  &= (\sum_{k=1}^n A_{j,k} B_{k,1}, \dots, \sum_{k=1}^n A_{j,k} B_{k,p}, )\\
                    &= (A_{j,1},\dots,A_{j,n})B\\
                    &= A_{j,.}B
    \end{align*}

\end{proof}


\subsection*{13}
We know that 
$$ (AA)_{j,k} = \sum_{r = 1}^n A_{j,r} A_{r,k} $$ 

$$ (A(AA))_{j,k} = \sum_{r = 1}^n A_{j,r} (AA)_{r,k}$$ 

$$ (A^{3})_{j,k} = \sum_{p = 1}^n A_{j,r}( \sum_{x = 1}^n A_{r,x} A_{x,k}  )$$ 

$$ (A^{3})_{j,k} = \sum_{p = 1}^n \sum_{x = 1}^n A_{j,r} A_{r,x} A_{x,k}$$ 






\subsection*{16}
\begin{proof}
    

$\implies$ 

Take $A$ with rank $1$ then we can decompose it to matirces $R$ and $C$ s.t, 
$$ R = (m \times  1), C = (1 \times  n) $$ 

Let $R = (c_1,\dots,c_m)^T$ and $C = (d_1,\dots,d_n)$

So we have $$A_{jk} = (RC)_{jk} = \sum_{n = 1}^1 R_{m,1}C_{1,n}$$
$$ = c_jd_k $$ 

$\impliedby$

If $A_{j,k} = c_jd_k$ then we can write  $A$ in terms of two matrices $R$  times $C$ such that R = $(c_1,\dots,c_m)^T$ and $C = (d_1,\dots,d_m)$

\end{proof}


\subsection*{3}
\begin{proof}
    
We need to show, $a \iff b$

$a \implies b$.

If  $T$ is invertible we know that it is injective and surjective. If $T$ is surjective then we know that null T = 0. So we have, 
$$ dim range(T) = dim (V) $$ 

To show that $T(v_1),\dots,T(v_n)$ is a basis of V we need to show that it is linearly independent and spans $V$. 

To show linear independence we need to show that $c_1,\dots,c_n = 0$ if,
$$ c_1(Tv_1) + \dots + c_nT(v_n) = 0 $$ 

This is equal to, 
$$ T(c_1v_1 + \dots + c_nv_n) = 0 $$

We know that the null T is \{0\} so we have, 
$$ c_1v_1+ \dots + c_nv_n = 0 $$ 

We know that $v_1,\dots,v_n$ is a basis for $V$. So it is linearly independent which means that, $c_1,\dots,c_n = 0$. So we have $T(v_1),\dots,T(v_n)$ is a linearly independent set.

The lenght of our list is the same as the length of the basis. Which means that we have linearly independent set of vectors are also a basis for $V$.


Another way of showing spanning is taking any $v \in V$ we can write it as $a_1v_1,\dots,a_nv_n = v$. We can applying $T$ on both sides and show that $ v$ can be represnted as a linear combination of $T(v_k)$


$a \impliedby b$

If $Tv_1,\dots,Tv_n$ is a basis for $V$. 

We need to show that $T$ is injective and surjective. First consider an arbitary $v \in V$ such that $T(v) = 0$. We know that  $v = a_1v_1+\dots+a_nv_n$ where $v_1,\dots,v_n$ is a basis for $V$. So we get, 
$$ T(a_1v_1+\dots+a_nv_n) = 0 $$ 
$$ a_1T(v_1) + \dots + a_nT(v_n) = 0 $$ 

We know that $Tv_1,\dots,Tv_n$ is a basis which means that its linearly independent. So w have, $a_1,\dots,a_n = 0$. But if $a_1,\dots,a_n$ = 0 then we have $v = a_1v_1+\dots+a_nv_n = 0$. Which means that for any $T(v) = 0$ means that $v = 0$. This means that it is injective.

We already know that if we have $V \rightarrow V$ such that both the dimensiosn are same then injective  means that its surjective which means that it is invertible.

We also can show that any $w \in V$ can be written as $T(v) = w$ which m eans that it is surjective.

Take a  $w \in V$ so $Tw = T(a_1v_1+ \dots + a_nv_n) = a_1T(v_1) + \dots + a_nT(v_n)$. We know that $T(v_1),\dots,T(v_n)$ is a basis which  means that any vector $v \in V$ can be repreesnted as  alinear combination of  $T(v_1),\dots,T(v_n)$. So we showed that any vector $v \in V$ can be represented by a $w \in V$ such that $T(w) = v$ which means that its surjective.


\end{proof}
 
\subsection*{5}
\begin{proof}
    $\impliedby$
We know that $T$ is invertible so its injective and surjective. We know for every  $u \in U$, $T(u) = S(u)$. 

We need to show $S$ is injective. First consider $u_1,u_2$ so we need to showm $S(u_1) = S(u_2) \implies u_1 = u_2$. 

If $Su_1 = Su_2$ then we can say $T(u_1) = T(u_2)$. However we nkow that T is injective so this means that $u_1 = u_2$. Hence we show that $S$ has to be injective.

$\implies$

We have $S$ is injective and maps a subspace of $V$, $U$ onto $V$. We need to show that there exists a linear map $T$ from $V$ to itself such that it is an invertible linear map.

First consider the basis of $U$ as $u_1,\dots,u_k$. We can extend the basis from this to, 
$$ u_1,\dots, u_k, v_{k+1},\dots,v_n $$ 



Let us define our lienar map T such that $T(u) = S(u)$ if  $u \in U$ or in other words if $u =a_1u_1+ \dots + a_ku_k$. And define $T(v) = v$ if $v \in V - U$


We need to show $T(v'_1) = T(v'_2) \implies v'_1 =v'_2$. Let $v'_1 = a_1u_1+ \dots a_nv_n$ and $v'_2 = b_1u_1+ \dots + b_nv_n$.
So we have, 
\begin{align*}
    T(v'_1) &= T(v'_2)\\
    T(a_1u_1+ \dots a_nv_n) &= T(b_1u_1+ \dots b_nv_n)\\
    T(a_1u_{1})+ \dots+ a_nT(v_n) &= T(b_1v_1)+ \dots +b_nT(v_n)\\
    (a_1- b_1)T(u_{1})+ \dots+ (a_n - b_n)T(v_n) &= 0\\
\end{align*}


TO DO LATER 
\subsection*{9}
If $T$ is surjective then there exists a map $S:W \rightarrow V$, TS is the identity map. Or that $T(S(v)) = v$ for  $v \in V$.

Now let  $U = range(S)$ we need to show that  $T_{|U}$ is injective and surjective.

1. Injective.

Consider a $u \in $ null $T|_U$. So $$T_U(u) = 0$$ $$ T_U(S(w)) = w = 0$$

But 
$$ S(w) = u $$  and $w = 0$ 

which means that $u = 0$.

2. Surjective

For any $w \in W$ we have $v = S(w) \in U$ such that $T(S(w)) = w \in W$ .

So we show that  $\exist v = S(w)$ for any $w \in W$ 

Hence we show that it is isomorphic

\end{proof}

\subsection*{11}
$\implies$

We have $ST$ is invertible. Lets assume the contrary that either $S$ is not invertible or $T$ is not invertible.

1. S is not invertible. Means $S$ is not surjective. We know that  $ST$ is invertibel which means that $\forall v \in V$ $\exists v'$ s.t. $STv' = v$. Now let  $Tv' = v''$. This means that $\forall v \in V, \exists v'' \in V $  s.t. $S(v'') = v$. But this makes $S$ surjective which contradicts our assumption.

2. T is not invertible means that $T$ is not injective or surjective. We know that $ST$ is injective and surjective. If $T$ is not injective then $\exists v\in V$ s.t. $T(v) = 0$. Now this means  $S(T(v)) = S(0) = 0$ or that  $STv = 0$ for some $v \ne 0$. But this makes ST not injective and hence not invertible which contradicts our assumption.  

So by proof by contradiction our assumption ust be wrong and both S and T are invertible.


$\impliedby$

We have S and T are invertible. We need to show that $ST$ is injective and surjective.

1. Injective

\quad We need to show that null ST = \{0\}. Consider $v \in $ null ST  such that $STv = 0$. We know that $S$ is injective hence $Tv = 0$. But  $T$ is injective as well. Hence $v = 0$. So null ST = 0. Hence $ST$ is injective

2. Surjective

\quad We need to shwo that $\forall v\in V$, $\exists v' \in V$s.t. $STv' = v$. We know that $S$ is surjective this means that for any $v \in V$, $\exists v'' $ s.t. $S(v'') = v$. We also know that $T$ is surjective this means that for any  $v'' \in V$ $\exists v' \in V$ s.t. $T(v') = v''$. So we have  $\exists v'$ s.t. $S(T(v')) = v$ or  $ST$ is surjective. 

Hence we show that $ST$ is invertible.


\subsection*{12}
We have $STU = I$ This means that $ST$ and  $U$ are both invertible which means that $S$ and $T$ are also invertibel. So $\exists S^{-1}$ and $U^{-1}$.

\begin{align*}
    STU &= I\\
    TU &= S^{-1}\\
    U &= T^{-1}S^{-1}\\
    US &= T^{-1}S^{-1}S\\
    T^{-1} &= US\\
\end{align*}

\subsection*{15}
If $Tv_1,\dots,Tv_m$ spans V that menas that m $\ge$ dim V.

1. m = $\dim V$

Assume $v_1,\dots,v_m$ is not a lin ind set of vectors. Then, $\exists a_1,\dots a_m$ some not zero such that, 
$$ a_1v_1 + \dots + a_mv_m = 0 $$ .

So we get, 
$$ a_1(Tv_1) + \dots + a_mT(v_m) = 0 $$ where some a not zero.

But this implies that $Tv_1,\dots,Tv_m$ list is not lin independent which is contradictory.

Hence we have $v_1,\dots,v_m$ is a lin independent set of vectors with dimension smae as basis of V which makes $v_1,\dots,v_m$ a basis of V or spanning.



2. If $m > $ dim V.

Then we can remove vectors from our list to make it spanning and lin independent and spanning, consider the smaller list as follows, 
$$ Tv_1,\dots,Tv_k $$ 
Using similar reasoning we know that  
$$ v_1,\dots,v_k $$  has to be linearly independent as well. However in this case we have lin independent vectors whose number is the same as that of a basis which makes it a basis as well. Hence our list of vectors $v_1,\dots,v_k$ is spanning. So we can continue extending this basis to $v_1,\dots,v_m$ and still have it spanning.











\end{document}

