\documentclass[a4paper]{report}
\usepackage[utf8]{inputenc}
\usepackage[T1]{fontenc}
\usepackage{textcomp}

\usepackage{url}

% \usepackage{hyperref}
% \hypersetup{
%     colorlinks,
%     linkcolor={black},
%     citecolor={black},
%     urlcolor={blue!80!black}
% }

\usepackage{graphicx}
\usepackage{float}
\usepackage[usenames,dvipsnames]{xcolor}

% \usepackage{cmbright}

\usepackage{amsmath, amsfonts, mathtools, amsthm, amssymb}
\usepackage{mathrsfs}
\usepackage{cancel}

\newcommand\N{\ensuremath{\mathbb{N}}}
\newcommand\R{\ensuremath{\mathbb{R}}}
\newcommand\F{\ensuremath{\mathscr{F}}}
\newcommand\Z{\ensuremath{\mathbb{Z}}}
\renewcommand\O{\ensuremath{\emptyset}}
\newcommand\Q{\ensuremath{\mathbb{Q}}}
\newcommand\C{\ensuremath{\mathbb{C}}}
\let\implies\Rightarrow
\let\impliedby\Leftarrow
\let\iff\Leftrightarrow
\let\epsilon\varepsilon

% horizontal rule
\newcommand\hr{
    \noindent\rule[0.5ex]{\linewidth}{0.5pt}
}

\usepackage{tikz}
\usepackage{tikz-cd}

% theorems
\usepackage{thmtools}
\usepackage[framemethod=TikZ]{mdframed}
\mdfsetup{skipabove=1em,skipbelow=0em, innertopmargin=5pt, innerbottommargin=6pt}

\theoremstyle{definition}

\makeatletter

\declaretheoremstyle[headfont=\bfseries\sffamily, bodyfont=\normalfont, mdframed={ nobreak } ]{thmgreenbox}
\declaretheoremstyle[headfont=\bfseries\sffamily, bodyfont=\normalfont, mdframed={ nobreak } ]{thmredbox}
\declaretheoremstyle[headfont=\bfseries\sffamily, bodyfont=\normalfont]{thmbluebox}
\declaretheoremstyle[headfont=\bfseries\sffamily, bodyfont=\normalfont]{thmblueline}
\declaretheoremstyle[headfont=\bfseries\sffamily, bodyfont=\normalfont, numbered=no, mdframed={ rightline=false, topline=false, bottomline=false, }, qed=\qedsymbol ]{thmproofbox}
\declaretheoremstyle[headfont=\bfseries\sffamily, bodyfont=\normalfont, numbered=no, mdframed={ nobreak, rightline=false, topline=false, bottomline=false } ]{thmexplanationbox}


\declaretheorem[numberwithin=chapter, style=thmgreenbox, name=Definition]{definition}
\declaretheorem[sibling=definition, style=thmredbox, name=Corollary]{corollary}
\declaretheorem[sibling=definition, style=thmredbox, name=Proposition]{prop}
\declaretheorem[sibling=definition, style=thmredbox, name=Theorem]{theorem}
\declaretheorem[sibling=definition, style=thmredbox, name=Lemma]{lemma}



\declaretheorem[numbered=no, style=thmexplanationbox, name=Proof]{explanation}
\declaretheorem[numbered=no, style=thmproofbox, name=Proof]{replacementproof}
\declaretheorem[style=thmbluebox,  numbered=no, name=Exercise]{ex}
\declaretheorem[style=thmbluebox,  numbered=no, name=Example]{eg}
\declaretheorem[style=thmblueline, numbered=no, name=Remark]{remark}
\declaretheorem[style=thmblueline, numbered=no, name=Note]{note}

\renewenvironment{proof}[1][\proofname]{\begin{replacementproof}}{\end{replacementproof}}

\AtEndEnvironment{eg}{\null\hfill$\diamond$}%

\newtheorem*{uovt}{UOVT}
\newtheorem*{notation}{Notation}
\newtheorem*{previouslyseen}{As previously seen}
\newtheorem*{problem}{Problem}
\newtheorem*{observe}{Observe}
\newtheorem*{property}{Property}
\newtheorem*{intuition}{Intuition}


\usepackage{etoolbox}
\AtEndEnvironment{vb}{\null\hfill$\diamond$}%
\AtEndEnvironment{intermezzo}{\null\hfill$\diamond$}%




% http://tex.stackexchange.com/questions/22119/how-can-i-change-the-spacing-before-theorems-with-amsthm
% \def\thm@space@setup{%
%   \thm@preskip=\parskip \thm@postskip=0pt
% }

\usepackage{xifthen}

\def\testdateparts#1{\dateparts#1\relax}
\def\dateparts#1 #2 #3 #4 #5\relax{
    \marginpar{\small\textsf{\mbox{#1 #2 #3 #5}}}
}

\def\@lesson{}%
\newcommand{\lesson}[3]{
    \ifthenelse{\isempty{#3}}{%
        \def\@lesson{Lecture #1}%
    }{%
        \def\@lesson{Lecture #1: #3}%
    }%
    \subsection*{\@lesson}
    \testdateparts{#2}
}

% fancy headers
\usepackage{fancyhdr}
\pagestyle{fancy}

% \fancyhead[LE,RO]{Gilles Castel}
\fancyhead[RO,LE]{\@lesson}
\fancyhead[RE,LO]{}
\fancyfoot[LE,RO]{\thepage}
\fancyfoot[C]{\leftmark}
\renewcommand{\headrulewidth}{0pt}

\makeatother

% figure support (https://castel.dev/post/lecture-notes-2)
\usepackage{import}
\usepackage{xifthen}
\pdfminorversion=7
\usepackage{pdfpages}
\usepackage{transparent}
\newcommand{\incfig}[1]{%
    \def\svgwidth{\columnwidth}
    \import{./figures/}{#1.pdf_tex}
}

% %http://tex.stackexchange.com/questions/76273/multiple-pdfs-with-page-group-included-in-a-single-page-warning
\pdfsuppresswarningpagegroup=1

\author{Aamod Varma}
\setlength{\parindent}{0pt}


\title{Linear Alebgra HW04}
\author{Aamod Varma}
\begin{document}
\maketitle
\date{}


\section*{2B}
\subsection*{Problem 4}
(a). We are given $U = \{(z_1,z_2,z_3,z_4,z_5) \in \C^5 : 6z_1 =z_2, z_3+2z_4+3z_5=0\}$

The contraints are as follows, $6z_1 = z_2$ and $z_3+2z_4+3z_5=0$

So we can rewrite each $z$ as  \[
z_1 = \frac{z_2}{6}, z_2=z_2, z_3=-2z_4-3z_5, z_4=z_4,z_5=z_5
.\] 

We see we have two dependent variables and three independent variables which means our basis will be of length $3$ dependent on $z_2,z_4,z_5$ as follows, \[
    (\frac{1}{6}, 1,0,0,0), (0,0,-2,1,0), (0,0,-3,0,1)
.\] 

(b). We need to extend this basis onto $\C^5$. We know from (a) that our dependent variables are  $z_1$ and $z_3$. So to extend our basis we need to be able to make these vectors our indepdent. For this we can add the following two vectors, \[
    (1,0,0,0,0), (0,0,1,0,0)
.\] 

These additions are linearly independent because we can't represent these vectors as a linearly combination of our previous list (in our first list it was necessarily true that $z_1 = \frac{z_2}{6}$, so if $z_1=1, z_2\neq 0$, similary reasoning for $z_3$). We also know this new list spans $\C^5$ because our new additions give us control over the dependent variables from our previous list (we could also argue that because it is a linearly independent set of vectors and we have  $\dim(\C_5)$ of them.


(c). We need to find a subspace $W$ such that  $U \oplus W = \C^5$. Take  $W$ from above as,  \[
    W = {(1,0,0,0,0),(0,0,1,0,0)}
.\] 

First we need to show that $W + U =\C^5$.  That every vector in $\C^5$ can be represented as $v = u + w, u \in U, w \in W$ 

Now, if $u \in U, u = a_1u_1+a_2u_2 + a_3u_3$ and if $w \in W, w = b_1w_1+b_2w_2$. 

So $$v = a_1u_1+a_2u_2+a_3u_3 + b_{1}w_1+b_2w_2$$

But we know from above that  $u_1,u_2,u_3,w_1,w_2$ is a basis for $\C^5$. Which means that the linear combination of these vectors can reprsent every vector in  $\C^5$. So we show that all of  $v \in \C^5$ can be written as a vecotr  $u \in U$ plus a vector  $w \in W$.


\subsection*{Problem 5}
If $V = W + U$ we can say that $\forall v \in V$,  \[
    v = u + w \text{ for }u\in U, w \in W
.\] 

Now, $u$ can be written as a linear combinatino of vectors in $U$ and  similar can be done for $w$.

So let $u = a_1u_1 + \dots + a_nu_n$ and $w = b_1w_1 + \dots + b_mw_m$. So we have a linear combination of $n + m$ vectors.  We know that $\dim(V) \leq n + m$ because  $\dim(V) \leq $ length of any spanning set in  $V$.

If  $n + m > \dim V$. Then we can reduce it to a linearly independent set of vector such that it still spans  $V$. So now we have a basis of $V$ that consists of vectors that are either in $U$ or  $W$. Or in other words our basis are vectors in  $U \cup W$.

If $n + m = \dim V$ then we already have a linearly independent set of vectors that span $V$ which consists of vectors either in  $U$ or  $V$. Which meanst hat the basis are vectors in $U \cup W$.

So we have shown that there exists a basis of  $V$ in  $U \cup W$ if  $U + W = V$.

\subsection*{11}
We know that $v_1,\dots,v_n$ is a basis for $V$. We need to show that it is also a basis for  $V_C$. Now $V_\C$ is defined by $V \times V$ such that $(x,y) = x + iy \in V_\C$. 

So we need to show that any vector of the form $u + iw \in V_C$ can be represented by a linear combinatino of  $v_1,\dots,v_n$.

First we know that $u \in V, w \in V$. So we can write  $u = a_1v_1+\dots+a_nv_n$, similarly $w = b_1v_1 + \dots + b_nv_n$.

Now because we also define scalar multiplication with complex numbers we can write, \[
a_1v_1+ \dots + a_nv_n + i(b_1v_1 + \dots + b_nv_n) = u + iw
.\] 
Or, \[
    \forall (u,w) \in V_C, u + iw = (a_1+ib_1)v_1 + \dots + (a_n + ib_n)v_n
.\] 
So we showed that we can represent all elements of $V_C$ as a linear combination of our vectors  $v_1, \dots, v_n$


\section*{2C}
\subsection*{Problem 1}
We know that $\dim(\R^2) = 2$ which means that for a given subspace $V$ we have three cases, \[
\dim(V) = 0,
\dim(V) = 1,
\dim(V) = 2
.\] 

If $\dim(V) = 0$ then our vector space if  $V = \{0\}$ by definition.

If  $\dim(V) = 1$ then that means our vector space contains one vector so $V$ is spanned by  $\{v\}$. First we knwo that $0 \in V $ as $V$ is a subspace (we can take the coefficient to be 0). Now for any vector $v \in V, kv \in V$. We know that this defines any line in $\R^2$ that goes through the origin.

If $\dim(V) = 2$ we also know that  $U \subseteq V$.  If $U \subseteq V$ and  $\dim(U) = \dim(V)$ then we know that  $ U = V$. So, $U$ determines  $\R^2$


\subsection*{Problem 4}
(a). A basis of $U$ would be one where $p''(6) = 0$. First we know that a basis of  $P_4(R)$ is  $1,x,x^2,x^3$ which can also be written as $1,(x-6),(x-6)^2,(x-6)^3$ where $x \in R$

So any $p$ is written as $$p(x) = 1a_1 + a_2(x-6) + a_3(x-6)^2 + a_4 (x - 6)^3$$
$$p''(6) = 2a_3$$
So we see that for it to be equal to 0, $a_3 = 0$. Which means our basis is, \[
1, (x - 6), (x-6)^3
.\] 

(b). As we discussed above, adding $(x-6)^2$ to the list will give us a basis for $P_4(R)$

So our basis is, $$1, (x-6),(x-6)^2, (x-6)^3$$


(c). Our subspace $W$ would be spanned by  $(x-6)^2$. We first show that $W + U = P_4(R)$. To do this we need to show any  $p \in P_4(R)$ can be represented as,  \[
p = u + w, u \in U, w \in W
.\] 

We know for $u \in U, u = a_1 + a_2(x-6) + a_3(x-6)^3$ and for $w \in W, w= b_1(x-6)^2$.

So, \[
p = a_1 + a_2(x-6) + a_3(x-6)^3 + b_1(x-6)^2
.\] 

Which is a linear combination of the basis of $P_4(R)$ which means that $u + w$ can represent any vector  $p \in P_4(R)$ and hence we can say $U + W = P_4(F)$

Now we need to show that  $ U \oplus W = P_4(F)$. To show this we can show that there is only one way of representing  $0$ as  $u + w$.

Now if  $u + w = 0$ as we did abov ewe can write,  
$$0 = a_1 + a_2(x-6) + a_3(x-6)^3 + b_1(x-6)^2$$

First we know that $a_3 = 0$ as we can't represent $x^3$ using any of the other terms. Similary we can show that $b_1 =0, a_2 = 0, a_1 = 0$. Hence the only way of reprenseting $0$ is to have all coefficients as 0.

Which means that $U \oplus W = P_4(R)$ 

\subsection*{Problem 8}
Given $v_1, \dots, v_m$ is linearly independent in $V$ and $w \in V$.
Take $U := span($




\end{document}
