\documentclass[a4paper]{report}
\usepackage[utf8]{inputenc}
\usepackage[T1]{fontenc}
\usepackage{textcomp}

\usepackage{url}

% \usepackage{hyperref}
% \hypersetup{
%     colorlinks,
%     linkcolor={black},
%     citecolor={black},
%     urlcolor={blue!80!black}
% }

\usepackage{graphicx}
\usepackage{float}
\usepackage[usenames,dvipsnames]{xcolor}

% \usepackage{cmbright}

\usepackage{amsmath, amsfonts, mathtools, amsthm, amssymb}
\usepackage{mathrsfs}
\usepackage{cancel}

\newcommand\N{\ensuremath{\mathbb{N}}}
\newcommand\R{\ensuremath{\mathbb{R}}}
\newcommand\F{\ensuremath{\mathscr{F}}}
\newcommand\Z{\ensuremath{\mathbb{Z}}}
\renewcommand\O{\ensuremath{\emptyset}}
\newcommand\Q{\ensuremath{\mathbb{Q}}}
\newcommand\C{\ensuremath{\mathbb{C}}}
\let\implies\Rightarrow
\let\impliedby\Leftarrow
\let\iff\Leftrightarrow
\let\epsilon\varepsilon

% horizontal rule
\newcommand\hr{
    \noindent\rule[0.5ex]{\linewidth}{0.5pt}
}

\usepackage{tikz}
\usepackage{tikz-cd}

% theorems
\usepackage{thmtools}
\usepackage[framemethod=TikZ]{mdframed}
\mdfsetup{skipabove=1em,skipbelow=0em, innertopmargin=5pt, innerbottommargin=6pt}

\theoremstyle{definition}

\makeatletter

\declaretheoremstyle[headfont=\bfseries\sffamily, bodyfont=\normalfont, mdframed={ nobreak } ]{thmgreenbox}
\declaretheoremstyle[headfont=\bfseries\sffamily, bodyfont=\normalfont, mdframed={ nobreak } ]{thmredbox}
\declaretheoremstyle[headfont=\bfseries\sffamily, bodyfont=\normalfont]{thmbluebox}
\declaretheoremstyle[headfont=\bfseries\sffamily, bodyfont=\normalfont]{thmblueline}
\declaretheoremstyle[headfont=\bfseries\sffamily, bodyfont=\normalfont, numbered=no, mdframed={ rightline=false, topline=false, bottomline=false, }, qed=\qedsymbol ]{thmproofbox}
\declaretheoremstyle[headfont=\bfseries\sffamily, bodyfont=\normalfont, numbered=no, mdframed={ nobreak, rightline=false, topline=false, bottomline=false } ]{thmexplanationbox}


\declaretheorem[numberwithin=chapter, style=thmgreenbox, name=Definition]{definition}
\declaretheorem[sibling=definition, style=thmredbox, name=Corollary]{corollary}
\declaretheorem[sibling=definition, style=thmredbox, name=Proposition]{prop}
\declaretheorem[sibling=definition, style=thmredbox, name=Theorem]{theorem}
\declaretheorem[sibling=definition, style=thmredbox, name=Lemma]{lemma}



\declaretheorem[numbered=no, style=thmexplanationbox, name=Proof]{explanation}
\declaretheorem[numbered=no, style=thmproofbox, name=Proof]{replacementproof}
\declaretheorem[style=thmbluebox,  numbered=no, name=Exercise]{ex}
\declaretheorem[style=thmbluebox,  numbered=no, name=Example]{eg}
\declaretheorem[style=thmblueline, numbered=no, name=Remark]{remark}
\declaretheorem[style=thmblueline, numbered=no, name=Note]{note}

\renewenvironment{proof}[1][\proofname]{\begin{replacementproof}}{\end{replacementproof}}

\AtEndEnvironment{eg}{\null\hfill$\diamond$}%

\newtheorem*{uovt}{UOVT}
\newtheorem*{notation}{Notation}
\newtheorem*{previouslyseen}{As previously seen}
\newtheorem*{problem}{Problem}
\newtheorem*{observe}{Observe}
\newtheorem*{property}{Property}
\newtheorem*{intuition}{Intuition}


\usepackage{etoolbox}
\AtEndEnvironment{vb}{\null\hfill$\diamond$}%
\AtEndEnvironment{intermezzo}{\null\hfill$\diamond$}%




% http://tex.stackexchange.com/questions/22119/how-can-i-change-the-spacing-before-theorems-with-amsthm
% \def\thm@space@setup{%
%   \thm@preskip=\parskip \thm@postskip=0pt
% }

\usepackage{xifthen}

\def\testdateparts#1{\dateparts#1\relax}
\def\dateparts#1 #2 #3 #4 #5\relax{
    \marginpar{\small\textsf{\mbox{#1 #2 #3 #5}}}
}

\def\@lesson{}%
\newcommand{\lesson}[3]{
    \ifthenelse{\isempty{#3}}{%
        \def\@lesson{Lecture #1}%
    }{%
        \def\@lesson{Lecture #1: #3}%
    }%
    \subsection*{\@lesson}
    \testdateparts{#2}
}

% fancy headers
\usepackage{fancyhdr}
\pagestyle{fancy}

% \fancyhead[LE,RO]{Gilles Castel}
\fancyhead[RO,LE]{\@lesson}
\fancyhead[RE,LO]{}
\fancyfoot[LE,RO]{\thepage}
\fancyfoot[C]{\leftmark}
\renewcommand{\headrulewidth}{0pt}

\makeatother

% figure support (https://castel.dev/post/lecture-notes-2)
\usepackage{import}
\usepackage{xifthen}
\pdfminorversion=7
\usepackage{pdfpages}
\usepackage{transparent}
\newcommand{\incfig}[1]{%
    \def\svgwidth{\columnwidth}
    \import{./figures/}{#1.pdf_tex}
}

% %http://tex.stackexchange.com/questions/76273/multiple-pdfs-with-page-group-included-in-a-single-page-warning
\pdfsuppresswarningpagegroup=1

\author{Aamod Varma}
\setlength{\parindent}{0pt}


\title{Linear Alebgra 3B}
\author{Aamod Varma}
\begin{document}
\maketitle
\date{}

\section*{3C}
\subsection*{Problem 1}
\begin{proof}
    Let us assume the contrary that the matrix of $T$ can be less than $\dim \range T$ non-zero entries. Now consider a basis of $W$ as $w_1,\dots,w_n$ such that $w_1,\dots,w_k$ spans $\range T$. Now because we have only $\dim \range T - 1$ non-zero entries in our matrix we have $r < k$ non-zero entries.

    So we would have a maximum of $r$ columns with non-zero entries in our matrix. By definition now, 
    \begin{align*}
        Tv_k = A_{1k}w_1 + \dots + A_{nk}w_n\\
    \end{align*}
    We also know that the definition of linear maps maps a vector in the basis of V to a vector in W. So,
    $$ Tv_1 = w_1 ,\dots,Tv_m = w_m$$ 

    However because we have a maximum of $r$ columns that are non-zero, we can only map  any v to a linear combination of $w_1,\dots,w_r$. But this means that $\dim range = r$ as any $v \in V$ is mapped to $r$ linearly independent vectors. But this contradicts our assumption that $\dim range = k$. Hence our assumption must be wrong.
\end{proof}



\subsection*{Problem 2}
\begin{proof}
    $\impliedby$ 

    First we know that if $\dim \range T = 1$ then  $\range T$ is spanned by a single arbitrary non-zero vector $w$. We nkow as $T$ is a linear map we can construct a basis $V$, $v_1,\dots,v_n$ as follows,


    First choose any $v_1 \in V$ such that $Tv_1 = b_1w$. Now let us extend this basis to $V$ as $v_1,\dots,v_m$. We change this basis to $v_1,v_2+v_1,\dots,v_m+v_1$. It is easy to show that this list is linearly independent and hence is a basis.
    
    For any vector in this basis we know that $T(v_k) = b_kw$ such taht $b_k \ne 0$ as $T(v_k) = T(v_{k'}) + T(v) = \lambda + b_1w$ where $\lambda$ may or may not be zero.
    where any $b \ne 0$ because for any $v$ such that $P(v) = 0$

    Hence we find, 
    $$ T(v_1) = b_1w,\dots,T(v_n) = b_nw $$

    Now we divide each side by $\frac{1}{b_k}$ to get, 
    $$ T(\frac{v_1}{b_1})= w,\dots,T(\frac{v_n}{b_n}) = w $$ 

    We can construct a basis of $W$ from $w$ to $w,\dots,w_n$. Hence we have our matrix as,  
    $$ T(v_1') = 1w + \dots + 0w_n $$ 
    $$ \dots$$
    $$ T(v_n') = 1w + \dots + 0w_n $$ 

    Now we know that $M(T)$ can be constructed with all 1s.
\end{proof}

\subsection*{Problem 3}
\begin{proof}
    (a). We know that the matrix of T is defined as, 
    $$ T(v_k) = A_{1k}w_1 + \dots + A_{nk}w_n $$ 
    and that of $S$ is defined as, 
    $$ S(v_k) =B_{1k}w_1 + \dots + B_{nk}w_n $$

    And for $(S+T)$ let it be, 
    $$ (S+T)v_k = C_{1k}w_1 + \dots + C_{nk}w_n $$ 
    We know that $(S+T)v = Sv + Tv$. So for the basis of $V$ given we have, 
    \begin{align*}
        (S+T)v_k &= Sv_k + Tv_k\\
                 &= (A_{1k}+B_{1k})w_1 + \dots + (A_{nk} + B_{nk})w_n
    \end{align*}

    So we have $C_{j,k} = A_{jk} + B_{jk}$. Which essentially means $M(S+T) = M(S) + M(T)$


    \vspace{2em}
    (b). We have $M(T)$ defined as  follows,  
    \begin{align*}
     (T) v_k  &= A_{11}w_n + \dots + A_{nk}w_n \\
     \lambda \times  (T)v_k &=  (\lambda A_{11}) w_1 + \dots + (\lambda A_{nk} w_n)\\
    \end{align*}
    Or in other words the matrix $\lambda M(T)$ is  defined as, 
    \begin{align*}
     \lambda \times  (T)v_k &=  (\lambda A_{11}) w_1 + \dots + (\lambda A_{nk} w_n)\\
    \end{align*}

    Now consider the matrix of $\lambda T$ we have, 
    \begin{align*}
     (T) v_k  &= A_{11}w_n + \dots + A_{nk}w_n \\
     (\lambda T)v_k &= (\lambda) \times  Tv_k\\
                    &= (\lambda A_{11}) w_1 + \dots + (\lambda A_{nk} w_n)\\
    \end{align*}

    Hence as each element of $\lambda M(T)$ is the same as that of $M(\lambda T)$ we have $M(\lambda T) = \lambda M(T)$


\end{proof}

\subsection*{Problem 4}
\begin{proof}
    Consider a basis of $P_3$ as $p_1,\dots,p_4$ and a basis of $P_2$ as $1 + x + x^2$. By definitino of linear map we have, 

    \begin{align*}
        Dp_1 &= 1 \\
        Dp_2 &= x\\
        Dp_3 &= x^2\\
        Dp_4 &= 0 \\
    \end{align*}

    So we have $p_1 = x, p_2 = x^2, p_3 = x^{3}, p_4 = 1$ as a basis of $P_3$


\end{proof}
\subsection*{Problem 5}
\begin{proof}
    Consider a basis of $V$ as $v_1,\dots,v_n$, we know as $T$ is a linear map we find, 
    $$ Tv_1 = w_1,\dots, Tv_n = w_n $$ 

    Let $\range T = k$ so we know that $w_1,\dots,w_n$ spans $\range T$ or in other words $w_1,\dots,w_n$ can be reduced to a basis of $\range T$ such that without loss of generality let them be ordered as $w_1,\dots,w_k$. 

    So we have, 
    \begin{align*}
        Tv_1 &= w_1\\
        &\dots\\
        Tv_k &= w_k
    \end{align*}

    Now we can find $v_{k+1},\dots,v_n \in \nul T$ such that $Tv_{k+1} = 0,\dots,Tv_n = 0$ as null of T is spanned by $n - k$ vectors.

    So let us extend our current list of  $v_1,\dots,v_k$ to $v_{k+1},\dots,v_n$. So our linear map is dfined as, 
    \begin{align*}
        Tv_1 &= w_1\\
        &\dots\\
        Tv_k &= w_k\\
        Tv_{k+1} &= 0\\
        &\dots\\
        Tv_{n} &= 0\\
    \end{align*}

    First let us extend our basis of $\range T$  to a basis of $W$ as $w_1,\dots,w_n$. Now let our definitio of $M(T)$ be as follows, 
    \begin{align*}
        Tv_1 &= 1w_1 + \dots + 0w_k + \dots + 0w_n\\
        &\dots\\
        Tv_k &= 0w_1 + \dots + 1w_k+ \dots + 0w_n\\
        Tv_{k+1} &= 0w_1 + \dots + 0w_k + \dots 0 w_n\\
        &\dots\\
        Tv_{n} &= 0w_1 + \dots + 0w_k + \dots 0w_n\\
    \end{align*}

    Hence based on how we define $M$ we have 1s at $A_{11},\dots,A_{kk}$ and $0$ 's everywhere else.

    % Where $w_1,\dots,w_k$ is  a basis of $\range T$. Now we know for any $v_x$ such that that $x > k$ that $T(v_x)$ can be written as a linear combination of $w_1,\dots,w_k$.

\end{proof}


\subsection*{Problem 6}
\begin{proof}
    First consider the case when $W = \range T$ and $\dim W = n$. And say we have basis of $V $ as $v_1,\dots,v_m$. Now consider $Tv_1 = w_1$. Let us extend this set of vectors to $w_1,\dots,w_n$. So we have, 
    $$ Tv_1 = w_1 $$ 
    $$ \dots $$ 
    $$ Tv_m = a_1w_1+\dots+ a_nw_n$$ 

    So we define our matrix as follows, 
    \begin{align*}
        Tv_1 &= 1w_1 + \dots + 0w_n \\
    &\dots \\
        Tv_m &= A_{1m}w_1+\dots+ A_{nm}w_n\\
    \end{align*}

    So for our first column we have all zeroes except for  $A_{11}$

    Now consider if $\dim \range T \le \dim \range W$. So $\exists v_k \in V$ such that $T(v_k) = 0 = 0w_1,\dots,0w_n$  for any basis of $W$. So we construct the matrix such that all in first column are zeroes.
\end{proof}
\subsection*{Problem 8}
\begin{proof}
    We need to show $(AB)_{j,.} = A_{j,.}B$

    First we have, 
         $$AB_{j,k} &= \sum_{r=1}^{n} A_{j,r}B_{r,k}\$$

         So, 
         \begin{align*}
             AB_{j,.}  &= (\sum_{r=1}^{r=n} A_{j,r}B_{r,1},  \dots, \sum_{r=1}^{r=n} A_{j,r}B_{r,n})\\
                       &= (A_{j,1},\dots,A_{j,n}) B\\
                       &= A_{j,.}B
         \end{align*}

\end{proof}

\subsection*{Problem 9}
\begin{proof}
    We have, 
    \begin{align*}
        (aB)_{1,k} &= \sum_{r=1}^{n} A_{1,n}B_{n,k}\\
        (aB) &= (\sum_{r=1}^{n} A_{1,n}B_{n,1},\dots,\sum_{r=1}^{n} A_{1,n}B_{n,1})\\
    \end{align*}

    Now we have, 
    $$ a_1B_{1,.} + \dots + a_nB_{n,.} = a_1B_{11}+ \dots + a_n B_{n,1} , \dots, a_1B_{n1} + \dots + a_n B_{nn}$$ 
    $$ = (\sum_{r=1}^{n} a_{n}B_{n,1},\dots,\sum_{r=1}^{n} a_{n}B_{n,1}) $$ 

    Hence we have our equality.
\end{proof}
\subsection*{Problem 10}
\begin{proof}
    Let $A = (1,0;0,0)$ and $B = (0,1;0,0)$. We have AB = (0,1;0,0) and $BA = (0,0;0,0)$
\end{proof}

\subsection*{Problem 11}
\begin{proof}
    Let $B + C = X$ such that,
    First be have $(B + C)_{j,k} = B_{j,k} + C_{j,k} = X_{j,k}$

    So we have, 
    $$ A(B + C) = AX $$ 
    \begin{align*}
        (AX)_{j,k} &= \sum_{r=1}^{n} A_{j,r} X_{r,k}\\
                   &= \sum_{r=1}^{n} A_{j,r}(B_{r,k} + C_{r,k})\\
                   &= \sum_{r=1}^{n} A_{j,r}(B_{r,k})  + A_{j,r}(C_{r,k})\\
                   &= \sum_{r=1}^{n} A_{j,r}(B_{r,k})  + \sum_{r=1}^{n} A_{j,r}(C_{r,k})\\
                   &= AB_{j,k} + AC_{j,k}
    \end{align*}

    Hence we have $A(B + C) = AB + AC$
\end{proof}

\subsection*{Problem 12}
\begin{proof}
    We know if $T$ and $S$ are linear map from $U,V$ and $V,W$ respectively then $M(ST) = M(S)M(T)$

    Let  $A = M(T), B = M(S), C = M(R)$ for a linear map $T,S,R$ So we have, 
    \begin{align*}
        (AB)C &= (M(T)M(S))M(R)\\
              &= (M(TS))M(R)\\
              &= M((TS)R)\\
              &= M(T(SR))\\
              &= M(T)M(SR)\\
              &= M(T)(M(S)M(R))\\
              &= A(BC)
    \end{align*}

\end{proof}

\subsection*{Problem 13}

\begin{proof}
    
We know that 
$$ (AA)_{j,k} = \sum_{r = 1}^n A_{j,r} A_{r,k} $$ 

$$ (A(AA))_{j,k} = \sum_{r = 1}^n A_{j,r} (AA)_{r,k}$$ 

$$ (A^{3})_{j,k} = \sum_{p = 1}^n A_{j,r}( \sum_{x = 1}^n A_{r,x} A_{x,k}  )$$ 

$$ (A^{3})_{j,k} = \sum_{p = 1}^n \sum_{x = 1}^n A_{j,r} A_{r,x} A_{x,k}$$ 


\end{proof}
\subsection*{Problem 14}
\begin{proof}
    To show that the functino is a linear map we need to show additvity and homogenity. Consider $A \in F^{m,n}$  and $B \in F^{m,n}$. 

    1. Additive. 

   We need to show that $(A+B)^{t} = A^{t} + B^{t}$ . First we know that $(A+B)_{j,k} &= A_{j,k} + B_{j,k}$

   \begin{align*}
       (A+B)^{t}_{j,k} &= (A+B)_{k,j}\\
                   &= A_{k,j} + B_{k,j}\\
                   &= A^{t}_{j,k} + B^{t}_{j,k}
   \end{align*}

   So we have $(A+B)^{t} = A^{t} + B^{t}$

    2. Scalar multiplication.

    We need to show that $(kA)^{t} = k A^{t}$

    We have $(kA)_{j,k} = k(A_{j,k})$ . So $$(kA)^{t}_{j,k} = (kA)_{k,j}$$
    $$ = k A_{k,j} $$ 

    So we have $(kA)^{t} = k A^{t}$


\end{proof}

\subsection*{Problem 15}
\begin{proof}
    We have, 
    $$ AC_{j,k} =  \sum_{r=1}^{n} A_{j,n} C_{n, k}$$

    So $AC^{t}_{j,k} = AC_{k,j} $ which is, 
    $$ = \sum_{r=1}^{n} A_{k,r} C_{r, j}$$  

    Now $C^{t}_{j,k} = C_{k,j} $ and $A^{t}_{j,k} = A_{k,j}$ . So $$(C^{t}A^{t})_{j,k} = \sum_{r=1}^{n} C_{r,j} A_{k,r}$$
    
    $$  = \sum_{r=1}^{n} A_{k,r} C_{r, j} $$ 
    $$ = (AC)^{t}_{j,k} $$ 


    

\end{proof}

\subsection*{Problem 16}
\begin{proof}
    $\impliedby$
    First we know that for any rank $k$ matrix we can write it as a prodcut of $RC$ such that $R = m \times  k$ and $C =k \times  n$ if $A$ is a $m \times n$ matrix. So if $A$ is a rnak 1 matrix we can write it as a $m \times  1$ times $1 \times n $ product of matrices.

    Let $C = (c_1,\dots,c_m)^{T}$ and $R = (d_1,\dots,d_n)$

    So we have $A_{j,k} = C_{j,1}R_{1,k} = c_jd_k$

    $\implies$
    We have $A_{j,k} = c_jd_k$. Let  $C \in F^{m,1}$ where $C_{j,1} = c_j$. Now we have  $A_{.,k} = d_k C$. So each column of  $A$ is a scalar multiplcatino of $C$. Which means that our matrix $A$ has rank 1.
\end{proof}



\subsection*{Problem 17}
\begin{proof}
    (a) $\implies$ (b)
    We need to show that $A_{.,1},\dots,A_{.,n}$ are linearly independent or that, if  
    $$ \lambda_1A_{.,1}  + \dots + \lambda_n A_{.,n} = 0$$ The only solution is all  $\lambda= 0$

    By definitino of matrix we have,  
    $$ Tv_k= A_{1,k}u_1 + \dots + A_{n,k}u_n $$ 

    Consider $\lambda_k Tv_k = \lambda_k A_{1,k}u_1 + \dots + \lambda_k A_{n,k}$ 

    Now as $T$ is injective we have $\lambda_k T v_k = 0$. This can only be true if $\lambda_k = 0$. Which means our columns are linearly independent.
\end{proof} 


\end{document}
