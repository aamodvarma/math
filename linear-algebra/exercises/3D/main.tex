\documentclass[a4paper]{report}
\usepackage[utf8]{inputenc}
\usepackage[T1]{fontenc}
\usepackage{textcomp}

\usepackage{url}

% \usepackage{hyperref}
% \hypersetup{
%     colorlinks,
%     linkcolor={black},
%     citecolor={black},
%     urlcolor={blue!80!black}
% }

\usepackage{graphicx}
\usepackage{float}
\usepackage[usenames,dvipsnames]{xcolor}

% \usepackage{cmbright}

\usepackage{amsmath, amsfonts, mathtools, amsthm, amssymb}
\usepackage{mathrsfs}
\usepackage{cancel}

\newcommand\N{\ensuremath{\mathbb{N}}}
\newcommand\R{\ensuremath{\mathbb{R}}}
\newcommand\F{\ensuremath{\mathscr{F}}}
\newcommand\Z{\ensuremath{\mathbb{Z}}}
\renewcommand\O{\ensuremath{\emptyset}}
\newcommand\Q{\ensuremath{\mathbb{Q}}}
\newcommand\C{\ensuremath{\mathbb{C}}}
\let\implies\Rightarrow
\let\impliedby\Leftarrow
\let\iff\Leftrightarrow
\let\epsilon\varepsilon

% horizontal rule
\newcommand\hr{
    \noindent\rule[0.5ex]{\linewidth}{0.5pt}
}

\usepackage{tikz}
\usepackage{tikz-cd}

% theorems
\usepackage{thmtools}
\usepackage[framemethod=TikZ]{mdframed}
\mdfsetup{skipabove=1em,skipbelow=0em, innertopmargin=5pt, innerbottommargin=6pt}

\theoremstyle{definition}

\makeatletter

\declaretheoremstyle[headfont=\bfseries\sffamily, bodyfont=\normalfont, mdframed={ nobreak } ]{thmgreenbox}
\declaretheoremstyle[headfont=\bfseries\sffamily, bodyfont=\normalfont, mdframed={ nobreak } ]{thmredbox}
\declaretheoremstyle[headfont=\bfseries\sffamily, bodyfont=\normalfont]{thmbluebox}
\declaretheoremstyle[headfont=\bfseries\sffamily, bodyfont=\normalfont]{thmblueline}
\declaretheoremstyle[headfont=\bfseries\sffamily, bodyfont=\normalfont, numbered=no, mdframed={ rightline=false, topline=false, bottomline=false, }, qed=\qedsymbol ]{thmproofbox}
\declaretheoremstyle[headfont=\bfseries\sffamily, bodyfont=\normalfont, numbered=no, mdframed={ nobreak, rightline=false, topline=false, bottomline=false } ]{thmexplanationbox}


\declaretheorem[numberwithin=chapter, style=thmgreenbox, name=Definition]{definition}
\declaretheorem[sibling=definition, style=thmredbox, name=Corollary]{corollary}
\declaretheorem[sibling=definition, style=thmredbox, name=Proposition]{prop}
\declaretheorem[sibling=definition, style=thmredbox, name=Theorem]{theorem}
\declaretheorem[sibling=definition, style=thmredbox, name=Lemma]{lemma}



\declaretheorem[numbered=no, style=thmexplanationbox, name=Proof]{explanation}
\declaretheorem[numbered=no, style=thmproofbox, name=Proof]{replacementproof}
\declaretheorem[style=thmbluebox,  numbered=no, name=Exercise]{ex}
\declaretheorem[style=thmbluebox,  numbered=no, name=Example]{eg}
\declaretheorem[style=thmblueline, numbered=no, name=Remark]{remark}
\declaretheorem[style=thmblueline, numbered=no, name=Note]{note}

\renewenvironment{proof}[1][\proofname]{\begin{replacementproof}}{\end{replacementproof}}

\AtEndEnvironment{eg}{\null\hfill$\diamond$}%

\newtheorem*{uovt}{UOVT}
\newtheorem*{notation}{Notation}
\newtheorem*{previouslyseen}{As previously seen}
\newtheorem*{problem}{Problem}
\newtheorem*{observe}{Observe}
\newtheorem*{property}{Property}
\newtheorem*{intuition}{Intuition}


\usepackage{etoolbox}
\AtEndEnvironment{vb}{\null\hfill$\diamond$}%
\AtEndEnvironment{intermezzo}{\null\hfill$\diamond$}%




% http://tex.stackexchange.com/questions/22119/how-can-i-change-the-spacing-before-theorems-with-amsthm
% \def\thm@space@setup{%
%   \thm@preskip=\parskip \thm@postskip=0pt
% }

\usepackage{xifthen}

\def\testdateparts#1{\dateparts#1\relax}
\def\dateparts#1 #2 #3 #4 #5\relax{
    \marginpar{\small\textsf{\mbox{#1 #2 #3 #5}}}
}

\def\@lesson{}%
\newcommand{\lesson}[3]{
    \ifthenelse{\isempty{#3}}{%
        \def\@lesson{Lecture #1}%
    }{%
        \def\@lesson{Lecture #1: #3}%
    }%
    \subsection*{\@lesson}
    \testdateparts{#2}
}

% fancy headers
\usepackage{fancyhdr}
\pagestyle{fancy}

% \fancyhead[LE,RO]{Gilles Castel}
\fancyhead[RO,LE]{\@lesson}
\fancyhead[RE,LO]{}
\fancyfoot[LE,RO]{\thepage}
\fancyfoot[C]{\leftmark}
\renewcommand{\headrulewidth}{0pt}

\makeatother

% figure support (https://castel.dev/post/lecture-notes-2)
\usepackage{import}
\usepackage{xifthen}
\pdfminorversion=7
\usepackage{pdfpages}
\usepackage{transparent}
\newcommand{\incfig}[1]{%
    \def\svgwidth{\columnwidth}
    \import{./figures/}{#1.pdf_tex}
}

% %http://tex.stackexchange.com/questions/76273/multiple-pdfs-with-page-group-included-in-a-single-page-warning
\pdfsuppresswarningpagegroup=1

\author{Aamod Varma}
\setlength{\parindent}{0pt}


\title{Linear Alebgra 3D}
\author{Aamod Varma}
\begin{document}
\maketitle
\date{}


\subsection*{Problem 1}
\begin{proof}
    We are given that $T^{-1}$ is the inverse of $T$ which menas that $TT^{-1} = I$ and $T^{-1}T = I$. Now we know that if $AB = BA = I$ then $B$ is the inverse of $T$. So similarly we see that $T$is the inverse of $T^{-1}$. Hence $(T^{-1})^{-1} = T$
\end{proof}
\subsection*{Problem 2}

\begin{proof}
   We need to show that $(ST)^{-1} = T^{-1}S^{-1}$. If $ST$ is invertiebl  we can find $M$ such that $STM =I$ and $MST = I$. Let  $M =T^{-1}S^{-1} $

   So we have, 
   $$ STM = ST(T^{-1}S^{-1}) $$ 
   $$ = STT^{-1}S^{-1} $$ 
   $$ = SS^{-1} $$ 
   $$ = I $$ 

   And, 
   $$ MST = T^{-1}S^{-1}ST $$ 
   $$ = T^{-1}T $$ 
   $$ =I $$ 

   Hence $ST$ is invertible and the inverse is  $T^{-1}S^{-1}$


\end{proof}



\subsection*{Problem 3}
\begin{proof}
    (a) $\implies$ (b)
    We know that $T$ is invertiebl hence it is both injective and surjective. We can say that $Tv_1,\dots,Tv_n$ spans range of $T$. But we nkow that $\range T = V$ as it is injective. We also know that $\dim V = n = \dim \range T$. 

    Hence we have n vectors $Tv_1,\dots,Tv_n$ that span $V$ and because the number of vectors are equal to basis length the vectors themselve form a basis of V.

    (b) $\implies $(c) 
    (c) is just a specific case of (b)

    (c) $\implies$ (a)
    For some basis $v_1,\dots,v_n$ we know that $Tv_1,\dots,Tv_n$ is a basis of $V$. So that means that for any $v \in V,$  v is in the span of $Tv_1,\dots,Tv_n$ or that it is in the range of T. This means that $V$ is surjective. As it is a map onto itself and T is surjective this means T is also injective. Hence T is invertiebl.
\end{proof}

\subsection*{Problem 5}
\begin{proof}
    $\impliedby$

    We have $T$ is an invertiebl linear map such that $Tu = Su, \forall u \in U$ and we need to show that $S$ is injective.

    For any $u_1,u_2 \in V$ we need to show that $Su_1 = Su_2 \implies u_1 = u_2$

    We know that $Su_1 = Tu_1$ and $Su_2 = Tu_2$. So we have $Tu_1 = Tu_2$. But we know that $T$ is invertibel hence it is also injective. So we have $u_1 = u_2$. Hence S is injective.


    $\implies$ 
    We know that $S$ is injective from $U$ to $V$, we need to construct a map $T$ such that $Tu = Su$ but T is invertible (both injective and surjective).

    First consider the basis of $U$ as $u_1,\dots,u_n$. We know S is injective hence null S =  \{0\}. So we know that $S $ can be written as, 
    \begin{align*}
        Su_1 = v_1,\dots,Su_n = v_n
    \end{align*}
    Such that $v_1,\dots,v_n$ span $\range S$ and as  $\dim \range S = \dim U = n$. $v_1,\dots,v_n$ are linearly independent.

    Now let us first extend our basis of U to a basis of $V$ as follows,$ u_1,\dots, u_m$ and extned our basis of $\range S$ to a basis of $V$ as $v_1,\dots,v_m$.

    Let us define a linear map as follows, 
    \begin{align*}
        Tu_1 &= Su_1 = v_1\\
             &\dots\\
        Tu_n &= Su_n = v_n\\
        Tu_{k}&= v_k \text{ for k >  n}
    \end{align*}

    Now based on our definition we have range $T$ is spanned by $v_1,\dots,v_n$ whichi spans $V$. Hence range T = V. Which both means that it is surjective and because null space is $\{0\}$ we also have injectivity. Hence T is invertible.

    Based on our definitino $T(u) = S(u)$ is also true, as for any $u = a_1u_1+ \dots a_nu_n$ we have, $S(u) =a_1Su_1 + \dots + a_nSu_n = a_1Tu_1+ \dots + a_nTu_n = T(a_1u_1+\dots + a_nu_n) = T(u)$
\end{proof}


\subsection*{Problem 9}
\begin{proof}
    We know $T$ is surjective we need to show there is a subspace $U$ of $V$ such that $T|_U$ is an isomophism of U onto W. 

    First because $T$ is surjection we have $\range T = W$. Consider $v_1,\dots,v_n$ is a basis on $V$ then we can say $T$ is defined as, 
    \begin{align*}
        Tv_1 = w_1,\dots,Tv_n = w_n 
    \end{align*}
    
    Now because $w_1,\dots,w_n$ span $W$ let us reduce it to a basis of $W$, take this as $w_1,\dots,w_k$ without loss of generality.  Now consider the subspace spanned by $v_1,\dots,v_k$. Now we need to show it is an isomophism or that it is injective and surjective.

    Firstly we know it is surjective as it maps to $w_1,\dots,w_k$ which we nkow span $W$. We need to show that it is injective.

    Because U is spanned by $n$ vectors and $W$ is spanned by $n$ vectors we nkow by rank nullity theorem that $\dim \nul T= 0$ hence  our map is injective. 

\end{proof}

\subsection*{Problem 11}
\begin{proof}
    $\implies$ We know that $ST$ is invertieble. Assume to the contrary that $S$ or $T$ is not invertiebl.

    (1). We have $S$ is not invertiebl and $ST = S(T(v))$  is invertible this means that $\forall v \in V$ we can find $v' \in V$ such that $S(T(v')) = v$.  So if $T(v') = v''$ then $S(v'') = v$ for any $v$. But this means that $S$ is surjective. However based on our assumption we nkow S is not surjective. So this contradicts our assumption.

    (2). Consider $T$ is not invertible $\implies$ T is not injective. If T is not injective the $\exists v \in V$ such that $Tv = 0$. So we have $STv = S(0) = 0$ which means that $STv = 0$  for some non-zero v which makes ST not injective and not invertiebl. Hence a contradiction.

    $\impliedby$ 
    We can show $\exists M = T^{-1}S^{-1}$ such that $STM =I, MST = I$
\end{proof}

\subsection*{Problem 12}
We have $STU = I$ this means that $ST$ and  $U$ are both invertible which means that $S$ and $T$ are also invertibel. So $\exists S^{-1}$ and $U^{-1}$.

\begin{align*}
    STU &= I\\
    TU &= S^{-1}\\
    U &= T^{-1}S^{-1}\\
    US &= T^{-1}S^{-1}S\\
    T^{-1} &= US\\
\end{align*}


\subsection*{Problem 13}
\begin{proof}
    Consider the backward shift operator $T((x_1,\dots)) = (x_2,\dots)$. It is not injective and hence not invertible. Let $S$ be $I$ and $U$ be the forward shift operator. We have $STU = I$ but $T$ is not invertible.
\end{proof}
\subsection*{Problem 14}
\begin{proof}
    RST is a map from V to V hence $RST$ is invertieble. This means that RST is injective. First lets show that  $T$ is invertible. Assume T is not invertiebl means that $T$ is not injective or $\exists v \in V$ such that $Tv = 0$ but this means that $RSTv = 0$ but this is a contradictino as RST is invertieble.  Now we know that $T$ is invertible so range $T$ is V.

    Now assume S is not injective $\exists v \in \range T = V$ such that $S(v) = 0$ but this gives us a contradictino as well as RST  is invertiebl. Hence $S$ is injective.

\end{proof}
\subsection*{Problem 15}
\begin{proof}
    We know that $Tv_1,\dots,Tv_m$ span V. So, 
    $$ a_1Tv_1 + \dots + a_nTv_m = 0 $$  only if $a_1,\dots,a_n = 0$

    Nonw lets assume the contrary that $v_1,\dots,v_n$ are not inveritble so there exists not all zero $b_1,\dots,b_n$ such that
    $$ b_1v_1 + \dots + b_nv_m = 0 $$ 

    Let us take $T(v)$ so,  
    $$ T(b_1v_1+ \dots + b_nv_n) = T(0) = 0 $$ 
    $$ b_1T(v_1) + \dots + b_nT(v_n) = 0 $$ 

    If we have $b_1,\dots,b_n$ all not zero then that means that $T(v_1) ,\dots,T(v_n)$ is not lin independent hence a contradiction.
\end{proof}
\subsection*{Problem 16}
\begin{proof}
    
    We know that $Tx$ in $F^{m,1}$ so $Tx = M(Tx)$,  
    $$ = M(T)M(x) $$ 
    $$ Ax $$ 
\end{proof}
\subsection*{Problem 20}
\begin{proof}
    
\end{proof}
\subsection*{Problem 21}



\end{document}
