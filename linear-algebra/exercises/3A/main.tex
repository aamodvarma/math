\documentclass[a4paper]{report}
\usepackage[utf8]{inputenc}
\usepackage[T1]{fontenc}
\usepackage{textcomp}

\usepackage{url}

% \usepackage{hyperref}
% \hypersetup{
%     colorlinks,
%     linkcolor={black},
%     citecolor={black},
%     urlcolor={blue!80!black}
% }

\usepackage{graphicx}
\usepackage{float}
\usepackage[usenames,dvipsnames]{xcolor}

% \usepackage{cmbright}

\usepackage{amsmath, amsfonts, mathtools, amsthm, amssymb}
\usepackage{mathrsfs}
\usepackage{cancel}

\newcommand\N{\ensuremath{\mathbb{N}}}
\newcommand\R{\ensuremath{\mathbb{R}}}
\newcommand\F{\ensuremath{\mathscr{F}}}
\newcommand\Z{\ensuremath{\mathbb{Z}}}
\renewcommand\O{\ensuremath{\emptyset}}
\newcommand\Q{\ensuremath{\mathbb{Q}}}
\newcommand\C{\ensuremath{\mathbb{C}}}
\let\implies\Rightarrow
\let\impliedby\Leftarrow
\let\iff\Leftrightarrow
\let\epsilon\varepsilon

% horizontal rule
\newcommand\hr{
    \noindent\rule[0.5ex]{\linewidth}{0.5pt}
}

\usepackage{tikz}
\usepackage{tikz-cd}

% theorems
\usepackage{thmtools}
\usepackage[framemethod=TikZ]{mdframed}
\mdfsetup{skipabove=1em,skipbelow=0em, innertopmargin=5pt, innerbottommargin=6pt}

\theoremstyle{definition}

\makeatletter

\declaretheoremstyle[headfont=\bfseries\sffamily, bodyfont=\normalfont, mdframed={ nobreak } ]{thmgreenbox}
\declaretheoremstyle[headfont=\bfseries\sffamily, bodyfont=\normalfont, mdframed={ nobreak } ]{thmredbox}
\declaretheoremstyle[headfont=\bfseries\sffamily, bodyfont=\normalfont]{thmbluebox}
\declaretheoremstyle[headfont=\bfseries\sffamily, bodyfont=\normalfont]{thmblueline}
\declaretheoremstyle[headfont=\bfseries\sffamily, bodyfont=\normalfont, numbered=no, mdframed={ rightline=false, topline=false, bottomline=false, }, qed=\qedsymbol ]{thmproofbox}
\declaretheoremstyle[headfont=\bfseries\sffamily, bodyfont=\normalfont, numbered=no, mdframed={ nobreak, rightline=false, topline=false, bottomline=false } ]{thmexplanationbox}


\declaretheorem[numberwithin=chapter, style=thmgreenbox, name=Definition]{definition}
\declaretheorem[sibling=definition, style=thmredbox, name=Corollary]{corollary}
\declaretheorem[sibling=definition, style=thmredbox, name=Proposition]{prop}
\declaretheorem[sibling=definition, style=thmredbox, name=Theorem]{theorem}
\declaretheorem[sibling=definition, style=thmredbox, name=Lemma]{lemma}



\declaretheorem[numbered=no, style=thmexplanationbox, name=Proof]{explanation}
\declaretheorem[numbered=no, style=thmproofbox, name=Proof]{replacementproof}
\declaretheorem[style=thmbluebox,  numbered=no, name=Exercise]{ex}
\declaretheorem[style=thmbluebox,  numbered=no, name=Example]{eg}
\declaretheorem[style=thmblueline, numbered=no, name=Remark]{remark}
\declaretheorem[style=thmblueline, numbered=no, name=Note]{note}

\renewenvironment{proof}[1][\proofname]{\begin{replacementproof}}{\end{replacementproof}}

\AtEndEnvironment{eg}{\null\hfill$\diamond$}%

\newtheorem*{uovt}{UOVT}
\newtheorem*{notation}{Notation}
\newtheorem*{previouslyseen}{As previously seen}
\newtheorem*{problem}{Problem}
\newtheorem*{observe}{Observe}
\newtheorem*{property}{Property}
\newtheorem*{intuition}{Intuition}


\usepackage{etoolbox}
\AtEndEnvironment{vb}{\null\hfill$\diamond$}%
\AtEndEnvironment{intermezzo}{\null\hfill$\diamond$}%




% http://tex.stackexchange.com/questions/22119/how-can-i-change-the-spacing-before-theorems-with-amsthm
% \def\thm@space@setup{%
%   \thm@preskip=\parskip \thm@postskip=0pt
% }

\usepackage{xifthen}

\def\testdateparts#1{\dateparts#1\relax}
\def\dateparts#1 #2 #3 #4 #5\relax{
    \marginpar{\small\textsf{\mbox{#1 #2 #3 #5}}}
}

\def\@lesson{}%
\newcommand{\lesson}[3]{
    \ifthenelse{\isempty{#3}}{%
        \def\@lesson{Lecture #1}%
    }{%
        \def\@lesson{Lecture #1: #3}%
    }%
    \subsection*{\@lesson}
    \testdateparts{#2}
}

% fancy headers
\usepackage{fancyhdr}
\pagestyle{fancy}

% \fancyhead[LE,RO]{Gilles Castel}
\fancyhead[RO,LE]{\@lesson}
\fancyhead[RE,LO]{}
\fancyfoot[LE,RO]{\thepage}
\fancyfoot[C]{\leftmark}
\renewcommand{\headrulewidth}{0pt}

\makeatother

% figure support (https://castel.dev/post/lecture-notes-2)
\usepackage{import}
\usepackage{xifthen}
\pdfminorversion=7
\usepackage{pdfpages}
\usepackage{transparent}
\newcommand{\incfig}[1]{%
    \def\svgwidth{\columnwidth}
    \import{./figures/}{#1.pdf_tex}
}

% %http://tex.stackexchange.com/questions/76273/multiple-pdfs-with-page-group-included-in-a-single-page-warning
\pdfsuppresswarningpagegroup=1

\author{Aamod Varma}
\setlength{\parindent}{0pt}


\title{Linear Alebgra 3A}
\author{Aamod Varma}
\begin{document}
\maketitle
\date{}


\section*{3A}

\subsection*{Problem 1}
\begin{proof}
    
We know for a linear map, $T(u + v) = T(u) + T(v)$ and $T(\lambda v ) = \lambda T(v)$

First we look at additivity, 

Consider an arbitrary  $u = (x_1,y_1,z_1)$ and $v = (x_2,y_2,z_2)$. So we have, 
$$ T(u + v) = T((x_1+x_2),(y_1+y_2),(z_1+z_2))$$ $$ = (2(x_1+x_2) - 4(y_1+y_2) + 3(z_1+z_2) + b, 6(x_1+x_2) + c(x_1+x_2)(y_1+y_2)(z_1+z_2)) $$ 


We need the above to be equal to, 
$$ T(u) + T(v) = (2x_1-4y_1+3z_1 + b, 6x_1 + cx_1y_1z_1) + (2x_2-4y_2+3z_2 + b, 6x_2 + cx_2y_2z_2) $$ 
$$ = (2(x_1+x_2) - 4(y_1+y_2) + 3(z_1+z_2) + 2b, 6(x_1+x_2) + c(x_1y_1z_1+x_2y_2z_2)$$ 

Comparing each of the terms we have, 
$$ 2(x_1+x_2) - 4(y_1+y_2) + 3(z_1+z_2)  + 2b  =  2(x_1+x_2) - 4(y_1+y_2) + 3(z_1+z_2) + b$$ 
$$ 2b = b $$ 
$$ b = 0 $$ 

Similarly comparing the second term we have, 
$$ 6(x_1+x_2) + c(x_1+x_2)(y_1+y_2)(z_1+z_2) =  6(x_1+x_2) + c(x_1y_1z_1+x_2y_2z_2) $$ 
$$ c(x_1+x_2)(y_1+y_2)(z_1+z_2) = c(x_1y_1z_1+x_2y_2z_2) $$ 
$$ c((x_1+x_2)(y_1+y_2)(z_1+z_2) - (x_1y_1z_1+x_2y_2z_2)) = 0 $$ 

For this to be true for any $x,y,z$ we need  $c = 0$. Hence for additvity we need $b = c = 0$

Now we check if $T(k v) = k T(v)$. Consider $v = (x,y,z)$. Then we have 
$$ T(k v) = T(k x,k y,k z) = (2kx - k4y + 3kz + b, 6kx + k^{3}cxyz) $$ 

We need this to be equal to $$kT(v) = k(2x - 4y + 3z + b, 6x + cxyz) = (2kx- 4ky + 3kz + bk, 6kx + kcxyz) $$

Comparing the terms we have, 
$$ 2kx - 4ky + 3kz + bk  = 2kx - 4ky + 3kz + b$$ 
$$ bk = b $$ 
$$b = 0$$

$6kx + kcxyz = 6kx+ k^{3}cxyz$ 
$$ c = k^2 c$$ 
$$ c = 0 $$ 

So we have $b = c = 0$

\end{proof}

\subsection*{Problem 2}

\begin{proof}
    Similar to (1) but take $p_1 = a_1 + b_1x$ and $p_2 = a_2 + b_2x$
\end{proof}


\subsection*{Problem 3}
\begin{proof}
    Consider the standard basis $e_1,\dots,e_n$ of $F^{n}$. That is  
    \begin{align*}
        e_1 &= (1,\dots,0)\\
        e_2 &= (0,1,\dots,0)\\
        &\dots\\
        e_n &= (0,\dots,1)
    \end{align*}

    We have 
    \begin{align*}
        T(x_1,\dots,x_n)  &= T(x_1(1,\dots,0), x_2(0,1,\dots,0),\dots x_n(0,\dots,1))\\
                      &=  T(x_1e_1,\dots,x_ne_n)\\
                      &= x_1T(e_1) + \dots + x_nT(e_n)
    \end{align*}    
    Let $T$ map $e_1$ to $(A_{11},\dots,A_{m1})$ and $e_n$ to $(A_{1n},\dots,A_{mn})$

    So we have, 
    \begin{align*}
        &= x_1(A_{11},\dots,A_{m1}) + \dots x_n(A_{1n},\dots,A_{mn})\\
        &= (x_1A_{11} + \dots + x_nA_{1n}, \dots , x_1A_{m1} + \dots + x_nA_{mn})
    \end{align*}
\end{proof}

\subsection*{Problem 4}
\begin{proof}
    Let us assume the contrary that $v_1,\dots,v_m$ is linearly dependent. This means that $\exists, a_1,\dots,a_m$ not all zero such that, 
    $$ a_1v_1 + \dots + a_mv_m = 0 $$ 

    Now let us apply the lienar map on this vector and we get, 
    $$ T(a_1v_1 + \dots + a_mv_m) = T(0) = 0 $$ 
    $$ a_1T(v_1) + \dots + a_mT(v_m) = 0 $$ 

    Here we see that $\exists$ scalars $a_1,\dots,a_m$ not all zero such that the linear combination of $T(v_1), \dots ,T(v_m)$ is equal to zero. This means that the list of vectors are linearly dependent. However we know that the list is linearly independent. Hence our assumption must be wrong and $v_1,\dots, v_m$ are actually linearly independent.
\end{proof}

\subsection*{Probelm 5}
\begin{proof}
We need to show addivitiy and homogenity.

(1). Additivity

We need to show for any $T_1, T_2 \in L(V,W)$  that $T_1 + T_2 \in L(V,W)$. In other words we need to show that $T_1 + T_2 $ is also a linear map.

Consider $v_1,v_2 \in V$ we have $$(T_1+T_2)(v_1+v_2) = T_1(v_1+v_2) + T_2(v_1 + v_2)$$
$$ = T_1(v_1) + T_2(v_1) + T_1(v_2) + T_2(v_2) $$ 
$$ = (T_1+T_2)(v_1) + (T_1+T_2)(v_2) $$ 

Hence we show that $T_1 + T_2$ is additive.

Now consider $v_1 \in V$ we have $(T_1+T_2)(\lambda v_1)$ we get, 
$$  = T_1(\lambda v_1) + T_2(\lambda v_1)  $$ 
$$ = \lambda T_1(v_1) + \lambda T_2(v_1) $$ 
$$ = \lambda (T_1v_1 + T_2v_1) $$ 
$$ = \lambda (T_1+T_2)v_1 $$ 

Which means that it is homogenous.

Hence we show that $(T_1+T_2) \in L(V,W)$ or that $L(V,W)$ is additive.

(2). Homogenous

Consider $T \in L(V,W)$ we  need to show that  $\lambda T$ is a linear map as well.

First we show that $\lambda T$ is additive. Consider $v_1,v_2$,we have, 
$$ (\lambda T) (v_1 + v_2) =\lambda (T)(v_1+v_2) $$ 
$$ = \lambda (Tv_1 + Tv_2) $$ 
$$ = \lambda Tv_1 + \lambda Tv_2 $$ 
Which shows that $\lambda T$ is additive.

Now we check homogenous, consider $v \in V$ and $k \in F$ we have,  
$$ (\lambda T)(kv) = \lambda (T)(kv) $$ 
$$ = \lambda k T(v) $$ 
$$ = k (\lambda T)v $$ 

Hence we show that $\lambda T$ is homogenous. This makes $\lambda T$ a linear map.

Therefore we show that  $L(V,W)$ is a vector space.



\end{proof}


\subsection*{Problem 6}
\begin{proof}
     
1. Associativity. We have $(T_1T_2)T_3 = T_1(T_2T_3)$

Consider the operation on a vector $v$ so we have,  $(T_1T_2)T_3v$ which is, 
$$ ((T_1T_2)(T_3(v)) = T_1(T_2(T_3(v)))$$ 

Now looking at the right side we have, $T_1(T_2T_3) = T_1(T_2(T_3(v)))$. So we showed tha tthe LHS is equal to the RHS.

2. Identity. Consider a vector $v$ we have,  
$$ TIv = T(I(v)) = T(v)$$ 

Now, 
$$ ITv = I(T(v))  = T(v) \text{ because $Iv = v, \forall v$}  $$ 

3. Distributive Property

To show that, 
$$ (S_1+S_2)T  = S_1T + S_2T$$ 

Consider an abitrary vector $v$ in the domain of $T$. We have,  
$$ (S_1+S_2)Tv = (S_1+S_2)(T(v)) $$ 
By definitino of addition of linera maps we have, 
$$= (S_1(T(v))) + (S_2(T(v)))$$ 

Simliary we have, 
$$ (S_1T + S_2T)v = S_1T(v) + S_2T(v) = S_1(T(v)) + S_2(T(v))$$ 

We see that the distributive property holds.


Now To show that $S(T_1+T_2) = ST_1 + ST_2$. Consider $v$ we have, 
$$ S(T_1+T_2)v = S(T_1(v) + T_2(v)) = S(T_1(v)) + S(T_2(v)) $$ 

And we have, 
$$ (ST_1 + ST_2)v = ST_1(v) + ST_2(v) = S(T_1(v)) + S(T_2(v))$$ 

We see that the property holds again.



\end{proof}

\subsection*{Problem 7}
\begin{proof}
    As $T$ is a linear map from V to itself and $V$ is one dimentional say with basis $\{v'\}$. Then $T$ is defined as  
    $$ T(v') = \lambda v' $$ for some $\lambda $

    Now we know  $\forall v \in V$ we can write v as a linear combination of the basis of $V$, or 
    $$ v = kv'  $$ for some $k \in F$.

    We have,  
    $$ T(v') = \lambda v' $$ 
    $$ kT(v') = k\lambda v' $$ 
    $$ T(kv') = \lambda (kv') $$ 
    $$ T(v) = \lambda (v) $$ 
\end{proof}

\subsection*{8}
\begin{proof}
    Consider the function that maps any vecotor $(x,y)$ to the $max(|x|,|y|)$. We can see that this satisfies homogeneity. For instance consider  $(2,6)$. Our function maps this to  $6$. Now consider  $(2 \times 3, 6 \times 3)$  which is mapped to $18$ which is  $3 \times  6$ as we saw above. 

    Now consider two vector $(1,0)$ and  $(0,4)$. Our function maps both these vectors to  $1$ and $4$ respectively. However it maps its sum  $(1,4)$ to  $4 \ne 4 + 1$. Hence it does not follow additvity. Hence not a linear space.
\end{proof}


\subsection*{Problem 9}
\begin{proof}
    Consider the functino that maps any complex number $x + iy$ to $x$. First we show this functino is linear.

    Consider two complex number $x_1 + iy_1$ and $x_2 + iy_2$. We have $$f(z_1 + z_2) = f((x_1 + x_2) + i(y_1 + y_2)$$
    $$ = x_1 + x_2 $$ 
    $$ = f(x_1 + iy_1) + f(x_2 + iy_2) $$ 
    $$ = f(z_1) + f(z_2) $$ 

    Now we show it is not homogenous.
    
    Consider $\lambda = i$ then we have,  
    $$ f(\lambda z_1) = f(-y_1 + ix_1) $$ 
    $$ = -y_1 $$ 

    however we know that $\lambda (fz_1) =  ix_1 \ne -y_1$

    Hence it is not homogenous.



\end{proof}

\subsection*{Problem 10}
\begin{proof}
    We show counter example. Assume $q = 1 + x$ ,  $p_1 = x$ and $p_2 = 2x$. We have, 
    $$ q(p_1) = q(x) = 1 + x $$ 
    $$ q(p_2) = q(2x) = 1 + 2x $$ 
    $q(p_1 + p_2) = q(3x) = 1 + 3x$

    It is easy to see that $1 + 3x \ne 2 + 3x$

    Hence  $T$ is not additive and not linear.
\end{proof}

\subsection*{Problem 12}
\begin{proof}
    First consider the basis of $U$ as $u_1,\dots,u_n$. Now let us extend this basis to $V$ as follows, $u_1,\dots, u_n, v_{n+1}, \dots, v_m$. We need to shwo that $T$ is not a linear map.

    We know that $T(v) = S(v) $ for any  $v \in U$. So we have  
    $$ T(u_1) = S(u_1) \ne 0 $$ 

    Now consider  $v_m$ we have  
    $$ T(v_m) = 0 $$  by definition.

    Now consider the sum of these vectors and we have $T(u_1+ v_m)$. We know that $u_1 + v_m$ cannot be in $U$ as it cannot be represented as a linear combination of $u_1,\dots,u_n$ as $v_m$ is linearl independent with $u_1,\dots,u_n$. Hence $u_1 + v_m \in V$  but $\not \in U$. Therefore by defintion we have  $T(u_1 + v_m) = 0$.

    However we know that $T(u_1) + T(v_m) = S(u_1)$. But $S(u_1) \ne 0$. Which shows us that $T$ is not additive. Hence $T$ is not a linear map.
\end{proof}


\subsection*{13}
\begin{proof}
    First consider the basis of $U$ as $u_1,\dots, u_n$. Now let us extend this basis of $U$ to span $V$ and we have $v_{n+1}, \dots, v_m$.

    Let us define our linear map for our basis as follows, 
    $$ T(u_1) = S(u_1),\dots,T(u_n) = S(u_n) $$  and 
    $$ T(v_{n+1}) = 0,\dots,T(v_{m}) = 0 $$ 

    So we have defined as linear map such that for any $u \in U$ say $a_1u_1 + \dots + a_nu_n$ we have, 
    $$ T(u) = T(a_1u_1 + \dots + a_nu_n) $$  
    $$ T(u) = a_1(Tu_1) + \dots + a_nT(u_n) $$ 
    $$ = a_1S(u_1) + \dots + a_nS(u_n) $$ 
    $$ = S(a_1u_1 + \dots + a_nu_n) $$ 
    $$ = S(u) $$ 
\end{proof}
\subsection*{Problem 14}
\begin{proof}
    We have V is finite dim and  $W$ is infinite dim. We need to show that $L(V,W)$ is infinite dimentional or in other words there isn't a basis for $L(V,W)$. We see that a new linear map  $T$ is independent from other maps if the range  $T$ is distinct from those spaces.

    % Let us shwo this by construction. Consider the basis of $V$ as $v_1,\dots,v_k$. Define $T_1$ such that, 
    % $$ Tv_1 = w_1,\dots,Tv_k = w_1 $$ for any $w_1 \in W$ .

    % Now for any $w_1$ we can choose $w_2$ such that $w_1,w_2$ is linearly independent so we define $T_2$ as 
    % $$ T_2v_1 = w_2,\dots,T_2v_k = w_2 $$ 

    % Because the range of $T_2$ is different from $T_1$ then 

    So it is enough to show that there isn't an upperbound on the number of linearly independent lienar maps in $L$. Or we need to shwo that for any  $n \in N$ we can consturct a linearly independent set of linear maps $T_1,\dots,T_n$

    We prove this by induction. First consider consider the base case $T_1$ that maps to any subspace of $W$. Now $T_1$ is linearly independent to itself.

    Now let us assume it is true for an artbirary $n$. That is the list  $T_1,\dots,T_n$ is lienarly independent.

\end{proof}

\subsection*{Problem 15}
\begin{proof}
Let us assume the contrary that we can consturct a linear map $T$ such that $Tv_k = w_k$ for any choice of $w_1,\dots,w_m \in W$. 

We know that $v_1,\dots,v_m$ is linearly dependent. So  $\exists v_k$ such that $v_k = a_1v_1 + \dots a_{k-1}v_{k-1}$

Now let us choose a choice of $w_1,\dots,w_n$ as follows, $n \ne k, w_n = 0$ and if $n = k$ then $w_n$ is any arbitary non-zero vector in $W$.

Based on our assumptino we can construct a map such that $Tv_k = w_k$ for any $k$ so we have,  
$$ T(v_1) = w_1 = 0 $$ 
$$ \dots $$ 
$$ T(v_k) = w_k  = w$$ 
$$ \dots $$ 
$$ T(v_n) = w_n = 0 $$ 

But we know that $T(v_k) = T(a_1v_1 + \dots + a_{k-1}v_{k-1})$. So because $T$ is linear we have, 
$$ = a_1T(v_1) + \dots + a_{k-1}T(v_{k-1}) $$ 
$$ = a_1 0+ \dots  _ a_{k-1} 0$$ 
$$ = 0 $$ 

So we have $T(v_k) = 0$. But we just showed above that  $T(v_k) = w$ such that $w \ne 0$. Hence we have a contradiction. So our assumption must be wrong and we cannot have a linear map that satifies  $Tv_k = w_k$ for any choice of $w_1,\dots,w_k$

\end{proof}


\end{document}

