\documentclass[a4paper]{report}
\usepackage[utf8]{inputenc}
\usepackage[T1]{fontenc}
\usepackage{textcomp}

\usepackage{url}

% \usepackage{hyperref}
% \hypersetup{
%     colorlinks,
%     linkcolor={black},
%     citecolor={black},
%     urlcolor={blue!80!black}
% }

\usepackage{graphicx}
\usepackage{float}
\usepackage[usenames,dvipsnames]{xcolor}

% \usepackage{cmbright}

\usepackage{amsmath, amsfonts, mathtools, amsthm, amssymb}
\usepackage{mathrsfs}
\usepackage{cancel}

\newcommand\N{\ensuremath{\mathbb{N}}}
\newcommand\R{\ensuremath{\mathbb{R}}}
\newcommand\F{\ensuremath{\mathscr{F}}}
\newcommand\Z{\ensuremath{\mathbb{Z}}}
\renewcommand\O{\ensuremath{\emptyset}}
\newcommand\Q{\ensuremath{\mathbb{Q}}}
\newcommand\C{\ensuremath{\mathbb{C}}}
\let\implies\Rightarrow
\let\impliedby\Leftarrow
\let\iff\Leftrightarrow
\let\epsilon\varepsilon

% horizontal rule
\newcommand\hr{
    \noindent\rule[0.5ex]{\linewidth}{0.5pt}
}

\usepackage{tikz}
\usepackage{tikz-cd}

% theorems
\usepackage{thmtools}
\usepackage[framemethod=TikZ]{mdframed}
\mdfsetup{skipabove=1em,skipbelow=0em, innertopmargin=5pt, innerbottommargin=6pt}

\theoremstyle{definition}

\makeatletter

\declaretheoremstyle[headfont=\bfseries\sffamily, bodyfont=\normalfont, mdframed={ nobreak } ]{thmgreenbox}
\declaretheoremstyle[headfont=\bfseries\sffamily, bodyfont=\normalfont, mdframed={ nobreak } ]{thmredbox}
\declaretheoremstyle[headfont=\bfseries\sffamily, bodyfont=\normalfont]{thmbluebox}
\declaretheoremstyle[headfont=\bfseries\sffamily, bodyfont=\normalfont]{thmblueline}
\declaretheoremstyle[headfont=\bfseries\sffamily, bodyfont=\normalfont, numbered=no, mdframed={ rightline=false, topline=false, bottomline=false, }, qed=\qedsymbol ]{thmproofbox}
\declaretheoremstyle[headfont=\bfseries\sffamily, bodyfont=\normalfont, numbered=no, mdframed={ nobreak, rightline=false, topline=false, bottomline=false } ]{thmexplanationbox}


\declaretheorem[numberwithin=chapter, style=thmgreenbox, name=Definition]{definition}
\declaretheorem[sibling=definition, style=thmredbox, name=Corollary]{corollary}
\declaretheorem[sibling=definition, style=thmredbox, name=Proposition]{prop}
\declaretheorem[sibling=definition, style=thmredbox, name=Theorem]{theorem}
\declaretheorem[sibling=definition, style=thmredbox, name=Lemma]{lemma}



\declaretheorem[numbered=no, style=thmexplanationbox, name=Proof]{explanation}
\declaretheorem[numbered=no, style=thmproofbox, name=Proof]{replacementproof}
\declaretheorem[style=thmbluebox,  numbered=no, name=Exercise]{ex}
\declaretheorem[style=thmbluebox,  numbered=no, name=Example]{eg}
\declaretheorem[style=thmblueline, numbered=no, name=Remark]{remark}
\declaretheorem[style=thmblueline, numbered=no, name=Note]{note}

\renewenvironment{proof}[1][\proofname]{\begin{replacementproof}}{\end{replacementproof}}

\AtEndEnvironment{eg}{\null\hfill$\diamond$}%

\newtheorem*{uovt}{UOVT}
\newtheorem*{notation}{Notation}
\newtheorem*{previouslyseen}{As previously seen}
\newtheorem*{problem}{Problem}
\newtheorem*{observe}{Observe}
\newtheorem*{property}{Property}
\newtheorem*{intuition}{Intuition}


\usepackage{etoolbox}
\AtEndEnvironment{vb}{\null\hfill$\diamond$}%
\AtEndEnvironment{intermezzo}{\null\hfill$\diamond$}%




% http://tex.stackexchange.com/questions/22119/how-can-i-change-the-spacing-before-theorems-with-amsthm
% \def\thm@space@setup{%
%   \thm@preskip=\parskip \thm@postskip=0pt
% }

\usepackage{xifthen}

\def\testdateparts#1{\dateparts#1\relax}
\def\dateparts#1 #2 #3 #4 #5\relax{
    \marginpar{\small\textsf{\mbox{#1 #2 #3 #5}}}
}

\def\@lesson{}%
\newcommand{\lesson}[3]{
    \ifthenelse{\isempty{#3}}{%
        \def\@lesson{Lecture #1}%
    }{%
        \def\@lesson{Lecture #1: #3}%
    }%
    \subsection*{\@lesson}
    \testdateparts{#2}
}

% fancy headers
\usepackage{fancyhdr}
\pagestyle{fancy}

% \fancyhead[LE,RO]{Gilles Castel}
\fancyhead[RO,LE]{\@lesson}
\fancyhead[RE,LO]{}
\fancyfoot[LE,RO]{\thepage}
\fancyfoot[C]{\leftmark}
\renewcommand{\headrulewidth}{0pt}

\makeatother

% figure support (https://castel.dev/post/lecture-notes-2)
\usepackage{import}
\usepackage{xifthen}
\pdfminorversion=7
\usepackage{pdfpages}
\usepackage{transparent}
\newcommand{\incfig}[1]{%
    \def\svgwidth{\columnwidth}
    \import{./figures/}{#1.pdf_tex}
}

% %http://tex.stackexchange.com/questions/76273/multiple-pdfs-with-page-group-included-in-a-single-page-warning
\pdfsuppresswarningpagegroup=1

\author{Aamod Varma}
\setlength{\parindent}{0pt}


\title{Linear Alebgra 3B}
\author{Aamod Varma}
\begin{document}
\maketitle
\date{}


\section*{3B}
\subsection*{Problem 1}
Let us define a linear  map $T: V^{5} \rightarrow V^{5}$ on any arbitary basis of $V$, $v_1,\dots,v_5$ as follows, 
$$ T(v_1) = 0 $$ 
$$ T(v_2) = 0 $$ 
$$ T(v_3) = 0 $$ 
$$ T(v_4) = v_4 $$ 
$$ T(v_5) = v_5 $$ 

So $T$ is a linear map such that $\dim \nul T = 3$ and $\dim \range T = 2$


\subsection*{Problem 2}
\begin{proof}
    We need to show $(ST)^2 = 0$ or that $STST = 0$.

    Consider $v \in V$ and let  $T(v) = v'$. Now $S(T(v)) = S(v') = v''$ which is in range of $S$ by definition.

    We are told that $\range S \subseteq \nul T$. This means that for any  $v \in \range S$ $T(v) = 0$. So because $v'' \in \range S$ we have $T(v'') = 0$. So we have,  
    $$ S(T(S(T(v))) = S(T(v'')) $$ 
    $$ = S(0) = 0 $$ 

    Hence we  show that for any arbitary chocie of $v \in V$ $(ST)^2 = 0$
\end{proof}


\subsection*{Problem 3}
\begin{proof}
    (a). If $\dim(\range T) = \dim V$ then $v_1,\dots,v_m$ spans $V$

    (b). If $\nul T = \{0\}$ then  $v_1,\dots,v_m$ is linearly independent.
\end{proof}

\subsection*{Problem 4}
\begin{proof}
    For a subspace we need three condition, existance of $0$ element, closure under addition adn closure under scalar multiplication. We show that the set doesn't satisfy the closure under addition. First consider any basis for $R^{5}$ as $v_1,\dots,v_5$ and a basis for $R^{4}$ as $u_1,\dots,u_4$

    Consider the following construction, $T_1: R^{5}\rightarrow R^{4}$  such that, 
    \begin{align*}
       T(v_1) = 0\\
       T(v_2) = 0\\
       T(v_3) = 0\\
       T(v_4) = u_1\\
       T(v_5) = u_2
    \end{align*}

    Now consider $T_2: R^{5}\rightarrow R^{4}$ such that, 

    $$ T(v_1) = u_3 $$ 
    $$ T(v_2) = u_4 $$ 
    $$ T(v_3) = 0 $$ 
    $$ T(v_4) = 0 $$ 
    $$ T(v_5) = 0 $$ 

    Now we show that $T_1 + T_2$ is not in the set. Now $T_3 = T_1 + T_2$ is defined as follows (by definition), 
    \begin{align*}
    T_3(v_1) = T_1v_1 + T_2v_1 = u_3 \\
    T_3(v_2) = T_1v_2 + T_2v_2 = u_4 \\
    T_3(v_3) = T_1v_3 + T_2v_3 = 0 \\
    T_3(v_4) = T_1v_4 + T_2v_4 = u_1 \\
    T_3(v_5) = T_1v_5 + T_2v_5 = u_2 \\
    \end{align*}

    We know that $u_1,\dots,u_4$ are linaerly independent. Which means that  the dimension of null space is $1$. Hence it is not within the conditions of our set.

    So closure under additino is not satisfied and hence the set iis not a subspace.
\end{proof}

\subsection*{Problem 5}
Consider the standard basis of $R^{4}$, $v_1 = (1,0,0,0), v_2 = (0,1,0,0), v_3 = (0,0,1,0), v_4 = (0,0,0,1)$. Now let our linear map be as follows, 
$$ T(v_1) = v_3 $$ 
$$ T(v_2) = v_4 $$ 
$$ T(v_3) = 0 $$ 
$$ T(v_4) = 0 $$ 

As $v_1$ and $v_2$ are linearly indepenent we see that the range of $T$ is spanned by two vectors $v_3,v_4$. Similary we see that the null space is spanned by $v_3,v_4$ as T maps these vectors to 0. Hence we get $\range T = \nul T$

\subsection*{Problem 6}
\begin{proof}
Let us assume $\exists T \in L(R^{5})$ such that $\range T = \nul T$. This implies that $\dim \range T = \dim \nul T$.  We know from the fundamental theorem of linear map that  
$$ \dim \range T  + \dim \nul T = \dim V $$ 

Let $\dim \range T = \dim \nul T = k$ such that $ k \in N$. So we have,  
$$ 2k = 5 $$ 
$$ k = 2.5 $$ 

Howver this means $k \not \in N$ which is a contradiction. This must mean our assumptino is wrong and hence it is not possible to find $T$ such that $\range T = \nul T$
\end{proof}


\subsection*{Problem 9}
\begin{proof}
    Consider, 
    $$ a_1T(v_1) + \dots + a_nT(v_n)= 0 $$ 
    To show that it is linearly independent we need to shwo that the only possible values for $a_1,\dots,a_n $ is if all are zero.

    Now let us rewrite this as follows, 
    \begin{align*}
        T(a_1v_1) + \dots + T(a_nv_n) &= 0\\
        T(a_1v_1 + \dots + a_nv_n) &= 0\\
    \end{align*}
    We know that $T$ is injective which means that null space of $T$ is $\{0\}$. This implies that  $a_1v_1 + \dots + a_nv_n = 0$. However  if this is the case the only choice for $a_1,\dots,a_n $ is if all are zero as we know that  $v_1,\dots,v_n$ is lienarly indepenent.

    Hence we show that the only choice of $a_1,\dots,a_n$ is if all are zero to satify the equation, 
    $$ a_1T(v_1) + \dots + a_nT(v_n) = 0 $$  which shows that the list $T(v_1), \dots, T(v_n)$ is linearly independent.
\end{proof}

\subsection*{Problem 10}
\begin{proof}
    To show that $Tv_1, \dots, Tv_n$ spans $\range T$ we need to shwo that any $w \in \range T$ can be writen as a linear combinatino of $Tv_1,\dots, Tv_n$.

    If $w \in \range T$ we that $\exists v\in V$ such that $T(v) = w$ by definition. As $v_1,\dots,v_n$ spans $V$ we know that any vector $v \in V$ can be written as a linear combinatiof these vectors so let, 
    $$ v = a_1v_1 + \dots + a_nv_n $$
    So we have $T(v) = w = T(a_1v_1 + \dots + a_nv_n)$

    \begin{align*}
        w = T(a_1v_1) + \dots + (a_nv_n)\\
        w = a_1T(v_1) + \dots + a_n(v_n)\\
    \end{align*}

    So we show that for any choice of $w \in \range T$ we can write it as a linear combinatio of vectors in the list $Tv_1,\dots,Tv_n$. Hence this implies that $Tv_1, \dots, Tv_n$ spans $\range T$
\end{proof}

\subsection*{Problem 11}
\begin{proof}
    First let us consider the basis of $\nul T$. Let that be  $u_1,\dots,u_n$. Now let us extend this basis to a basis of $V$. as $u_1,\dots,u_n, v_{n+1},\dots,v_m$. Let us define $U$ as the subspace defined by the basis $v_{n+1},\dots,v_m$.

    First we show that $U \cap \nul T = \{0\}$. So we have $v \in U$ and $v \in \nul T$. If this is the case we can write  $v $ as , 
    $$ v = a_1u_1 + \dots + a_nu_n $$  and 
    $$ v = b_1v_1 + \dots + b_mv_m $$ 

    So we have $a_1u_1 + \dots + a_nv_n + c_1v_1 + \dots + c_mv_m = 0$

    As we know that $u_1,\dots,v_m$ is linearly independet the only solutino is all coefficients is zero which implies $v = 0$.

    Now we show that  $\range T = \{Tu : u \in U\}$. Consider any  $v \in V$. Let $v = a_1u_1 + \dots + a_nu_n + b_1v_1 + \dots + b_mv_m$.

    Now we need to show that  for all $v \in V$ such that $T(v) \in \range T$ that $T(v) = T(u)$ for some $u \in U$. We have $T(v)$ whic his, 
    $$ T(v) =  T(a_1u_1 + \dots + b_mv_m) $$ 
    $$ = a_1T(u_1) + \dots +a_nT(u_n) + b_1T(v_1) + \dots + b_mT(v_m) $$ 
    $$ = b_1T(v_1) + \dots + b_mT(v_m) $$ 
    $$ = T(b_1v_1 + \dots + b_mv_m) $$ 
    $$ = T(u) $$ where $ u = b_1v_1 + \dots + b_mv_m$ which means that $ u \in U$.

    Hence we showed that  $\range T = \{Tu : u \in U\}$
\end{proof}

\subsection*{Problem 12}
\begin{proof}
    It is enough to show that $\dim \range T = \dim F^{2}$. We have, null space is spanned by 
    $$ (5,1,0,0), (0,0,7,1) $$ which makes $\dim \nul T = 2$. Using the rank nullity theorem we have $\dim \range T = 4 -2 = 2 $

    So the range of our linear map has the same dimension as the co-domain which means that our fuctino is surjective.
\end{proof}

\subsection*{Problem 13}
\begin{proof}
    
We have $\nul T = U$ which means that $\dim \nul T = 3$. Using the rank nullity theorem we have $\dim \range T = 5$. We also know that $\dim R^{5} = 5$. So because the range and co-domain have the same dimension this implies that our map is surjective.

\end{proof}
\subsection*{Problem 14}
\begin{proof}
    
Let us assume the null space is as shown in the question. We can see that this is spanned by the following vectors,  
$$ (3,1,0,0,0), (0,0,1,1,1) $$ 

This means that $\dim \nul T = 2$. So using the rank-nullity theorem we have  $\dim \range T = 5 - 2 = 3$. But our codomain is  $F^{2}$ so our assumption leads us to believe the range is a subspace of codomain but the range has higher dimension than the codomain. This  obviously cannot be the case. 

Hence it must be true that the null space cannot be as given.

\end{proof}
\subsection*{Problem 15}
\begin{proof}
    We know that $\range T$ and $\nul T$ are finite dimentional. Now consider $Tv_1,\dots,Tv_n$ span $\range T$. This means that for any  $v \in V$ we have, 
    \begin{align*}
        Tv &= a_1Tv_1 + \dots + a_nTv_n\\
         Tv &= T(a_1v_1 + \dots + a_nv_n)\\
         T(v - (a_1v_1 + \dots + a_nv_n)) &= 0\\
    \end{align*}

    Now this means that $v - (a_1v_1 + \dots + a_nv_n) \in \nul T$. As $\nul T$ is finite dimentional we have any $w \in \nul T = b_1w_1 + \dots + b_mw_m$. So we have $v = a_1v_1 + \dots + a_nv_n + b_1w_1 + \dots + b_mw_m$ for any  $v \in V$.  Hence $V$ is in the span of  afinite number of vectors whihc makes $V$ a finite dimentional vector space.
\end{proof}

\subsection*{Problem 16}
\begin{proof}
    $\impliedby$
    We are given an injecive linear map from $V$ to $W$ we need to show that $\dim V \le \dim W$.

    If $T$ is injective then we know that $\dim \nul T = 0$. So using the rank nullity theorem we have, 
    $$ \dim V = \dim \range T $$

    But we know that $\range T \subseteq W$ which means that  $\dim \range T \le \dim W$. We showed above that $\dim \range T = \dim V$ which means that $\dim V \le \dim W$.
 
    $\implies$
    We are given that $\dim V \le \dim W$ and we are to show that there exists a linear map that is injective from $V$ to $W$.

    If $\dim v \le \dim W$ consider any basis of $V$ as $v_1,\dots,v_n$ and similary choose linearly independent set of vectors from $W$ as $w_1,\dots,w_n$. Now we can construct a linear map from $V$ to $W$ such that $T(v_k) = w_k$. Because the range is spanned by $n$ linearly independent vectors we have $\dim \range T = n$ we also know that $\dim V = n$. So using the rank nullity theorem we have $\dim \nul T = 0$. 

    Hence we showed that there exists a linear map always if  $\dim V \le \dim W$.
\end{proof}

\subsection*{Problem 17}
\begin{proof}
    $\impliedby$

    We know know that our map $V$ to $W$ is surjective which implies that $\dim \range T = \dim W$. So using the rank nullity theorem we have,  
    $$ \dim V = \dim W + \nul T $$

    Case 1: $\nul T = 0$. We have $\dim V = \dim W$

    Case 2:  $\nul T \ne 0$. We have $\dim V > \dim W$

    So we have $\dim V \ge \dim W$

     $\implies$
     We have $\dim V \ge \dim W$, we need to show we can construct a surjective linear map fro  $V $ to $W$. Consider the basis for W as $w_1,\dots,w_n$. Now choose $n$ linearly independednt vectors from $V$,  $v_1,\dots,v_n$. We know this can be done as $V$ has greater than or equal to n linearly indepdnent vectors in its basis.

     Let our map be as follows, 
     $$ T(v_1) = w_1,\dots, T(v_n) = w_n, T(v_k) =0 $$  for $k > n$.

     So our range is spanned by  the basis for $W$. Which makes it equal to $W$. Hence we have a surjective map.

\end{proof}

\subsection*{Problem 18}
\begin{proof}
    $\impliedby$ 
    We need to show that $ \nul T = U$ implies that $\dim U \ge \dim V  - \dim W$. As $\dum \nul T = \dim U$ we have $\dim \nul T + \dim W \ge \dim V$. But we know that  $\dim \nul T = \dim V - \dim \range T$. So we have to show that  $\dim W \ge \dim \range T$.

    We know this is necessarily true.

    $\implies$
    We have  $\dim U + \dim W \ge \dim V$ and we have to show that $\exists T$ such that $\nul T = U$. Let $W$ be spanned by $w_1,\dots,w_m$. Now let the $\range T$ be spanned by $w_1,\dots w_k$. We can find $v_1,\dots,v_k$ such that $T(v_k) = w_k$. Now extend lin ind set of vecotrs of $V$ from $v_1,\dots,v_k$ to $v_1,\dots,v_m$ such that we added $n -k$ vectors. Such that we have $\dim V - \dim W $ = n - m. Because we know that $U \ge n - m$. We can choose at least $m - k$ (note that this is larger than $n -m$ as $m > k$) vectors from our added set of vectors such that  
$$ T(v_{k+1}) = 0 $$ 
$$ \dots $$ 
$$ T(v_{n}) = 0 $$ 

Hence given our condition we constructed a linear map from $V$ to $W$ such that $ \nul T = U$
\end{proof}

\subsection*{Problem 19}
\begin{proof}
    $\impliedby$
    We know because $T$ is injective for any $w \in W, \exists | v \in V$  such that $T(v) = w$. Or we can say that the dim range of $T$ is equal to dim $V$. Which means that they are both spanned by an equal number of vectors.

    Consnider the basis of $T$ as $v_1,\dots,v_n$. We have T defined as,
    $$ T(v_1) = w_1 $$ 
    $$ \dots $$ 
    $$ T(v_n) = w_n $$ 

    Such that $w_1,\dots,w_n$ span $\range T$ and because its dimension is equal to the basis dimenstino $w_1,\dots,w_n$ is a basis for $\range T$.

    Because $w_1,\dots,w_n$ is a basis of $\range T$ let us define a map $S$ from $W$ to $V$ as follows, 
    $$ S(w_1) = v_1 $$ 
    $$ \dots $$ 
    $$ S(w_n) = v_n $$ 

    Now we need to show that $ST$ is the identity operator. 

    Consider any $v \in V$ we can write $v = a_1v_1,\dots,a_nv_n$.

    So, 
     \begin{align*}
         Tv &= T(a_1v_1+\dots+a_nv_n)\\
        &= a_1(Tv_1)+ \dots+ a_n(Tv_n)\\
        &= a_1w_1+\dots+a_nw_n
    \end{align*}

    So $ST(v)$ we have, 
    \begin{align*}
        STv &= S(a_1w_1+\dots+a_nw_n)\\
        &= a_1(Sw_1) + \dots + a_n(Sw_n)\\
        &= a_1v_1,\dots,a_nv_n\\
        &= v
    \end{align*}

    So we showed that $\exists S$ such that $STv =  v$

    $\implies$ 
    We need to show that if there exist a map fro $W$ to $V$, $S$ such that $STv = v$ then $T$ is injective.

    Let us assume for the sake of contardiction that $T$ is not injective. That means $\exists v \ne 0$  such that $T(v) = 0$ (beacuse null T is not equal to just \{0\}). Hence we have, 
    $$ T(v) = 0 $$
    So we have, $ST(v) = S(0) = 0$. But we know that $ST$ is identity map on $ V$ which means that $STv = v$ $\forall v$. However we see that $v \ne 0$ which means that our assumptino must be wrong and T is injective.

\end{proof}


\subsection*{Problem 20}
\begin{proof}
    $\impliedby$ 
    We need show that $T$ is surjective implies that $\exists S$ such that $TS$ is the identity operator on $W$.

    We know that  $T$ is surjective this means that $\forall w \in W$ $\exists v \in V$ such that $T(v) = w$. Now consider the basis for  $W$ as $w_1,\dots,w_n$. We nkow that $\exists v$ for each one of these vectors, $v_1,\dots,v_n$ such that, 
    $$ T(v_1) = w_1,\dots,T(v_n) = w_n $$.

    Now let us define $S$ such that, 
    $$ S(w_1) = v_1,\dots,S(w_n) = v_n $$ 

    Now we need to show that $TS$ is the identity operator on $W$. Consider any $w \in W$ we have, $w = a_1w_1,\dots,a_nw_n$

    So we have, 
    \begin{align*}
        S(w) &= S(a_1w_1,\dots,a_nw_n)\\
             &= a_1S(w_1) + \dots + a_nS(w_n)\\
             &= a_1v_1+ \dots + a_nv_n
    \end{align*}

    Now 
    \begin{align*}
        TS(w) &= T(a_1v_1+\dots+a_nv_n) \\
              &= a_1T(v_1) + \dots + a_nT(v_n)\\
              &= a_1w_1 + \dots + a_nw_n\\
        &= w
    \end{align*}

    So we defined $S$ such that $TS$ is the identtiy map on $W$.


    $\implies$ 
    Assume for contradictino that $T$ is not surjective. Now this means that $\exists w \in W$ such that it is not in the range of $T$. So $w \not \in \range T$ or  $\not \exists v \in V $ such that $T(v) = w$. 

    Howevwe we know that  $TS$ is the identity operator on $W$ which means that for any $w \in W$ we have $TSw = w$.

    
    $$ T(S(w)) = w $$  
    Now let $S(w) = v' \in V$

    $$ T(v') = w $$ 

    Howver this implies that $w \in \range T$ which contradicts the fact that $ T$ is not surjective. Hence our assumption must be wrong and $T$ is surjective.
\end{proof}

\subsection*{Problem 22}
\begin{proof}
    Restrict $T$ to $\nul ST$ and call that $T'$. We have  $\dim \nul T' \le \nul T$. We know that $\dim T' = \nul T' + \range T'$  or that $\dim \nul ST = \nul T' + \range T'$. So we get,  
    $$ \dim \nul ST \le \nul T + \range T' $$ 

    But we also know that $\range T' \subseteq \nul S$ so  $\dim \range T' \le \dim \nul S$ which gives us $$\dim \nul ST \le \nul T + \dim \nul S$$
\end{proof}
\begin{proof}
    We know that 
    \begin{align*}
        \dim \nul ST &= \dim U  - \range ST\\
    \end{align*}

    But we know that $\dim U = \dim \nul T + \dim \range T$, so we have,
    $$ \dim \nul ST &= \dim \nul T + \dim \range T  - \range ST$$

    $\range T$ is the values that are the outputs of $T$. However these are the inputs of $S$. So we know that. So $\range T$ is the inputs of $S$ in $ST$ and $\range ST$ are the outputs of $ST$. So we can say that  $\dim \range T = \dim (\range ST \cap \range S) + \dim \nul S$ which gives us,  
    $$ \dim \nul ST = \dim \nul T + \dim (\range ST \cap \range S) + \dim \nul S - \dim (\range ST) $$ 

    We know that $\dim (\range ST \cap \range S) \le \dim (\range ST)$ hence we have, 
    $$ \dim \nul ST \le \dim \nul T + \dim \nul S $$ 
\end{proof}


\subsection*{Problem 23}
\begin{proof}
    We already know that $\dim \range ST \le \dim \range S$. Because for any  $v$ we have $S(T(v))$ which lies in the range of $S$.

    Now consider when $\dim \range T \le \dim \range S$.  We know that for a $v \in \range T$ ,  $S(v)$ is mapped to a vector in $\range S$. So if $v_1,\dots,v_n$ is the basis for $\range T$ which is smaller than that of $\range S$ then $S$ will only map to at most $n$ linearly independent vectors in $\range S$ which is smaller than $\dim \range S$. Hence we show that if  $\dim \range T \le \dim \range S $ then $\dim \range ST \le n$ which means $\dim \range ST \le \dim \range T$
\end{proof}


\subsection*{Problem 27}
\begin{proof}
    We are givne that $P^2 = P$ we can to show that $V = \nul P \oplus \range P$.

    First we shwo that  $\nul P \cap \range P = \{0\}$. Consider  $v \in \nul P \cap \range P$. That means that  $v \in \nul P $ and $v \in \range P$.  If $v \in \nul P$ we nkow that $P(v) = 0$ but we know that  $P(P(v)) = P(v)$. So  $P(0) = P(v)$. But if $v \in \range P$ then $\exists v'$ such that $P(v') = v$ whcih means taht  $P(P(v')) = P(v')$. So,  
    $$ P(v) = v $$

    But we know that $P(v) = P(0) = 0$ so  $P(v) = v \implies v = 0$

    Now we show that we can write any vector  $v \in V$ as a sum of vectors from null and range of P.

    Consider any $v \in V$. Now let $P(v) = v_1$ which means that $v_1 \in \range P$. So we have, 
    $$ P(v) = v_1 $$ 
    $$ P(P(v)) = P(v_1) = P(v) $$ 

    Now take $v - v_1$. We have, 
    $$ P(v - v_1) = P(v) - P(v_1) $$ 
    As we got  $P(v) = P(v_1)$ we have $P(v - v_1) = 0$ which means that $v - v_1 \in \nul P$. Hence we found two vectors, $v_1 \in \range P$  and $v - v_1 \in \nul T$ such that $v - v_1 + v_1 = v$ for any $v \in V$

\end{proof}



\subsection*{Problem 28}
\begin{proof}
    
    We need to show that for any $p' \in P(R)$ we can find $p \in P(R)$ such that $Dp = p'$ given that  $\deg p' = \deg p - 1$.

    Consider any arbitarry polynomial  $p' = a_1 + a_2x + \dots + a_nx^{n}$. We can find $p$ as follows, 
    $$ p = \int  p' = \int a_1 + \dots + a_nx^{n} = a_1x + \dots + \frac{a_n}{n+1}x^{n+1} + C$$  where $C$ can be any arbitarry constant.

    We see that $Dp$ defined as the differntiation map would map $p$  as follows, 
    $$ Dp = \frac{dp}{dx} = x_1 + \dots +a_nx^{n} = p'$$  such that $\deg p = n+1 $ and $\deg p' = n$ which satisfies that  $\deg Dp = \deg p - 1$
    $$  $$ 
\end{proof}

\subsection*{Problem 29}


\begin{proof}
    First let $\deg p = n$ such that $p = a_1+\dots+a_nx^{n}$ . Now let $q $ be a degree $n + 1$ polynomial such that $q = b_1 + \dots + b_{n+1}x^{n+1} $. So we have, 
    \begin{align*}
        q' &= b_2 + \dots + b_{n+1}(n+1)x^{n}\\
        q'' &= b_3 + \dots + b_{n+1}(n)(n+1)x^{n -1}\\
        5q'' + 3q' &= 5b_3 + 3b_2 + \dots + (5b_{n+1}n(n+1) +3(nb_n))x^{n-1} + 3b_{n+1}(n+1)x^{n}
    \end{align*}
    It is enough to show that $5q'' + 3q'$ can span $P(R)$ for any $n$. For this we need to show that the coefficients must be 0 ie. $b_2=\dots=b_n+1 = 0$

    We see that there is only one term affecting $x^{n}$ so for that to be $0$ there is no other choice but $b_{n+1} = 0$. But if $b_{n+1} = 0$  then for $x^{n-1}$ term we  need $b_n = 0$. So by inductino we can show that any $b$ must be equal to 0. Hence it is linearly indpendent.
\end{proof}


\end{document}
$ $$ 

            
