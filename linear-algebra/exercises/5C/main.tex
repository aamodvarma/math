\documentclass[a4paper]{report}
\usepackage[utf8]{inputenc}
\usepackage[T1]{fontenc}
\usepackage{textcomp}

\usepackage{url}

% \usepackage{hyperref}
% \hypersetup{
%     colorlinks,
%     linkcolor={black},
%     citecolor={black},
%     urlcolor={blue!80!black}
% }

\usepackage{graphicx}
\usepackage{float}
\usepackage[usenames,dvipsnames]{xcolor}

% \usepackage{cmbright}

\usepackage{amsmath, amsfonts, mathtools, amsthm, amssymb}
\usepackage{mathrsfs}
\usepackage{cancel}

\newcommand\N{\ensuremath{\mathbb{N}}}
\newcommand\R{\ensuremath{\mathbb{R}}}
\newcommand\F{\ensuremath{\mathscr{F}}}
\newcommand\Z{\ensuremath{\mathbb{Z}}}
\renewcommand\O{\ensuremath{\emptyset}}
\newcommand\Q{\ensuremath{\mathbb{Q}}}
\newcommand\C{\ensuremath{\mathbb{C}}}
\let\implies\Rightarrow
\let\impliedby\Leftarrow
\let\iff\Leftrightarrow
\let\epsilon\varepsilon

% horizontal rule
\newcommand\hr{
    \noindent\rule[0.5ex]{\linewidth}{0.5pt}
}

\usepackage{tikz}
\usepackage{tikz-cd}

% theorems
\usepackage{thmtools}
\usepackage[framemethod=TikZ]{mdframed}
\mdfsetup{skipabove=1em,skipbelow=0em, innertopmargin=5pt, innerbottommargin=6pt}

\theoremstyle{definition}

\makeatletter

\declaretheoremstyle[headfont=\bfseries\sffamily, bodyfont=\normalfont, mdframed={ nobreak } ]{thmgreenbox}
\declaretheoremstyle[headfont=\bfseries\sffamily, bodyfont=\normalfont, mdframed={ nobreak } ]{thmredbox}
\declaretheoremstyle[headfont=\bfseries\sffamily, bodyfont=\normalfont]{thmbluebox}
\declaretheoremstyle[headfont=\bfseries\sffamily, bodyfont=\normalfont]{thmblueline}
\declaretheoremstyle[headfont=\bfseries\sffamily, bodyfont=\normalfont, numbered=no, mdframed={ rightline=false, topline=false, bottomline=false, }, qed=\qedsymbol ]{thmproofbox}
\declaretheoremstyle[headfont=\bfseries\sffamily, bodyfont=\normalfont, numbered=no, mdframed={ nobreak, rightline=false, topline=false, bottomline=false } ]{thmexplanationbox}


\declaretheorem[numberwithin=chapter, style=thmgreenbox, name=Definition]{definition}
\declaretheorem[sibling=definition, style=thmredbox, name=Corollary]{corollary}
\declaretheorem[sibling=definition, style=thmredbox, name=Proposition]{prop}
\declaretheorem[sibling=definition, style=thmredbox, name=Theorem]{theorem}
\declaretheorem[sibling=definition, style=thmredbox, name=Lemma]{lemma}



\declaretheorem[numbered=no, style=thmexplanationbox, name=Proof]{explanation}
\declaretheorem[numbered=no, style=thmproofbox, name=Proof]{replacementproof}
\declaretheorem[style=thmbluebox,  numbered=no, name=Exercise]{ex}
\declaretheorem[style=thmbluebox,  numbered=no, name=Example]{eg}
\declaretheorem[style=thmblueline, numbered=no, name=Remark]{remark}
\declaretheorem[style=thmblueline, numbered=no, name=Note]{note}

\renewenvironment{proof}[1][\proofname]{\begin{replacementproof}}{\end{replacementproof}}

\AtEndEnvironment{eg}{\null\hfill$\diamond$}%

\newtheorem*{uovt}{UOVT}
\newtheorem*{notation}{Notation}
\newtheorem*{previouslyseen}{As previously seen}
\newtheorem*{problem}{Problem}
\newtheorem*{observe}{Observe}
\newtheorem*{property}{Property}
\newtheorem*{intuition}{Intuition}


\usepackage{etoolbox}
\AtEndEnvironment{vb}{\null\hfill$\diamond$}%
\AtEndEnvironment{intermezzo}{\null\hfill$\diamond$}%




% http://tex.stackexchange.com/questions/22119/how-can-i-change-the-spacing-before-theorems-with-amsthm
% \def\thm@space@setup{%
%   \thm@preskip=\parskip \thm@postskip=0pt
% }

\usepackage{xifthen}

\def\testdateparts#1{\dateparts#1\relax}
\def\dateparts#1 #2 #3 #4 #5\relax{
    \marginpar{\small\textsf{\mbox{#1 #2 #3 #5}}}
}

\def\@lesson{}%
\newcommand{\lesson}[3]{
    \ifthenelse{\isempty{#3}}{%
        \def\@lesson{Lecture #1}%
    }{%
        \def\@lesson{Lecture #1: #3}%
    }%
    \subsection*{\@lesson}
    \testdateparts{#2}
}

% fancy headers
\usepackage{fancyhdr}
\pagestyle{fancy}

% \fancyhead[LE,RO]{Gilles Castel}
\fancyhead[RO,LE]{\@lesson}
\fancyhead[RE,LO]{}
\fancyfoot[LE,RO]{\thepage}
\fancyfoot[C]{\leftmark}
\renewcommand{\headrulewidth}{0pt}

\makeatother

% figure support (https://castel.dev/post/lecture-notes-2)
\usepackage{import}
\usepackage{xifthen}
\pdfminorversion=7
\usepackage{pdfpages}
\usepackage{transparent}
\newcommand{\incfig}[1]{%
    \def\svgwidth{\columnwidth}
    \import{./figures/}{#1.pdf_tex}
}

% %http://tex.stackexchange.com/questions/76273/multiple-pdfs-with-page-group-included-in-a-single-page-warning
\pdfsuppresswarningpagegroup=1

\author{Aamod Varma}
\setlength{\parindent}{0pt}


\title{Linear Algebra 5C}
\author{Aamod Varma}
\begin{document}
\maketitle
\date{}

\section*{5C}
\subsection*{Problem 1}
\begin{proof}
   If $F =  C$ then $T$ has an upper triangular matrix regardless.

   If $F = R$ then consider $T(x,y) = (-y,x)$. We have $T^2 + I = 0$. So  the minimal polynomial of T does not have any real eigenvalues. 

   However $T' = T^2 $ has the upper triangular matrix $-I$
\end{proof}


\subsection*{Problem 2}
\begin{proof}
   (a). We have $(A + B)_{jk} = A_{jk} + B_{jk}$

   As both $A$ and $B$ are upper triangular matrices we know that $A_{jk} = B_{jk} = 0$ for $j > k$. Hence $(A + B)_{jk} = 0$ for $j > k$. And $(A + B)_{kk} = A_{kk} + B_{kk} = \alpha_k + \beta_k$

   (b). We have, 
   $$ (AB)_{jk} = \sum_{n=1}^{m} A_{jn}B_{nk} $$ 

   First consider if $j > k$. We have, 
   $$ \sum_{n=1}^{k} A_{jn}B_{nk} + \sum_{n=k+1}^{m} A_{jn}B_{nk} $$ 

   In the first sum we have $j > k$ and $k > n$ so $j > n$ which means that $A_{jn} = 0$ so the sum goes to zero. In the second sum we have $n > k$ so $B_{nk} = 0$ so that goes to zero. Hence the sum is always 0 for any choice of $j,k$ if $j > k$.

   This shows that $AB$ is an upper triangular matrix.

   Now if $j = k$ we have, 
   $$ (AB)_{kk} = \sum_{n=1}^{m} A_{kn} B_{nk} = A_{kk} B_{kk}  = \alpha_k \beta_k$$ 


\end{proof}

\subsection*{Problem 3}
\begin{proof}
   (1).
  We know if T is an upper triangular matrix then the minimal polynomial of $T$ can be written as $(z - \lambda_1)\dots(z - \lambda_n)$. We also know that if T is invertible then  its minimal polynomial is, 
  $$ (z - \frac{1}{\lambda_1}) \dots(z - \lambda_n) $$ 

  Because it is of this form we can create an upper triangular matrix with the reciprocals on the diagonal.

  (2). 
  The existance of an upper triangular matrix for $T$ implies that for the basis $v_1,\dots,v_n$ we can write, 
  \begin{align*}
     Tv_1 &= \lambda_1v_1\\
     Tv_2 &= a_1v_1 + \lambda_2 v_2\\
          &\dots\\
     Tv_n &= b_1v_1 + \dots + \lambda_nv_n
  \end{align*}

  Now let us apply $T^{-1}$ on each side and we get, 

  \begin{align*}
     v_1 &= \lambda_1 T^{-1}v_1\\
     v_2 &= a_1T^{-1}v_1 + \lambda_2 T^{-1}v_2\\
         &\dots\\
     v_n &= b_1T^{-1}v_1 + \dots +T^{-1} \lambda_nv_n
  \end{align*}

  Rearranging the term we get,  

  \begin{align*}
     T^{-1}v_1  &= \frac{v_1}{\lambda_1} \\
     T^{-1}v_2 &= \frac{v_2}{\lambda_2} - \frac{a_1}{\lambda_2}T^{-1}v_1 \\
         &\dots\\
     T^{-1} v_n &=\frac{v_n}{\lambda_n}  - \frac{b_1}{\lambda_n}T^{-1}v_1+ \dots 
  \end{align*}

  Going from the beginning we have $T^{-1}v_1 \in span(v_1), T^{-1}v_2 \in span(v_1,v_2)$ and going forward  we get $T^{-1}v_k \in span(v_1,\dots,v_{k - 1})$. This makes our matrix upper triangular. 

  We see that for any $k \in \{1,\dots,n\}$ that the term before $v_k$ for $T^{-1}v_k$ is $\frac{1}{\lambda_k}$. Hence our diagonal is $\frac{1}{\lambda_k}$ for the $v_k$.

\end{proof}
\subsection*{Problem 6}
\begin{proof}
   If $F = C$ that means that there exists an upper triangular matrix with respect to some basis of $V$. Let this basis be $v_1,\dots,v_n$ where $n = \dim V$.

   Now this means that for any  $k$,  $span(v_1,\dots,v_k)$ is invariant under $T$ as $T(v_k) \in span(v_1,\dots,v_k), T(v_{k-1}) \in span(v_1,\dots,v_{k-1}),\dots$
\end{proof}
\subsection*{Problem 7a}
\begin{proof}
   (a). Consider the list $(v,Tv,\dots,T^{\dim V}v)$. As the dimension is $\dim V + 1$ there is some smallest $k$ such that $T^{k + 1} \in span(v,\dots,T^{k})$. Which makes $U = span(v,\dots,T^{k})$ invariant under $T$. Let $p_v$ be the minimal polynomial of $T_{|U}$ so we see that, 
   $$ p_v(T)v = p_v(T_{|U})v = 0 $$ 

   We know that the degree cannot be smaller than $k$ so it is of least degree.
\end{proof}
\subsection*{Problem 8b}
\begin{proof}
   We have $T^2v + 2Tv + 2v = 0$

   So the minimal polynomial is either $z^2 + 2z + 2 = 0$ or a polynomial multiple of it whose roots are $-1 + i$ or $-1 -i$.

   Which means that eigenvalues of  $T$ are the same so it must appear on the diagonal of A.
\end{proof}
\subsection*{Problem 9}
\begin{proof}
   Now let $T$ be the linear map associated with $B$. Because $F = C$ there exists some basis of $V$ in which there is a linear map $C$ which is upper triangular. Let this basis be $v_1,\dots,v_n$. If $B$ is a matrix defined on the basis $u_1,\dots,u_n$. Then we can define $A = M(T, (v_1,\dots,v_n),(u_1,\dots,u_n))$ such that $A^{-1}BA = M(T, (v_1,\dots,v_n), (v_1,\dots,v_n)) = C$.
\end{proof}
\subsection*{Problem 10}
\begin{proof}
   $a \implies b$

   If the matrix is lower triangular then we can say that, 
   \begin{align*}
      Tv_1 &= a_1v_1,\dots, a_nv_n\\
      \dots\\
      T(v_n) &= b_nv_n
   \end{align*}

   So we see that for any $j$ we have  $Tv_j \in span(v_j,\dots,v_n)$ but $span(v_j,\dots,v_n) \subset span(v_k,\dots,v_n)$. So for any  $v \in span(v_k,\dots,v_n)$ we have $Tv \in span(v_1,\dots,v_n)$ which makes the span invariant.

   $b \implies c$
   If the span is invariant then it follows.

    $c \implies b$ 
    If $c$ is true that means that we can write $Tv_1,\dots,Tv_k$ in the way we wrote above which means we can make a lower triangular matrix.
\end{proof}

\end{document}
