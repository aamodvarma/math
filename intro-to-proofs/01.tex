\chapter*{Real Numbers}
\begin{definition}[Properties of real numbers] Properties of $\R$ are
    
    % (a).
    
    % (b).
    
    (d). $\exists$ an order on  $\R$ which means $\forall x,y \in \R, x < y \text{ or }, x > y, \text{ or } x = y$
    Ordering follows the following properties, 

    \qquad (1). $x < y, y < z \Rightarrow x < z$ (transitivity)
    
    \qquad (2). $x < y \Rightarrow x + z < y + z, \forall z \in \R$
    
    \qquad (3). $x < y, z > 0 \Rightarrow xz < yz$
\end{definition}

\begin{theorem}
    $xy = 0 \iff x = 0 \text{ or } y = 0 $ 
    \begin{proof} 
        $\Leftarrow$
        Without loss of generality take, $x = 0$
        Then we get,  \[
            0y
        .\] 
        We can write this as, \[
            (0 + 0)y = 0y + 0y
        .\] 
        So, \[
        0y = 0y + 0y
        .\] 
        Or, m

        $\Rightarrow$

        Assume the contrary that, $x \neq 0 \text{ and } y \neq 0$ 
        We have, $xy = 0$.
        Without loss of generality we take the multiplicative inverse of  $x$ so,
         \[
        \frac{xy}{x} = \frac{0}{x}
        .\] 
        We showed that $0(k) = 0$ so  $y = 0$

        Which contradicts our assumption, hence our asusmptoin must be wrong and  $x = 0 \text{ or } y = 0$
        
    \end{proof}
\end{theorem}
\begin{theorem}
    $(-)x = -x$
    \begin{proof}
        We start with $(-1)x$ and add $x$ to both sides so,  \[
            (-1)x + x = x (1 - 1) = 0x = 0
        .\] 
        So we showed that $(-1)x$ is the additive identity of $x$. 

        We know that the additive identity is unique for any  $x$

        Therefore,  $(-1)x = -x$
    \end{proof}
\end{theorem}

\begin{theorem}
    $\forall x < y, z < 0$  \[
    xz > yz
    .\] 

    
\end{theorem}

\begin{theorem}
    $\forall x \in \R, x^2 \geq 0$ and if $x \neq 0$ then  $x^2> 0$
\end{theorem}

\begin{theorem}
    $x^2 = - (-x^2)$

    Case 1, $x > 0$: 
    $$x > 0$$
    $$x\times x > x$$
    $$x\times x > 0x$$
    $$ x^2 > 0$$
    
    Case 2, $x < 0$:
\end{theorem}

\begin{eg}
    $\forall a,b > 0$  \[
        \frac{a + b}{2} \geq \sqrt{ab}
    \begin{proof}
        \[
            0 \leq (\sqrt{a} - \sqrt{b})^2 = a - 2\sqrt{ab} + b
        .\]      
    \end{proof}
\end{eg}

\begin{eg}
    $x^2 - x + 1$
\end{eg}



\begin{theorem}
    $\forall x,y \in \R$ we have,  \[
        |x| \geq x \text{ and } |x + y| \leq |x| + |y|
    .\] 

    \begin{proof}
        
    \end{proof}
\end{theorem}


\subsection*{Proof related to Sets}
\begin{theorem}
    \[
    A \cup B \backslash (A \cap B) = (A \backslash B) \cup (B \backslash A)
    .\] 

    \begin{proof}
        We need to show that, \[
     A \cup B \backslash (A \cap B) \subseteq  (A \backslash B) \cup (B \backslash A)
        .\] 
        and,
      \[
      (A \backslash B) \cup (B \backslash A) \subseteq A \cup B \backslash (A \cap B) 
      .\] 
    \end{proof}
\end{theorem}



\begin{theorem}
    $A \subseteq B \iff A \cup B = B$
     \begin{proof}
        $\Rightarrow$   
        Take $\forall x \in A \cup B$, so either
        
        Case 1, $x \in A$:

        We know that by deifnition if, $A \subseteq B$ then for  $x \in A, x \in B$ so  $x \in B$
        
        Case 2, $x \in B$:
        If $x \in B$ then we don't need to go further.


        So we get $\forall x \in A \cup B$,  $x \in B$

        $\impliedby$

        $\forall x \in A \Rightarrow x \in A \cup B = B$

        So,  $x \in B$ which means that, $A \subseteq B$
        
    \end{proof}
\end{theorem}



