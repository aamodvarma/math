\documentclass[a4paper]{article}

\usepackage[margin=0.6mm]{geometry}
\usepackage[table]{xcolor}

\begin{document}
    \pagestyle{empty}%
    \noindent
    \begin{tabular}{@{}c@{}}
    \begin{minipage}[t][\paperheight][t]{0.49\paperwidth}%
        \textbf{Integers (mod n)}\\
        $Z_n$ denotes the integers mod n. 
         \begin{enumerate}
            \item If $a,b \in Z_n$, $a + b$ (mod n) is  $k$ s.t. $a + b \equiv k$ (mod n)
            \item Usual arithmetic hold but not all have multiplicative inverse. Eg. In $Z_8, 2$ does not have a multiplicative inverse. $\not \exists k$ s.t. $2k \equiv 1$ (mod 8). 
        \end{enumerate}


        The following hold for $Z_n$, 
        \begin{enumerate}
            \item $a + b \equiv b + a $ ( mod n) and same for  $ab$
            \item  $(a + b) + c \equiv a + (b + c)$ (mod n) and same for  $(ab)c$
            \item $a + 0 \equiv a$ (mod n) and  $a \cdot 1\equiv a$ (mod n)
            \item $\exists -a$ s.t. $a + (-a) \equiv 0$ (mod n)
            \item $gcd(a,n) = 1 \iff \exists b$ s.t.  $ab \equiv 1$ (mod n)
        \end{enumerate}

        \textbf{Symmetry}
        \begin{enumerate}
            \item A triangle has 6 symmetries, essentially $3!$ the permutations of the vertices.
            \item In the multiplication table for symmetries of a triangle, for every motion there is an inverse.
        \end{enumerate}



        \textbf{Groups}\\
        A group $(G, \circ)$ is a set $G$ with a law of composition $(a,b) \rightarrow a \circ b$ s.t.
        \ \begin{enumerate}
            \item $\forall a,b \in G, a \circ b \in G$
            \item $(a \circ b) \circ c = a \circ ( b \circ c)$
            \item $\exists e \in G$ s.t. $\forall a \in G, e\circ a = a\circ e = e$
            \item  $\forall a \in G, \exists a^{-1} s.t., a\circ a^{-1} = a^{-1} \circ a = e$
        \end{enumerate}         
        If $a \circ b = b \circ a $ then the group is abelian.

        \begin{itemize}
            \item $Z$ is a group under addition.
            \item $(Z_n, +)$ is a group but  $Z_n$ is not with modular multiplication
            \item The group of units $U(n)$ is all $a \in Z_n$ that are coprime with $n$. So $U(8) = \{1,3,5,7\}$. It is a group under multiplication.
            \item $M_2(R)$ is set of all $2 \times 2$ matrices. Then $GL_2(R)$ the general linear group is the subset that consists of all invertible matrices. The identity of the group is just $I$. It is a non-abelian group.
            \item $S_3$ the symmetry group is $\{(1), (12), (23), (13), (123), (132)\}$
        \end{itemize}


    A group is finite if it has finite order. The order of a finite group is number of elements. $|Z_5| = 5$

    \textbf{Properties of Groups}
    \begin{itemize}
        \item Inverse is unique; Identity is unique
        \item $(ab)^{-1} = b^{-1}a^{-1}$; $(a^{-1})^{-1}= a$
        \item $ba = ca \lor ab = ac \Rightarrow b = c$ 
    \end{itemize}

    Law of exponents hold as follows, 
    \begin{itemize}
        \item $g^{m}g^{n} = g^{m + n}$; $(g^{m})^{n} = g^{mn}$       
        \item $(gh)^{n} = (h^{-1}g^{-1})^{-n}$
    \end{itemize}
    $(gh)^{n} = g^{n}h^{n}$ only if $G$ is abelian.

    \textbf{Subgroups}\\
    A subgroup $H$ of $G$ is a subset $H$ s.t. $H$ is a group under the same operation on $G$. Every group with at least two elements will have at least two subgroups. $H = \{e\}$ (trivial subgroup) and $H = G$ (proper subgroup)

    - $Q^{*} = \{p / q: p,q \ne 0\}$ is a subgroup of $R^{*}$
    \end{minipage}%
    \end{tabular}%
    \begin{tabular}{@{}c@{}}
    \begin{minipage}[t][\paperheight][t]{0.49\paperwidth}%
        - $SL_2(R)$ is a subset of $GL_2(R)$ s.t. determinant is 1. It is a sugroup of $GL_2(R)$

    A subset $H$ of $G$ can be a group but not a subgroup (essentially a group under a diff operation)

    \textbf{Subgroup Theorems}\\
    H is a subgroup of $G$ if and only if
    \begin{itemize}
        \item $e$ of  $G$ is in $H$
        \item  $h_1,h_2 \in H$ then $h_1 \circ h_2 \in H$
        \item $h \in H$ then $h^{-1} \in H$
    \end{itemize}

    $H$ is a subgroup of $G$ if and only if $H \ne \phi$ and for $g,h \in H, gh^{-1} \in H$

    \textbf{Cyclic Subgroups}\\
    If $G$ is a group and $a$ is an element of $G$ then, $\langle a \rangle = \{a^{k}: k \in Z\}$ is a subgroup of $G$, the smallest containing $a$.

    We call $a$ the generator of the subgroup. The order of $a$ is the smallest $n$ such that $a^{n} = e$ and $|a| = n$. If order of  $a$ is $|G|$ then $a$ is a generator of $G$.

    Eg. Both 1 and 5 generate  $Z_6$. 1 generates  $Z $ and any $Z_n$

    \begin{itemize}
        \item If a generates  $G$ then $a^{k} = e$ if and only if n divides $k$ if G is of order $n$.
        \item if $a \in G$ is a generator. If $b = a^{k}$ then order of b is $n / d$ where  $d = $ gcd(k,n)
    \end{itemize}

    To find the order of any element $a \in G$ we have, $|a| = n / gcd(a,n)$

    A normal subgroup is $N$ s.t. $g \in G$ and $n \in N$ we have $gng^{-1} \in N$
    \textbf{Cosets}\\
    A left coset of  $H$ with a given $g \in G$ is, $gH = \{gh: h \in H\}$\\
    A right coset of  $H$ with a given $g \in G$ is, $Hg = \{hg h \in H\}$
    The following are equivalent for $g_1,g_2 \in G$
    \begin{enumerate}
        \item $g_1H = g_2H$
        \item $Hg_1^{-1} = Hg_2^{-1}$
        \item $g_1H \subset g_2H$
        \item $g_2 \in g_1H$
        \item $g_1^{-1}g_2 \in H$
    \end{enumerate}

    The cosets of a subgroup partition the larger group $G$ (always).

    The \textbf{index} of H in G is the number of left cosets of H in G which is  $[G:H]$

    Eg.  $G = Z_6$ and $H = \{0,3\}$ then  $[G:H] = 3$

    \textbf{lagranges theorem}\\
    $H$ is a subgroup of G and with $g \in G$ the map $\phi: H \rightarrow gH$, $\phi(h) = gh$ is bijective. So $|H| = |gH|$. 

    \begin{enumerate}
        \item  If  $H$ is a subgroup of $G$ then $|G| / |H| = [G : H]$
        \item Order of $g \in G$ must divide number of elements in $G$
        \item  $|G| = p$ then any $g \in G \ne e$ is a generator and $G$ is cyclic.
    \end{enumerate}

    \textbf{Homomorphisms}\\
    A homomorphism between $(G,\circ_1)$ and $(H,\circ_2)$ is a map $\phi: G \rightarrow H$ s.t. $\phi(g_1 \circ_1 g_2) = \phi(g_1) \circ_2 \phi(g_2)$

    The relation is stronger if its isomorphic.

    Eg. $\phi: Z \rightarrow G$ s.t.  $\phi(n) = g^{n}$ is a homomorphism from $Z$ to $G$.

    The following hold, 
     \begin{enumerate}
        \item $e$ is identity of $G_1$ then $\phi(e)$ is of $G_2$
        \item For any $g \in G_1$, $\phi(g^{-1}) = [\phi(g)]^{-1}$ 
        \item $H_1$ is a subgroup of $G_1$ then $\phi(H_1)$ is a subgroup of $G_2$
        \item If $H_2$ is a subgroup of $G_2$ then $\phi^{-1}(H_2) = \{g \in G_1: \phi(g) \in H_2\}$ is a subgroup of $G_1$.
    \end{enumerate} 


    The subgroup $H = \phi^{-1}(\{e\})$ is called the kernal of $\phi$

    \end{minipage}%
    \end{tabular}%
\end{document}
