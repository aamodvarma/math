\documentclass[a4paper]{report}
\usepackage{multirow}
\input{preamble.tex}
\title{Intro to Proofs: HW10}
\author{Aamod Varma}
\graphicspath{ {./} }
\begin{document}
\maketitle
\date{}
\subsection*{Problem 2}
\begin{proof}
    First we see that for (a.) there is no identity element hence it cannot be a group. However for b., c. and d. we have the identity element $e = a$
    \vspace{1em}
    Now we see that for (d.) there is no inverse element such that it gives us the identity. The identity is $e = a$ and there is no element $d^{-1}$ such that $d \circ d^{-1} = a$. Hence $(d.)$ is not a group.

    So (b) and (c) form a group because we have an inverse element, identity element and is associated as well.
\end{proof}

\subsection*{Problem 4}
All four side of a rhombus are of equal length, so we can both rotate and reflect and keep it symmetrical. A rhombus has the following symmetries, 
\begin{enumerate}
    \item Identity (I)
    \item 180 degree rotation around the center (R)
    \item Reflection over vertical axis (V)
    \item Reflecting over horizontal axis. (H)
\end{enumerate}

So the table would look like,
\begin{center}
\begin{tabular}{|c|c|c|c|c|}
\hline
    & \(I\) & \(R\) & \(V\) & \(H\) \\ \hline
\(I\) & \(I\) & \(R\) & \(V\) & \(H\) \\ \hline
\(R\) & \(R\) & \(I\) & \(H\) & \(V\) \\ \hline
\(V\) & \(V\) & \(H\) & \(I\) & \(R\) \\ \hline
\(H\) & \(H\) & \(V\) & \(R\) & \(I\) \\ \hline
\end{tabular}
\end{center}

We first see using the table that we have an identity element, an inverse as well as its associative because its similar to the $\Z_4$ Cayley tables.

We also see that the Cayley table for a rectangle is similar to that of a rhombus and is the same as above.


\subsection*{Problem 7}
\begin{proof}
    First we show its a group and then that its abelian.

    1. Identity

    We have, 
    $$a \ast b = a + b + ab$$
    So let $a + b + ab = a$ then, 
    \begin{align*}
        a + b + ab &= a\\
                b + ab &= 0\\
                b(1 + a) &= 0
    \end{align*}
    So for all $a$ if the rhs has to be zero then  $b = 0$. Hence we have, 
    $$ a \ast b = a \text{ if } b = 0$$  so, 
    $$ a \ast 0 = a $$ 
    which means we have an identity element.


    2. Inverse, 

    We need $b$ such that $a \ast b = 0$

    So,  
    \begin{align*}
        a \ast b = a + b + ab &= 0\\
                            b(1 + a)&= -a\\
                            b &= -\frac{a}{1 + a}
    \end{align*}

    Hence we found $b$ such that $a \ast b = 0$. Which is the existence of inverse  $a^{-1} = -\frac{a}{1 + a}$ 

    3. Associativity. 

    We need to show that, 
    $$ (a \ast b) \ast c = a \ast (b \ast c) $$ 

    First the left hand side evaluates to, 
    $$ (a \ast b) \ast c = (a + b + ab) \ast c = (a + b + ab) + c + c(a + b + ab) $$ 
    $$ = a + b + c + ab + ac + bc + abc $$ 

    And the right hand side evaluates to,
    $$ a \ast (b + c + bc) = a + b + c + bc + a(b + c + bc) $$ 
    $$ = a + b + c + ab + bc + ac + abc  $$ 

    So both sides evaluate to the same thing hence it is associative.

    Now to show its abelian we need to show that, 
    $$ a \ast b = b \ast a $$ 

    So consider $a \ast b = a + b + ab$ and we have  $ b \ast a = b + a + ba$

    It is easy to see that this is equal because of associativity and commutativity of the reals. Hence our group is an abelian group.
\end{proof}

\subsection*{Problem 10}
\begin{proof}
    Consider $A = \begin{bmatrix} 1 & x & y  \\ 0 & 1 & z \\ 0 & 0 & 1\end{bmatrix}$

    1. Identity

    We need to find $B$ such that $AB = A$. So we have,  
    $$ \begin{bmatrix} 1 & x & y  \\ 0 & 1 & z \\ 0 & 0 & 1\end{bmatrix} \begin{bmatrix} 1 & x' & y'  \\ 0 & 1 & z' \\ 0 & 0 & 1\end{bmatrix} = \begin{bmatrix} 1 & x & y  \\ 0 & 1 & z \\ 0 & 0 & 1\end{bmatrix} $$           

    So by definition we have, 
    $$ \begin{bmatrix} 1 & x & y  \\ 0 & 1 & z \\ 0 & 0 & 1\end{bmatrix} = \begin{bmatrix} 1 & x + x' & y + y' + xz'  \\ 0 & 1 & z + z' \\ 0 & 0 & 1\end{bmatrix} $$ 

    Or that $x + x' = x \implies x' = 0$, $z + z' = z \implies z' = 0$ and lastly,   $y + y' + xz' = y$ but we have  $z' = 0$ hence $y + y' = y \implies y' = 0$

    So we have our identity ,  
    $$I =  \begin{bmatrix} 1 & 0 & 0  \\ 0 & 1 & 0 \\ 0 & 0 & 1\end{bmatrix}  $$ 



    2. Inverse

    We need to find $B$ such that $AB = I$. So we have, 
    $$  \begin{bmatrix} 1 & x & y  \\ 0 & 1 & z \\ 0 & 0 & 1\end{bmatrix}  \begin{bmatrix} 1 & x' & y'  \\ 0 & 1 & z' \\ 0 & 0 & 1\end{bmatrix}  =  \begin{bmatrix} 1 & 0 & 0  \\ 0 & 1 & 0 \\ 0 & 0 & 1\end{bmatrix}  $$ 

    By definition we have, 
    $$  \begin{bmatrix} 1 & x + x' & y + y' + xz'  \\ 0 & 1 & z + z' \\ 0 & 0 & 1\end{bmatrix}   =  \begin{bmatrix} 1 & 0 & 0  \\ 0 & 1 & 0 \\ 0 & 0 & 1\end{bmatrix} $$ 

    This means that $x + x' = 0 \implies x' = -x$,  $z + z' = 0 \implies z = -z'$. And we have,  $y + y' + xz' = 0, y' = -y - xz' = -y + xz$

    So we have,  
    $$ B = \begin{bmatrix} 1 &- x & - y + xz  \\ 0 & 1 & -z  \\ 0 & 0 & 1\end{bmatrix} $$ 


    such that $AB = I$ hence  $B$ is our inverse $A^{-1}$ 


    3. Associativity, 

    we need to show that $(AB)C = A(BC)$

    Let  $$B = \begin{bmatrix} 1 & x' & y'  \\ 0 & 1 & z' \\ 0 & 0 & 1\end{bmatrix}$$ and 
 $$C = \begin{bmatrix} 1 & x'' & y''  \\ 0 & 1 & z'' \\ 0 & 0 & 1\end{bmatrix}$$

 We have, 
 $$ AB = \begin{bmatrix} 1 & x + x' & y + y' + xz'  \\ 0 & 1 & z + z' \\ 0 & 0 & 1\end{bmatrix} $$ 

So, $$(AB)C = \begin{bmatrix} 1 & x + x' + x'' & (y + y' + xz') + y'' +(x + x')z''  \\ 0 & 1 & z + z' + z'' \\ 0 & 0 & 1\end{bmatrix} $$

$$(AB)C = \begin{bmatrix} 1 & x + x' + x'' & y + y' + y''  + xz'  +xz'' + x'z''  \\ 0 & 1 & z + z' + z'' \\ 0 & 0 & 1\end{bmatrix} $$

Now, 
$$BC = \begin{bmatrix} 1 &  x' + x'' & y' + y'' + x'z'' \\ 0 & 1 &  z' + z''\\ 0 & 0 & 1\end{bmatrix} $$


So we have, 
$$ A (BC) =  \begin{bmatrix} 1 & x + x' + x'' & y + (y' + y'' + x'z'') + x(z' + z'')  \\ 0 & 1 & z + z' + z'' \\ 0 & 0 & 1\end{bmatrix}  $$ 
$$ A (BC) =  \begin{bmatrix} 1 & x + x' + x'' & y + y' + y'' + x'z'' + xz' + xz''  \\ 0 & 1 & z + z' + z'' \\ 0 & 0 & 1\end{bmatrix}  $$ 


It is easy to see that $(AB)C = A(BC)$
\end{proof}


\subsection*{Problem 14}
\begin{proof}
    1. Identity

    We need, $(b,n)$ such that, 
    $$ (a,m) \circ (b,n) = (a,m) $$ 

    So we have, 
    \begin{align*}
        (a,m) \circ (b,n) &= (ab,m + n)\\
        (a,m) &= (ab, m + n)
    \end{align*}

    So $ab = a \implies b = 1$ and  $m + n = m \implies n = 0$

    So our identity element is  $(b,n) = (1,0)$
    


    2. Inverse

    We need $(b,n)$ such that, 
    $$ (a,m) \circ (b,n) = (1,0) $$ 

    So we have, 
    \begin{align*}
        (a,m) \circ (b,n) &= (ab, m + n)\\
         (1,0) &= (ab, m + n)        
    \end{align*}

    SO $ab = 1 \implies b = 1 /a$ ( if a not equal to 0 which is true as  $a \in R^{*}$) and  $ m + n = 0 \implies n = -m$

    So we have  $(b, n) = (1 /a, -m)$




    3. Associtivity

    We need to show that, 
    $$ ((a,m) \circ (b, n) ) \circ (c, o) = (a,m) \circ ((b,n) \circ (c,o)) $$ 

    For the left hand side we have, 
    \begin{align*}
        ((a,m) \circ (b, n) ) \circ (c, o)  &= (ab, m + n) \circ (c, o)\\
                                            &= (abc, m + n + o)
    \end{align*}

    For the right hand side we have, 
    \begin{align*}
        (a,m) \circ ((b, n) ) \circ (c, o)  &= (a, m) \circ (bc, n + o)\\
                                            &= (abc, m + n + o)
    \end{align*}

    We see that the left side and right side equal to the same thing.

    Hence it is associative

    Therefore $G$ is a group under this operation.
\end{proof}

\subsection*{Problem 17}
\begin{proof}
    1. We have the cyclic group $\Z_8$ which are the integer's modulo  $8$ and on addition. The element are, 

    $$ \{0,1,2,3,4,5,6,7\} $$ 

    2. We have the symmetries of a square. We have two main operations on a square to preserve symmetry (1). Rotate by $90$ degrees (r) and (2). reflection (s) and (3) The identity

    So the elements of this group is, 
    $$ \{e, r, r^2, r^{3}, s, sr, s r^{2}, s r^{3}\} $$ 
\end{proof}


\subsection*{Problem 25}
    \begin{proof}
        We show this by induction. 

        First we check for the case of $n = 1$. We have, 
        \begin{align*}
            ab^{1}a^{-1} &= (aba^{-1})^{1}\\
            aba^{-1} &= aba^{-1}
        \end{align*}

        Now assume its true for $n = k$ so we have, 
        $$ ab^{k}a^{-1}= (aba^{-1})^{k} $$ 

        Now we need to show that $n = k + 1$ holds as well or that, 
        $$ ab^{k + 1} a^{ -1} = (a ba ^{-1}) ^{k + 1} $$ 


        Going back to $n = k$ case we have, 
        \begin{align*}
            ab^{k}a^{-1} &= (aba^{-1})^{k} \\
            ab^{k}a^{-1} (aba^{-1}) &= (aba^{-1})^{k} (aba^{-1})\\
            ab^{k}a^{-1}a b a^{-1} &= (aba^{-1})^{k + 1}\\
            ab^{k} b a^{-1} &= (aba^{-1})^{k + 1}\\
            ab^{k + 1}a^{-1} &= (aba^{-1})^{k + 1}\\
        \end{align*}

        Which is the $n = k + 1$ case. Hence by induction we show that it is true for all $n \in N$. 
        
        Now we consider the case when  $n = 0$. Which is, 
        $$ ab^{n}a^{-1} = (aba^{-1})^{n} $$ 
        $$ a a^{-1} = 1 $$  and  
        $$ (aba^{-1})^{0}  = 1 $$ 

        So it is true.

        Lastly we see that case for $n < 0$. This is equivalent to, doing induction for $n \in N$ for, 
        $$ ab^{-n}a^{-1} = (aba^{-1})^{-n} $$
                
        So first we see the base case which is  when $n = 1$ we have, 
        $$ ab^{-1}a^{-1} = (aba^{-1})^{-1} $$ 

        The right hand side becomes, 
        $$ ab^{-1}a^{-1} $$ based on how $^{-1}$ is distributed. 
        Hence it is true for the $n = 1$ case.



        Now consider the case for $n = k$. We assume that, 
        $$ ab^{-k}a^{-1} = (aba^{-1})^{- k} $$ 

        We need to show it also holds true for the $n = k + 1$ case that is, 
        $$ ab^{-(k  + 1)}a^{-1} = (aba^{-1})^{-(k + 1)} $$ 


        The $n = k$ case gives us, 
        $$ ab^{-k}a^{-1} = (aba^{-1})^{- k} $$ 

        Now we multiply $(aba^{-1})^{-1}$ on both sides and we get, 
        $$ ab^{-k}a^{-1} (aba^{-1})^{-1} = (aba^{-1})^{-k} (aba^{-1})^{-1} $$ 
        $$ ab^{-k}a^{-1} ab^{-1}a^{-1} = (aba^{-1})^{-(k + 1)} $$                   
        $$ ab^{-k}b^{-1}a^{-1} = (aba^{-1})^{-(k + 1)} $$                   
        $$ ab^{-(k + 1)}a^{-1} = (aba^{-1})^{-(k + 1)} $$                   
    \end{proof}

    Which is the $n = k + 1$ case. 

    Hence we show its true for $n > 0, n = 0$ and $n < 0$




\subsection*{Problem 33}
\begin{proof}
    We need to show that, 
    $$ ab = ba $$  for any elements $a,b$ in our group.

    We have, 
    \begin{align*}
        (ab)^2 &= a^2b^2\\
        ab ab &= a^2 b^2\\
        a^{-1}abab  &= a^{-1} a^2 b^2\\
        bab &= ab^2\\
        bab b^{-1} &= ab^2 b^{-1}\\
        ba &= ab
    \end{align*}

    Hence we show its commutative. So its an abelian group.
\end{proof}


\end{document}
