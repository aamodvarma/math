\documentclass[a4paper]{report}
\usepackage{multirow}
\input{preamble.tex}
\title{Intro to Proofs: HW07}
\author{Aamod Varma}
\graphicspath{ {./} }
\begin{document}
\maketitle
\date{}
\subsection*{11.5.3}
\begin{tabular}{ |c|c|c|c|c| } 
\hline
\cdot & [0] & [1] & [2] & [3]\\
\hline
\multirow{1}{1em}{[0]} & [0] & [0]& [0]& [0]\\ 
\hline
\hline
\multirow{1}{1em}{[1]} & [0] & [1]& [2]& [3]\\ 
\hline
\hline
\multirow{1}{1em}{[2]} & [0] & [2]& [0]& [2]\\ 
\hline
\hline
\multirow{1}{1em}{[3]} & [0] & [3]& [2]& [1]\\ 
\hline
\end{tabular}
\vspace{5mm}


\begin{tabular}{ |c|c|c|c|c| } 
\hline
+ & [0] & [1] & [2] & [3]\\
\hline
\multirow{1}{1em}{[0]} & [0] & [1]& [2]& [3]\\ 
\hline
\hline
\multirow{1}{1em}{[1]} & [1] & [2]& [3]& [0]\\ 
\hline
\hline
\multirow{1}{1em}{[2]} & [2] & [3]& [0]& [1]\\ 
\hline
\hline
\multirow{1}{1em}{[3]} & [3] & [0]& [1]& [2]\\ 
\hline
\end{tabular}
\subsection*{11.5.6}
For $\Z_6$ this is not true as we can take $[a] = [2] \ne [0]$ and $[b] = [3] \ne [0]$ and we have,  
$$ [a] \cdot [b] = [2] \cdot [3] = [6] = [0] $$

However for $\Z_7$ we cannot find equivalence classes $[a],[b]$ such that one of them is not $[0]$. Becuase for $\Z_7$ either a is a multiple of 7 or it is not. If,

1. a is a multiple of 7 then  $[a] = [0]$ by definition.

2. a is not a multiple of 7 then $a$ and 7 are coprime which means that  if $[a][b] = [a][c]$ then $[b]=[c]$. Here $[c] = 0$ which means that $[a][b] = [0] = [a][c]$. We have  $a$ is coprime with $n = 7$ which means that $[b] = [c] = [0]$.

Hence either  $[a] = 0$ or $[b] = 0$


\subsection*{11.5.7}
\begin{enumerate}
\item 
    $[8] + [8] = [16] = [7]$
\item $[24] + [11] = [35] = [8]$

\item $[21] \cdot [15] = [315] = [0]$

\item $[8][8] = [64] = [1]$

\end{enumerate}


\subsection*{12.1.1}
Domain is $A =\{0,1,2,3,4\}$\\
Range is $\{2,3,4\}$\\
 $f(2) = 4$ and $f(1) = 3$

\subsection*{12.1.4}
\begin{align*}
    f_1 = \{(a,0),(b,0),(c,0)\}\\
    f_2 = \{(a,0),(b,0),(c,1)\}\\
    f_3 = \{(a,0),(b,1),(c,0)\}\\
    f_4 = \{(a,0),(b,1),(c,1)\}\\
    f_5 = \{(a,1),(b,0),(c,0)\}\\
    f_6 = \{(a,1),(b,0),(c,1)\}\\
    f_7 = \{(a,1),(b,1),(c,0)\}\\
    f_8 = \{(a,1),(b,1),(c,1)\}
\end{align*}
 
\subsection*{12.1.5}
A relation that is not a functino is, 
$$ R = \{(a,d),(a,e),(b,d),(c,d),(d,d)\} $$ 

\subsection*{12.1.8}
First we know that the set is a relation because it is a subset of $\Z \cross \Z$ by definition.

Now we need to show that $\forall x \in \Z$ there exists only one ordered pair of the form  $(x,y) \in f$.

We know all the pairs in our set are such that $x + 3y = 4$. So for any x,  
$$ 3y = 4 - x $$ 
$$ y = \frac{4-x}{3} $$

However we see that for $y \in \Z$ we need $x \equiv 4$ (mod 3). However we know that this isn't true  $\forall x \in \Z$. For instance take  $x = 2$ then  there isn't a $y \in Z$ such that that $(x,y) \in f$.

Hence because we can't assign a  $y \in \Z$ for all $x \in \Z$ f is not a funciton.



\subsection*{12.2.5}
1. For injectivity we need to shwo that for any $y \in \Z$ if $f(x) =f(x')$ that means that $x = x'$.

Consider  $f(n) = 2n + 1$ and  $f(n') = 2n' + 1$. We have, 
\begin{align*}
    f(n) = f(n')\\
    2n + 1 = 2n' + 1\\
    2n = 2n'\\
    n = n'
\end{align*}
which means that it is injective.


2. For surjective we need to show that for all $y \in \Z$ there exists an $x \in \Z$ such that $f(x) = y$.

Consider an arbitrary  $y \in Z$ we have 
\begin{align*}
   y = 2n + 1\\
   y - 1 = 2n\\
   n = \frac{y - 1}{2}
\end{align*}

We see that for  $n \in \Z$ we need $2 | y-1$. However this isn't true  $\forall y \in \Z$. For instance take any $y = 2k$ then there doesn't exists an $n$ such that $f(n) = y$

Hence  $f$ is not surjective.


\subsection*{12.2.6}
We have $f(m,n) = 3n - 4m$. To show injectivity we need to show for any  $f(m,n) = f(m',n')$ we have $n = n', m = m'$.

If, 
\begin{align*}
   f(m,n) = f(m',n')\\
   3n -4m = 3n' - 4m'\\
   3(n - n') = 4(m - m')
\end{align*}

However now consider $n' = 1$ and $n = 5$ and $m = 4$ and $n' = 1$ and we have  $12 = 12$. 

So it is not injective.


Now we need to check surjectivity.

We need to show that for all  $y \in \Z$, $\exists m,n$ such that $f(m,n) = 3n - 4m = y$.

Consider  $y = 2k$ we have,  
$$ 3n - 4m = 2k $$ We can choose $n = 2k$ and $m = k$

For $y = 2k + 1$ we have, 
$$ 3n - 4m = 2k + 1 $$ 

We cannot find $m,n$ for all kwhich means that this isn't surjective.




\subsection*{12.2.10}
$f: \R - \{1\} \rightarrow \R - \{1\}$ defined by  $f(x) = (\frac{x+1}{x-1})^{3}$ 

1. Injectivity.

We   need to show $f(x) = f(x') \implies x = x'$,  
\begin{align*}
(\frac{x + 1}{x - 1})^{3} = (\frac{x' + 1}{x' - 1})^{3}\\
\end{align*}

Because we know that the terms are real we can say that, 
\begin{align*}
    \frac{x + 1}{x - 1} &= \frac{x' + 1}{x' - 1}\\
    x x' - x + x' -1 &= x x' + x - x' - 1\\
    2x &= 2x'\\
    x &= x'
\end{align*}

So it is injective.

2. Bijective.

We have need to show for all $y$ exists $x$ such taht $f(x) = y$.

We have, 
\begin{align*}
    y &= (\frac{x + 1}{x - 1})^{3}\\
    y^{\frac{1}{3}} &= \frac{x + 1}{x - 1}\\
    y^{\frac{1}{3}}x - y^{\frac{1}{3}} &= x  +  1\\
    y^{\frac{1}{3}}x - x&=   y^{\frac{1}{3}} +  1\\
    x(y^{\frac{1}{3}} - 1)&=   y^{\frac{1}{3}} +  1\\
    x&=   \frac{y^{\frac{1}{3}} +  1}{(y^{\frac{1}{3}} - 1)}\\
\end{align*}

So for all $y$ if $y\ne 1$ we have $x = \frac{y^{\frac{1}{3}} + 1}{y^{\frac{1}{3}}-1}$ and we have, 
\begin{align*}
    f(x) &= f(  \frac{y^{\frac{1}{3}} + 1}{y^{\frac{1}{3}}-1})\\
         &= \bigg (\frac{\frac{y^{\frac{1}{3}} + 1 + y^{\frac{1}{3}} - 1}{y^{\frac{1}{3}}}}{\frac{y^{\frac{1}{3}}+ 1 - y^{\frac{1}{3}} + 1}{y^{\frac{1}{3}}}}\bigg )^{3}\\
         &= \bigg (\frac{y^{\frac{1}{3}} + 1 + y^{\frac{1}{3}} - 1}{y^{\frac{1}{3}}+ 1 - y^{\frac{1}{3}} + 1}\bigg )^{3}\\
         &= \bigg (\frac{y^{\frac{1}{3}} + 1 + y^{\frac{1}{3}} - 1}{y^{\frac{1}{3}}+ 1 - y^{\frac{1}{3}} + 1}\bigg )^{3}\\
         &= (\frac{2y^{\frac{1}{3}}}{2})^{3}\\
         &= (y^{\frac{1}{3}})^{3}\\
         &= y
\end{align*}


Hence f is surjective.


\subsection*{12.2.14}
We have $\theta(X) = \overline{X}$

1. Injectivity. We need to show $\theta(X) = \theta(X') \implies X = X'$

We have,  
\begin{align*}
    \theta(X) &= \theta(X')\\
    \overline{X} &= \overline{X'}\\
    X &= X'
\end{align*}

Hence $\theta$ is injective.

2. Surjectivity.

For all  $Y \in P(X)$ we need to find $X \in P(X)$ such that $\theta(X) = Y $. We have,  
$$ \overline{X} = Y $$ 
$$ X = \overline{Y} $$ 

So we ahve $X$ such that $\theta(X) = Y$
$$ \theta(X) = \theta(\overline{Y}) = \overline{\overline{Y}}  = Y$$ 

Hence it is surjective.


\subsection*{12.2.16}
Total number of functions are: $7^{5}$

Number of injective functions are: $\frac{7!}{(7 - 5)!} = 7 \cdot 6 \cdot 5 \cdot 4 \cdot 3$

Total surjective functions are: 0 

So total bijective functinos are : 0






 


\end{document}
