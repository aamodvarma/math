\documentclass[a4paper]{report}
\usepackage{multirow}
\usepackage[utf8]{inputenc}
\usepackage[T1]{fontenc}
\usepackage{textcomp}

\usepackage{url}

% \usepackage{hyperref}
% \hypersetup{
%     colorlinks,
%     linkcolor={black},
%     citecolor={black},
%     urlcolor={blue!80!black}
% }

\usepackage{graphicx}
\usepackage{float}
\usepackage[usenames,dvipsnames]{xcolor}

% \usepackage{cmbright}

\usepackage{amsmath, amsfonts, mathtools, amsthm, amssymb}
\usepackage{mathrsfs}
\usepackage{cancel}

\newcommand\N{\ensuremath{\mathbb{N}}}
\newcommand\R{\ensuremath{\mathbb{R}}}
\newcommand\F{\ensuremath{\mathscr{F}}}
\newcommand\Z{\ensuremath{\mathbb{Z}}}
\renewcommand\O{\ensuremath{\emptyset}}
\newcommand\Q{\ensuremath{\mathbb{Q}}}
\newcommand\C{\ensuremath{\mathbb{C}}}
\let\implies\Rightarrow
\let\impliedby\Leftarrow
\let\iff\Leftrightarrow
\let\epsilon\varepsilon

% horizontal rule
\newcommand\hr{
    \noindent\rule[0.5ex]{\linewidth}{0.5pt}
}

\usepackage{tikz}
\usepackage{tikz-cd}

% theorems
\usepackage{thmtools}
\usepackage[framemethod=TikZ]{mdframed}
\mdfsetup{skipabove=1em,skipbelow=0em, innertopmargin=5pt, innerbottommargin=6pt}

\theoremstyle{definition}

\makeatletter

\declaretheoremstyle[headfont=\bfseries\sffamily, bodyfont=\normalfont, mdframed={ nobreak } ]{thmgreenbox}
\declaretheoremstyle[headfont=\bfseries\sffamily, bodyfont=\normalfont, mdframed={ nobreak } ]{thmredbox}
\declaretheoremstyle[headfont=\bfseries\sffamily, bodyfont=\normalfont]{thmbluebox}
\declaretheoremstyle[headfont=\bfseries\sffamily, bodyfont=\normalfont]{thmblueline}
\declaretheoremstyle[headfont=\bfseries\sffamily, bodyfont=\normalfont, numbered=no, mdframed={ rightline=false, topline=false, bottomline=false, }, qed=\qedsymbol ]{thmproofbox}
\declaretheoremstyle[headfont=\bfseries\sffamily, bodyfont=\normalfont, numbered=no, mdframed={ nobreak, rightline=false, topline=false, bottomline=false } ]{thmexplanationbox}


\declaretheorem[numberwithin=chapter, style=thmgreenbox, name=Definition]{definition}
\declaretheorem[sibling=definition, style=thmredbox, name=Corollary]{corollary}
\declaretheorem[sibling=definition, style=thmredbox, name=Proposition]{prop}
\declaretheorem[sibling=definition, style=thmredbox, name=Theorem]{theorem}
\declaretheorem[sibling=definition, style=thmredbox, name=Lemma]{lemma}



\declaretheorem[numbered=no, style=thmexplanationbox, name=Proof]{explanation}
\declaretheorem[numbered=no, style=thmproofbox, name=Proof]{replacementproof}
\declaretheorem[style=thmbluebox,  numbered=no, name=Exercise]{ex}
\declaretheorem[style=thmbluebox,  numbered=no, name=Example]{eg}
\declaretheorem[style=thmblueline, numbered=no, name=Remark]{remark}
\declaretheorem[style=thmblueline, numbered=no, name=Note]{note}

\renewenvironment{proof}[1][\proofname]{\begin{replacementproof}}{\end{replacementproof}}

\AtEndEnvironment{eg}{\null\hfill$\diamond$}%

\newtheorem*{uovt}{UOVT}
\newtheorem*{notation}{Notation}
\newtheorem*{previouslyseen}{As previously seen}
\newtheorem*{problem}{Problem}
\newtheorem*{observe}{Observe}
\newtheorem*{property}{Property}
\newtheorem*{intuition}{Intuition}


\usepackage{etoolbox}
\AtEndEnvironment{vb}{\null\hfill$\diamond$}%
\AtEndEnvironment{intermezzo}{\null\hfill$\diamond$}%




% http://tex.stackexchange.com/questions/22119/how-can-i-change-the-spacing-before-theorems-with-amsthm
% \def\thm@space@setup{%
%   \thm@preskip=\parskip \thm@postskip=0pt
% }

\usepackage{xifthen}

\def\testdateparts#1{\dateparts#1\relax}
\def\dateparts#1 #2 #3 #4 #5\relax{
    \marginpar{\small\textsf{\mbox{#1 #2 #3 #5}}}
}

\def\@lesson{}%
\newcommand{\lesson}[3]{
    \ifthenelse{\isempty{#3}}{%
        \def\@lesson{Lecture #1}%
    }{%
        \def\@lesson{Lecture #1: #3}%
    }%
    \subsection*{\@lesson}
    \testdateparts{#2}
}

% fancy headers
\usepackage{fancyhdr}
\pagestyle{fancy}

% \fancyhead[LE,RO]{Gilles Castel}
\fancyhead[RO,LE]{\@lesson}
\fancyhead[RE,LO]{}
\fancyfoot[LE,RO]{\thepage}
\fancyfoot[C]{\leftmark}
\renewcommand{\headrulewidth}{0pt}

\makeatother

% figure support (https://castel.dev/post/lecture-notes-2)
\usepackage{import}
\usepackage{xifthen}
\pdfminorversion=7
\usepackage{pdfpages}
\usepackage{transparent}
\newcommand{\incfig}[1]{%
    \def\svgwidth{\columnwidth}
    \import{./figures/}{#1.pdf_tex}
}

% %http://tex.stackexchange.com/questions/76273/multiple-pdfs-with-page-group-included-in-a-single-page-warning
\pdfsuppresswarningpagegroup=1

\author{Aamod Varma}
\setlength{\parindent}{0pt}


\title{Intro to Proofs: HW10}
\author{Aamod Varma}
\graphicspath{ {./} }
\begin{document}
\maketitle
\date{}
\subsection*{Problem 37}
\begin{proof}
    We have $H = \{2^{k}: k \in \Z\}$ we need to show its a subgroup of $Q^{*}$. First we know that $H \subseteq Q^{*}$  because $2^{k} \in Q^{*}, \forll k \in \Z$. Now we need to show that $H$ is a group under multiplication.

    1. Closed under the operation. Consider an arbitrary element $x \in H = 2^{k_1}$ and consider $y \in H = 2^{k_2}$. Now we show that $x \cdot y  \in H$. We have, 
    $$ x \cdot y = 2^{k_1} \cdot 2^{k_2} = 2^{k_1 + k_2} $$ 

    As $k_1+k_2 \in \Z$ we have $2^{k_1 + k_2} \in H$. Hence H is closed under multiplication.

    2. Identity element. We need to show there exists $e \in H$ such that for any $g \in H$ we have $g \cdot e = g$. Consider  $k = 0$ we have $1 = 2^{0} \in Z$. Now we have for any $g \in H = 2^{k'}$, 
    $$ g \cdot e = 2^{k'} \cdot 2^{0} = 2^{k'} \cdot 1 = 2^{k'} = g $$ 

    Hence we have defined an identity element $e = 2^{0} = 1$

    3. Inverse element. We need to show that for any $g \in H, \exists g' \in H$ such that $g \cdot g' = e$. Consider  $g \in H, g = 2^{k}$. Now for any $g$ let, 
    $$ g' = 2^{-k} \in H \text{ as $-k \in \Z$} $$ 

    So we have, 
    $$ g \cdot g' = 2^{k} \cdot 2^{-k} = 2^{k - k} = 2^{0} = 1 = e $$ 

    Hence we found $g'$ for any $g$ such that $g \cdot g' = e$ which is our inverse.

    4. Associatively. We need to show that, 
    $$ 2^{k_1}\cdot (2^{k_2} \cdot 2^{k_3}) = (2^{k_1}\cdot 2^{k_2}) \cdot 2^{k_3} $$ 

    We have, 
    \begin{align*}
        2^{k_1}\cdot (2^{k_2} \cdot 2^{k_3})&= 2^{k_1} \cdot (2^{k_2 + k_3})\\
                                            &= 2^{k_1 + (k_2 + k_3)}\\
                                            &= 2^{(k_1 + k_2) + k_3}\\
                                            &= 2^{k_1 + k_2} \cdot 2^{k_3}\\
                                            &= (2^{k_1} \cdot 2^{k_2}) \cdot 2^{k_3}\\
    \end{align*}

    Hence we show associativity. 

\end{proof}


\subsection*{Problem 43}

\begin{proof}
    It is enough to show that for any $x,y \in SL_2(\Z)$ that $xy^{-1} \in SL_2(\Z)$. So first consider $x,y \in SL_2(\Z)$.

    Let, 
    $$ x = \begin{bmatrix}  a & b \\ c & d\end{bmatrix} $$ 
    and
    $$ y = \begin{bmatrix}  e & f \\ g & h\end{bmatrix} $$ 

    We have, 
    $$ y^{-1} = \begin{bmatrix}  h & -f \\ -g & e\end{bmatrix} $$ 

    Such that, 
    $$ xy^{-1} = \begin{bmatrix}  ah - bg & -af + be \\ hc - gd & -fc + de\end{bmatrix} $$ 

    We know each element in $xy^{-1}$ are integers because we're only multiplying and adding integers together. Now we need to show that $det(xy^{-1}) = 1$

    We know, 
    $$ det(xy^{-1}) = det(x) det(y^{-1}) $$

    But we also known that $det(y^{-1}) = \frac{1}{det(y} = \frac{1}{1} = 1$  and that $det(x) = 1$. So we get, 
    $$ det(xy^{-1}) = 1 \cdot 1 = 1 $$ 

    So we have $xy^{-1} \in SL_2(\Z)$


\end{proof}
\subsection*{Problem 45}
\begin{proof}
    Consider two subgroups $H,F \subseteq G$. We need to show that  $H \cap F$ is a subgroup of G as well. First we have $H \cap F \subseteq G$. Now we need to show its a group.

    It is enough to show that for any $x,y \in H \cap F$ that $xy^{-1} \in H \cap F$. First we know for any $x , y \in H \cap F$ that means that $x,y \in H $ and $x,y \in F$. But we know that  $H$ and $F$ are subgroups. Hence if $x,y \in H$ that means that $xy^{-1} \in H$. Similarly if $x,y \in F$ we have $xy^{-1} \in F$. So now we have, $xy^{-1} \in F$ and $xy^{-1} \in H$. This means that $xy^{-1} \in H \cap F$. So we have shown that $H \cap F$ is a subgroup.
\end{proof}
\subsection*{Problem 46}

\begin{proof}
    Consider the group $Z$ and the subgroup $A = \{2^{k} : k \in \Z\}$ and $B = \{3^{k}: k \in \Z\}$. Both these are groups because they are orbits. Now consider their union $A \cup B$. 

    Take  $x \in A, x = 2$  and $y \in B, y = 3$. We see that  $xy = 2 \cdot 3 = 6 \not \in A \cup B$ because  $ \not \exists k \in Z$ such that $2^{k} = 6$ or $3^{k} = 6$. Hence it is not closed under the operation. Therefore is not a group.
\end{proof}. 
\subsection*{Problem 52}
\begin{proof}
    Consider $S_3$ which is group of all permutations on three elements.
    
    $$ S_3 = \{e, (12), (13), (23), (123),(132)\} $$ 
    this group is non-abelion because $(12)(23) \ne (23)(12)$. Hoover consider the subgroup,  
    $$ \{e, (12)\} $$ 
    Now this group is ableian  because the elements commute. Hence we disprove the claim.
\end{proof}

\end{document}
