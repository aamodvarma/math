\section*{Disproofs}


If we need to show existance, $\exists x. P(x)$. We can show using, 

1. Direct constructions

2. Indirectly (contradiction). For instance we can show that, $\forall x, \not P(x)$ is false


\begin{eg}
    $\exists a,b,c \in R - Q \text{ s.t. } a^b^c \in Q$
\end{eg}

\begin{eg}
    Pigeonhole principle

    Suppose there are $m$ balls in $n$ boxes, $m > n \geq 1$ then, $\exists$ a box where there are at least, $\frac{m}{n} + 1$ balls 
\end{eg}

\begin{proof}
    Assume pigeonhole is false.

    Then, there are at most $\frac{m}{n}$ balls in each box.

    In case 1 where $\frac{m}{n} \not \in N \implies$  total balls $\leq n [\frac{m}{n}] = \frac{nm}{n} = m$ which is a contradiction.

    In case 2 where $\frac{m}{n}\in N$ there are at most $\frac{m}{n} - 1$ balls in each box.
    So total number of balls are $\frac{nm}{n} - n = m - n$ which is contradictory.
\end{proof}


To disprove $\forall x P(x)$ we can show that, $\exists \not P(x)$



\[

.\] 
\begin{eg}
    100 can't be written as the sum of two even integers and an odd integer.
\end{eg}
\begin{proof}
    Suppose it's false $\implies \exists a,b,c \in Z$ s.t. $2 | a, 2 | b, 2 \not | c$ and  $100 = a + b + c$

    But,  $2 | a, 2 | b \implies 2 | a + b$ but  $2 \not | c \implies 2 \not | (a + b) + c = 100$

    So we get,  $2 \not | 100$ which is a contradiction.

    Which means that the original statmeent is true.
\end{proof}


\begin{eg}
    $\not \exists$ the smallest positive real number
    
    The smallest positive real number is defined as $x \in R$ s.t.  $x > 0$ and  $\forall y > 0, x \leq y$
\end{eg}
\begin{proof}
    Let's assume it is true which mean that $\exists x \in R$ s.t.  $x > 0$ and  $\forall y > 0$,  $x \leq y$

    We know that $x > 0 \implies \frac{x}{2} > 0$ 

    So if we set $y = \frac{x}{2}$ then we get \[
    x \leq \frac{x}{2}
    .\] 
    Which is a contradiction.

    Hence it cannot be the case that there exists the smalest positive number.
\end{proof}
\begin{eg}
    $\not \exists f(x): $ a polynomial with integer coefficients s.t.  $\forall n, f(n)$ is prime
\end{eg}
\begin{proof}
    Consider the general form of a polynomial, 
    $$ f(x) = a_1x^n + \dots + a_n $$ 

    Case 1: $a_n = 0$

    If  $a_n=0$ then for any $x > 1$ we can take $x$ common and get $$f(x) = x(a_1x^{n-1}+\dots+a_{n-1})$$

    So we get a factor $x \ne 1$

    Case 2:  $a_n = 1$

    In this case we can just plug  $x = 0$ and we get $f(x)$ is neither prime or composite

Case 3: $a_n > 1$
????
    
\end{proof}

\begin{eg}
    Let $f(x) = x^3 + 2x - 5$ then $\exists \text { unique }x_0 \in [1,2]$ s.t. $f(x_0) = 0$
\end{eg}
\begin{proof}
    Using intermediate value theorem.

    $$f(1) = -2$$
    $$f(2) = 7$$

    So because $-2 < 0 < 7$ we know that there must exists an $x_0 \in [1,2]$ s.t. this is the case.


    To show unique we need to show its strictly increasing. Or in other words, we need to show for every $x_1,x_2 \in [1,2], x_1 \le x_2 \implies f(x_1) \le f(x_2)$


    So we need to show that, 
    $$ x_1^3 + 2x_1 - 5 \le x_2^3 + 2x_2 - 5 $$ 
    $$ x_1^3 + 2x_1  \le x_2^3 + 2x_2 $$ 
    $$ (x_1^3 - x_2^3) + 2(x_1-x_2)  \le 0$$


    It is enough to show that both $x_1^3 - x_2^3$ and $x_1 - x_2$ are smaller than or equal to $0$.

    
    $$ x_1^3 - x_2^3 = (x_1 - x_2)(x_1^2 + x_1x_2 + x_2^2) = k (x_1 - x_2) \le 0 $$ 

    Similarly, $$2(x_1 - x_2) \le 0 \text{ a }  x_1-x_2 \le 0$$

    So we have, $x_1^3 - x^3 + 2(x_1 - x_2) \le 0$

    Which tells us that our function is strictly increasing which implies that we only have a unique $x_0 \in [1,2]$

\end{proof}


