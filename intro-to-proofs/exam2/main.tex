\documentclass[a4paper]{article}

\usepackage[margin=0.6mm]{geometry}
\usepackage[table]{xcolor}

\begin{document}
    \pagestyle{empty}%
    \noindent
    \begin{tabular}{@{}c@{}}
    \begin{minipage}[t][\paperheight][t]{0.49\paperwidth}%
        \textbf{Chapter 11: Relations}\\
        Relation: $R \subseteq A \times  A$. $(x,y) \in R$ is $xRy$

        \qquad Reflexive: $xRx$. Symmetric:  $xRy \Rightarrow yRx$. 

        Transitive: $xRy, yRz \Rightarrow xRz$.

        Equivalence Relation: Reflexive, Symmetric and Transitive

        \qquad Equivalence class containing  $a$ is the subset $\{x \in A: xRa\}$ of  $A$ consisting of all elements of $A$ that relate to $a$. Denoted by $[a]$
        $$ [a] = \{x \in A: xRa\} $$ 
        $$ [a] = [b] \iff aRb $$ 
    Partition: Non-empty subsets of $A$ such that union of all subsets equal $A$ and intersection of any is $\phi$

    \textbf{Integer Modulo n:} For $n \in N$ equivalence classes of the relation $\quiv $ (mod $n$ ) are $[0],\dots,[n-1]$ The integers modulo n is the set $Z_n\{[0],[1],\dots,[n-1]\}$. Following hold, 
    $$ [a]+[b] = [a + b], [a]\cdot[b]=[ab] $$ 



    If we have $[a][b] = [0]$ and the classes are for integer mod $n$. If $n$ is prime then either $[a] = [0]$ or $[b]= [0]$. If $n$ is composite then  $a$ or $b$ could be its factors.

        \textbf{Chapter 12: Functions}\\
        Function: $f:A \rightarrow B$ is a relation $f \subseteq A \times  B$ s.t. $\forall a \in A$ there is exactly one ordered pair $(a,b) \in f$ or $f(a) = b$.
        
        \begin{itemize}
            \item A is domain
            \item B is co domain
            \item $\{f(a): a \in A\}$ is range
        \end{itemize}

        A function $f: A \rightarrow B $ is, 
        \begin{itemize}
            \item injective: $f(x_1) = f(x_2) \Rightarrow x_1 = x_2$ OR $x_1\ne x_2 \Rightarrow f(x_1) \ne f(x_2)$
            \item surjective: $\forall y \in B, \exists x \in A, f(x) = y$ 
            \item bijective: injective and surjective
        \end{itemize}


        \textbf{Pigeonhole Principle}
        Given $f: A \rightarrow B$
         \begin{itemize}
            \item If $|A| > |B|$ then $f$ is not injective.
            \item If $|A| < |B|$ then $f$ is not surjective.
        \end{itemize}

        Examples:

        \qquad 1. Show if $a \in \N, \exists k,l$ s.t. $10|a^{k}-a^{l}$ 

        $A = N$ and $B = \{0,\dots,9\}$ and the function  $f:A \rightarrow B$ such that it maps $k \in A$ to the last digit of  $a^{k}$ which will be in $B$. 

        \qquad 2. Show any set of 7 integers contain pair of integers whose sum or difference is divisible by 10.

        $A = \{a_1,\dots,a_7\}$ and $B = \{\{0\}, \{1, 9\},\{2,8\},\{3,7\},\{4,6\},\{5\}\}$. Let  $f:A \rightarrow B$ such that it maps any of the numbers to the set in $B$ that contain its last digit. 


        \textbf{Composition}

    If $f:A \rightarrow B$ and $g:B \rightarrow C$ then, $g \circ f :A \rightarrow C$ is $g(f(x))$

    \begin{itemize}
\item $(h \circ g) \circ f = h \circ (g \circ f)$
    \end{itemize}

   Properties, let $f: A \rightarrow B$, $g: B \rightarrow C$ consider $g \circ f$

   \textbf{1. Show $f$ is injective if $g \circ f$ is injective.}
    
     Assume $f$ is not injective, so $\exists a_1\ne a_2$ s.t. $f(a_1) = f(a_2)$. Now, $g(f(a_1)) = g(f(a_2))$ but as $g\circ f$ is injective this implies $a_1 = a_2$ which contradicts our assumption.

    \textbf{2. Show $g$ is surjective if $g \circ f$ is surjective.}

     Definition implies that $\forall c \in C, \exists a \in A$ s.t. $g(f(a)) = c$. Let  $f(a) = b \in B$. So we have $\forall c \in C, \exists b \in B, g(b) = c \Rightarrow g$ is surjective.

    \textbf{3. Show $f,g$ is bijective $\Rightarrow g \circ f$ is bijective.}
    
    (a). Injectivity: Consider $g(f(a_1)) = g(f(a_2))$. As $g$ is injective we have $ f(a_1) = f(a_2)$, as $f$ is injective we have $a_1 = a_2 \Rightarrow g\circ f$ is injective.

    

    \end{minipage}%
    \end{tabular}%
    \begin{tabular}{@{}c@{}}
    \begin{minipage}[t][\paperheight][t]{0.49\paperwidth}%

    (b). Surjectivity: As  $g$ is surjective $\forall c \in C,\exists b \in B$ s.t. $g(b) = c$. As  $f$ is surjective $\forall b \in B, \exists  a \in A$ s.t. $f(a) = b$. So we have  $\forall c \in C, \exists a \in A$ s.t. $g(f(a)) = c$
        \textbf{Inverse}

        $f:A \rightarrow B$ is bijective $\iff f^{-1}$ is a function from $B \rightarrow A$

        If $A \rightarrow B$ is bijective the inverse is $f^{-1}:B \rightarrow A$ such that $f\circ f^{-1} = i_A$ and $f^{-1} \circ f = i_A$


        \textbf{Image and Preimage}

        If $f:A \rightarrow B$ then, 
        \begin{itemize}
            \item If $X \subseteq A$ image of $X$ is $f(X) = \{f(x): x \in X\} \subseteq B$
            \item If $Y \subseteq B$ preimage of $Y$ is $f^{-1}(Y) = \{x: f(x) \in Y\} \subseteq A$
        \end{itemize}


        Examples ($f:A \rightarrow B$),

        \textbf{1. $f(f^{-1}(Y)) \ne X$ in general}

        If $f$ is not injective the pre-image could contain $a_1$ and $a_2$ but $X$ could only contain $a_1$.

        2. $f$ is injective $\iff X = f^{-1}(f(X)), \forall X \subseteq A$ 
        To show backward direction assume $f$ is not injective and construct $X$ such that it is not true.


        \textbf{Chapter 14: Cardinality}

        $|A| = |B| \iff \exists f: A \rightarrow B$ and $f$ is bijective.

        \textbf{1. $|N| = |Z|$}:   $f: N \rightarrow Z$, $f(n) = \frac{(-1)^{n} (2n - 1) + 1}{4}$

        \textbf{2. $|N| \ne |R|$ }: We can show by using diagonal table.

        \textbf{3. $|(0,\infty)| = |(0,1)|$}: Let $f(x) = \frac{x}{x+1}$ 

        \textbf{4. $|R| = |(0,1)|$}: $|R| = |(0,\infty)|$ by  $g(x) = 2^{x}$ then we use (3.)


        \textbf{5.} $|Q| = |N|$. We show  $|Q^{+}| = |N|$ and $|Q^{-1}| = |N|$ and use union.




        \textbf{Countable And Uncountable Sets}
        
        $|N|\ne |R|$ as there is not bijection from $N \rightarrow R$
        
        $A$ is \textbf{countably infinite} if $|N| = |A|$ or if there is a bijection from $N$ to $A$ else its uncountable.

        A set $A$ is countable infinite $\iff$ its elements can be arranged in an infinite list $a_1,a_2,\dots$ 

        Eg.

        \qquad We can show that $Q$ is countable by plotting a 2x2  graph of all rationals and drawing a snake path from the top left which represents the list $a_1,a_2,\dots$

        If $A$ and $B$ are countably infinite then so is $A \times  B$.
        \qquad True because we can draw 2x2 "matrix" and draw snake from top left indicating each of the elements.

        If $A$ and $B$ are countably infinite, their union in countable infinite.

        \textbf{Power Set: $|A| < |P(A)|$}

        We show this by constructing a $B = \{x \in A: x \not \in f(x)\}$. There is no  $x \in A$ that belongs to this set hence there cannot be a surjection so no bijection. Implies $|A| < |P(A)|$ as we have an injection but no bijection.



        \textbf{Functions to use for bijections in uncountable sets}
        \begin{enumerate}
            \item $\ln(x)$ maps from $R^{+}$ to $R$ as if its smaller than 1 its negative. 
            \item $e^{x},2^{x}$ maps from $R$ to $R^{+}$
            \item $\frac{kx}{x+1}$ maps from $(0,\infty)$ to $(0,k)$. 
            \item Diagonal argument can be used for uncountable sets (set of infinite sequences, reals, etc)
            \item $R \rightarrow R \times R$. Injection from $R$ to $R \times  R$ we have $f(x) = (x, 0)$. For injection from  $R \times  R$ to $R$ we can interleave the decimals for $a,b$ in  $(a,b) \in R \times  R$ to create a new decimal for $R$.
            \item $[0,1) \rightarrow (0,1)$.  $f(x) = \frac{1}{4} + \frac{1}{2}x$ from  $[0,1)$ to  $(0,1)$
            \item $R \times  R = \{(x,y): \text{a condition on x,y}\}$. Easy to show injection to right to left. For left to right we can use $\frac{1}{1 + x}$ and map to a square than can fit in our contour.
        \end{enumerate}


        \textbf{Theorems}
        \begin{enumerate}
            \item A is countable if we can list the elements of $A$ as $a_1,a_2,\dots$
            \item $A$ and $B$ are countable then $A \times  B$ are countable. 
            \item An infinite subset of a countably infinite subset is countably infinite.
            \item $U \subseteq A$ and  $U$ is uncountable  then $A$ is uncountable.
        \end{enumerate}



        
        
    \end{minipage}%
    \end{tabular}%
\end{document}
