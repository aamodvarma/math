\documentclass[a4paper]{report}
\usepackage[utf8]{inputenc}
\usepackage[T1]{fontenc}
\usepackage{textcomp}

\usepackage{url}

% \usepackage{hyperref}
% \hypersetup{
%     colorlinks,
%     linkcolor={black},
%     citecolor={black},
%     urlcolor={blue!80!black}
% }

\usepackage{graphicx}
\usepackage{float}
\usepackage[usenames,dvipsnames]{xcolor}

% \usepackage{cmbright}

\usepackage{amsmath, amsfonts, mathtools, amsthm, amssymb}
\usepackage{mathrsfs}
\usepackage{cancel}

\newcommand\N{\ensuremath{\mathbb{N}}}
\newcommand\R{\ensuremath{\mathbb{R}}}
\newcommand\F{\ensuremath{\mathscr{F}}}
\newcommand\Z{\ensuremath{\mathbb{Z}}}
\renewcommand\O{\ensuremath{\emptyset}}
\newcommand\Q{\ensuremath{\mathbb{Q}}}
\newcommand\C{\ensuremath{\mathbb{C}}}
\let\implies\Rightarrow
\let\impliedby\Leftarrow
\let\iff\Leftrightarrow
\let\epsilon\varepsilon

% horizontal rule
\newcommand\hr{
    \noindent\rule[0.5ex]{\linewidth}{0.5pt}
}

\usepackage{tikz}
\usepackage{tikz-cd}

% theorems
\usepackage{thmtools}
\usepackage[framemethod=TikZ]{mdframed}
\mdfsetup{skipabove=1em,skipbelow=0em, innertopmargin=5pt, innerbottommargin=6pt}

\theoremstyle{definition}

\makeatletter

\declaretheoremstyle[headfont=\bfseries\sffamily, bodyfont=\normalfont, mdframed={ nobreak } ]{thmgreenbox}
\declaretheoremstyle[headfont=\bfseries\sffamily, bodyfont=\normalfont, mdframed={ nobreak } ]{thmredbox}
\declaretheoremstyle[headfont=\bfseries\sffamily, bodyfont=\normalfont]{thmbluebox}
\declaretheoremstyle[headfont=\bfseries\sffamily, bodyfont=\normalfont]{thmblueline}
\declaretheoremstyle[headfont=\bfseries\sffamily, bodyfont=\normalfont, numbered=no, mdframed={ rightline=false, topline=false, bottomline=false, }, qed=\qedsymbol ]{thmproofbox}
\declaretheoremstyle[headfont=\bfseries\sffamily, bodyfont=\normalfont, numbered=no, mdframed={ nobreak, rightline=false, topline=false, bottomline=false } ]{thmexplanationbox}


\declaretheorem[numberwithin=chapter, style=thmgreenbox, name=Definition]{definition}
\declaretheorem[sibling=definition, style=thmredbox, name=Corollary]{corollary}
\declaretheorem[sibling=definition, style=thmredbox, name=Proposition]{prop}
\declaretheorem[sibling=definition, style=thmredbox, name=Theorem]{theorem}
\declaretheorem[sibling=definition, style=thmredbox, name=Lemma]{lemma}



\declaretheorem[numbered=no, style=thmexplanationbox, name=Proof]{explanation}
\declaretheorem[numbered=no, style=thmproofbox, name=Proof]{replacementproof}
\declaretheorem[style=thmbluebox,  numbered=no, name=Exercise]{ex}
\declaretheorem[style=thmbluebox,  numbered=no, name=Example]{eg}
\declaretheorem[style=thmblueline, numbered=no, name=Remark]{remark}
\declaretheorem[style=thmblueline, numbered=no, name=Note]{note}

\renewenvironment{proof}[1][\proofname]{\begin{replacementproof}}{\end{replacementproof}}

\AtEndEnvironment{eg}{\null\hfill$\diamond$}%

\newtheorem*{uovt}{UOVT}
\newtheorem*{notation}{Notation}
\newtheorem*{previouslyseen}{As previously seen}
\newtheorem*{problem}{Problem}
\newtheorem*{observe}{Observe}
\newtheorem*{property}{Property}
\newtheorem*{intuition}{Intuition}


\usepackage{etoolbox}
\AtEndEnvironment{vb}{\null\hfill$\diamond$}%
\AtEndEnvironment{intermezzo}{\null\hfill$\diamond$}%




% http://tex.stackexchange.com/questions/22119/how-can-i-change-the-spacing-before-theorems-with-amsthm
% \def\thm@space@setup{%
%   \thm@preskip=\parskip \thm@postskip=0pt
% }

\usepackage{xifthen}

\def\testdateparts#1{\dateparts#1\relax}
\def\dateparts#1 #2 #3 #4 #5\relax{
    \marginpar{\small\textsf{\mbox{#1 #2 #3 #5}}}
}

\def\@lesson{}%
\newcommand{\lesson}[3]{
    \ifthenelse{\isempty{#3}}{%
        \def\@lesson{Lecture #1}%
    }{%
        \def\@lesson{Lecture #1: #3}%
    }%
    \subsection*{\@lesson}
    \testdateparts{#2}
}

% fancy headers
\usepackage{fancyhdr}
\pagestyle{fancy}

% \fancyhead[LE,RO]{Gilles Castel}
\fancyhead[RO,LE]{\@lesson}
\fancyhead[RE,LO]{}
\fancyfoot[LE,RO]{\thepage}
\fancyfoot[C]{\leftmark}
\renewcommand{\headrulewidth}{0pt}

\makeatother

% figure support (https://castel.dev/post/lecture-notes-2)
\usepackage{import}
\usepackage{xifthen}
\pdfminorversion=7
\usepackage{pdfpages}
\usepackage{transparent}
\newcommand{\incfig}[1]{%
    \def\svgwidth{\columnwidth}
    \import{./figures/}{#1.pdf_tex}
}

% %http://tex.stackexchange.com/questions/76273/multiple-pdfs-with-page-group-included-in-a-single-page-warning
\pdfsuppresswarningpagegroup=1

\author{Aamod Varma}
\setlength{\parindent}{0pt}


\title{MATH3012}
\author{Aamod Varma}
\usepackage{graphicx}
\graphicspath{ {./}}
\begin{document}
\maketitle




\section*{Keller and Trotter}
\section*{7.7}
\subsection*{Problem 1}
We have $147$ third graders and three flavors $M = mint, S = strawberry, C = chocolate$. We are given the following,  
$$ M = 60, C= 103, S=50, M \cap S = 30, M \cap C = 40, C \cap S = 25, M \cap C \cap S = 18 $$

So if $P_1, P_2, P_3$ are the properties respectively we need to find the elements that don't satisfy any of these properties. Using inclusion exclusion we have,  
$$ N(S)  = N(\phi) - N(S_1) - N(S_2) - N(S_3) + N(S_1 \cup S_2) + N(S_2 \cup S_3) + N(S_1 \cup S_3) - N(S_1 \cup S_2 \cup S_3)$$ 

This is, 
$$ 147 - 60 - 103 - 50 + 30 + 40 + 25 - 18 = 11$$ 

\subsection*{Problem 4}
So we have three properties $P_1, P_2, P_3$ and let each of them be respectively number divisible by $2,3$ and  $5$. SO we need numbers less than 100 that satisfy each of these properties. First we find numbers that don't satisfy these properties this would be,
$$ N(S)  = N(\phi) - N(S_1) - N(S_2) - N(S_3) + N(S_1 \cup S_2) + N(S_2 \cup S_3) + N(S_1 \cup S_3) - N(S_1 \cup S_2 \cup S_3)$$ 
where $S_i$ is the set of elements that satisfy  $P_i$.

 We have $N(S_1) = 50, N(S_2) = 33, N(S_2) = 20$. Now numbers divisible by both 3 and 2 means divisible by 6 which is $16$, 3 and 5 means divisibility by 15 which is 6 in number and 2 and 5 means 10 which is 10 in number and lastly 2 3 and 5 means 30 which is 3 in number.

 So we get,  
 $$ N(S) = 100 - 50 - 33- 20 + 16 + 6 + 10 - 3 = 26 $$ 

 So the number that doesn't satisfy any of them are 26.
\subsection*{Problem 7}
We have, 
$$ x_1 + x_2 + x_3 + x_4 = 32 $$ with $0 \le x_i \le 10$ for each $i$

The solution will be the total integer solutions minus the integral solutions where $x_i \ge 11$. Firstly total sol is,  
$$ {32}  \choose {3}$$ 


Now let $X_1,\dots,X_4$ represent the number of solutions for which $x_i \ge 11$ . Using inclusion exclusion we have,  
\begin{align*}
    N(X_1 \cup \dots \cup X_4) &= \sum_{i \le 4} |X_i|\\
                               &-  \sum_{1 \le i < j \le 4} |X_i \cap X_j|\\
                               &+ \sum_{1 \le i < j < k \le 4} |X_i \cap X_j \cap X_k|\\
                               &- |X_1 \cap \dots \cap X_4|
\end{align*}

Now we can easily see that the last two are zero as if each have at least 11 elements then we'll go over 32. So only the first two are relevant.

For the first we take an arbitrary $i$ and set it as greater than equal to 11. This gives us, 
$$ {32 - 11 + 4 - 1 \choose 3 }= {24 \choose 3 }$$ 

We have four choices of $i$ so we get,  
$$ 4 {24 \choose 3} $$ 

For the second term we take $i,j$ and set them greater than 11. This gives us,  
$$ {32- 22 + 4 - 1 \choose 3} ={ 13 \choose 3 }$$ 

There are $4 \choose 2$ ways to choose  $i,j$ which is  6. Which gives us,  
$$ 6 {13 \choose 3} $$ 

So our final answer will be, 
$$ {35 \choose 3} - 4 {24 \choose 3} + 6 {13 \choose 3} $$ 

\subsection*{Problem 10}
We have 268 students. Let our properties be  $P_m, P_k, P_j$. We have  $37$ that do not speak any of the languages. And the following, 
 \begin{align*}
     |M| &= 174\\
     |J| &= 139\\
     |K| &= 112\\
     |M \cap J| &= 102\\
     |M \cap K| &= 81\\
     |J \cap K| &= 71\\
\end{align*}

We need to find $|M \cap J \cap K|$. We know that,  
$$ 268 - |M| - |J| - |K| + |M \cap J | + |J  \cap K | + |K \cap M| - |M \cap J \cap K| = 37  $$ 

This gives us, 
$$ -|M \cap J \cap K| = 37 - 268 + 174 + 139 + 112 - 102 - 81 - 71 $$ 

Which gives us, 
$$ |M \cap J \cap K| = 60 $$ 
\subsection*{Problem 14}
We need the number of surjections from an 8 element set to a 6 element set.

Using the formula we have, 
$$ S(8,6) = \sum_{k = 0}^6 (-1)^{k} {6 \choose k} (6 - k)^{n} $$ 

This gives us,
\begin{align*}
    S(8,6) &= {6 \choose 0} (6 - 0)^{8} - {6 \choose 1} (6 - 1)^{8} + {6 \choose 2} (6 - 2)^{8} - {6 \choose 3} (6 - 3)^{8} + {6 \choose 4} (6 - 4)^{8} - {6 \choose 5} (6 - 5)^{8}\\
 &= {6 \choose 0} (6)^{8} - {6 \choose 1} (5)^{8} + {6 \choose 2} (4)^{8} - {6 \choose 3} (3)^{8} + {6 \choose 2} (2)^{8} - {6 \choose 1} \\
 &=  6^{8} - {6 \choose 1} (5^{8} - 1) + {6 \choose 2} (4^{8} + 2^{8}) - {6 \choose 3} (3)^{8}  \\
\end{align*}
\subsection*{Problem 15}
We have 10 distinct objects to be given to 4 distinct groups. So it is equivalent to the number of surjections from 10 element set to 4 element set. This is just $S(10, 4)$
\subsection*{Problem 19}
The number of rearrangements for 9 elements are $d_9$ which is,  
$$ d_9 = \sum_{k = 0}^{9} (-1)^{k} {9 \choose k}(9 - k)! $$ 
\subsection*{Problem 24}
We need $\phi(756)$ which is,  

The prime factors of $756$ are  $2, 3, 7$
$$ 756 \prod_{i = 1}^{m} \frac{p_i - 1}{p_i} = 756 \times \frac{1}{2} \times \frac{2}{3} \times   \frac{6}{7} = 216 $$ 
\section*{8.8}
\subsection*{Problem 2}
\textbf{a.} The generating function can be written as $G(x) = x + x^2 + x^{3} + x^{4} +  \dots$. But we know that, 
$$ 1 + x + x^2 + \dots = \frac{1}{1 - x} $$  

so we have, 
\begin{align*}
    G(x) =  x + x^2 + x^{3} + \dots &= \frac{1}{1-x} - 1 \\
                                    &= \frac{x}{1 - x} 
\end{align*}

\textbf{b.} We have $1, 0, 0, 1, 0, 0, \dots$. The generating function would be,  
$$ G(x) = 1 + x^{3} + x^{6} + \dots $$ 

But we know that, 
$$ \frac{1}{1 - x} = 1 + x + x^2 + x^{3} $$ 

which means that, 
$$ \frac{1}{1 - x^{3}} = 1 + x^{3} + x^{6} + \dots $$ 

Hence we have, 
$$ G(x) = \frac{1}{1-x^{3}} $$ 

\textbf{c. }  We have $1,2,4,8,16,32,\dots$. The generating function  would look like,  
$$ G(x) = 1 + 2x^2 + 4x^{3} + 8x^{3} + \dots = \sum_{n = 0}^{\infty} 2^{n} x^{n} = \sum_{n = 0}^{\infty} (2x)^{n} $$ 

But we know that, 
$$ \frac{1}{1-x} = \sum_{n = 0}^{\infty} x^{n} $$ 

So we have, 
$$ G(x) = \frac{1}{1 - 2x}  $$ 

\textbf{d. }  We have $0,0,0,0,1,1,1,\dots$. Which looks like,  
$$ G(x) = x^{4} + x^{5} + \dots = x^{4} (1 + x^2 + x^{3} + \dots) = x^{4} \frac{1}{1-x}$$ 
$$ = \frac{x^{4}}{1 - x} $$ 


\textbf{e. } We have $1, -1, 1, -1, \dots$. Generating function would be,  
$$ G(x) = 1 - x + x^2 - x^{3} + x^{4} + \dots = \sum_{n = 0}^{\infty} (-1)^n x^{n} $$ 

But we know that, 
$$ \frac{1}{1-x} = \sum_{n=0}^{\infty} x^{n}$$

Which means that, 
$$ \frac{1}{1 + x} = \sum_{n=0}^{\infty} (-x)^{n} = \sum_{n=0}^{\infty} (-1)^{n} x^{n} $$ 

Which means we have, 
$$  G(x) = \frac{1}{1 + x}$$ 


\textbf{f. } We have, $\displaystyle \displaystyle 2^8,2^7\binom{8}{1}, 2^6\binom{8}{2},\dots,\binom{8}{8},0,0,0,\dots$. 

Our generating function would be, 
$$ 2^{8}x^{0} + 2^{7}{8 \choose 1}x + \dots + 2^{0}{8 \choose 8}x^{8} $$ 
We know that, 
\begin{align*}
    (1 + x)^{p} &= \sum_{n=0}^{p} {p \choose n} x^{n}\\
    \bigg(1 + \frac{x}{2}\bigg)^{8} &= \sum_{n=0}^{p} {8 \choose n} \bigg(\frac{x}{2}\bigg)^{n}\\
    2^{8}\bigg(1 + \frac{x}{2}\bigg)^{8} &= \sum_{n=0}^{p} {8 \choose n} 2^{8 - n}x^{n} \\
\end{align*}

which is $G(x)$ so we have,  
$$ G(x)  =  (2 + x)^{8}$$

\textbf{g.}  We have $\displaystyle 1,1,1,0,0,1,1,1,1,1,1,1,1,1,\dots$ so, 
$$ G(x) = 1 + x + x^2 + x^{5} + x^{6}+ \dots $$ 

we can separate it as follows and get,
\begin{align*}
    G(x) &= (1 + x + x^2) + x^{5}(1 + x + x^{2} + \dots)\\
         &= \frac{1 - x^{3}}{1 - x} + x^{5}\frac{1}{1 - x}\\
         &= \frac{1 - x^{3} + x^{5}}{1 - x}
\end{align*}

\textbf{h.}  We have $\displaystyle 0,0,0,1,2,3,4,5,6,\dots$ so, 
\begin{align*}
    G(x) &= x^{3} + 2x^{4} + 3x^{5} + \dots \\
         &= x^{3}( 1 + 2x + 3x^2 + 4x^{3} + \dots)\\
         &= x^{3} (\frac{1}{(1-x)^2}) \\
         &= \frac{x^{3}}{(1 - x)^2}
\end{align*}

\textbf{i.} 
We have $\displaystyle 3,2,4,1,1,1,1,1,1,\dots$. The generating function would be, 
$$ G(x) = 3 + 2x + 4x^2  + x^{3}(1 + x + x^2 + \dots)$$  
$$ = 3 + 2x + 4x^2  + \frac{x^{3}}{1 - x}$$

We can combine these as, 
$$ \frac{3 - x + 2x^2 - 3x^{3}}{1 - x} $$ 

\textbf{j. } We have $\displaystyle 0,2,0,0,2,0,0,2,0,0,2,0,0,2,\dots$ whose generating function would be,  
\begin{align*}
    G(x)  = 2x + 2x^{4} + 2x^{7} &= 2x(1 + x^{3}  + x^{6} + \dots)\\
\end{align*}

But we know that, 
$$ \frac{1}{1 - x} = 1 + x + x^2 + \dots $$ 

so, 
$$ \frac{1}{1 - x^{3}} = 1 + x^{3} + x^{6} + \dots $$ 

So we have, 
$$ G(x) = 2x( \frac{1}{1-x^{3}}) = \frac{2x}{1- x^{3}} $$ 

\textbf{k.}  We have $\displaystyle 6,0,-6,0,6,0,-6,0,6,\dots$ whose generating function is, 
\begin{align*}
    G(x) &= 6 -6x^2 + 6x^{4} + \dots \\
         &= 6( 1 - x^{2} + x^{4} + \dots)
\end{align*}
We know, 
$$ \frac{1}{1 + x} = 1 - x + x^2 + \dots $$ 
so, 
$$ \frac{1}{1 + x^2} = 1 - x^2  + x^{4} + \dots $$ 

We get, 
$$ G(x) = 6(\frac{1}{1 + x^2}) = \frac{6}{1 + x^2}$$



\textbf{l.}  We have $\displaystyle \displaystyle 1,3,6,10,15,\dots,\binom{n+2}{2},\dots$ which gives us,  
\begin{align*}
    G(x) &= \sum_{n=0}^{\infty} {n + 2\choose 2} x^{n} \\
         &= \sum_{n=0}^{\infty} \frac{(n + 2)!}{2! n!} x^{n}\\
         &=\frac{1}{2} \sum_{n=0}^{\infty} (n + 2)(n + 1) x^{n}\\
         &= \frac{1}{2} \bigg(\frac{1}{1-x^{}} \bigg)^{''}\\
         &= \frac{1}{(1-x)^{3}}
\end{align*}
\subsection*{Problem 3}
\textbf{a.} The $n'th$ term would be, 
$$ 10 \choose n $$ 


\textbf{b.}  We have $\frac{1}{1- x^{4}}$ which is, 
$$ \sum_{k=0}^{\infty} x^{4k}$$

So if $4 | n$ then  $a_n = 1$ else  $a_n = 0$

\textbf{c. } We have $\frac{x^{3}}{1 - x^{4}}$ which is, 
$$ \sum_{k=0}^{\infty} x^{4k + 3} $$ 

So if $n \equiv 3 \mod 4$ then  $a_n = 1$ else  $a_n = 0$ 

\textbf{d.}  We have $\frac{1 - x^{4}}{1 - x}$ which is equivalent to $1 + x + x^2 + x^{3}$ so if $1 \le n \le 4$ then $a_n = 1$ else $a_n = 0$

\textbf{e. }  We have, 
$$ \frac{1 + x^2 - x^{4}}{1 - x} $$  which is, 
$$ \frac{1 - x^{4}}{1 - x} + \frac{x^2}{1 - x} $$ 

We know the first term is $1 + x + x^2 + x^{4}$. And the second term will be, 
$$ x^2 (1 + x + x^2 + \dots) = x^2 + x^{3} + x^{4} + \dots $$ 

So combining we get, 
$$ 1 + x + 2x^2 + 2x^{3} + x^{4} + \dots $$ 

so for $n = 3,4$ we have $a_n = 2$ else $a_n = 1$

\textbf{f.} We have, 
$$ \frac{1}{1 - 4x} = \sum_{n=0}^{\infty} (4x)^{n} $$ 

So the $n'th$ term will be $4^{n}$

\textbf{g.}  We have $\frac{1}{1 + 4x}$ 
whis is, 
$$ \frac{1}{1+ 4x} = \sum_{n=0}^{\infty} (-4x)^{n} $$ 

So the $n'th$ term of the sequence will be  $(-4)^n$
\textbf{h.} We have, 
$$ \frac{x^{5}}{(1 - x)^{4}}$$ 


We can write this as, 
\begin{align*}
&= x^{5} \frac{1}{6}(\sum_{n=0}^{\infty} (n + 3)(n +  2)(n + 1) x^{n})\\
&= \sum_{n=0}^{\infty} \frac{1}{6}(n + 3)(n +  2)(n + 1) x^{n + 5}\\
&= \sum_{n=5}^{\infty} \frac{1}{6} (n - 2 )(n - 3)(n - 4) x^{n}\\
\end{align*}

So the $n'th$ term if $n \ge 5$  will be $\frac{1}{6}(n - 2)(n - 3)(n - 4) = {n - 2 \choose 3}$ and if smaller than 5 then 0.


\textbf{i}. We have, 
$$ \frac{x^2 + x + 1}{1- x^{7}}  = (x^2 + x + 1) (1 + x^{7} + x^{14} + \dots)$$ 
$$ = 1 + x + x^2 + x^{7} + x^{8} + x^{9} + x^{14} + x^{15} + x^{16} + \dots$$ 

So if $n \equiv 0,1,2 \pmod 7$ then  $a_n = 1$ else $a_n = 0$

\textbf{j.}  We have, 
\begin{align*}
    3x^{4} + 7x^{3} - x^2 + 10 + \frac{1}{1 - x^{3}} &= 3x^{4} + 7x^{3} - x^2 + 10 + \sum_{n=0}^{\infty} x^{3n}\\
                                                     &= 3x^{4} + 7x^{3} - x^2 + 10  +(1 + x^{3} + x^{6} + \dots)\\
                                                     &= 3x^{4} + 8x^{3} - x^2 + 11 + x^{6} + \dots\\
                                                     &=11 - x^2+ 8x^{3} + 3x^{4}  + x^{6} + \dots\\
\end{align*}

So for $n = 0, a_n = 11, n = 2, a_n = -1, n = 3, a_n = 8, n = 4, a_n = 3, 3 | n $ and  $n \ge 6$  n is $1$ else  $n =0$
\subsection*{Problem 5}

We need to create a bunch of $n$ balloons from $W, G, B$ balloons such that it contains at least 1W, 1G and at most 2B. So we have, 
$$ W + G + B = 10 \qquad 1\le W, 1\le G, 0 \le B \le 2$$ 

To find the solution we can find the generating function of each of the variables and multiply them together. For $W$ and $G$ we have, 
$$ 0 + x + x^2 + \dots = \frac{x}{1 - x}$$ 

And for $B$ we have,  
$$ 1 + x + x^2 $$ 

Multiplying them all we have, 
$$ \frac{x^2}{(1 - x)^2}  (1 + x + x^2) = \frac{x^2 + x^{3} + x^{4}}{(1 - x)^2}$$ 

Now we need to find the coefficient of the $10th$ or in general the $n'th$ term. We know that, 
$$ \frac{1}{1 - x} = \sum_{n=0}^{\infty} x^{n} $$  so we have, 
$$ \frac{1}{(1 - x)^2} = \sum_{n=0}^{\infty} (n + 1)x^{n} $$ 

This gives us, 
\begin{align*}
    \frac{x^2 + x^{3} + x^{4}}{(1 - x)^2} &= \sum_{n=0}^{\infty} (n + 1)x^{n + 2} + \sum_{n=0}^{\infty} (n + 1)x^{n + 3} + \sum_{n=0}^{\infty} (n + 1)x^{n + 4} \\
                                          &= \sum_{n=2}^{\infty} (n - 1)x^{n} + \sum_{n=3}^{\infty} (n - 2) x^{n} + \sum_{n=4}^{\infty} (n - 3)x^{n}
\end{align*}

So the coefficient of the $n'th$ term is  $n - 1 + n - 2+ n - 3 = 3n - 6$. So for  $n = 10$ we have $30 - 6= 24$

\subsection*{Problem 11}
We have dollar coins and $1,2,5$ dollar bills. For each one of tem we have, 
\begin{align*}
    \frac{1}{1 - x} = \sum_{n=0}^{\infty} x^{n}\\
    \frac{1}{1 - x} = \sum_{n=0}^{\infty} x^{n}\\
    \frac{1}{1 - x^2} = \sum_{n=0}^{\infty} x^{2n}\\
    \frac{1}{1 - x^{5}} = \sum_{n=0}^{\infty} x^{5n}\\
\end{align*}

The number of ways to make 100 with these would be the coefficient to $x^{100}$ in,
$$ \frac{1}{(1 - x)^2(1 - x^2)(1 - x^{5})} $$ 

which is $19006$


\subsection*{Problem 17}

We need to partition 10 into odd parts which is equivalent to the number of partitions of $n$ into distinct parts which is, 
$$ D(x) = \prod_{n = 1}^{\infty} (1 + x^{n}) $$ 

So we need the coefficient of the $x^{10}$ term of this, 
$$ (1 + x)(1 + x^2)(1 + x^{3}) \dots (1 + x^{10}) \dots $$ 

Which would be $10$
\subsection*{Problem 20}
\textbf{a. }  $a_n = 5^{n}$ so we have, 

$$ e^{5x} = \sum_{n=0}^{\infty} 5^{n}x^{n} \frac{1}{n!} $$ 

So we have, $G(x) = e^{5x}$


\textbf{b. }  We have $a_n = (-2)^{n}$

$$ e^{-2x} = \sum_{n=0}^{\infty} \frac{(-2x)^{n}}{n!} = \sum_{n=0}^{\infty} (-2)^{n} x^{n} \frac{1}{n!} $$ 

So, 
$$ G(x) =e^{-2x}$$

\textbf{c. }  $a_n = 3^{n + 2} = 9 \cdot 3^{n}$. Now, 

$$ e^{3x} = \sum_{n=0}^{\infty} 3^{n}x^{n} \frac{1}{n!} $$ 
$$ 9e^{3x} = \sum_{n=0}^{\infty} 3^{n + 2} x^{n} \frac{1}{n!} $$ 

So, 
$$ G(x) = 9e^{3x} $$ 

\textbf{d. } $a_n = n!$ 
We have, 
$$ G(x) = \sum_{n=0}^{\infty} a_n x^{n} \frac{1}{n!} $$ 
$$ G(x) = \sum_{n=0}^{\infty} n! x^{n} \frac{1}{n!} $$ 

So, 
$$ G(x) = \sum_{n=0}^{\infty} x^{n} = \frac{1}{1 - x} $$ 

\textbf{e.}  $a_n = n$ so,  
$$ G(x) = \sum_{n=0}^{\infty} n x^{n} \frac{1}{n!} = \sum_{n=0}^{\infty} x^{n} \frac{1}{(n - 1)!} $$ 
$$ = x \sum_{n=0}^{\infty} x^{n} \frac{1}{n!} = x e^{x} $$ 


 \textbf{f. } $\frac{1}{1 + n}$
 So, 
 $$ G(x) = \sum_{n=0}^{\infty} \frac{1}{n + 1} x^{n} \frac{1}{n!} = x^{n} \frac{1}{(n + 1)!} $$ 
 $$ = \frac{1}{x} \sum_{n=0}^{\infty} x^{n } \frac{1}{n!} = \frac{e^{x}}{x} $$ 
\subsection*{Problem 26}

For each condition the following generating function are there, 
\begin{enumerate}
\item We have 5 vowels and for each vowel at least one time. For each vowel we have $a_n = 1, n \ge 1$. Which is $G(x) = e^{x} - 1$ this gives us for 5 vowels,
$$ G(x) = (e^{x} - 1)^{5}$$ 
\item T must appear at least three times so $T \ge 3$ which is so, 
$$ G(x) = (e^{x} - (1 + x + \frac{x^2}{2!})) $$ 
\item $Z$ appears at most three times so $$G(x) = \sum_{n=0}^{3} x^{n}\frac{1}{n!} = 1 + x +\frac{x^2}{2} + \frac{x^{3}}{6}$$
\item The rest of the letter (26 - 7 = 19) can appear how many ever times which gives us, 
$$ G(x) = e^{19x} $$ 
\item Each even digit appears even times, there are 5 even digits so we have the sequence for an even digits as $1, 0, 1, 0, \dots$
$$ e^{x} = \sum_{n=0}^{\infty} x^{n} \frac{1}{n!} $$ 
$$ e^{-1} = \sum_{n=0}^{\infty} -1^{n}x^{n}\frac{1}{n!} $$ 

So we have, 
$$ G(x) = \frac{e^{x} + e^{-x}}{2} = 1 + \frac{x^2}{2!} + \frac{x^{4}}{4!} + \dots $$ 

For 5 even we'll have, 
$$ G(x) =  \bigg  (\frac{e^{x} + e^{-x}}{2}  \bigg )^{5}$$ 
\item Each odd digits appears odd times, there are 5 odd digits, for each one the sequence is $0, 1, 0, 1, \dots$. 

$$ G(x) =  \bigg  (\frac{e^{x} - e^{-x}}{2}  \bigg )^{5}$$ 
\end{enumerate}

So multiplying all of this would give us the generating function which is, 
$$ (e^{x} - 1)^{5}\bigg (e^{x} - \bigg (1 + x + \frac{x^2}{2!} \bigg) \bigg) \bigg ( 1 + x +\frac{x^2}{2} + \frac{x^{3}}{6} \bigg) e^{19x}  \bigg  (\frac{e^{x} + e^{-x}}{2}  \bigg )^{5}  \bigg  (\frac{e^{x} - e^{-x}}{2}  \bigg )^{5}$$ 

\end{document}
