\documentclass[a4paper]{report}
\usepackage[utf8]{inputenc}
\usepackage[T1]{fontenc}
\usepackage{textcomp}

\usepackage{url}

% \usepackage{hyperref}
% \hypersetup{
%     colorlinks,
%     linkcolor={black},
%     citecolor={black},
%     urlcolor={blue!80!black}
% }

\usepackage{graphicx}
\usepackage{float}
\usepackage[usenames,dvipsnames]{xcolor}

% \usepackage{cmbright}

\usepackage{amsmath, amsfonts, mathtools, amsthm, amssymb}
\usepackage{mathrsfs}
\usepackage{cancel}

\newcommand\N{\ensuremath{\mathbb{N}}}
\newcommand\R{\ensuremath{\mathbb{R}}}
\newcommand\F{\ensuremath{\mathscr{F}}}
\newcommand\Z{\ensuremath{\mathbb{Z}}}
\renewcommand\O{\ensuremath{\emptyset}}
\newcommand\Q{\ensuremath{\mathbb{Q}}}
\newcommand\C{\ensuremath{\mathbb{C}}}
\let\implies\Rightarrow
\let\impliedby\Leftarrow
\let\iff\Leftrightarrow
\let\epsilon\varepsilon

% horizontal rule
\newcommand\hr{
    \noindent\rule[0.5ex]{\linewidth}{0.5pt}
}

\usepackage{tikz}
\usepackage{tikz-cd}

% theorems
\usepackage{thmtools}
\usepackage[framemethod=TikZ]{mdframed}
\mdfsetup{skipabove=1em,skipbelow=0em, innertopmargin=5pt, innerbottommargin=6pt}

\theoremstyle{definition}

\makeatletter

\declaretheoremstyle[headfont=\bfseries\sffamily, bodyfont=\normalfont, mdframed={ nobreak } ]{thmgreenbox}
\declaretheoremstyle[headfont=\bfseries\sffamily, bodyfont=\normalfont, mdframed={ nobreak } ]{thmredbox}
\declaretheoremstyle[headfont=\bfseries\sffamily, bodyfont=\normalfont]{thmbluebox}
\declaretheoremstyle[headfont=\bfseries\sffamily, bodyfont=\normalfont]{thmblueline}
\declaretheoremstyle[headfont=\bfseries\sffamily, bodyfont=\normalfont, numbered=no, mdframed={ rightline=false, topline=false, bottomline=false, }, qed=\qedsymbol ]{thmproofbox}
\declaretheoremstyle[headfont=\bfseries\sffamily, bodyfont=\normalfont, numbered=no, mdframed={ nobreak, rightline=false, topline=false, bottomline=false } ]{thmexplanationbox}


\declaretheorem[numberwithin=chapter, style=thmgreenbox, name=Definition]{definition}
\declaretheorem[sibling=definition, style=thmredbox, name=Corollary]{corollary}
\declaretheorem[sibling=definition, style=thmredbox, name=Proposition]{prop}
\declaretheorem[sibling=definition, style=thmredbox, name=Theorem]{theorem}
\declaretheorem[sibling=definition, style=thmredbox, name=Lemma]{lemma}



\declaretheorem[numbered=no, style=thmexplanationbox, name=Proof]{explanation}
\declaretheorem[numbered=no, style=thmproofbox, name=Proof]{replacementproof}
\declaretheorem[style=thmbluebox,  numbered=no, name=Exercise]{ex}
\declaretheorem[style=thmbluebox,  numbered=no, name=Example]{eg}
\declaretheorem[style=thmblueline, numbered=no, name=Remark]{remark}
\declaretheorem[style=thmblueline, numbered=no, name=Note]{note}

\renewenvironment{proof}[1][\proofname]{\begin{replacementproof}}{\end{replacementproof}}

\AtEndEnvironment{eg}{\null\hfill$\diamond$}%

\newtheorem*{uovt}{UOVT}
\newtheorem*{notation}{Notation}
\newtheorem*{previouslyseen}{As previously seen}
\newtheorem*{problem}{Problem}
\newtheorem*{observe}{Observe}
\newtheorem*{property}{Property}
\newtheorem*{intuition}{Intuition}


\usepackage{etoolbox}
\AtEndEnvironment{vb}{\null\hfill$\diamond$}%
\AtEndEnvironment{intermezzo}{\null\hfill$\diamond$}%




% http://tex.stackexchange.com/questions/22119/how-can-i-change-the-spacing-before-theorems-with-amsthm
% \def\thm@space@setup{%
%   \thm@preskip=\parskip \thm@postskip=0pt
% }

\usepackage{xifthen}

\def\testdateparts#1{\dateparts#1\relax}
\def\dateparts#1 #2 #3 #4 #5\relax{
    \marginpar{\small\textsf{\mbox{#1 #2 #3 #5}}}
}

\def\@lesson{}%
\newcommand{\lesson}[3]{
    \ifthenelse{\isempty{#3}}{%
        \def\@lesson{Lecture #1}%
    }{%
        \def\@lesson{Lecture #1: #3}%
    }%
    \subsection*{\@lesson}
    \testdateparts{#2}
}

% fancy headers
\usepackage{fancyhdr}
\pagestyle{fancy}

% \fancyhead[LE,RO]{Gilles Castel}
\fancyhead[RO,LE]{\@lesson}
\fancyhead[RE,LO]{}
\fancyfoot[LE,RO]{\thepage}
\fancyfoot[C]{\leftmark}
\renewcommand{\headrulewidth}{0pt}

\makeatother

% figure support (https://castel.dev/post/lecture-notes-2)
\usepackage{import}
\usepackage{xifthen}
\pdfminorversion=7
\usepackage{pdfpages}
\usepackage{transparent}
\newcommand{\incfig}[1]{%
    \def\svgwidth{\columnwidth}
    \import{./figures/}{#1.pdf_tex}
}

% %http://tex.stackexchange.com/questions/76273/multiple-pdfs-with-page-group-included-in-a-single-page-warning
\pdfsuppresswarningpagegroup=1

\author{Aamod Varma}
\setlength{\parindent}{0pt}


\title{MATH3012}
\author{Aamod Varma}
\begin{document}
\maketitle


\section*{Page 121}
\subsection*{Problem 13}
(a). $p(0,0): 0 - 0 = 0 + 0^2$ which is $ 0 = 0$ which is \textbf{true}

(b). $p(1,1): 1 - 1 = 1 + 1^2$ which is $0 = 2$ which is  \textbf{false} 

(c). $p(0,1): 1 - 0 = 1 + 0^2$ which is $1 = 1$ which is  \textbf{true}     

(d). For any $y$ we have, 
$$ y - 0 = y + 0^2 \text{ or } y = y $$  which  means that, 
$$ p(0,y) $$  is \textbf{true}


(e). We need to check the existance $\exists y$ s.t. $p(1,y)$ or that,  
$$ y - 1 = y + 1 $$ which is equivalent to, 
$$ 2 = 0 $$  which is \textbf{false}  so, 
$$ \not \exists y \text{ such that }p(1,y) $$ 


(f). Need to check if $\forall x, \exists y$  s.t. $p(x,y)$

Let us take any arbitrary $x$ we have, 
\begin{align*}
    y - x &= y + x^2 \\
    0 &= x + x^2 \\
\end{align*}

Which is not true for any $x$, hence we can't find a  $y$ for any $x$ such that $p(x,y)$ is true. So it is \textbf{false} 

(g). Need to show $\exists y$ such that  $\forall x$ we have $p(x,y)$

So let the  $y'$  be such a $y$ where this is true so we have for any $x \in Z$
$$ y' - x = y' + x^2 $$  
$$  0 =  x^2  + x$$  

Now we can choose $x = 1$ and see that is not true - so we get a contradiction. Hence $\not \exists y $ such that  $\forall x$ we have  $p(x,y)$. So \textbf{false} 


(h). To show that $\forall y$ we have $x$ such that $p(x,y)$. Take any  $y$ so we have, 
$$ y - x = y + x^2 $$ 
$$ 0 = x + x^2 $$ 

A solution to this can be if $x = 0$. So we see that for any $y \in Z$ we have $x = 0$ such that, 
$$ p(x,y) $$ So the statement is \textbf{true} 


\subsection*{Problem 14}

(a). We need to see if $\forall x, \exists y, \exists z (x = 7y + 5z)$

So essentially for any choice of $x$ we need to be able to find $y,z$ such that $7y + 5z = x$. First we see that $GCD(7,5) = 1$. This means that we have $y,z$ such that, 
$$ 7y' + 5z' = 1 $$ 
More specifically if $y' = 3, z' = -4$ we have $7 \cdot 3 - 5 \cdot 4 = 1$.

So now for any x we also have,  
\begin{align*}
    x(7\cdot 3 + 5\cdot -4) &= x \cdot 1 = x\\
    7\cdot 3x + 5\cdot -4x &= x
\end{align*}

So using any choice of $x$ we have shown, $\exists y = 3x$ and $\exists z = -4x$ such that, 
$$ 7y + 5z = x $$ 

Statement is \textbf{True} 

(b). This statement is \textbf{False}. First we see that $4$ and $6$ share common factor of 2. Hence GCD is not 1. So $ \not \exists y,z$ such that, 
$$ 4y + 6z = 1 $$ 

This is a counterexample to the claim. In general however because both $y,z$ are integers that means $4y $ and  $6z$ are even and their sum would be even. So for any choice of $x$ which is odd cannot be represented.


\section*{Page 277-278}
\subsection*{Problem 4}
Consider writing the  set $S$ as the set of sets, 
$$ A = \{\{3\}, \{7, 103\}, \dots, \{55\}\} $$ 
We know that $|A| = 14$. Now consider the function  $f$ that maps a subset $S'$ to $A$. Now we have $f$ is not injective i.e. two elements in $S'$ mapping to the same element in $A$ (only possible for sets except 3,55) if $|S'| > |A|$ and the smallest would be  $|A| + 1 = 14 + 1 = 15$. So  $|S'|$ must at least be  $15$ to ensure that two elements in $S'$ are mapped to the same in $A$ which must imply that they add up to 110.

\subsection*{Problem 6}
\begin{proof}
    Consider the set, 
    $$A =  \{\{1,2\},\{3,4\},\dots,\{199,200\}\} $$ 
    Such that $|A| = 100$

    Now if we choose  $101$ integers then at least two integers will be in the same set in $A$. So at least two integers will be consecutive elements. And we know  $ GCD(n,n+1) = 1$.
\end{proof}

\subsection*{Problem 10}

First let us divide our triangle into 9 different smaller equilateral triangles of side $1 /3$. Now let us consider a function that takes a set of points in  our main triangle and assigns it to one of the 9 triangles. The smallest cardinality of our domain would be  $10$ for our function to be not injective. So this would mean that at least two points would be assigned the same triangle. As sides of the triangle are $1 /3$ in length that means that those points must at least be smaller  than $1 /3$
\subsection*{Problem 14}
Let $x_i$ be the total number of resume that he sent out from the start to the $i$'th day such that $1 \le i \le 42$. We know that,  
$$ 1 \le x_1 < \dots < x_{42} \le 60 $$ 
$$ x_1 + 23  < \dots <x_{42} + 23 \le 83 $$ 

Together we have $84$ numbers between $1$ and $83$. So that must mean that for some $x_i$ we have $x_j$ such that, 
$$ x_i = x_j + 23 $$ 

Or that, 
$$ x_i - x_j = 23 $$ 

Which is means that there are cumulative days between $i$ and $j$ where he sent out exactly 23 resumes.



\section*{Keller and Trotter}
\subsection*{Chapter 2}

\subsection*{Problem 16}
(a). $${62 \choose 4}$$

(b).$${67 \choose 4}$$

(c). We consider another variable $x_6$ to maintain the balance. As the inequality is not strict $x_1,\dots,x_6$ can equal to 0. Hence our solution is, 
$$ 68 \choose 63 $$ 

(d).
We need $x_2$ to be at least 10. So let us first allocate $x_2$ with $10$ let this be $x_2'$
$$ x_1 + x_2' + x_3 + x_4 + x_5 = 53$$ 

So now we need to find solutions to the above which is, 
$$ 57 \choose 53 $$ 

(e). First we have total solutions as, 
$$ 67 \choose 4 $$ 

Now we have solutions where $x_2 >= 10$ as, 
$$ 57 \choose 4 $$ 
So the difference is, 
$$ {67 \choose 4} - {57 \choose 4} $$ 

\subsection*{Problem 20}
Consider choosing a team $k$ in size with a singular member in the team as the captain. Now we can choose $k$ people from $n$ using, 
$$ n \choose k $$ 
Now as any of the $k$ members can be a captain we have, 
$$ k {n \choose k} $$ 


Now another way of choosing the team is to find the number of ways to choose a $k-1$ members from a smaller set of $n - 1$ people adding on a captain (the one not in the $n - 1$). Ways to choose $k - 1$ teams from an arbitrary $n - 1$ group is, 
$$ {n -1 \choose k - 1 } $$ 

Now we have  $n$ ways to have a captain sit out to make $n - 1$. So for each member we have another group to give us, 
$$ n {n - 1\choose k - 1} $$ 

\subsection*{Problem 29}
Our multinomial formula gives us, 
$$ {100 \choose k_1,k_2,k_3} (2x)^{k_1}  (3y^2)^{k_2}  z^{k_3}$$ 

So we have $k_1 = 15, k_2 = 60, k_3 = 25$. 
$$ {100 \choose 15,60,25} 2^{15} x^{15}  3^{60}y^{120} z^{25} $$ 

So our coefficient must be, 
$$ {100 \choose 15,60,25} 2^{15} 3^{60}  $$ 
\subsection*{Problem 31}
(a). We have OVERNUMEROUSNESSES which has, 
$$ O: 2, V:1, E:4, R:2, N:2, U:2, M:1, S:4  $$ 
Total letters are $18$ so number of rearrangements are,  
$$ \frac{18!}{2!2!2!2!4!4!} $$ 

(b). We have OPHTHALMOOTORHINOLARYNGOLOGY which has, 
$$(P= 1), (M= 1), (I= 1), (T= 2), (A= 2), (R= 2), (N= 2), (Y= 2), (G= 2), (H= 3), (L= 3), (O= 7)$$

and total number of letters are 28 so our answer is, 
$$ \frac{28!}{7!3!3!2!2!2!2!2!2!} $$ 


(c). We have HONORIFICABILITUDINITATIBUS which has, 
$$(H= 1), (R= 1), (F= 1), (C= 1), (L= 1), (D= 1), (S= 1), (O= 2)$$ $$ (N= 2), (A= 2), (B= 2), (U= 2), (T= 3), (I= 7)$$

And total letters of, 27 so our answer is, 
$$ \frac{27!}{7!3!2!2!2!2!2!} $$ 


\section*{Section 3.11}
\subsection*{Problem 1}
We have for a given $n$ the number of identifier can be constructed in three different ways, ones that begin with any upper other than D and followed by a valid identifier of $n - 1$ is, 
$$ 25 \times r(n-1)  $$  in number

Ones that begin with $1C,2K$ or  $7J$ and followed by any valid  identifier of $n - 2$ is, 
$$ 3 \times  r(n - 2) $$ 

Ones that begin with $D$ and followed by string  of $n - 1$ is, 
$$ 10^{n-1} $$ 

So together we have, 
$$ r(n) = 25  r(n-1)  + 3   r(n- 2)  + 10^{n- 1}$$ 

We have, 
$$ r(2) = 25(26) + 3(1) + 10^{1} =  663$$ 
$$ r(3) = 25(663)  + 3(26) + 10^{2} = 16753$$  
$$ r(4) = 25(16753)  + 3(663) + 10^{3} = 421814$$  
$$ r(5) = 25(421814)  + 3(16753) + 10^{4} = 10605609$$  


\subsection*{Problem 3}
First we have $g(1) = 3$ and  $g(2) = 9 $ and $g(3) = 26$ as everything except 102 is legal. Now we partition our string of length  $n$ into three different cases,

1. If it ends with a 1. Then we have $g(n - 1)$ valid strings

2. If it ends with a 0. Then we have $g(n - 1)$ valid strings

3. If it ends with a $2$ then we cannot have $10$ preceding it. The number of strings ending with $2$ would be $g(n -1)$ and the number of (n-1) length strings ending with 10 would be $g(n - 3)$. So number would be $g(n - 1) - g(n - 3)$.

So  our solution is, 
$$ 3g(n - 1) - g(n - 3) $$ 

\subsection*{Problem 4}
First we have $t(1) = 1, t(2) = 1 + 4 = 5$ and  $t(3) = 5 + 4 \times 1 + 2$. Now our recursion we have three cases, first we have the nth column complete free and the $n - 1$ is filled completely. This gives us, 
$$ t(n - 1)  \text{ tiles}$$ 

Now consider n - 2 columns are filled and we can fill the remaining 2 columns by placing an L and a square in 4 ways. This gives us, 
$$ 4t(n - 2) $$ 

Now consider $n - 3$ columns are filled and we fill the remaining three columns just by using two L tiles. We can orient it 2 ways. This gives us, 
$$ 2t(n - 3) $$ 

So in total we have, 
$$ t(n - 1) + 4t(n - 2) + 2t(n - 3) \text{ total tiling's} $$ 
\subsection*{Problem 6}
We need to find $d = gcd(5544, 910)$ and integers  $a,b$ such that $5544a + 910b = d$. Using euclidean algorithm we have,  
\begin{align*}
    5544 &= 910 \times 6 + 84\\
    910 &= 84 \times 10 + 70\\
    84 &= 70 \times 1 + 14\\
    70 &= 14 \times 5 + 0\\
\end{align*}

As remainder is 0 in the last our GCD is $14$ so  $d = 14$. Now working backwards we have,  

\begin{align*}
    70 &= 84 - 14\\
    910 &= 84 \times 10 + 84 - 14\\
    910 &= 84 \times 11 - 14\\
    5544 &= 910 \times  6 + (910 + 14) /11\\
    5544 \times 11 &= 910 \times  67+ 14 \\
    14 &= 5544 \times  11 - 910 \times  67
\end{align*}

So we have $a = 11$ and $b = -67$


\subsection*{Problem 8}
We know that if gcd of $m,n$ is  $d$ then there exists  $a',b'$ such that  $ma' + nb' = d$. Now if are given that  $am + bn = 36$ we could have some  $a = ka'$ and  $b = kb'$ such that  $36 = kd$. So we know that,  
$$d \le 36$$ 

Now as $k$ should be an integer as well this means that $$d | 36$$. So the gad will be one of the divisors of 36 (including itself).
\subsection*{Problem 10}
Our base case is when $n = 4$. We see that $2^{4} = 16$ and $4! = 24$. We verify that  $16 < 24$. Now consider  an arbitrary case where  $n = k$  and assume it holds true, so we have, 
$$ 2^{k} < k! $$ 

Take the $n = k + 1$ case we have, 
$$ 2^{k + 1} < (k + 1)! $$ 

Now if we have, 
$$ 2^{k} < k! $$ 
Let us multiply $(k + 1)$ on both sides to get,  
\begin{align*}
    2^{k} (k + 1) &< k! (k + 1) \\
    2^{k} (k + 1) &< (k+ 1)!\\
\end{align*}

As $k > 4$ we also have, 
$$ 2^{k} 2 < 2^{k} (k + 1) $$ 

So we get, 
$$ 2^{k} 2 = 2^{k + 1} < 2^{k} (k + 1) < (k + 1)!$$ 

or, 
$$ 2^{k + 1} < (k + 1)! $$ 

Which is the $k + 1$ case. Hence we show that if its true for an arbitrary  $n > 4$ then it must be true for $n + 1$. Hence by inducting we show its true for all $n \ge 4$
\subsection*{Problem 11}
We have, 
$$ \sum_{n=0}^{n} 2^{i}= 2^{n + 1} - 1 $$ 

First let us assume the base case which is $n = 1$. We have, 

$$ \sum_{n=0}^{1} 2^{i} = 2^{0} + 2^{1} = 1 + 2 = 3 $$
and, 
$$ 2^{1 + 1} - 1= 4 - 1= 3 $$ 

So it is true for the $n = 1$ case.

Now let us assume it is true for an arbitrary $n = k$ case which gives us, 
$$ \sum_{i=0}^{k} 2^{i} = 2^{k + 1} - 1 $$ 

We need to show its true for $k + 1$  case which is, 
$$ \sum_{i=0}^{k + 1} 2^{i} = 2^{k + 2} - 1 $$ 

Our $n = k$ case gives us, 
$$ \sum_{i=0}^{k} 2^{i} = 2^{0} + \dots + 2^{k} = 2^{k + 1} - 1$$

Let us add $2^{k + 1}$ on both sides to get, 
$$  2^{0} + \dots + 2^{k} + 2^{k + 1} = 2^{k + 1} - 1 + 2^{k + 1}$$
$$  2^{0} + \dots + 2^{k} + 2^{k + 1} = 2 \times  2^{k + 1} - 1 $$
$$  2^{0} + \dots + 2^{k} + 2^{k + 1} =  2^{k + 2} - 1 $$

We can rewrite the left hand side to get,
$$  2^{0} + \dots + 2^{k + 1} = \sum_{i = 0}^{k + 1} 2^{i} = 2^{k + 2} - 1 $$ 

Which is the case for $n = k + 1$. Hence we show that if true for $n = k$ then it must be true for $n = k + 1$. So by induction we see that it must be true for all  $n \ge 1$.
$$  $$ 

\subsection*{Problem 12}
\textbf{Method 1.}
First we see that, 
$$ 7 \equiv 1 \mod 3 $$ 

Similarly we have, 
$$ 4 \equiv 1 \mod 3 $$ 

So we known that $7^{n} \equiv 1^{n} \mod 3$ or $7^{n} \equiv 1 \mod 3$. Similarly we have $4^{n}\equiv 1 \mod 3$. This  gives us, 
$$ 7^{n} - 4^{n} \equiv 0 \mod 3 $$  or that, 
$$ 3 | 7^{n} - 4^{n} $$  for any positive $n$.


\textbf{Method 2.}
We need to show that for all positive integers $n$ we have, 
$$ 3 | 7^{n} - 4^{n} $$ 

First for base case we have $n = 1$ to get, 
$$ 3 | 7 - 4 \implies 3 | 3  \text{ which is true}$$ 

Now assume true for arbitrary $n = k$ to get, 
$$ 7^{k} - 4^{k} = 3m \text{ for some $m \in Z$ }$$ 

We need to show that for some $m'$ that, 
$$ 7^{k + 1} - 4^{k + 1} = 3m' $$ 

First we have, 
$$ 7^{k} - 4^{k} = 3m$$

We multiply both sides by $7$ to get, 
\begin{align*}
    7 \times  7^{k} - 7 \times  4^{k} &= 3(7m)\\
     7^{k + 1} - (4 + 3) \times  4^{k} &= 3(7m)\\
     7^{k + 1} - 4  \times  4^{k} - 3 \times 4^{k} &= 3(7m)\\
     7^{k + 1} -4^{k + 1} - 3 \times 4^{k} &= 3(7m)\\
     7^{k + 1} - 4^{k + 1}  &= 3(7m) + 3 \times 4^{k}\\
     7^{k + 1} - 4^{k + 1}  &= 3(7m + 4^{k})\\
\end{align*}

Or in other words $3 | 7^{k + 1} - 4^{k + 1}$  which is the $n = k + 1$ case.

Hence we concluded by inducting it is true for any positive $n$.
\subsection*{Problem 15}
We need to show the following is true for all positive integers, 

$$ 9 | n^{3} + (n + 1)^{3} + (n + 2)^{3} $$ 
Check base case first for $n = 1$ we have, 
$$ 1 + 2^{3} + 3^{3} = 36 $$  and $9 | 36$ so base case is true.

Assume true for $n = k$ case we have, 
$$ 9 | k^{3} + (k + 1)^{3} + (k + 2)^{3} $$ 

We need to show true for $n = k + 1$ case or, 
$$ 9 | (k + 1)^{3} + (k + 2)^{3} + (k + 3)^{3} $$ 

From the $n = k $ case we know that $\exists m \in Z$ such that,
$$  k^{3} + (k + 1)^{3} + (k + 2)^{3} = 9m $$ 

Now let us add $(k + 3)^3$ to both sides to get, 
\begin{align*}
    k^{3} + (k + 1)^{3} + (k + 2)^{3} + (k + 3)^{3} &= 9m  + (k + 3)^{3}\\
    (k + 1)^{3} + (k + 2)^{3} + (k + 3)^{3} &= 9m  + (k + 3)^{3} - k^{3}  \\
\end{align*}

Now consider the last two terms in the right side which is, 
$$ (k + 3)^{3} - k^{3} = k^{3} + 3^{3} + 3k^2 \times 3  + 3 \times 3^2 k - k^{3}  = 3^{3} +9k^2 + 3^{3}k$$ 

We can write this as, 
$$ 9 (3 + k^2 + 3k) = 9m' \text{ where $m' = 3 + 3k + k^2$} $$ 

So this means that, 
$$ (k + 1) ^{3} + (k + 2)^{3} + (k + 3)^{3} &= 9m + 9m' = 9(m + m')$$ 

or that, 
$$ 9 | (k + 1) ^{3} + (k + 2)^{3} + (k + 3)^{3} $$  which is the case for $n = k + 1$

So we show that if $n = k$ is true then $n = k + 1$ must be true. Hence by induction we show that it must be true for all  $n \ge 1$
\subsection*{Problem 17}
We have, 
$$ f(n) = 2f(n - 1) - f(n - 2) + 6 $$ for $n \ge 2$ and  $f(0) = 2, f(1) = 4$. We need to show using mathematical induction that,  
$$ f(n) = 3n^2 - n + 2$$ 

First we verify the base case for $0,1,2$. To get, 
$$ f(0) = 2, f(1) = 3 - 1 + 2= 4 $$ 

Using recursion we have, $f(2) = 8 - 2 + 6 = 12$ and using formula we have,  
$$ 12 - 2 + 2 = 12 $$ 

Hence it is true for $n = 2$. Now let us assume it is true for that case $n - 1$ and $n$. So we have, 
$$f(n) = 3n^2 - n + 2$$
$$f(n - 1) = 3(n - 1)^2 - n + 3$$

We need to show that it is true for the $n + 1$ case or that, 
$$ f(n + 1) = 2f(n) - f(n - 1) + 6 = 3(n + 1)^2 - n + 1$$

Plugging in the solutions of the $f(n),f(n - 1)$ into the recursive algorithm for $n + 1$ we have, 
\begin{align*}
    f(n + 1) = 2f(n) - f(n - 1) &= 2(3n^2 - n + 2) - (3(n - 1)^2 -n + 3) + 6\\
                                &= 6n^2 -2n + 4 - (3(n^2 + 1 - 2n) -n + 3) + 6\\
                                &= 6n^2 -2n + 4 - (3n^2 + 6 - 7n ) + 6\\
                                &= 6n^2 -2n + 4 - 3n^2 - 6 + 7n  + 6\\
                                &= 6n^2 -2n + 4 - 3n^2  + 7n\\
                                &= 3n^2 + 5n + 4 \\
                                &= 3n^2 + 6n - n + 3 + 1 \\
                                &= 3(n^2 + 2n  + 1)- n + 1 \\
                                &= 3(n^2 + 2n  + 1)- n + 1 \\
                                &= 3(n + 1)^2- n + 1 \\
\end{align*}

This is the case for $n + 1$. Hence we show that assuming true for $n - 1$ and $n$ we show that  $n + 1$ is true. Hence by strong inducting we show it is true for any $n \ge 0$.

\subsection*{Problem 19}

We have $x \in \R$ and  $x > -1$. To show for all $n \ge 0$ that,  
$$ (1 + x)^{n} \ge 1 + nx $$ 

We check base case first with $n  = 0$ to get, 
$$ (x)^{0} \ge 1 + 0 $$ 
$$ 1 \ge 1 $$ 
which is true.

Now we assume true for case $n = k$ to get, 
$$ (1 + x)^{k} \ge 1 + kx \text{ for $x > -1$} $$ 

We need to show true for case $ n= k + 1 $ which is, 
$$ (1 + x)^{k + 1} \ge 1 + (k + 1)x \text{ for $x > -1$} $$ 


We have, 
$$ (1 + x)^{k} \ge 1 + kx$$

We know that $x > -1$ which means that $x + 1 > 0$. So multiplying  $1 + x$ on both sides we get, 
$$ (1 + x)^{k + 1} \ge  (1 + x)(1 + kx) = 1 + kx + x + (k + 1)x $$ 

Looking at the right side we see that, 
$$ 1 + kx + x + kx^2 = 1 + (k + 1)x + kx^2 \ge 1 + (k + 1)x $$ 

Putting the above two together we get, 
$$ (1 + k)^{k + 1} \ge 1 + (k + 1)x $$ 

which is the case for $n = k + 1$. Hence we show if true for  $n = k$ then it must be true for $n = k + 1$. Hence by induction we show that it must be true for all $n \ge 0$.









\end{document}
