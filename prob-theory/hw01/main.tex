\documentclass[a4paper]{report}
\usepackage[utf8]{inputenc}
\usepackage[T1]{fontenc}
\usepackage{textcomp}

\usepackage{url}

% \usepackage{hyperref}
% \hypersetup{
%     colorlinks,
%     linkcolor={black},
%     citecolor={black},
%     urlcolor={blue!80!black}
% }

\usepackage{graphicx}
\usepackage{float}
\usepackage[usenames,dvipsnames]{xcolor}

% \usepackage{cmbright}

\usepackage{amsmath, amsfonts, mathtools, amsthm, amssymb}
\usepackage{mathrsfs}
\usepackage{cancel}

\newcommand\N{\ensuremath{\mathbb{N}}}
\newcommand\R{\ensuremath{\mathbb{R}}}
\newcommand\F{\ensuremath{\mathscr{F}}}
\newcommand\Z{\ensuremath{\mathbb{Z}}}
\renewcommand\O{\ensuremath{\emptyset}}
\newcommand\Q{\ensuremath{\mathbb{Q}}}
\newcommand\C{\ensuremath{\mathbb{C}}}
\let\implies\Rightarrow
\let\impliedby\Leftarrow
\let\iff\Leftrightarrow
\let\epsilon\varepsilon

% horizontal rule
\newcommand\hr{
    \noindent\rule[0.5ex]{\linewidth}{0.5pt}
}

\usepackage{tikz}
\usepackage{tikz-cd}

% theorems
\usepackage{thmtools}
\usepackage[framemethod=TikZ]{mdframed}
\mdfsetup{skipabove=1em,skipbelow=0em, innertopmargin=5pt, innerbottommargin=6pt}

\theoremstyle{definition}

\makeatletter

\declaretheoremstyle[headfont=\bfseries\sffamily, bodyfont=\normalfont, mdframed={ nobreak } ]{thmgreenbox}
\declaretheoremstyle[headfont=\bfseries\sffamily, bodyfont=\normalfont, mdframed={ nobreak } ]{thmredbox}
\declaretheoremstyle[headfont=\bfseries\sffamily, bodyfont=\normalfont]{thmbluebox}
\declaretheoremstyle[headfont=\bfseries\sffamily, bodyfont=\normalfont]{thmblueline}
\declaretheoremstyle[headfont=\bfseries\sffamily, bodyfont=\normalfont, numbered=no, mdframed={ rightline=false, topline=false, bottomline=false, }, qed=\qedsymbol ]{thmproofbox}
\declaretheoremstyle[headfont=\bfseries\sffamily, bodyfont=\normalfont, numbered=no, mdframed={ nobreak, rightline=false, topline=false, bottomline=false } ]{thmexplanationbox}


\declaretheorem[numberwithin=chapter, style=thmgreenbox, name=Definition]{definition}
\declaretheorem[sibling=definition, style=thmredbox, name=Corollary]{corollary}
\declaretheorem[sibling=definition, style=thmredbox, name=Proposition]{prop}
\declaretheorem[sibling=definition, style=thmredbox, name=Theorem]{theorem}
\declaretheorem[sibling=definition, style=thmredbox, name=Lemma]{lemma}



\declaretheorem[numbered=no, style=thmexplanationbox, name=Proof]{explanation}
\declaretheorem[numbered=no, style=thmproofbox, name=Proof]{replacementproof}
\declaretheorem[style=thmbluebox,  numbered=no, name=Exercise]{ex}
\declaretheorem[style=thmbluebox,  numbered=no, name=Example]{eg}
\declaretheorem[style=thmblueline, numbered=no, name=Remark]{remark}
\declaretheorem[style=thmblueline, numbered=no, name=Note]{note}

\renewenvironment{proof}[1][\proofname]{\begin{replacementproof}}{\end{replacementproof}}

\AtEndEnvironment{eg}{\null\hfill$\diamond$}%

\newtheorem*{uovt}{UOVT}
\newtheorem*{notation}{Notation}
\newtheorem*{previouslyseen}{As previously seen}
\newtheorem*{problem}{Problem}
\newtheorem*{observe}{Observe}
\newtheorem*{property}{Property}
\newtheorem*{intuition}{Intuition}


\usepackage{etoolbox}
\AtEndEnvironment{vb}{\null\hfill$\diamond$}%
\AtEndEnvironment{intermezzo}{\null\hfill$\diamond$}%




% http://tex.stackexchange.com/questions/22119/how-can-i-change-the-spacing-before-theorems-with-amsthm
% \def\thm@space@setup{%
%   \thm@preskip=\parskip \thm@postskip=0pt
% }

\usepackage{xifthen}

\def\testdateparts#1{\dateparts#1\relax}
\def\dateparts#1 #2 #3 #4 #5\relax{
    \marginpar{\small\textsf{\mbox{#1 #2 #3 #5}}}
}

\def\@lesson{}%
\newcommand{\lesson}[3]{
    \ifthenelse{\isempty{#3}}{%
        \def\@lesson{Lecture #1}%
    }{%
        \def\@lesson{Lecture #1: #3}%
    }%
    \subsection*{\@lesson}
    \testdateparts{#2}
}

% fancy headers
\usepackage{fancyhdr}
\pagestyle{fancy}

% \fancyhead[LE,RO]{Gilles Castel}
\fancyhead[RO,LE]{\@lesson}
\fancyhead[RE,LO]{}
\fancyfoot[LE,RO]{\thepage}
\fancyfoot[C]{\leftmark}
\renewcommand{\headrulewidth}{0pt}

\makeatother

% figure support (https://castel.dev/post/lecture-notes-2)
\usepackage{import}
\usepackage{xifthen}
\pdfminorversion=7
\usepackage{pdfpages}
\usepackage{transparent}
\newcommand{\incfig}[1]{%
    \def\svgwidth{\columnwidth}
    \import{./figures/}{#1.pdf_tex}
}

% %http://tex.stackexchange.com/questions/76273/multiple-pdfs-with-page-group-included-in-a-single-page-warning
\pdfsuppresswarningpagegroup=1

\author{Aamod Varma}
\setlength{\parindent}{0pt}


\title{Probability Theory: HW1}
\author{Aamod Varma}
\begin{document}
\maketitle
\date{}


\section*{Exercise 1.10}
Given $A, B \in \F$ and we need to show that  $A \triangle B \in \F$. Now  if  $x \in A \triangle B$ then we know that  $x \in (A \cup B) \setminus (A \cap B)$. By definition we have  $A \cup B \in \F$ (closure under countable union) and we also have  $A^{c}, B^{c} \in \F \text{(closure under complement)} \implies (A^{c} \cup B^{c}) \in \F \implies (A \cap B)^{c} \implies A \cap B \in \F  $. So now let $C = A \cup B$ and  $D = A \cap B$. It is enough to show that if $C,D \in \F$ then  $C \setminus D \in \F$. We have  $C \setminus D = C \cap D^{c}$. We know $D^{c} \in \F$ and $\F$ is closed under intersection as shown above which means that  $C \cap D^{c} \in F \implies C \setminus D \in \F \implies (A \cup B) \setminus (A \cap B) \in \F \implies A \triangle B \in \F $

\section*{Exercise 1.17}

First given that $\F$ is the powerset of  $\Omega$. 

1. We have $\Q(A) = \sum_{i: \omega_i \in A} p_i$ for  $A \in \F$ and we know that  $p_i \ge 0$ for any  $i$ so sum of non-negative numbers are also non-negative which means that  $\Q(A) \ge 0$ for $A \in \F$

2. We have $\Q(\Omega) = \sum_{i: \omega_i \in \Omega} p_i = p_1 + \dots + p_n = 1$. Similarly we have $\Q(\phi) = \sum_{i: \omega_i \in \phi} p_i = 0$.

3. We need to show that given disjoint events  $A_1, A_2, \dots \in \F$ we have, $\Q\big ( \bigcup_{i = 1}^{\infty} A_i \big) = \sum_{i = 1}^{\infty} \Q(A_i)$.

% Now let's assume we have $k$ disjoint subsets of $\Omega$ we have,
\begin{align*}
    \Q\big ( \bigcup_{i = 1}^{\infty} A_i \big)  &= \Q(A_1 \cup A_2 \dots ) \\ &= \sum_{i: \omega_i \in (A_1 \cup A_2 \dots)} p_i \\
                                            &\text{Now since $A_1,\dots$ are pairwise disjoint we can write,}\\
                                            &=  \sum_{i: \omega_i \in (A_1)} p_i   + \sum_{i: \omega_i \in (A_2)} p_i  + \dots\\
                                            &= \Q(A_1) + \Q(A_2)  + \dots \\
                                            &= \sum_{i = 1}^{\infty} \Q(A_i)
\end{align*}
\section*{Exercise 1.21}
We need to find, 
\begin{align*}
    &P(A \cap B \cap C^{c})  +  P(A \cap B^{c} \cap C) +  P(A^{c} \cap B \cap C) \\&=P((A \cap B)\setminus  C)  + P((A \cap C)\setminus  B)  + P((C \cap B)\setminus  A) \\
    &= P(A \cap C) - P(A \cap B \cap  C)  + P(A \cap B) - P(A \cap B \cap  C)  + P(B \cap C) - P(A \cap B \cap  C) \\
    &= .3 - .1 + .4 - .1 + .2 - .1 = .6
\end{align*}
\section*{Exercise 1.27}
First the ways to distribute $4$ aces among 4 players would be $4!$. Now with the remaining  $48$ cards, the ways to split it among 4 people random is, ${48 \choose 12}{36 \choose 12}{24 \choose 12}{12 \choose 12}$. Similalry the total ways to split $52$ cards among  $4$ people w $13$ each would be ${52 \choose 13}{39 \choose 13}{26 \choose 13}{13 \choose 13}$. So the probability would be, 
\begin{align*}
\frac{{48 \choose 12}{36 \choose 12}{24 \choose 12}4!}{{52 \choose 13}{39 \choose 13}{26 \choose 13}} = 0.1055
\end{align*}


\section*{Exercise 1.30}
\section*{Exercise 1.44}
\section*{Exercise 1.52}

\section*{Problem 9}
\section*{Problem 14}
\section*{Problem 17}


\end{document}
