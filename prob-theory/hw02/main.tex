\documentclass[a4paper]{report}
\usepackage[utf8]{inputenc}
\usepackage[T1]{fontenc}
\usepackage{textcomp}

\usepackage{url}

% \usepackage{hyperref}
% \hypersetup{
%     colorlinks,
%     linkcolor={black},
%     citecolor={black},
%     urlcolor={blue!80!black}
% }

\usepackage{graphicx}
\usepackage{float}
\usepackage[usenames,dvipsnames]{xcolor}

% \usepackage{cmbright}

\usepackage{amsmath, amsfonts, mathtools, amsthm, amssymb}
\usepackage{mathrsfs}
\usepackage{cancel}

\newcommand\N{\ensuremath{\mathbb{N}}}
\newcommand\R{\ensuremath{\mathbb{R}}}
\newcommand\F{\ensuremath{\mathscr{F}}}
\newcommand\Z{\ensuremath{\mathbb{Z}}}
\renewcommand\O{\ensuremath{\emptyset}}
\newcommand\Q{\ensuremath{\mathbb{Q}}}
\newcommand\C{\ensuremath{\mathbb{C}}}
\let\implies\Rightarrow
\let\impliedby\Leftarrow
\let\iff\Leftrightarrow
\let\epsilon\varepsilon

% horizontal rule
\newcommand\hr{
    \noindent\rule[0.5ex]{\linewidth}{0.5pt}
}

\usepackage{tikz}
\usepackage{tikz-cd}

% theorems
\usepackage{thmtools}
\usepackage[framemethod=TikZ]{mdframed}
\mdfsetup{skipabove=1em,skipbelow=0em, innertopmargin=5pt, innerbottommargin=6pt}

\theoremstyle{definition}

\makeatletter

\declaretheoremstyle[headfont=\bfseries\sffamily, bodyfont=\normalfont, mdframed={ nobreak } ]{thmgreenbox}
\declaretheoremstyle[headfont=\bfseries\sffamily, bodyfont=\normalfont, mdframed={ nobreak } ]{thmredbox}
\declaretheoremstyle[headfont=\bfseries\sffamily, bodyfont=\normalfont]{thmbluebox}
\declaretheoremstyle[headfont=\bfseries\sffamily, bodyfont=\normalfont]{thmblueline}
\declaretheoremstyle[headfont=\bfseries\sffamily, bodyfont=\normalfont, numbered=no, mdframed={ rightline=false, topline=false, bottomline=false, }, qed=\qedsymbol ]{thmproofbox}
\declaretheoremstyle[headfont=\bfseries\sffamily, bodyfont=\normalfont, numbered=no, mdframed={ nobreak, rightline=false, topline=false, bottomline=false } ]{thmexplanationbox}


\declaretheorem[numberwithin=chapter, style=thmgreenbox, name=Definition]{definition}
\declaretheorem[sibling=definition, style=thmredbox, name=Corollary]{corollary}
\declaretheorem[sibling=definition, style=thmredbox, name=Proposition]{prop}
\declaretheorem[sibling=definition, style=thmredbox, name=Theorem]{theorem}
\declaretheorem[sibling=definition, style=thmredbox, name=Lemma]{lemma}



\declaretheorem[numbered=no, style=thmexplanationbox, name=Proof]{explanation}
\declaretheorem[numbered=no, style=thmproofbox, name=Proof]{replacementproof}
\declaretheorem[style=thmbluebox,  numbered=no, name=Exercise]{ex}
\declaretheorem[style=thmbluebox,  numbered=no, name=Example]{eg}
\declaretheorem[style=thmblueline, numbered=no, name=Remark]{remark}
\declaretheorem[style=thmblueline, numbered=no, name=Note]{note}

\renewenvironment{proof}[1][\proofname]{\begin{replacementproof}}{\end{replacementproof}}

\AtEndEnvironment{eg}{\null\hfill$\diamond$}%

\newtheorem*{uovt}{UOVT}
\newtheorem*{notation}{Notation}
\newtheorem*{previouslyseen}{As previously seen}
\newtheorem*{problem}{Problem}
\newtheorem*{observe}{Observe}
\newtheorem*{property}{Property}
\newtheorem*{intuition}{Intuition}


\usepackage{etoolbox}
\AtEndEnvironment{vb}{\null\hfill$\diamond$}%
\AtEndEnvironment{intermezzo}{\null\hfill$\diamond$}%




% http://tex.stackexchange.com/questions/22119/how-can-i-change-the-spacing-before-theorems-with-amsthm
% \def\thm@space@setup{%
%   \thm@preskip=\parskip \thm@postskip=0pt
% }

\usepackage{xifthen}

\def\testdateparts#1{\dateparts#1\relax}
\def\dateparts#1 #2 #3 #4 #5\relax{
    \marginpar{\small\textsf{\mbox{#1 #2 #3 #5}}}
}

\def\@lesson{}%
\newcommand{\lesson}[3]{
    \ifthenelse{\isempty{#3}}{%
        \def\@lesson{Lecture #1}%
    }{%
        \def\@lesson{Lecture #1: #3}%
    }%
    \subsection*{\@lesson}
    \testdateparts{#2}
}

% fancy headers
\usepackage{fancyhdr}
\pagestyle{fancy}

% \fancyhead[LE,RO]{Gilles Castel}
\fancyhead[RO,LE]{\@lesson}
\fancyhead[RE,LO]{}
\fancyfoot[LE,RO]{\thepage}
\fancyfoot[C]{\leftmark}
\renewcommand{\headrulewidth}{0pt}

\makeatother

% figure support (https://castel.dev/post/lecture-notes-2)
\usepackage{import}
\usepackage{xifthen}
\pdfminorversion=7
\usepackage{pdfpages}
\usepackage{transparent}
\newcommand{\incfig}[1]{%
    \def\svgwidth{\columnwidth}
    \import{./figures/}{#1.pdf_tex}
}

% %http://tex.stackexchange.com/questions/76273/multiple-pdfs-with-page-group-included-in-a-single-page-warning
\pdfsuppresswarningpagegroup=1

\author{Aamod Varma}
\setlength{\parindent}{0pt}


\title{Probability Theory: Hw1}
\author{Aamod Varma}
\begin{document}
\maketitle
\date{}

\section*{Problem 2.10}
We need to show that the indicator function $1_E$ is a discrete random variable. First we need to show that $1_E(\Omega)$ is countable. 

\vspace{1em}

We have, 
$$ 1_E(\omega) = \begin{cases} 1 \quad \text{if } \omega \in E \\ 0 \quad \text{if } \omega \not \in E \end{cases} $$ 

So for any $\omega \in \Omega$ we have either  $1_E(\omega) = 1$ or $1_E(\omega) = 0$ hence we have  $X(\Omega) = \{0, 1\}$  which is countable. 

\vspace{1em}

Now we need to show that $\forall a \in \R$ we have, $\{ \omega: X(\omega) = a\}  \in \F$. We see for $a = 1$  we have $\{\omega : X(\omega) = 1\} = E$ and we assume that $E \in \F$. Similarly we have  for $a = 0$ that $\{\omega : X(\omega) = 0\} = \{\omega: \omega \not \in E\} = E^{c}$. Using the properties of $\F$ we know that  $E^{c} \in \F$. Lastly if $a \ne 1, 0$ we have $\{\omega: X(\omega) \ne 1,0\} = \phi $ and we know that $\phi \in E$.

\vspace{1em}

Hence, we show that by definition the indicator function is a discrete random variable.





\section*{Problem 2.11}
\begin{enumerate}
	\item $U(\omega) = \omega$
	
	First we check if $U(\Omega)$ is countable. As  $U(\omega) = \omega$ we have, 
	$$ U(\Omega) = U(\{1, \dots, 6\} )  = \{U(\omega): \omega \in \Omega\} =  \{1, 2, 3, 4, 5, 6\} $$ 
	which is countable. Now we check if for any $a \in \R$ it's preimage is in the family of events.

	 \vspace{1em}
	
	 Take $a = 1$ we have  $\{\omega: X(\omega) = 1\} = \{1\} \subset \Omega $. However, we see that $\{1\} \not \in \F$ which means that it fails the condition and hence $U$ is not a discrete random variable.

	\item $V(\omega) = \begin{cases}1 \quad \text{if $\omega$ is even}\\0 \quad \text{if $\omega$ is odd}  \end{cases}$

	We see that $V$ maps all $\omega \in \Omega$ to either  $0, 1$. Hence, we have,  
	$$ V(\Omega) = V(\{1, \dots, 6\} ) = \{V(\omega) : \omega \in \Omega\}  $$ 

	Now, as $\omega \in \Omega$ can either be even or odd we have,  $\{V(\omega) : \omega \in \Omega\} = \{0, 1\}$ which is countable.

	\vspace{1em}
	
	Now, consider any $a \in \R$ we need to see if it's preimage  is in the family of events. For $a = 1$ we have  $\{\omega \in \Omega: X(\omega) = 1\} = \{\omega: \omega \text{ is even} \} = \{2,4,6\}$ and we see that $\{2,4,6\} \in \F $. Similarly for $a = 0$ we have  $\{\omega \in \Omega: X(\omega) = 0\} = \{\omega: \omega \text{ is odd} \} = \{1,3,5\}$ and we see that $\{1,3,5\} \in \F$. And lastly for $a \ne 1,0$ we have $\{\omega: X(\omega) \ne 1, 0\} = \phi \in \F$. Hence, $V$ satisfies both conditions making it a discrete random variable.
	
	\item $U(\omega) = \omega^2$
		First we check if $W(\Omega)$ is countable. We have,  
		$$ W(\Omega) = \{W(\omega): \omega \in \Omega\} = \{1^2, 2^2, \dots, 6^2\} = \{1, 4, 9, 16, 25, 36\}    $$ 

		which is obviously countable.

		\vspace{1em}
		
		Now, consider any $a \in \R$ and we check the preimage of  $a$. Take $a = 1$ we have  $\{\omega: X(\omega) = 1\} = \{1\} $. But we see that $\{1\} \not \in \F $. Hence $W$ is not a discrete random variable.

	
\end{enumerate}
\section*{Problem 2.24}

We have $X$ a discrete random variable having geometric distribution. Which means that, 
$$ \P(X = k) = p(1 - p)^{k - 1} $$ 

We need to find $\P(X > k)$ or  $\P(X = k + 1) + \P(X = k + 2) + \dots$ which is  $\sum_{n=1}^{\infty} \P(X = k + n)$ as $X $ is geometric we have, 
\begin{align*}
	\sum_{n=1}^{\infty} \P(X = k + n) &= 	\sum_{n=1}^{\infty} p (1 - p)^{k + n - 1}\\
					  &= p \sum_{n=1}^{\infty} (1- p)^{k + n - 1}\\
					  &= p ((1 - p)^{k} + (1 - p)^{k + 1}  + \dots)\\
\end{align*}

Now using sum of geometric series we have, 
\begin{align*}
	\P(X > k)&= p ((1 - p)^{k} + (1 - p)^{k + 1}  + \dots)\\
		 &= p\frac{(1 - p)^{k}}{p} \\
		 &= (1 - p)^{k}
\end{align*}

\section*{Problem 4} 
We need value of $c$ and  $\alpha$ such that,  
$$ p(k) = \begin{cases} ck^{\alpha} \quad \text{ for $k = 1,2, \dots$} \\ 0 \quad \text{ otherwise} \end{cases} $$ 

is a mass function.

\vspace{1em}

A mass function is defined as $p(x) = \P(X = x)$ if  $X$ is a discrete random variable. So we have  $\P(X = k) = ck^{\alpha}$ if $k = 1,2,3,\dots$ else  $\P(X = k) = 0$. So we need  $\sum_{k=1}^{\infty} \P(X = n) = ck^{\alpha} = 1$. So we have, 
\begin{align*}
	\sum_{k=1}^{\infty} ck^{\alpha}  = 1\\
	c \sum_{k=1}^{\infty} k^{\alpha} = 1\\
\end{align*}

Now the summation only converges if $\alpha < -1$. Assume it converges to  $m$ then we can define  $c = \frac{1}{m}$. 


\section*{Problem 5}

We need to show that $\P(X > m + n \mid X > m) = \P(X > n)$. In geometric distribution we know that  $\P(X = k) = p(1 - p)^{k - 1}$ so we have $\P(X > n) = (1 - p)^{n}$. Similarly we get, 
\begin{align*}
	\P(X > m + n \mid X > m) = \frac{\P((X > m) \cap (X > m + n))}{\P(X > m)}
\end{align*}

Now if $X > m$ and  $X > m + n$ as the first is included in the second it is equivalent to  $X > m + n$ so we have,  
\begin{align*}
	\P(X > m + n \mid X > m) &= \frac{\P((X > m) \cap (X > m + n))}{\P(X > m)}\\
				 &=  \frac{\P(X > m + n)}{\P(X > m)}\\
				 &= \frac{(1 - p)^{m + n}}{(1 - p)^{m}}\\
				 &= (1 - p)^{m + n - m} = (1 - p)^{n}
				 &= \P(X > n)
\end{align*}


For the lack of geometric property we need $\P(X > m + n) = \P(X > n) \P(X > m)$

\section*{Problem 7}
We have $c$ types of coupons with each coupon equally likely as the others. We need to find probability of first $n$ coupons do not form a complete set.

\vspace{1em}

Let us begin by defining a discrete random variable $X : \Omega \to \R$ defined as the number of coupons collected before getting a complete set. So we have $\P(X = n)$ is the probability that we get a complete set within the first $n$ coupons. So we need to find  $\P(X > n)$ as that  is the probability that we get  a complete set only if we take more than $n$ coupons. So we have, 
 \begin{align*}
	 \P(X < c) &= 0\\
	 \P(X = c) &= c!\\
	 \P(X = c + 1) &= {c + 1 \choose c} c! c\\
	 \P(X = n) &= {n \choose c} c! c^{n - c}
\end{align*}

So our solution is, 
\begin{align*}
	\P(X > n) &= \sum_{k = 1}^{\infty}{n + k \choose c} c! c^{n + k - c}\\
	\P(X > n) &= c! c^{n - c}\sum_{k = 1}^{\infty}{n + k \choose c} c^{k}\\
\end{align*}



\end{document}
