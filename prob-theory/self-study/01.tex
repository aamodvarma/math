\chapter{Introduction}
\section{Sample Spaces and Sigma-Algebras}

\begin{definition}[Sample Space]
    A sample space $\Omega$ is any set and its elements are called outcomes.
\end{definition}

\begin{eg}
    Flip a coin twice, sample space is, 
    $$ \{HH, HT, TH, TT\} $$ 
\end{eg}

Events are subsets of sample space such that, 

\qquad 1. The whole space $\Omega$ should be an event (The event that something happened).

\qquad 2. If an event $A \in \Omega$ then $A^c \in \Omega$

\qquad 3.  If $A,B \in \Omega$ then $A \cup B \in \Omega$


\begin{definition}[Algebra]
    An algebra is a collection $\Sigma$ of subsets of $\Omega$ satisfying the following, 

    1. $\Omega \in \Sigma$

    2.  If  $A \in \Sigma$ then $A^c \in \Sigma$ 

    3. If $A,B \in \Sigma$ then $A \cup B \in \Sigma$
\end{definition}

\begin{definition}[Sigma-algebra]
    A sigma-algebra is an algebra such that if whenever $A_1,A_2,\dots \in \Sigma$ we also have $\bigcup_{n = 0}^{\infty} A_n \in \Sigma$ we call $\Sigma$ a sigma-algebra.
\end{definition}
\begin{remark}
    The key differnet is a sigma-algebra allows for a countably infinite union and intersection of elements while a ordinary algebra allows for a finite intersection and union.
\end{remark}
\begin{remark}
    
\end{remark}

Some consequences are, 
    \item 1. $\phi \in \Sigma$
    \item 2. If $A,B \in \Sigma$ then $A\cap B \in \Sigma$ 
        \begin{proof}
            $A\cap B= (A^c \cup B^c)^c$
        \end{proof}
    \item 3. If $\Sigma$ is a sigma-algebra then $A_1,\dots \in \Sigma$ means that $\bigcap_{n=1}^{\infty} \in \Sigma$
         \begin{proof}
            $\bigcap_n A_n=  (\bigcup_n A_n^c)^c$
        \end{proof}


\begin{eg}
\item 1. If $\Omega$ is any set, then $\{\phi, \Omega\}$ is a sigma-algebra (the trivial sigma-algebra)

\item 2. If $\Omega$ is any set, then the power set $P(\Omega)$ is a sigma-algebra.

\item 3. Let  $\Omega = (0,1]$ and define $\Sigma$ as finite disjoint unions of half-open intervals.
\end{eg}



Consider $\Sigma_0 = \{(a_1,b_1] \cup \dots \cup (a_n,b_n]: n\in \N, 0 \le a_i \le b_i \le 1, \forall i, (a_i,b_i]\cap (a_j,b_j] = \phi, \forall i \ne j\}$
\begin{prop}
    $\Sigma_0$ is an algebra but not a sigma-algebra.
\end{prop}
\begin{proof}
    $\Omega = (0,1] \in \Sigma_0$. 

    If  $A \in \Sigma_0$, wirte it as $A = (a_1,b_1] \cup \dots \cup (a_n,b_n], a_1 \le b_1 \le a_2 \le b_2 \dots \le a_n \le b_n$

    Then $A^c = (0,a_1] \cup (b_1,a_2] \cup \dots \cup (b_n,1] \in \Sigma_0$

    Now if $A, A^c \in \Sigma_0$ we have, 
    \begin{align*}
        A = (a_1,b_1] \cup \dots \cup (a_n,b_n]\\
        A^c = (a_1',b_1'] \cup \dots \cup (a_m', b_m']
    \end{align*}
    
\end{proof}


\begin{definition}
    If $A$ is a collection of subsets of $\Sigma$ then the sigma-algebra generated by $A$ written as $\sigma(A)$, is the intersection of all sigma-algebras that contain A.
\end{definition}
\begin{proof}
    If $e$ is a collection of sigma-algebras of $\Omega$ then $\bigcup_{\Sigma \in e} \Sigma$ is a sigma-algebra
\end{proof}


\begin{eg}
\item 1. The sigma-alg generated by $\{\phi\}$ is $\{\phi, \Omega\}$

\item 2. The sigma-algebra generated by open subsets of  $\R^d$ is called the Bore sigma-algebra

\item 3. If $A \subset B$ then $\sigma(A) \subset \sigma(B)$

\item 4. If $\Sigma$ is a sigma-algebra then $\sigma(\Sigma) = \Sigma$

\item 5. $\sigma(A)$ is the "smallest sigma-algebra containing A". If $\Sigma$ is some sigma-algebra s.t. $A \subset \Sigma$ then $\sigma(A) \subset \Sigma$
\end{eg}


