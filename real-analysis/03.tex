\begin{proof}
    It suffices to find $m,n \in Z$ such that, 
    \begin{align*}
    a < \frac{m}{n} < b
    \end{align*}


    Step 1: find $n$.

    Note that   $ b - a > 0$  and  $b -a \in R$. By (2) we have  $n$ s.t.  $b - a > \frac{1}{n}$. Now fix such an $n$. 

    Step 2. find  $m$ for the fixed $n$.

    % We take  $m$ to be the smallest integer $m > na$

    Without loss of generality take  $na > 0$ and by (1), $m_0 \in \N$ s.t. $M_0 > na$. Then consider a finite set $\{0, 1, \dots, M_0\}$. Now take  $k$ in this set and compare with $na$. Take  $m$ to be the smallest one such that $m > na$.

    So we have  $ m > na \ge m - 1$

    Step 3: Check if  $m, n$ work, 

    We have, 
    \begin{align*}
        m > na \ge m  -1 \\
        \frac{m}{n} > a \text{ and } \frac{m}{n} \le a + \frac{1}{n}\\
    \end{align*}


    But we have $b - a > \frac{1}{n}$ so $b > a + \frac{1}{n}$ which gives us, 
    $$ a < \frac{m}{n} \le a + \frac{1}{n} < b $$ 


\end{proof}


\begin{theorem}
    $\exists s \in R$ such that $s^2  = 2$
\end{theorem}
\begin{proof}
    Let $A = \{x > 0, x \in \R, s.t. x^2 < 2\}$. Clearly $A \subseteq \R$ and is nonempty. We have $A$ is bounded above as $2$ is an upper bound.
 
    By AoC $\sup A \in \R$  exists and set  $s = \sup A$. Claim  $s^2 = 2$.  

    Now we will prove this by contradiction by showing it cannot be the case that $s^2 < 2 $ or $s^2 > 2$. 

    Now assume that $s^2 < 2$ and let $0 < \delta = 2 - s^2$. We will show that there is some $\epsilon > 0$ such that  $(s + \epsilon)^2 < 2$ (i.e. $s$ cannot be a supremum)
    \begin{swork}[10cm]
        Scratchwork

        We want to find $\epsilon$ to satisfy the  $(s + \epsilon)^2 < 2$, for this we work backwards.

        \begin{align*}
            (s + \epsilon)^2 &< 2\\
            s^2 + \epsilon^2 + 2s\epsilon &< 2
        \end{align*}

        We have $s < s^2 < 2$ (as $s$ is definitely greater than $1$). So $2s < 4$ to get,  
        \begin{align*}
            s^2 + \epsilon^2 + 2s\epsilon <  s^2 + \epsilon^2 + 4\epsilon  &< 2\\
        \end{align*}

        Now let's assume that $\epsilon < 1$ as if  $\epislon \ge 1$ works then trivially  $\epsilon < 1$ works as well. So we have  $\epsilon^2 < \epsilon$ so, 
        \begin{align*}
            s^2 + 5\epsilon &< 2    \\
                            5\epsilon &< 2 - s^2\\
                            \epsilon &< \frac{\delta}{5}
        \end{align*}

        Now we can take $\epsilon = \min \{\frac{\delta}{10}, 1\}$.
    \end{swork}


    If we take $\epsilon = \min \{ 1, \frac{\delta}{10}\}$  then we have, 
    \begin{align*}
        (s + \epsilon)^2 &= s^2 + \epsilon^2 + 2s\epsilon\\
        (s + \epsilon)^2  &\le s^2 +   \epsilon^2 + 2s\epsilon  \le s^2 + \delta < 2
    \end{align*}


\end{proof}

\begin{ex}
   Show $s^2 > 2$  is impossible. We similarly show that we can find an $\epsilon$ such that  $(s - \epsilon)^2 > 2$
\end{ex}




\section{Cardinality} 

\begin{definition}
    % [Countability]
    We say that two  sets  $A$ and $B$ have the same cardinality if there is a bijective function  $f: A \to B$  . We write $A \sim B$
\end{definition}
\begin{remark}
    Types,
    \begin{enumerate}
        \item  We say $A$ is finite if $A \sim \{1, 2, \dots, n\} $ for some $n \in N$
        \item We say $A$ is countable  (countably infinite) then $A \sim N$
        \item An infinite set that is not countable is called unountable.
    \end{enumerate}
\end{remark}

\begin{eg}
$E = \{2, 4, \dots\}$, we show $E \sim N$.

Take  $f : N  \to E$ defined as  $f(n) =2n $.
\end{eg}
\begin{eg}
 $N \sim Z$

    Take $f: N \rightarrow Z$ s.t.  $f(n) = \frac{n - 1}{2} $ if $n$ is odd else $-\frac{n}{2}$.
\end{eg}
\begin{eg}
    $(-1, 1) \sim \R$  

    Take $f(x) = \frac{x}{x^2 - 1}$
\end{eg}

\begin{theorem}
    Following are true,
    \begin{enumerate}
        \item $\Q$ is countable
        \item $\R$ is uncountable
    \end{enumerate}
\end{theorem}

\begin{proof}
    For $\Q$ define  $A_1 = \{0\}$ and $$A_n = \{\pm \frac{p}{q}: p + q = n, p, q \in \N \text{ and } p,q \text{ coprime}\}$$


    Note that $A_n$ is finite and $\forall x \in \Q$ we can find a unique  $n \in \N$ s.t. $x \in A_n$

    Now map elements of $A_0, A_1, \dots$ iterative with $1, 2, 3, \dots$. So by construction, any element from $A_n$ will be listed. So there is a bijection between $\Q$ and  $\N$ since  $\Q = \bigcup_{n - 1}^{\infty} A_n$ and $A_n$'s are disjoint, so  $\Q \sim \N$


    Now we show the reals are not countable. Assume that $\R$ is countable and suppose there is a bijective funciton $f: \N \to \R$. Let  $x_1 = f(1), x_2= f(2), \dots$

    Then $\R = \{x_1, x_2, \dots,\} $. Let $I$ be a closed interval  $I_1 \subseteq \R$ s.t. $x_1 \not \in I_1$ and similarly find $I_2 \subseteq I_1$ such that $x_2 \not \in I_2$. Similarly define for all $n$ such that  $I_{n + 1} \subseteq I_n$ closed interval such that  $x_n \not \in I_{n + 1}$. 

    Since $\forall n_0 \in N$ we have $x_{n_0} \not \in I_{n_0}$ so $xn_0 \not \in \bigcap_{n = 1}^{\infty} I_n$. But $R = \{x_1,x_2,  \dots\}$ so we have $\bigcap_{n = 1}^{\infty} I_n = \phi$. However, this is a contradiction with the nested interval property.
\end{proof}


\begin{theorem}
    Following are true,
    \begin{enumerate}
        \item $A \subseteq B$ if $B$ is countable them $A$ is either countable or finite. 
        \item If $A_n$ is countable $\forall n \in \N$ then $\bigcup_{n = 1}^{\infty} A_n$ is countable.
    \end{enumerate}
\end{theorem}
