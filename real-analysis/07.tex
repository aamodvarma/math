\section{Cauchy Criterion}
\begin{definition}[Cauchy Criterion]
	A sequence $(a_n)$ is Cauchy if $\forall \epsilon > 0, \exists N$ s.t. $|a_n - a_m| < \epsilon, \forall n, m > N$.
\end{definition}

\begin{theorem}
    A sequence $(a_n)$ is convergent if and only if it is Cauchy.
\end{theorem}
\begin{proof}
	($\Rightarrow$) Let $(a_n)$ be a convergent sequence that converges to $a$. Then $\forall \epsilon > 0, \exists N$ s.t. $|a_n - a| < \frac{\epsilon}{2}, \forall n > N$. Now for $n, m > N$ we have,
	\begin{align*}
		|a_n - a_m| &= |(a_n - a) - (a_m - a)|\\
			    &\le |a_n - a| + |a_m - a| \le \frac{\epsilon}{2} + \frac{\epsilon}{2} 
	\end{align*}
	Which completes our proof.
\end{proof}

\begin{lemma}
Every Cauchy sequence is bounded
\end{lemma} 
\begin{proof}
	Take $\epsilon = 1$, then we have $N$ such that $|x_m - x_n| < 1$ for all $m, n > N$ so we can write, 
	\[ 
		|x_m - x_{N + 1}| < 1, \quad \forall m > N
	.\]
	So $|x_m| = |x_m - x_{N + 1} + X_{N + 1}| < 1 + |X_{N + 1}|$. Now just pick $\max\{|x_1|, \dots, |x_{N - 1}|, |x_N| + 1\}$.
\end{proof}

Now we can prove the other direction of the theorem.

\begin{proof}
	($\Leftarrow$) Let $(a_n)$ be a Cauchy sequence. By the lemma above, we know that $(a_n)$ is bounded. So by the Bolzano-Weierstrass theorem, tells us that it contains a convergent subsequence. So consider a subsequence $\{x_n_k\} $ s.t. $\lim_{k \to \infty} x_n_k = x$.

	\vspace{1em}
	
	Now, $\forall \epsilon > 0,$ we can find $N_1$ such that $\forall k > N_1$, $|x_n_k - x| < \frac{\epsilon}{2}$. And we have $N_2$ s.t. $\forall m,n > N_2$ we have $|x_n - x_m| < \frac{\epsilon}{2}$. Now take $N = \max \{N_1, N_2\}$. We have, 
	$$ |x_n - x| = |x_n - x_n_k + x_n_k - x| \le |x_n - x_n_k| + |x_n_k - x| $$ 

	If we take $k > N$, note that  $n_k \ge k > N$, so  $|x_n_k - x| < \frac{\epsilon}{2}$ and $|x_n - x_n_k| < \frac{\epsilon}{2}$. It follows that $|x_n - x| \le \epsilon$

	\vspace{1em}

	So $$\lim_{n \to \infty} x_n = x$$

\end{proof}

\vspace{1em}

\hline
\vspace{1em}


\begin{eg}
	$M.C.T. \implies N.I.P$

	Consider your nested intervals and have  $I_n = [a_n, b_n]$ then  $\{a_n\} $ is an increasing sequence and it is bounded above $(b_1)$ is an upper bound. So using M.C.T it converges to some $x$.
\end{eg}

\begin{eg}
	NIP $\implies$ AoC


\end{eg}
