\chapter{Introduction}
\section{Logic and proofs}
Types of proofs,
\begin{enumerate}
    \item Direct proof
    \item Argument by contradiction
    \item Induction
    \item Contrapositive  (we show$\neg B$ \implies $\neg A$)
\end{enumerate}

\begin{theorem}
    $a = b \iff \forall \epsilon > 0, |a - b | < \epsilon$ 
\end{theorem}
\begin{proof}
    1. To show, $a = b \implies \forall \epsilon > 0, |a - b | < \epsilon$.

    Suppose $a = b$ so  $|a - b |= 0$. We have  $\forall \epsilon > 0$ so
    $|a - b | = 0 < \epsilon$
    
    2. To show, $\forall \epsilon > 0, |a - b | < \epsilon \implies a = b$

    Now assume this is not true, or that $a \ne b$ so  $a - b \ne 0$ this means that there is a non-zero number  $k$ such that  $|a - b| = \epsilon_0$. Now take $\epsilon = \frac{\epsilon_0}{2}$. This gives us, 
    $ |a - b | = \epsilon_0 > \epsilon $  which contradicts the statement. Hence our assumption is false and we prove the results.
\end{proof}



\begin{eg}[Induction]
    $x_1 = 1$ and $x_{n + 1} = \frac{1}{2}x_n + 1, \forall n \in \Z$. Show $x_{n + 1} \ge x_n \forall n \in \N$ 


    Define $S = \{n \in \N, s.t. x_{n + 1} \ge x_n\}$ clearly, $S \subseteq N$.


    $x_1 = 1$ and  $x_2 = \frac{x_1}{2} + 1 = 1.5$. This gives us $x_2 > x_1$ so $1 \in S$

    Suppose $n \in S$  and $x_{n + 1} \ge x_n$. Note that,
    \begin{align*}
        x_{n + 2} &= \frac{1}{2}x_{n + 1} + 1\\
        x_{n + 1} &= \frac{1}{2}x_{n} + 1\\
    \end{align*}

    Then $x_{n + 2} = \frac{1}{2}x_{n + 1} + 1 \ge \frac{1}{2} x_n + 1 = x_{n + 1}$ or $x_{n + 2} \ge x_{n + 1}$ which means  $n + 1 \in S$. So by induction we have  $S = N$ and  $x_{n + 1} \ge x_n, \forall n \in \N$
\end{eg}


\pagebreak
\section{Real Numbers}
Number systems,
\begin{enumerate}
    \item Natural numbers $\N$

        $1,2,3, \dots$

        Can't do subtraction
    \item Integers  $\Z$

         $\dots, -3, -2, -1, 0, 1, 2, 3 \dots$ 


         Can't do division

     \item Rationals $\R$

         \{$\frac{p}{q}$ where $p,q \in \Z$ but  $q \ne 0$\}

    Now we have $\N \subseteq \Z \subseteq \R$

    But other numbers are still not captured, 


\begin{eg}
    $\sqrt{2}$ is not defined in  $\R$.
    However if we  define $x_1 = 2$, $x_{n + 1} = \frac{1}{2}(x_n + \frac{2}{x_n})$. We know $x_{n + 1} \in \R, \forall n \in \N$ (we can then show that $x_n \to \sqrt{2}$).
\end{eg}


\begin{theorem}
    $\sqrt{2}$ is not rational
\end{theorem}
\begin{proof}
    Argue by contradiction
\end{proof}

\item Real numbers $\R$

We will define $\R$ as $\Q$ with the gaps filled in.
\end{enumerate}


\begin{definition}[Axiom of completeness]
    Every non-empty subset of $\R$ that is bounded above has a least upper bound called the supremum.
\end{definition}
Let $S \subseteq \R$ and  $S$ is bounded above. If there is  $u \in \R$ such that $s \le u, \forall s \in S$ then  $S$ is bounded above by $u$ (\textit{Similar for bounded below})

\begin{definition}[Least upper bound or supremum]
    We say $u \in \R$ is the least upper bound for  $S$ if, 
    
    \begin{enumerate}
        \item If $u$ is an upper bound for  $S$
        \item  $u \le v$ for any other upperbound  $v$ of  $S$.
    \end{enumerate}
\end{definition}
\textit{Similar for greatest lower bound or infimum}
