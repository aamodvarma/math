\chapter{Diophantine Equations}
\begin{definition}[Diophantine equation]
    Any equation with one or more variables to be solved in the integers is called a \emph{Diophantine equation}
\end{definition}
\begin{eg}
    
    \[
        5x^2 -2x +1 = 0 \text{  for $x \in \Z$}
    \]
\end{eg}

\section{Linear Diophantine Equation}

\begin{definition}[Linear Diophantine Equation]
    Let $a_{1}, \dots, a_n \in \Z$ and $ b \in \Z$ with $a_i \ne 0$. A Diophantine equation of the fomr, 
    \[
        a_{1}x_{1} + \dots + a_n x_n = b
    \]
    is called a \emph{Linear Diophantine Equation}
\end{definition}

\begin{theorem}
    Let $ax = b$ be a linear D.E in the variable $x$. If $a \mid b$ then  there is a unique solution $x = \frac{b}{a}$ else there is no solution.
\end{theorem}

\begin{theorem}
    Let $ax + by = c$ be a linear D.E in variables $x, y$. Let $d = (a, b)$. If $d \nmid c$ there are no solutions. Else there are infinitely many solutions. Furthermore, if $x_{0},y_{0} \in \Z$ is a particular solution, then all solutions $x,y$ are given by, 
    \begin{align*}
        x &= x_{0} + (b / d)n\\
        y &= y_{0} - (a / d)n
    \end{align*}
\end{theorem}

\begin{proof}
    Since $d \mid a$ and $d \mid b$ then $d \mid c$ by Prop 1.2, thus if $d \nmid c$ then we have no solution. By Prop 1.11 there exists $r, s \in \Z$ such that, 
    \[
        d = (a,b) = ra + sb
    \]
    Further, if $d \mid c$, then $c = dq$. So we may write, 
    \[
        c = (ra + sb)q = a(rq) + b(sq)
    \]

    And thus $x = rq, y = sq$ is a particular solution.

    \vspace{1em}
    
    Let $x_{0}, y_{0}$ be any solution and let $x, y$ be given as shown above. Then, 
    \begin{align*}
        ax + by &= a(x_{0} + (b / d) n) + b (y_{0} - (a / d)n)\\
                &= ax_{0} + by_{0} = c
    \end{align*}

    So $x,y$ is an integer solution for any integer $n$.

    \vspace{1em}
    
    Now we show that every solution is of this form. Let $x, y$ a solution. Note, 
    \begin{align*}
        (ax + by) - (ax_{0} + by_{0}) &= 0\\
        a(x - x_{0}) &= b(y_{0} - y) 
    \end{align*}

    Divide both by $d$ and we have, 
    \begin{align*}
        (a / d) (x - x_{0}) = (b / d)(y_{0} - y)
    \end{align*}
    Thus we have $(a / d) \mid (b / d) (y_{0} - y)$ and we get $(a / d) \mid (y_{0} - y)$ and we get, 
    \[
        y = y_{0} - (a / d)n
    \]

    and similarly we get, 
    \[
        x = x_{0} + (b / d)n
    \]
\end{proof}
\begin{eg}
    Determine if $803x + 154y = 22$ has any solutions and calculate all of them.
\end{eg}
\begin{sol}
    We have $(803, 154) = 11$ so $803 \cdot 5 - 26 \cdot 154 = 11$.
\end{sol}

\section{Nonlinear Diophantine Equations}

We can show that some equations are not solvable using the following method,

\vspace{1em}
If a D.E has solutions, then the equation when viewed as a congruence modulo any modulus, will also have solutions. The contrapositive of this is that if a particular congruence is not solvable then neither is the original.

\begin{eg}
    Show that $3x^2 + 2 = y^2$ is not solvable. We have,
    \[
    3x^2 - y^2 = -2
    \]
    Assume it's solvable, then it's solvable modulo $3$ so we have, 
    \[
    y^2 \equiv 2 \pmod 3
    \]
    But this says that $2$ is a residue modulo $3$ which is untrue. So there are no solutions.
\end{eg}

\begin{eg}
    Show that $7x^{3} + 2 = y^{3}$ has no solutions. 

    \vspace{1em}
    
    Consider equation modulo $7$ we have, 
    \[
        y^{3} \equiv 2 \pmod 7
    \]

\end{eg}
\begin{eg}
    Prove that $x^2 + y^2 + 1 = 4z$ has no solutions.


    \vspace{1em}

    Modulo $4$ we have $x^2 + y^2 \equiv 3 \pmod 4$. Do odd even thingy
\end{eg}

\begin{eg}
   Show  x^2 + 2y^2 \equiv 5 \pmod 8
   4m^2 + 4m + 3
\end{eg}


