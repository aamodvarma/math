\documentclass[12pt]{exam}
\usepackage{amsmath,amstext,amssymb,amsthm}   
\usepackage{enumerate}
%%%%%%%%%%%%%%%%%%%%%%%%%%%%%%
%%%%%%%%%%%%%%%%%%%%%%%%%%%%%%
%%%%%%%%%%%%%%%%%%%%%%%%%%%%%%
 


\begin{document}


 \begin{center}
 {\Large Homework 6, Math 4150 }
 \\
 \end{center}


\begin{questions}
\question Exercise Set 5.1, \#9.
Let $p$ be an odd prime number and let $r$ be an integer with $p \nmid r$. Prove that $r$ is a primitive 
root modulo $p$ if and only if $r^{(p-1)/q} \not \equiv 1 \pmod{p}$ for all prime divisors $q$ of $p-1$.

\textbf{Solution.}

($\Rightarrow$) Given that $r$ is a primitive root modulo $p$ we know the order of $r$ is $\phi(p) = p - 1$ as $p$ is prime. Now we know that for any $k$ for which we  have $r^{k} \equiv 1 \pmod p$ we have $ord_p r \mid k$. Now assume that we have for some prime divisor of $p - 1$, q (so note that $q \ne 1$) for which we get, 
\[
    r^{(p - 1) / q} \equiv 1 \pmod p
\]

Now this means that we have $p - 1 \mid (p - 1) / q$. But as $q \ne 1$ meaning $q \ge 2$ we have $\frac{p - 1}{q} < p - 1$ and this means that $p -1 \mid (p - 1) / q$ is not possible as a larger number cannot divide a smaller number. Hence, we cannot find any $q$ for which we have $r^{(p - 1) / q} \equiv 1 \pmod p$ which means that for all prime divisors $q$ of $p - 1$ we have $r^{(p - 1) / q} \not \equiv 1 \pmod p$


($\Leftarrow$)
Now assume for all prime divisors $q$ of $p - 1$ we have $r^{(p - 1) / q} \not \equiv 1 \pmod p$. Now using eulers theorem we know $r^{(p - 1)} \equiv 1 \pmod p$. Now we need to show that $p - 1$ is the smallest positive integer for which we have equivalency with $1$. Assume there exists a smaller number say $k$ so we have $ord_p r = k$ and $k \mid p - 1$ or $km = p - 1$. Now choose $q$ for which it's a prime divisor of $m$ then we have,

\begin{align*}
    r^{(p - 1) / q}  \equiv r^{km / q} \equiv (r^{k})^{m / q} \equiv 1 &\not \equiv 1 \pmod p\\
\end{align*}

A contradiction. Hence, there cannot be any $k < p - 1$ for which we have $r^{k} \equiv 1 \pmod p$ and we have that the order is $p - 1$ which means that $r$ is a primitive root.

\newpage
\question  Exercise Set 5.2, \#11(d). 
Find all incongruent integers having order $10$ modulo $61$
\text{[HINT: First find a primitive root modulo $61$]}.

\textbf{Solution.}

First we need a primitive root modulo $61$. We have $\phi(61) = 60$ so possible orders are $1, 2, 3, 5, 6, 10, 12, 15, 20, 30, 60$
\vspace{1em}

    Try 2: We have $2, 4, 8, 32, 3, 48, 9, 11, 47, 60=-1, 1$. 

Hence we have $2$ is a primitive root. Now we know the following is true, 
\begin{align*}
    ord_{61} (2^{i}) &= \frac{ord_{61} (2)}{(ord_{61} (2), i)}\\
                     &=  \frac{60}{(60, i)}\\
\end{align*}

We need all the numbers with order $10$ so we need numbers with $ord_{61} (2^{i}) = 10$ note that all possible numbers will be of form $2^{i}$ as $2$ being a primitive root means that the powers of it will form a reduced residue class. So we have, 
\begin{align*}
    \frac{60}{(60, i)} &= 10\\
    (60, i)&= 6\\
    (10, \frac{i}{6}) &= 1
\end{align*}

Now the numbers smaller than $10$ coprime to it are $1, 3, 7, 9$ hence we have $i = 6, 18, 42, 54$

\vspace{1em}

So all incongruent integers having order $10$ modulo $61$ are $2^{6}, 2^{18}, 2^{42}, 2^{54}$ which are $3, 27, 41, 52$ modulo $61$ respectively.


\newpage
\question Exercise Set 5.2, \#12.
Let $p$ be an odd prime number.
\begin{parts}
\item Prove that any primitive root $r$ modulo $p$ is a quadratic non-residue modulo $p$.
Deduce that $r^{(p-1)/2} \equiv -1 \pmod{p}$.

\textbf{Solution.}

We have $r^{(p - 1) / 2} \equiv k \pmod p$. Now squaring both sides we get $r^{p -1} \equiv k^2 \equiv 1 \pmod p$. Now we have two solutions for $k$ either $k = 1, -1$. If $k = 1$ then we have $r^{(p - 1) / 2} \equiv 1 \pmod p$ but this means that $r$ is not a primtive root as the order is smaller than $p - 1$, hence this means we have $k = -1$ or that we have, 
\[
    r^{(p - 1) / 2} \equiv -1 \pmod p
\]

which means that $r$ is a quadratic non-residue modulo $p$.

\item Prove that there are exactly $\tfrac{p-1}{2}-\phi(p-1)$ incongruent quadratic non-residues modulo $p$
that are not primitive roots modulo $p$.

\textbf{Solution.}

First we know that there are $\frac{p - 1}{2}$ quadratic residues and non-residues. Now we know that every primitive root is a quadratic non-residue and hence is in the list of $(p - 1) / 2$. Moreover the count of the number of primitive roots are $\phi(p - 1)$ which are also non-residues. So the total non-residues that are not primitive roots are $\frac{p - 1}{2} - \phi(p - 1)$
\end{parts}

\newpage
\question Exercise Set 5.3, \#24(c).
\begin{parts}
\item Find a primitive root that works modulo $13^m$ for every positive integer $m$. Justify your choice.
\end{parts}

\textbf{Solution.}

First we know that $2$ is a primitive root modulo $13$. And we claim that $2$ is also a primitive root for any $13^m$ where $m$ is a positive integer. We do this by induction. The base case is that it is a root for $m = 1$, i.e. $ord_{13} 2 = 12$. Now assume true for some case $m = k$. So we have $ord_{13^{k}} 2 = \phi(13^{k}) = 12 \cdot 13^{k - 1}$. We need to show that for case $k + 1$ it also holds true.

\vspace{1em}

Note that for $13^{k + 1}$ the orders must divide $12 \cdot 13^{k}$ and further note if order modulo $13^{k}$ is $12 \cdot 13^{k - 1}$ then modulo $13^{k + 1}$ cannot be smaller than that. Assume for contradiction that for some $n < 12 \cdot 13^{k - 1}$ is the order for $13^{k + 1}$ then we have as $13^{k + 1} \mid 2^{n} - 1$ we also have $13^{k} \mid 2^{n} - 1$ or that $2^{n} \equiv 1 \pmod {13^{k}}$ which means that the order modulo $13^{k}$ is smaller the $12 \cdot 13^{k - 1}$, a contradiction. So the only possible orders of 2 modulo $13^{k + 1}$ are $12 \cdot 13^{k}, 12 \cdot 13^{k - 1}$.

\vspace{1em}

Now it is enough to show that $12 \cdot 13^{k - 1}$ does not work for $13^{k + 1}$ which would mean that $2$ is a primitive root modulo $13^{k + 1}$ as well. Now note for $13$ we have $2^{12} \equiv 1 \pmod 13$. And note that modulo $13^{2}$ we have $2^{12} \equiv 40 \not \equiv 1 \pmod 13^2$. So for $k = 1$ it fails. Now chaining this we can say for any arbitrary $k$ going up a level to $k + 1$ the order of the previous doesn't work. Hence $12 \cdot 13^{k - 1}$ doesn't work and the order must be $12 \cdot 13^{k}$ which makes $2$  a primitive root.

\vspace{1em}

Hence $2$ is a primitive  for any $13^{m}$ where $m$ is a positive integer.

\newpage 
\question Exercise Set 5.4, \#30(a),(e).
Use indices to find all incongruent solutions of each congruence below.
\begin{parts}
\item $8x^7 \equiv 5 \pmod{13}$

    \textbf{Solution.}

    Let $r$ be a primitive root. Then $ind_r 8x^{7} \equiv ind_r 5 \pmod {12}$  we can expand it as follows,
    \begin{align*}
        ind_r 8 + ind_r x^{7} &\equiv ind_r 5 \pmod {12}\\
        ind_r 8 + 7 ind_r x &\equiv ind_r 5 \pmod {12}\\
        7 ind_r x  &\equiv ind_r 5 - ind_r 8 \pmod {12}
    \end{align*}

    Now a primitive root modulo $13$ is $2$. So we have $x$ such that
    
    \[
        7 ind_2 x  &\equiv ind_2 5 - ind_2 8 \pmod {12}
    \]

    And we have $ind_2 5 = 9$ and $in_2 8 = 3$. So, 
    \begin{align*}
        7 ind_2 x &\equiv  6 \pmod {12}
    \end{align*}

    Note $7$ has an inverse as we have $12 \cdot 3 - 7 \cdot 5 = 1$ so inverse of $7$ modulo $12$ is $-5$. Hence we have, 
    \[
        ind_2 x &\equiv -30 \equiv 6 \pmod {12}
    \]
    Now this means that we have $x \equiv 2^{6} \equiv 12 \pmod {13}$

\item $7x^5 \equiv 2 \pmod{17}$

    \textbf{Solution.}

    We have  $\phi(17) = 16$. We can easily check the $3$ is a primitive root modulo $17$. Now note that,
    \begin{align*}
        ind_3 7 x^{5} &\equiv ind_3 2 \pmod {16}\\
        ind _3 7 + 5 ind_3 x &\equiv ind_3 2 \pmod {16}\\
    \end{align*}

    Now we have $ind_3 7 = 11$ and $ind_3 2 = 14$ so, 
    \begin{align*}
    5 ind_3 x &\equiv 3 \pmod {16}
    \end{align*}

    Now $(16, 5) = 1$ and $16 + 5 \cdot -3 = 1$ so inverse of $5$ modulo $16$ is $-3$ which gives us, 
    
    \[
    ind_3 x \equiv 7 \pmod {16}
    \]

    So we have $x \equiv 3^{7} \equiv 11 \pmod {17}$

\end{parts}

\newpage 
\question Exercise Set 5.4, \#35. 
Let $p$ be a prime number and let $r$ and $s$ be primitive roots modulo $p$. Let
$a \in \mathbb{Z}$ with $p \nmid a$.
\begin{parts}
\item Prove that $\text{ind}_s a \equiv (\text{ind}_r a )( \text{ind}_s r) \pmod{p-1}$.
\newline
(This corresponds to the change-of-base formula for logarithms).

\textbf{Solution.}

Let $ind_s a = x, ind_r a = y, ind_s r = z$. Now by definition we have, 
\[
    s^{x} \equiv a, r^{y} \equiv a, s^{z} \equiv r \pmod p
\]

So we get putting the second into the third we have, 
\begin{align*}
    (s^{z})^{y} &\equiv r^{y} \pmod p\\
           s^{yz} &\equiv r^{y} \equiv a \pmod p
\end{align*}

But we also know that $s^{x} \equiv a \pmod p$. This gives us,
\begin{align*}
    s^{x} \equiv s^{yz} \pmod p
\end{align*}

which means we have, 
\begin{align*}
    x &\equiv yz \pmod {p - 1}\\
ind_s a & \equiv (ind_r a)(ind_s r) \pmod {p - 1}
\end{align*}



\item Prove that $\text{ind}_r(p-a) \equiv \text{ind}_r a+\frac{p-1}{2} \pmod{p-1}$.
(This congruence yields all indices relative to $r$ after half of these indices are computed).

\textbf{Solution.} 

Let $ind_r (p - a) = x$ so $r^{x} \equiv p - a \pmod p$ and $r^{x} \equiv -a \equiv a r^{(p - 1) / 2} \pmod p$. Note that we have $r^{(p - 1) / 2} \equiv -1 \pmod p$ as $r$ is a primitive root implying that it's not a quadratic residue. So we have $ind_r (p - 1) \equiv x \equiv ind_r (a r^{(p - 1) / 2}) \pmod {p - 1}$. Now note that expanding the last we get $ind_r (a r^{(p - 1) / 2}) \equiv ind_r a + ind_r r^{(p - 1) / 2} \equiv ind_r a + (p - 1) / 2 \pmod {p -1} $. So we get, 
\[
    ind_r (p - a) \equiv ind_r a + (p - 1) / 2 \pmod {p - 1}
\]
\end{parts}

\newpage 
\question
Exercise Set 5.4, \#36.
\begin{parts}
\part Let $p$ be an odd prime number. Prove that the congruence $x^4 \equiv -1 \pmod{p}$ is solvable if and only if $p \equiv 1 \pmod{8}$.

\textbf{Solution.}

($\Rightarrow$) We have $x^{4} \equiv -1 \pmod p$ so $x^{8} \equiv 1 \pmod p$, but note that the squareroot of $x^{8}$, i.e. $x^{4}$ is not equivalent to $1$ hence the order of $x$ is $8$. But we know the order must divide $p - 1$ so we have $8 \mid p - 1$ or that $p \equiv 1 \pmod 8$

($\Leftarrow$) We have $p \equiv 1 \pmod 8$ or $8 \mid p - 1$. So $8$ is a possible order for $p$ and we can further say that we can find an element smaller than $p$ for which we have $x^{8} \equiv 1 \pmod p$ i.e. with order $8$ as we have $\phi(8)$ elements with order $8$. But if we have $x^{8} \equiv 1 \pmod p$ and $8$   is the order then we have $x^{4} \equiv \pm \pmod p$ but it cannot be equivalent to $1$ as it would mean the order is $4$ hence we have $x^{4} \equiv -1 \pmod p$.

\part Prove that there are infinitely many primes expressible of the form $8n+1$ where $n$ is a positive integer.
\newline 
[Hint: Parallel the proof from Question 6 on Homework 5.]

\textbf{Solution.}

Assume we have finitely many primes say $p_{1}, \dots, p_r$ now consider $N = 16p_{1}^4 \dots p_r^4 + 1$. Now take $k = 4p_{1}\dots p_r$ so we have $N = k^{4} + 1$ or that $k^{4} \equiv -1 \pmod N$. Now let $p$ be a prime factor of $N$ so we also have $k^{4} \equiv -1 \pmod p$. From (a) we know this means that $p \equiv 1 \pmod 8$ so $p$ is expressible in the form $8n + 1$ for some $n \in Z$. But this means that $p$ is in our list $\{p_{1}, \dots,  p_r\}$ say $p_i$. So $p_i$ is a factor of $N$ which means $p_i \mid k^{4} + 1$ expanding this we get $p_i \mid 16p_1^{4} \dots p_r^{4} + 1$. However, note that $p_i \mid 16p_{1}^{4} \dots p_r^{4}$ as $p_i$ is in the list of primes. Which means we need $p_i \mid 1$. But $p_i$ is an odd prime greater than $1$ and hence this is not possible. A contradiction. Hence it cannot be true that there are finitely many primes of the form $8n + 1$.
\end{parts}


\newpage 
Blank page:






\end{questions}

\end{document} 


 
%%%% don't delete the last line!
\end{document}
