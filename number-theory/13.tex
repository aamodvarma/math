
\begin{ex}
    Find all the quadratic residues modulo 23. We need, 
    
    $$
        x^2 \equiv a \pmod 23
    $$

    1, 4, 9, 16, 2, 13, 3, 18, 12, 8, 6
\end{ex}



\begin{theorem}[Euler's Criterion]
    Let $p$ be an odd prime and let $a \in \Z$ with $p \nmid a$. Then, 
    \begin{align*}
        \left ( \frac{a}{p} \right ) \equiv a^{\frac{p - 1}{2}} \pmod p
    \end{align*}
\end{theorem}
\begin{proof}
    Suppose that $\left ( \frac{a}{p} \right ) = 1$. Then we have, 
    $$
        x^2 \equiv a \pmod p
    $$
    has a solution for some $x = x_{0}$. So we have, 
    \begin{align*}
        a^{\frac{p - 1}{2}} \equiv (x_{0}^2)^{\frac{p - 1}{2}} \equiv x_{0}^{p - 1} \equiv 1 \equiv \left ( \frac{a}{p} \right ) \pmod p
    \end{align*}

    Now let $\left ( \frac{a}{p} \right ) = -1$ so $a$ is not a quadratic residues mod $p$. Since $p \nmid a$ for each $1 \le i \le p - 1$ the congruence $ij \equiv a \pmod p$ has a solution $j$ with $1 \le j \le p - 1$. We have $ \ne j$ as otherwise $a$ is a quadratic residues. 
    \vspace{1em}
    
    Thus we can pair the residues $1, 2, \dots, p - 1$ into $\frac{p - 1}{2}$ pairs $(i, j)$ such that ,
    \begin{align*}
        ij \equiv a \pmod p
    \end{align*}

    So this gives us, 
    $$
        1 2 \dots (p - 1) \equiv (p - 1)! \equiv a^{\frac{p - 1}{2}} \equiv -1 \pmod p
    $$

    by Wilson's theorem.
\end{proof}



\begin{eg}
    Calculate $\left ( \frac{3}{7} \right )$. We have, 
    $$
        \left ( \frac{3}{7} \right ) \equiv 3^{3} \equiv 27 \equiv -1 \pmod 7
    $$
\end{eg}
\begin{prop}
    Let $p$ be an odd prime with $a, b \in \Z$ such that $p \nmid a$ and $p \nmid b$. Then, 
    \begin{enumerate}
        \item $\left ( \frac{a^2}{p} \right ) = 1$
        \item If $b \equiv a \pmod p$ then $\left ( \frac{a}{p} \right ) = \left ( \frac{b}{p} \right )$
        \item $\left ( \frac{ab}{p} \right ) = \left ( \frac{a}{p} \right ) \left ( \frac{b}{p} \right )$
    \end{enumerate}
\end{prop}

\begin{proof}
    1. We have $a^2 \equiv x^2 \pmod p$ has a solution $x = a$.
    \vspace{1em}
    
    2. The congruence $x^2 \equiv a \pmod p$ is solvable if and only if $x^2 \equiv b \equiv a \pmod p$ is solvable.

    \vspace{1em}
    
    3. By Euler's criterion we have,
    \begin{align*}
        \left ( \frac{ab}{p} \right ) \equiv (ab)^{\frac{p - 1}{2}} \equiv a^{\frac{p - 1}{2}} b^{\frac{p - 1}{2}} \equiv \left ( \frac{a}{p} \right ) \left ( \frac{b}{p} \right ) \pmod p
    \end{align*}

    Since the only values are $\pm 1$ the congruence implies equality.
\end{proof}



\subsection{Further Properties}

If $a = \pm 2^{a_{0}} p_{1}^{a_{1}} \dots p_r^{a_r}$, then
\begin{align*}
    \left ( \frac{a}{p} \right ) = \left ( \frac{\pm 1}{p} \right )\left ( \frac{2}{p} \right )^{a_{0}} \left ( \frac{p_{1}}{p} \right )^{a_{1}} \dots \left ( \frac{p_r}{p} \right )^{a_r}
\end{align*}

To evaluate $\left ( \frac{a}{p} \right )$ we only need to understand $\pm 1, 2, p_{1}, \dots, p_r$ over $p$.

\begin{theorem}
        Let $p$ be an odd prime. Then $\left ( \frac{-1}{p} \right )\equiv (-1)^{\frac{p - 1}{2}} = \begin{cases}
            1 \quad \text{ if $p = 1 \pmod 4$ }\\
            -1 \quad \text{ if $p \equiv 3 \pmod 4$ }
        \end{cases}$
\end{theorem}
\begin{proof}
    The first equivalence follows from Euler's Criterion. For the second we compute. If $ \equiv 1 \pmod 4$ then $p = 1 + 4k$ for some $k \in \Z$ so, 
    \begin{align*}
        \left ( \frac{-1}{p} \right ) \equiv (-1)^{4k} \equiv (-1)^{2k} \equiv 1 \pmod p
    \end{align*}

    Similarly if $p = 3 + 4k$ then, 

    \begin{align*}
        \left ( \frac{-1}{p} \right ) \equiv (-1)^{2k + 1} \equiv (-1)^{2k + 1} \equiv -1 \pmod p
    \end{align*}
\end{proof}

\begin{lemma}[Gauss' Lemma]
    Let $p$ be an odd prime and let $a \in \Z, p \nmid a$. Let $n$ be the number of least positive residues of the integers in, 
    
    $$
        a, 2a, 3a, \dots, \frac{(p - 1)}{2} a
    $$
    that are greater than $\frac{p}{2}$. Then we have, 
    $$
        \left ( \frac{a}{p} \right ) = (-1)^{n}
    $$
\end{lemma}
\begin{proof}
    Let $r_{1}, \dots, r_n$ be the least positive residues among $a, 2a, \dots, \frac{p - 1}{2}a$ greater than $\frac{p}{2}$ and let $s_1, \dots, s_m$ be the ones less than $\frac{p}{2}$. Note, none of the $r_i, s_j$ are $0$ mod $p$. Consider the $\frac{p - 1}{2}$ integers given by the following list, 
    $$
        p - r_{1}, p - r_{2}, \dots, p - r_n, s_{1}, \dots, s_m
    $$
    This is the set of residues $1, 2, \dots, \frac{p  -1}{2}$ in some order. All elements satisfy $\ge 1$ and are less than equal to $\frac{p - 1}{2}$ since they are all less than $\frac{p}{2}$. Thus it suffices to show that there are no duplicates. If $p - r_i \equiv p - r_j \pmod p$ then $r_i \equiv r_j \pmod p$ so $r_i = r_j$ but that means that $ak_i \equiv ak_j \pmod p$ but as $(a, p) = 1$ we have $k_i = k_j$ but they are distinct.
    \vspace{1em}
    
    By a similar argument there is not repetition among the $s_j$. The only other possibility is to have $p - r_i \equiv s_j \pmod  p$. This is, 
    $$
        -k_i a \equiv k_j a \pmod p
    $$
    for some $1 \le k_i, k_j \le \frac{p - 1}{2}$. So we have $-k_i \equiv k_j \pmod p$. But we have $p - k_i \ge p / 2 > \frac{p - 1}{2} > k_j$  so the congruence is impossible. So the list is just, 
    $$
        1, 2, \dots, \frac{p - 1}{2}
    $$
    Thus multiplying them we have,
    \begin{align*}
        1 \cdot 2 \dots \frac{p - 1}{2} \equiv \frac{(p - 1)}{2}! \pmod p\\
        (-1)^{n} r_{1} \dots r_n s_1 \dots s_m \equiv \frac{p - 1}{2}! \pmod p\\
        (-1)^{n} (a)(2a) \dots (\frac{p - 1}{2}a)\equiv \frac{p - 1}{2}! \pmod p\\
        (-1)^{n} a^{\frac{p - 1}{2}} \frac{(p - 1)!}{2} \equiv \frac{(p - 1)!}{2} \pmod p\\
        (-1)^{n} a^{\frac{p - 1}{2}} \equiv 1 \pmod p\\
        a^{\frac{p - 1}{2}} \equiv (-1)^{n}  \pmod p\\
\left ( \frac{a}{p} \right ) \equiv (-1)^{n}  \pmod p\\
    \end{align*}
    and $\left ( \frac{a}{p} \right ) = (-1)^{n}$
\end{proof}


\begin{ex}
    Calculate, $\left ( -\frac{1}{13} \right ), \left ( \frac{2}{17} \right ), \left ( -\frac{14}{1} \right ), \left ( \frac{18}{23} \right )$
\end{ex}


\begin{theorem}
    Let $p$ be an odd prime. Then,
    \[
        \left ( \frac{2}{p} \right ) = (-1)^{\frac{p^2 - 1}{8}} = \begin{cases}
            1 \quad \text{ if $p \equiv 1,7 \pmod 8$ }\\
            -1 \quad \text{ if $p \equiv 3,5 \pmod 8$ }\\
        \end{cases}
    \]
\end{theorem}
\begin{proof}
    By Gauss' Lemma, we have $\left ( \frac{2}{p} \right ) = (-1)^{n}$ where $n$ is the nmber of least positive residues of, 
    \[
        2, 2 \cdot 2, 3 \cdot 2, \dots, \frac{p - 1}{2} \cdot 2
    \]

    Let $k \in \Z$ with $1 \le k \le \frac{p - 1}{2}$. Note, $2k < \frac{p}{2}$ if and only if $k < \frac{p}{4}$, so there are $\lfloor \frac{p}{4} \rfloor$ values of $k$ for which $2k < \frac{p}{2}$. Thus, there are $\frac{p - 1}{2} - \lfloor \frac{p}{4} \rfloor$ values of $k$ for which $2k > \frac{p}{2}$. Thus we have $n = \frac{p - 1}{2} - \lfloor \frac{p}{4}\rfloor$. To show that $\frac{p^2 - 1}{8}$ and $\frac{p - 1}{2} - \lfloor \frac{p}{4} \rfloor$ always have the same parity. Consider the four cases, $p \equiv 1, 3, 5, 7 \pmod 8$.


    \vspace{1em}
    
    1. For $p \equiv 1 \pmod 8$. We have $p = 8m + 1$ for some $m \in \Z$. So we have $\frac{p - 1}{2} = \frac{8m}{2} = 4m$ and $\frac{p}{4} = 2m + \frac{1}{4}$ whose floor is $2m$. So we have $4m - 2m = 2m$ which is even. Now $\frac{p^2 - 1}{8}$ is even as well.

    \vspace{1em}
    2.

    3. 

    4.

    Finally, the last equality follows by a similar case analysis.

\end{proof}
\begin{eg}
    For $\left ( \frac{2}{23} \right )$ is $(-1)^{\frac{23^2 - 1}{8}} = 1$
\end{eg}

\section{Quadratic Reciprocity}
% \setcounter[theorem][1]
\begin{theorem}[Law of Quadratic Reciprocity]
    Let $p,q $ be odd distinct primes. Then,
    
    \[
        \left ( \frac{p}{q} \right ) \left ( \frac{q}{p} \right ) = (-1)^{\frac{p - 1}{2} \cdot \frac{q - 1}{2}} = \begin{cases}
            1 \quad \text{ if either $p,q \equiv 1 \pmod 4$ } \\ -1 \quad \text{ $p \equiv q \equiv 3 \pmod 4$ }
        \end{cases}
    \]
\end{theorem}
\begin{remark}
    Q.R siplifies the calculation of $\left ( \frac{p}{q} \right )$
\end{remark}

\begin{remark}
    Which primes are quadratic residues mod 17, i.e eval $p$ such that $\left ( \frac{p}{17} \right ) = 1$ (note that this has a finite solution for $p$). Now consider for which primes $p$ is $17$ a quadratic residue for. (Here we have infinite possibilities for $p$).
\end{remark}

\begin{eg}
    Compute $\left ( \frac{7}{53} \right )$. We have $53 \equiv 1 \pmod 4$ so we get $\left ( \frac{7}{53} \right ) \left ( \frac{53}{7} \right ) = 1$. So both have to be equal which we get as,
    
    \[
        \left ( \frac{7}{53} \right ) = \left ( \frac{53}{7} \right ) = \left ( \frac{4}{7} \right ) = 1
    \]
\end{eg}

\begin{eg}
    Compute $\left ( \frac{-158}{101} \right ) =  \left (\frac{-1}{101} \right)   \left (\frac{158}{101} \right) = 1 \cdot \left ( \frac{158}{101} \right ) = \left ( \frac{57}{101} \right ) = \left ( \frac{3}{101} \right )\left ( \frac{19}{101} \right )$.

    \vspace{1em}
    
    We have $101 \equiv 1 \pmod 4$ so we have, the above is equal to,
    \begin{align*}
        \left ( \frac{-158}{101} \right ) &= \left ( \frac{101}{3} \right ) \left ( \frac{101}{19} \right )\\
                                          &= \left ( \frac{2}{3} \right ) \left ( \frac{6}{19} \right )\\
                                          &= (-1) (-1) (- \left ( \frac{19}{3} \right ))\\
                                          &= -1
    \end{align*}
\end{eg}

\begin{lemma}
    Let $p$ be an odd prime. Let $a \in \Z, p \nmid a$ and $a$ is odd. Let, 
    \[
        N = \sum_{j = 1}^{\frac{p - 1}{2}} \lfloor \frac{ja}{p} \rfloor
    \]

    Then $\left ( \frac{a}{p} \right ) = (-1)^{N}$
\end{lemma}
\begin{eg}
    Consider $\left ( \frac{7}{11} \right )$. Here we have,
    
    \[
        N = \sum_{j = 1}^{5} \left \lfloor \frac{7j}{11} \right \rfloor = 0 + 1 + 1 +2 + 3 = 7
    \]

    And we have $(-1)^{7} = -1$
\end{eg}
\begin{proof}
    Let $r_{1}, r_{2}, \dots, r_n$  be the least non-negative residues of $a, 2a, \dots, \frac{p - 1}{2} a$ that are $> \frac{p}{2}$. Likewise, let $s_{1}, \dots, s_m$ be the remaining residues that are $< \frac{p}{2}$.  Note $r_{1}, \dots, r_n, s_{1}, \dots, s_m$ are all distinct mod $p$ as they come from $a, 2a, \dots, \frac{p - 1}{2}$. This means the fractions $\frac{r_{i}}{p}, \frac{s_j}{p}$ are also all distinct. Then,
    
    \begin{align*}
        ja &= p - \frac{ja}{p}\\
           &= p \left (\left \lfloor \frac{ja}{p}\right  \rfloor + \frac{remainder}{p} \right ) \\
           &= p \left \lfloor \frac{ja}{p} \right \rfloor + remainder
    \end{align*}

    here the remainders are exaclty $r_{1}, \dots, r_n, s_{1}, \dots, s_m$. So we have,
    \begin{align*}
        \sum_{j = 1}^{\frac{p - 1}{2}} ja &= \sum_{j = 1}^{\frac{p - 1}{2}} p \left \lfloor \frac{ja}{p} \right \rfloor + \sum r_i +  \sum s_j
    \end{align*}

    Note also that, 
    \begin{align*}
        \sum_{j = 1}^{\frac{p - 1}{2}} j &= \sum_{j = 1}^{\frac{p - 1}{2}} (p - r_i) + \sum s_j\\
                                         &= p_n - \sum r_i + \sum s_j
    \end{align*}

    Note, subtracting this from above we have,
    \begin{align*}
        \sum_j ja - \sum_j j &=  \sum_{j = 1}^{\frac{p - 1}{2}} p \left \lfloor \frac{ja}{p} \right \rfloor + \sum r_i +  \sum s_j - ( p_n - \sum r_i + \sum s_j)\\
                             &=  \sum_{j = 1}^{\frac{p - 1}{2}} p \left \lfloor \frac{ja}{p} \right \rfloor  - p_n + 2 \sum r_i
    \end{align*}

    Now since $a$ is odd we have,
    \begin{align*}
        \sum_{j = 1}^{\frac{p - 1}{2}} p \left \lfloor \frac{ja}{p} \right \rfloor - pn &\equiv 0 \pmod 2\\
        \sum_{j = 1}^{\frac{p - 1}{2}} p \left \lfloor \frac{ja}{p} \right \rfloor  &\equiv pn \pmod 2\\
        \sum_{j = 1}^{\frac{p - 1}{2}} \left \lfloor \frac{ja}{p} \right \rfloor  &\equiv n \pmod 2
    \end{align*}

    So $N \equiv n pmod n$ and hence we  have $(-1)^{N} = (-1)^{n}$ so by gausses we have, 
    \[
        \left ( \frac{a}{p} \right ) = (-1)^{n} = (-1)^{N}
    \]
\end{proof}























