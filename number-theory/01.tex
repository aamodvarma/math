\chapter{Divisibility and Factorization}

\section{Divisibility}

\begin{definition}[Divisibility]
Let $a, b \in \Z$, then $a$ divides  $b$ and we write, $a \: |  \: b$, if there exists $c \in \Z$ such that,  $ b = ac $. We also say $a$ is a divisor of $b$ or a factor. We write  $a \not | \: b$ to say a does not divide  $b$
\end{definition}
\begin{eg}
    \begin{enumerate}
    \item $3 | 6$ as  $c = 2 \in \Z$ such that  $3 \cdot 2 = 6$
    \item $3 | -6$ as  $c = -2 \in \Z$ such that  $3 \cdot 2 = 6$
    \item If $a \in \Z$ then  $a | 0$ as for all a $c = 0$ will give us $a \cdot 0 = 0$
    \item $0 \: | \: 0$ as for  any $c \in \Z$ it holds true.
    \end{enumerate}
\end{eg}

\begin{prop}
    Let $a, b ,c \in \Z$. If  $a | b$ and  $b | c$, then  $a | c$
\end{prop}
\begin{proof}
   If $a | b$  then we have $c_1$ such that $a c_1 = b$ by definition. If  $ b | c$ then we have  $ b c_2 = c$ by definition. So we have, 
   \begin{align*}
       bc_2&= c\\
       ac_1c_2 &= c\\
       ac_3 &= c \quad \text{taking $c_3 = c_1c_2$}
   \end{align*}
   which by definition implies that $a | c$
\end{proof}


\begin{prop}
    Let $a, b , c, m ,n \in \Z$. If  $ c | a$ and  $c | b$ then  $c | am + bn$.
\end{prop}
\begin{proof}
    If $c | a$ then exists $ c_1 $ such $c c_1  = a$ similarly exists $c_2$ such that $cc_2 = b$. Now we have, 
    \begin{align*}
        cc_1 &= a\\
        cc_1m &= am
    \end{align*}
    and
    \begin{align*}
        cc_2 &= b\\
        cc_2n &= bn
    \end{align*}

    which gives us $am + bn = c(c_1m + c_2n) = cc_3$ which by definition implies that $c  | am + bn$
\end{proof}

\begin{definition}[Greatest integer function]
    Let $x \in \R$, the greatest integer function of  $x$, denoted   $[x]$ or $\lfloor x \rfloor$ is the greatest integer  less than or equal to $x$. 
\end{definition}
\begin{eg}
    \begin{enumerate}
        \item If $a  \in \Z$ then $[a] = a$ (The converse that if $[a] = a$ then  $a \in \Z$ is also true.)
        \item $[\pi] = 3, [e] = 2, [-1.5] = -2, [-\pi] = -4$
    \end{enumerate}
\end{eg}

\begin{lemma} 
    Let $x \in R$ then  $ x - 1 <  [x] \le x$ 
\end{lemma}

\begin{proof}
    Suppose to the contrary that $[x] \le x - 1$ then $[x] < [x] + 1\le x$.  However $[x] + 1 \in \Z$ which mmakes  $[x] + 1$ the greatest integer lesser than $x$. But this contradicts the definition hence we have $x - 1 < [x]$. 
\end{proof}

\begin{theorem}[The Division Algorithm]
    Let $a, b \in \Z$ with  $b > 0$. Then there exists unique  $q, r$ such that,  
    $$ a = bq + r \qquad 0 \le r < b $$ 
\end{theorem}

\begin{proof}
    1. Existence

    Let $q = [\frac{a}{b}]$ and $r = a - b [\frac{a}{b}]$. Now by construction we have, $a = bq + r$. Now we show that  $ 0 \le r < b$. By Lemma we have, 
     \begin{align*}
         \frac{a}{b} - 1 < [\frac{a}{b}] &\le \frac{a}{b}\\
         b - 1 > -b[\frac{a}{b}] &\ge -a\\
         b - a > -b[\frac{a}{b}] &\ge -a\\
         b  > a -b[\frac{a}{b}] = r &\ge 0\\
    \end{align*}

    2. Uniqueness

    Assume there are $q_1, q_2, r_1, r_2$ such that,  
    $$ a = bq_1 + r_1 \quad  a = bq_2 + r_2 $$ 
    We have, 
    \begin{align*}
        0 &= a - a \\
        &= (bq_1 + r_1 )- (bq_2 + r_2) \\
        &= b(q_1 - q_2) + (r_1 - r_2)
    \end{align*}
    Now, 
        $$r_2 - r_1&= b(q_1 - q_2)$$ so now we have $b | r_2 - r_1$, but we know that $-(b - 1) \le r_2 - r_1 \le b - 1$ which means that $r_2 - r_1 = 0$ which implies that $r_1 = r_2$. Similarly we have $b(q_1 - q_2) = r_2 - r_1 = 0$ which means that $q_1 - q_2 = 0$ or $q_1 = q_2$
\end{proof}
\begin{note}
     $r = 0$  if and only if $b | a$
\end{note}
\begin{eg}
    Suppose $a = -5, b = 3$ then we have,  
    $$ q = [\frac{a}{b}] =[-\frac{5}{3}] = -2 $$ 

    And 
    $$ r = a - b[\frac{a}{b}] = -5 = 3(-2) = 1 $$ 
    So $-5 = 3 \cdot -2 + 1$
\end{eg}
\begin{note}
    We can also write $-5 = -3 \cdot 1 - 2$. However this doesn't contradicts the uniqueness as  $r = -2$ is not in the bounds defined in our definition.
\end{note}


\begin{definition}
    Let $n \in \Z$, then  $n$ is even if $2 | n$ and odd otherwise.
\end{definition}
