\chapter{Applications}

\section{Cryptography; RSA Encryption}
\textbf{The RSA Encryption Scheme}
\begin{enumerate}
    \item The Public Key

        We have $p, q$ are distinct odd primes (very large) with $m = pq$ and $e$ a positive integer such that $(e, \phi(m)) = 1$. The pair $(e, m)$ is the public key.

        \vspace{1em}
        
        Here $p, q, \phi(m)$ are not disclosed, only $e, m$.

    \item Formatting

        Each letter of plain text can be converted to a numerical encoding (say position in the alphabet). Now format these numerical versions into blocks of maximal even length s.t each block of digits viewed as a single positive integer is less than $m$.

    \item Encryption Scheme

        Each block $P$ viewed as a positive integer is encrypted as, 
        \[
            P^{e} \equiv C \pmod m
        \]

        to obtain block $C$ viewed as a single positive integer.
\end{enumerate}


\textbf{RSA Decryption Scheme}

\begin{enumerate}
    \item Decryption Key

        If $(e, m)$ is the public key, then $(d, m)$ is the private key where $d$ is the inverse of $e$ modulo $\phi(m)$.

    \item Decryption Scheme

        Now each block $C$ viewed as a single positive integer can be decrypted by, 
        \[
            C^{d} \equiv P \pmod m
        \]

    \item Deformatting

        Replace each two-digit block with it's alphabetical form.

\end{enumerate}



\textbf{Theory}
Given that $P^{e} \equiv C \pmod m$. Since $d$ is the inverse of $e$ modulo $\phi(m)$ we have $ed \equiv 1 \pmod \phi(m)$ or, 
\[
    ed = k \phi(m) + 1 \quad \text{ for some } k \in \Z
\]
Then if $(P, m) = 1$ we have, 
\[
    C^{d} \equiv (P^{e})^{d} \equiv P^{k \phi(m) + 1} \equiv (P^{\phi(m)k}) P \equiv P \pmod m
\]

The last is true because of Euler's Theorem. So we have $C^{d} \equiv P \pmod m$.


\begin{remark}
    What if $P, m$ are not coprime?
\end{remark}
