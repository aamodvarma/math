\documentclass[a4paper]{report}
\input{preamble.tex}
\title{Number Theory: HW1}
\author{Aamod Varma}
\begin{document}
\maketitle
\date{}
    
\section*{Problem 1}

(a). Given $a,b,c \in \Z$ and  $c \ne 0$, we need to show that  $a | b$ if and only if  $ac | bc$.

    First we show that $a | b$ implies  $ac | bc$. If $a | b$ then  there exists some $x \in \Z$ such that  $ax = b$. Now multiply $c$ on both sides to get $acx = bc$. This means that there is some $x \in \Z$ for which  $ac$ multiplied by  $x$ is $bc$. In other words by definition we have  $ac | bc$.


    Now we show that $ac | bc$ implies that  $a | b$. If  $ac | bc$, by definition we have  some $x \in \Z$ such that  $ac x = bc$. Now, because  $c \ne 0$ we can divide  $c$ from both sides to get  $ ax = b$. Again by definition this means that  $b$ is a multiple of  $a$ or that $a | b$.


(b). Consider $a = 3$ and  $b = 5$. If  $c = 0$ we have  $ac = 0 $ and  $bc = 0$. Now  $0 | 0$ is true because for any choice of  $x \in \Z$ we have  $0x = 0$. So we have  $ac | bc$. However, we can easily see that  $3 | 5$ is not true. So this counterexample shows that the statement if and only if doesn't hold if  $c = 0$


\section*{Problem 2}
We have $a, m, n$ are positive integers with  $a > 1$. We need to show that  $a^{m} - 1 | a^{n} - 1$ if and only if $m | n$.


First we show that if $m | n$ then  $a^{m} - 1 | a^{n} - 1$. If $m | n$ then we have for some  $x \in \Z$ that  $mx = n$ which means that  $a^{mx} - 1 = a^{n} - 1$. We can write the left hand side as $(a^{m})^{x} - 1 = (a^{m} - 1)(a^{m}^{(x - 1)}+a^{m}^{(x - 2)}+ \dots+ 1) = (a^{m} - 1)(k)$ where $k = (a^{m}^{(x - 1)}+a^{m}^{(x - 2)}+ \dots+ 1)$. This give us, 
$$ (a^{m} - 1) k = a^{n} - 1 $$ 

Or that $a^{m} - 1 | a^{n} - 1$

Now, we show that  $a^{m} - 1 | a^{n} - 1 $ implies that $m | n$. If $a^{m} - 1 | a^{n} - 1$ then that means $\exists x\in \Z$ such that  $(a^{m} - 1)x = a^{n} - 1$. Now as $n > m$ (we know this because $x$ is a positive integer, which means that $a^{n} > a^{m}$ which means that $n > m$) we can take $n = qm + r$ for some  $q,r \in \N$ where $r < m$. Now we can write, 
\begin{align*}
    a^{n} - 1 &= a^{qm}a^{r}- 1\\
              &= a^{qm}a^{r} - a^{r} + a^{r} - 1\\
              &= (a^{qm} - 1)a^{r}+ (a^{r} - 1)\\
\end{align*}

Now we know that $a^{m} - 1 | a^{n} - 1$ by assupmtion and we also know from the above proof that $a^{m} - 1 | a^{qm} - 1$ as $m | qm$. Hence, this also must mean that  $a^{m} - 1  | a^{r} - 1$. However, by construction we have $r < m$ based on how we constructed  $n$ as $n = qm + r$. This meanss that $a^{m} - r > a^{r} - 1$. A larger mumber divides a smaller number only when the smaller number is zero. SO we have $a^{r} - 1 = 0$. Or that $r = 0$. This gives us,  $n = qm + r = qm + 0 = qm$ which implies that  $m | n$.
\



\section*{Problem 3}
Give $n \in \Z$,


(a). To show that $3 | n^{3} - n$.

First we rewrite $n^{3} - n = n(n^2 - 1) = n(n + 1)(n - 1)$

We have three cases, 

\textbf{Case 1:} $n = 3q + 0$ for some  $q \in Z$

 Here $n  = 3q$ so we have  $$n(n + 1)(n - 1) = 3q(3q + 1)(3q - 1) = 3(k)$$  where $k = q(3q + 1)(3q - 1) $. Hence we have $3 | n^{3} - n$

\textbf{Case 2:} $n = 3q + 1$ for some  $q \in Z$

 Here $n = 3q + 1$ so we have  $$n^{3} -n = n(n + 1)(n - 1) = (3q + 1 )(3q + 2)(3q) = 3k$$ where $k = (3q + 1 )(3q + 2)q$ so we have $3 | n^{3} - n$

\textbf{Case 3:} $n = 3q + 2$ for some  $q \in Z$

 Here $n = 3q + 2$ so we have  $$n^{3} - n = n (n + 1)(n - 1) = (3q + 2)(3q + 3)(3q + 1) = 3 (3q + 2)(q + 1)(3q + 1) = 3k$$ where $k =(3q + 2)(q + 1)(3q + 1)  $ which means that $3 | n^{3} - n$

So, in all three cases we have $3 | n^{3} - n$

(b). Similar to above we have 5 cases as any number can only leave reminaders $0, 1, 2, 3, 4$ when divided by 5. We can also expand $$n^{5} - n = n(n^{4} - 1) = n(n^2 + 1)(n + 1)(n - 1)$$


Now the five cases are,


\textbf{Case 1:}  $n = 5q + 0$

 Here $n = 5q$ so  $$n^{5} - n = n(n^2 + 1)(n + 1)(n - 1) = 5q(5q^2 + 1)(5q + 1)(5q - 1) = 5k$$ where $k = q(5q^2 + 1)(5q + 1)(5q - 1)$ which gives us $5 | n^{5} - n$

\textbf{Case 2:}  $n = 5q + 1$

 Here $n = 5q + 1$ so  $$n^{5} - n = n(n^2 + 1)(n + 1)(n - 1) = (5q + 1)((5q + 1)^2 + 1)(5q + 2)(5q) = 5k$$ where $k =  (5q + 1)((5q + 1)^2 + 1)(5q + 2)q$ which gives us $5 | n^{5} - n$

\textbf{Case 3:}  $n = 5q + 2$

 Here $n = 5q + 2$ so  
 \begin{align*}
     n^{5} - n = n(n^2 + 1)(n + 1)(n - 1) &= (5q + 2)((5q + 2)^2 + 1)(5q  + 3)(5q + 1)\\
                                          &= (5q + 2)(25q^2 + 4 + 20q + 1)(5q + 3)(5q + 1)\\
                                          &= (5q + 2)5(5q^2 + 1 + 4q)(5q + 3)(5q + 1)\\
                                          &= 5k
 \end{align*}
 where $k = (5q + 2)(5q^2 + 1 + 4q)(5q + 3)(5q + 1)$ so we have  $5 | n^{5} - n$

 \textbf{Case 4:}  $n = 5q + 3$

 Here $n = 5q + 3$ so  
 \begin{align*}
     n^{5} - n = n(n^2 + 1)(n + 1)(n - 1) &= (5q + 3)((5q + 3)^2 + 1)(5q + 4)(5q + 2) \\
                                          &=  (5q + 3)(25q^2 + 9 + 30q + 1)(5q + 4)(5q + 2) \\
                                          &=  (5q + 3)(25q^2 + 10 + 30q)(5q + 4)(5q + 2) \\
                                          &=  5(5q + 3)(5^2 + 2 + 6)(5q + 4)(5q + 2) \\
                                          &=  5k
 \end{align*}

 where $k= (5q + 3)(5^2 + 2 + 6)(5q + 4)(5q + 2) $ which gives us  $5 | n^{5} - n$


 \textbf{Case 5:}  $n = 5q + 4$

 Here $n = 5q + 4$ so  
 \begin{align*}
     n^{5} - n = n(n^2 + 1)(n + 1)(n - 1) &= (5q + 4)((5q + 4)^2 + 1)(5q + 5)(5q + 3)\\
                                          &=  5(5q + 4)((5q + 4)^2 + 1)(q + 1)(5q + 3)\\
                                          &= 5k
 \end{align*}
 where $k = (5q + 4)((5q + 4)^2 + 1)(q + 1)(5q + 3)$ which means that $5 | n^{5} - n$


 So in all three cases we have $5 | n^{5} - n$





In all cases we have $5 | n^{5} - n$

(c). We need to either prove or disprove that $4 | n^{4} - n$. We will disprove the statement by giving a counter example. Consider $n = 3$. Here we have  $n^{4} - n = 81 - 3 = 77$. We see that $77 = 4 \times 19 + 1$ which means that $4 \not | 77 $ and hence disproves the statement.



\section*{Problem 4}

We have $a, n$ are positive integers with  $a > 1$. We need to show that  $a^{n} + 1$ is prime then $a$ is even and $n$ is a power of 2.

Let us assume the contrary that $a$ is odd or $n$ is not a power of 2. 

Case 1 we have $n$ is odd. This means that $a^{n}$ is odd (as odd times odd is always odd). If $a^{n}$ is odd then $a^{n} + 1$ is even and not equal to 2(as $a > 1$ we have $a^{n} + 1 > 2$). And we know that 2 is the only even prime number. This means that $a^{n} + 1$ is composite which breaks our assumption that it is prime. Hence, $n$ cannot be odd.

Case 2 we have $n$ not a power of 2. If $a$ is odd we already showed that $a^{n} + 1$ is composite regardless of $n$ a power of 2 or not. Now if $a$ is even we have $a = 2x$ for some odd  $x$. Now we have  $(2x)^{n} + 1$ where $n$ is not a power of 2. We have $2^{n}(2m + 1)^{n} + 1$. We can expand this as, 
$$ 2^{n}( (2m)^{n} + n (2m)^{n - 1} + n\frac{n - 1}{2}(2m)^{n - 2} + \dots + n (2m) + 1) + 1 $$ 


\section*{Problem 5}

    Given that $n^2 + 1$ is prime. We know from above that $n$ has to be even as if its odd the number would be composite. Now assume to the contrary that $n^2 + 1$  is not expressibel int he form $4k + 1$ with integer $k$. This means that it's expressible as either  $4k, 4k + 2, 4k + 3$.

    If $n^2 + 1 = 4k$ or$ 4k + 2$ then $n^2 + 1$ is even making it composite and not prime. If $n^2 + 1 = 4k + 3 $  then $n^2 = 4k  + 2 = 2(2k + 1)$. However we know that $n^2$ is a perfect square and perfect squares cannot have an odd number of factors of 2  (it must have an even exponent for every prime factor). Hence this must mean that $n^2$ cannot be of the form $4k + 2$ or that  $n^2 + 1$ cannot be $4k + 3$. So, in all three cases we show that  $n^2 + 1$ cannot be prime and of that form. So our assumption must be wrong and $n^2 + 1 = 4k + 1$

    % then n^2 = 4k + 2 = 2 (2k + 1). And 2k + 1 is odd which means that n^2 must have a prime factor that is odd which means that n must have a prime factor that is odd which make n^2 odd which is contradictory as n^2is even as it is 2(2k + 1). This means that n^2 + 1 cannot be a prime of form 4k + 3. Hence in all three cases n^2 + 1 cannot be a prime number which implies the initial assumption is wrong and n^2 + 1 must be of the form 4k + 1



\section*{Problem 6}

(a). We have $GCD(a, b) = 1$. We need to show that  $a | c, b | c $ implies that  $ab | c$.

If $ GCD(a, b)= 1$ we have  some $m,n \in \Z$ such that $am + bn = 1$. If  $a | c, b | c$ then we have  some  $x,y \in \Z$ such that  $ax = c$ and  $by = c$.

We have, 
\begin{align*}
    am + bn &= 1\\
    cam + cbn &= c\\
    byam + axbn &= c \quad \text{ as $c = ax = by$}\\
    ab(ym + xn) &= c \\
    ab(z) &= c \\
\end{align*}
which by definition mean that $ab | c$


(b). Consider if $a = 5$ and  $b = 10$, so we have  $(a, b) = 5$.  We see that if $c = 20$ we have, $a = 5 | 20$ and  $b = 10 | 20$. However we see that $ab = 50 \not | 20$.


(c). We have $a_1, \dots, a_n \in \Z$ which are pairwise relatively prime numbers. We need to show that if $a_j | c$ then  $a_1\dots a_n | c$.

First we prove a primliminary result that given relatively prime numbers $a_1,\dots,a_n$, the gcd of the product of a subset of these numbers is relatively prime with numbers outside the subset. In other words we show that, 
$$ gcd(a_1\dots a_i, a_k) = 1 \text{ if $k > i$ } $$ 


We will do this by induction. For the base case we have $i = 1$  for which this is trivially true by construction (as all the numbers are pairwise relatively prime). Now consider the case for some arbitrary $m$. So we have,  
$$ gcd(a_1 \dots a_m, a_{m + 1}) = 1 $$ 

We need to show this is true for $m + 1$. Now let  $a_1 \dots a_m = x, a_{m + 1} = y, a_{k} = z$ where $k > m + 1$

So we have,

\begin{align*}
    xm_1 + zn_1 &= 1 \text{ for some $m_1,n_1 \in \Z$}\\
    ym_2 + zn_2 &= 1\text{ for some $m_2,n_2 \in \Z$}\\
\end{align*}

Now multiplying these two together we have, 
$$ xym_1m_2 + z(xm_1n_2 + zn_1n_2 + yn_1m_2) = 1 $$ 
$$ xym_3 + z(n_3) = 1 \text{ where $m_3 = m_1m_2, n_3 = xm_1n_2 + zn_1n_2 + yn_1m_2$} $$ 

This by definition means that $gcd(xy, z) = 1$. Or expending it gives us,

$$ gcd(a_1\dots a_{m}a_{m + 1}, a_{m + 2}) = 1 $$ 


Which is the case for $n = m + 1$


Now we prove the initial statement inductively.  Consider  $i = 1$ for which the statement is trivially true. Now consider the statement is true for some arbitrary $i$  so we have  $a_1\dots a_i | c$ given $a_1 | c,\dots,a_i | c$. Now, consider $a_{i + 1}$. Let $a_1 \dots a_i = a'$ so we have  $gcd(a', a_{i + 1}) = 1$ based on the proof above. Similarly we have $a' | c$ and $a_{i + 1} | c$ by assumption. So based on the proof in (a) this means that $a' a_{i + 1}| c$ or that $a_1 \dots a_{i + 1} | c$ which is the case for $i + 1$. Hence we compute the inductive step and show that it must be true for any  $i$.



\section*{Problem 7}
We have $a, b \in \Z$. We need to show that  $(a, b) = 1$ if and only if  $(a + b, ab) = 1$. 

First we show the if condition. So we have, 
$$ (a + b, ab) = 1 $$  which means that, 
$$ (a + b) m + abn = 1 \text{ for some $m,n \in \Z$} $$ 

Now we have,
\begin{align*}
    am + bm + abn &= 1\\
    a(m + bn) + bm  &= 1\\
    ax + by &= 1
\end{align*}

By definition this means that $(a, b) = 1$


Now we show the only if condition.


Let's assume the contrary that $(a + b, ab ) = x > 1$ so we have,  
$$ x | a + b \text{ and } x | ab $$

If $x | ab$ let $p$ be a prime dividing $x$ then it must mean either  $p | a$ or $p | b$ as  $p$ can't divide  $a$ and  $b$ as   $gcd(a, b) = 1$. So assume without loss of generality that $ p | a$. Now we know that $x | a + b$ which means $p | a + b$. But if $p | a + b$ and $p | a$, then that must mean  $p$ also divides their difference or that $p | (a + b) - a$ or  $ p | b$. However, then we get that  $p | a$ and  $p | b$  or that $a$ and $b$ are not coprime as $p \ne 1$ which is a contradiction as we know that $gcd(a, b) = 1$. Hence, our assumption  must be wrong and it must be true that $(a + b, ab) = 1$. 


\section*{Problem 8}
We need to compute $gcd(441, 1155) $ using the euclidean algorithm,

 
\begin{align*}
    1155 &= 441 \times 2 +  273\\
    441 &= 273 \times 1 + 168\\
    273 &= 168 \times 1 + 105\\
    168 &= 105 \times 1 + 63\\
    105 &= 63 \times 1 + 42\\
    63 &= 42 \times 1 + 21\\
    42 &= 21 \times2 + 0
\end{align}
This gives us the GCD as $21$

To find the linear combination we go backwards to get,
\begin{align*}
    1 \times 105 &= 63 + (63 - 21)= 2\times63 - 21\\
    2 \times 168 &= 2 \times 105 + (105 + 21) = 3 \times 105 + 21\\
    3 \times 273 &= 3 \times 168 + (2 \times 168 - 21) = 5 \times 168 - 21\\
    5 \times 441 &= 5 \times 273  + (3 \times 273 + 21) = 8 \times 273 + 21\\
    8 \times 1155 &= 16 \times 441 + (5 \times 441 - 21) = 21 \times 441 - 21\\
\end{align*}
This gives us, 
    $$21\times 441 - 8 \times 1155 = 21$$




\section*{Problem 9}
We need to find two rations with denominators $11$ and  $13$ whose sum is  $\frac{7}{143}$. Consider the rationals to be $\frac{p}{11}$ and $\frac{q}{13}$ so we have, 
\begin{align*}
    \frac{p}{11} + \frac{q}{13} &= \frac{7}{143} \\
    \frac{13p + 11q      }{143} &= \frac{7}{143}\\
    13p + 11q &= 7
\end{align*}

We know that $13$ and  $11$ have gcd of  $1$ so there exists $m, n$ such that  $13m + 11n = 1$. 

 \begin{align*}
    13 &= 11 \times 1 + 2\\
    11 &= 2 \times 5 + 1\\
    5 \times 13 &= 5 \times 11  + (11 - 1)\\
    6 \times 11 -5 \times 13 &= 1
\end{align*}


Multiplying this by 7 we get, 
$$ 11 \times 42 + 13 \times -35= 7 $$ 


Hence we get $p = -35$ and $q = 42$ to get,  
$$ \frac{-35}{11} + \frac{42}{13} = \frac{7}{143} $$ 


\end{document}
