\begin{corollary}
	Consider the linear congruence $ax \equiv b \pmod m$, and let $d = gcd(a, m)$. If $d \mid b$, then there are exactly $d$ incongruent solutions modulo $m$ given by, 
	$$ x = x_0 + \bigg (\frac{m}{d}n \bigg ), \quad  n= 0,1,2,\dots, d - 1$$ 
	and $x_0$ is any particular solution.
\end{corollary}
	
\begin{eg}
	Find all incongruent solutions to $16x \equiv 8 \pmod 28$. Here we have $d = gcd(a, m) =gcd(16, 28) = 4$. We see that $4 \mid 8$. Now we find a particular solution. Working backwards we have $4 = 2 \cdot 16+ (-1) \cdot 28$ so $8 \cdot 16 + (-2) \cdot 28$. Then $x_0 = 4$ is a solution, and we have all solutions given by,
	\begin{align*}
		x = 4 + \bigg(\frac{28}{4} \bigg)	n, \quad n = 0,1,2,3
	\end{align*}

	Which gives us $x = 4, 11, 18, 25$
\end{eg}

\begin{definition}
	Any solution of $ax \equiv 1 \pmod m$ is call the \emph{multiplicative inverse} of $a$ modulo $m$.
\end{definition}

\begin{corollary}
	The congruence $ax \equiv 1 \pmod m$ has a solution if and only if $(a, m) = 1$
\end{corollary}

\section{Chinese Remainder Theorem}

\begin{eg}
	Find a positive integer having a remainder of 2 when divided by 3, a remainder of 1 when divided by 4, and a remainder of 3 when divided by 5. So this means,
	\begin{align*}
		x \equiv 2 &\pmod 3\\
		x \equiv 1 &\pmod 4\\
		x \equiv 3 &\pmod 5
	\end{align*}
\end{eg}
\begin{theorem}
	Let \( m_1, m_2, \dots, m_n \) be pairwise relatively prime and let \( b_1, \dots, b_n \in \Z \).  Then this system,

	\begin{align*}
		x \equiv b1 &\pmod (m_1)\\
			    &\vdots\\
		x \equiv bn &\pmod (m_n)\\
	\end{align*}
\end{theorem}
\begin{proof}
	Let \( M = m_1, \dots, m_n \) and \( M_i = M/m_i\). Then \( M_i, m_i = 1 \). There are solutions to each system \( M_ix_i \equiv 1 \pmod m \) denoted \( x_i = \overline M_i \). Now consider \( x = b_1M_1\overline M_1 +  b_2M_2\overline M_2 + \dots +  b_nM_n\overline M_n \).

	\vspace{1em}

	Note that,
	\begin{align*}
		x &\equiv 0 + \dots + b_iM_i\overline M_i + \dots + 0 \pmod m_i \\
		  & \equiv b_i \pmod m_i
	\end{align*}

	This gives existence. For uniqueness, let \( x' \) be another solution. Then \( x' \equiv b_i \pmod m_i \) for each \( 1 \le i \le n \). Then \( x \equiv x' \pmod m_i \). Then \( m_i \mid x - x' \). So \( M \mid x - x' \) since \( m_i \) are pairwise relative prime and \( x \equiv x' \pmod M \)
\end{proof}



\begin{eg}[Continued]
	We have, 
	\begin{align*}
		x \equiv 2 &\pmod 3\\
		x \equiv 1 &\pmod 4\\
		x \equiv 3 &\pmod 5
	\end{align*}
	We have \( M = 3 \cdot 4 \cdot 5 = 60 \) and \( M_1 = 20, M_2 = 15, M_3 = 12 \). So we need to solve, 
	\begin{align*}
		20y_1 \equiv 1 &\pmod 3\\
		15y_2 \equiv 1 &\pmod 4\\
		12y_3 \equiv 1 &\pmod 5
	\end{align*}

	For each we have \(  7 \cdot 3  - 20= 1 \), \( 4 \cdot 4 - 15 = 1 \) and \(  5 \cdot 5 - 2 \cdot 12 = 1\). So \( y_1 = -1 = 32, y_2 = -1 = 3, y_3 = -2 = 3 \). 

	So, \[
		x = 2 \cdot 20 \cdot 2 + 1 \cdot 15 \cdot 3 + 3 \cdot 12 \cdot 3 = 233
	.\] 
	And we have \( 233 \equiv 53 \pmod {60} \) which means \( 53 \) is the least positive solution.
\end{eg}



\begin{lemma}
	Let \( p \) be a prime and let \( a \in \Z \). Then \( a \) is it's own inverse modulo \( p  \iff a \equiv \pm 1 \pmod p\)
\end{lemma}
\begin{proof}
	Suppose \( a \) is it's own inverse so \( a = \overline a \). Then \( a^2 \equiv 1 \pmod p \) then \( p \mid a^2 - 1 \) so \( p \mid (a + 1)(a - 1) \) so we have either \( p \mid (a + 1) \) or \( p \mid (a - 1) \). In both cases we have either \( a \equiv \pm 1 \pmod p \)

	\vspace{1em}

	Now suppose \( a \equiv  \pm 1 \pmod p \). Squaring both sides we get \( a^2 \equiv 1 \pmod p \) so \( a = \overline a \).
\end{proof}

\begin{theorem}[Wilson's Theorem]
	Let \( p \) be a prime. Then \( (p - 1)! \equiv -1 \pmod p \)
\end{theorem}
\begin{proof}
	Easily check for \( p = 2, 3 \). Suppose \( p > 3 \) is a prime. Then each \( 1 \le a \le p - 1\) has a unique inverse modulo \( p \) and this inverse is distinct from \( a \) if \( 2 \le a \le p - 2 \). Pair each such integer with its inverse   modulo \( p \) say \( a, a' \). The product of all these primes is \( (p - 2)! \) and \( (p - 2)! \equiv 1 \pmod p \) and we get \( (p -1)! \equiv (p-1)(p - 2)! \equiv (p -1) \equiv -1 \pmod p \).

	\vspace{1em}

	The converse is also true. 
\end{proof}
\begin{prop}
	Let \( n \in \Z \) with \( n > 1 \). If \( (n - 1)! \equiv -1 \pmod n \) then \( n  \) is prime.
\end{prop}
\begin{proof}
	Suppose \( n = ab \) with \( 1 \le a < n \). It suffices to show that \( a = 1 \). Since \( a < n \) so \( a \mid (n - 1)! \). Also \( n \mid (n - 1)! + 1  \). Now since \( a \mid n \) we have \( n \mid (n - 1)! + 1 \). But we know \( a \mid (n - 1)! \) so we need \( a \mid 1 \) which means \( a = 1 \).
\end{proof}
\begin{eg}
	Take \( p = 11 \) then, \( 11 - 1 \equiv 10! \pmod {11} \). By previous Lemma, 10 and 1 are their own inverses. For the other numbers between \( 2\) and \( 9 \), we can pair them with their inverses like \( 2\iff 6, 3 \iff 4, 5 \iff 9, 7 \iff 8 \) which means, \[
		(11 - 1)! \equiv 10 \cdot 1 \equiv -1 \pmod {11}
	.\] 
\end{eg}


\begin{definition}
	A prime \( p \) is a \emph{Wilson Prime} if \( (p - 1)! \equiv -1 \pmod {p^2}\). The first few are, \[
		5, 13, 563
	.\] 
\end{definition}

\begin{theorem}[Fermat's Little Theorem]
	Let \( p \) be a prime and let \( a \in \Z \) then if \( p \nmid a \) then \[ a^{p - 1} \equiv 1 \pmod p \]
\end{theorem}
\begin{proof}
	
\end{proof}
