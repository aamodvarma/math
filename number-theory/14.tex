
\begin{proof}
    of Theorem 4.9

    Without loss of generality, assume $p > q$. Consider the picture, the number of lattice points in the rectangle OABC. This is clearly $\frac{p -1}{2} \cdot \frac{q - 1}{2}$.

    \vspace{1em}
    
    1. The line ON has slope $\frac{q}{p}$. In particular, ON contains no lattice points. 
    
    \vspace{1em}
    
    2. The y-coordinate of M is $\frac{p - 1}{2} \cdot \frac{q}{p} = \frac{q}{2} - \frac{q}{2p}$. This lies between the consecutive integers are $\frac{q - 1}{2}$ and $\frac{q + 1}{2}$.  We have,
    \begin{align*}
        \frac{q - 1}{2} = \frac{q}{2} - \frac{1}{2} < \frac{q}{2} - \frac{q}{2p} < \frac{q}{2} < \frac{q + 1}{2}
    \end{align*}

    The number of lattice points in the rectangle OABC, not on the axis and below ON is,
    \begin{align*}
        N_{1} = \sum_{j = 1}^{\frac{p - 1}{2}} \left \lfloor \frac{jq}{p}\right \rfloor 
    \end{align*}

    Likewise, the number of lattice points above ON is,

    \begin{align*}
        N_{2} = \sum_{j = 1}^{\frac{q - 1}{2}} \left \lfloor \frac{jp}{q}\right \rfloor 
    \end{align*}

    Thus, the total number of lattice points in question is $N_{1} + N_{2} = \frac{p - 1}{2} \cdot \frac{q - 1}{2}$.

    \vspace{1em}
    
    From Lemma 4.10 we have $\left ( \frac{p}{q} \right ) \left ( \frac{q}{p} \right ) = (-1)^{N_{2}} (-1)^{N_{1}} = (-1)^{\frac{p - 1}{2} \cdot \frac{q - 1}{2}}$
\end{proof}

\subsection*{Characterizing Particular Primes}
Note, we have characterized the primes for which $-1$ and $2$ are quadratic residues.

\vspace{1em}

\begin{eg}
For primes is $3$ a quadratic residues. So for what $p$ is $\left ( \frac{3}{p} \right ) = 1$.

\vspace{1em}

1. $\left ( \frac{3}{p} \right ) = \begin{cases}
    \left ( \frac{p}{3} \right )  &\text{ if $p \equiv 1 \pmod 4$ } \\
    - \left ( \frac{p}{3} \right )  &\text{ if $p \equiv 3 \pmod 4$ } \\
\end{cases}$

\vspace{1em}


2. Tabulate quadratic residues  and quadratic non-residues mod $3$. We have $1^2 = 1, 2^2 = 1 \pmod 3$. So only quadratic resiude is $1$. 

\vspace{1em}


3. Analyze cases,

Suppose $p \equiv 1 \pmod 4$. Then, $\left ( \frac{p}{3} \right ) = 1$ if $p \equiv 1 \pmod 3$

\vspace{1em}

Suppose $ p \equiv 3\pmod 4$. Then, $\left ( \frac{p}{3} \right ) = -1$ if $p \equiv 2 \pmod 3$.

\vspace{1em}

4. Chinese remainder theorem,

In case 1 we have $p \equiv 1 \pmod 4, p \equiv 1 \pmod 3$ and case 2 we have $p \equiv 3 \pmod 4, p \equiv 2 \pmod 3$. For case 1 we get $p \equiv 1 \pmod {12}$ and for case 2 we have $p = 3 \cdot 3 \cdot \overline{3} + 2 \cdot 4 \cdot \overline{4} = -1$.
\vspace{1em}

5. We have $\left ( \frac{3}{p} \right ) = 1$ if and only if $p \equiv \pm 1 \pmod {12}$ 
\end{eg}

\begin{eg}
    Characterize the primes $p$ for which both 2 and 3 are quadratic residues. So we want $p$ such that, 
    \[
        \left ( \frac{2}{p} \right ) = \left ( \frac{3}{p} \right ) = 1
    \]

    We have $\left ( \frac{2}{p} \right ) \equiv 1 $ if $p \equiv \pm 1  \pmod 8$ and $\left ( \frac{3}{p} \right ) \equiv 1$ iff $p \equiv \pm 1 \pmod {12} $. This is equivalent to $p \equiv \pm 1 \pmod {24}$
\end{eg}


\begin{eg}
    Characterize the primes $p$ for which $13$ is a quadratic residue. So we want $p$ such that, 
    \[
        \left ( \frac{13}{p} \right ) = 1
    \]

    We have $\left ( \frac{13}{p} \right ) = \left ( \frac{p}{13} \right )$. Now we need to find squares modulo $13$ which is $1^2  = 1, 2^2 = 4, 3^2 = 9, 4^2 = 3, 5^2 = 12, 6^2 = 10$. 13 is a quadratic residue mod $p$ if and only if $p \equiv 1, 3, 4, 9, 10, 12 \pmod {13}$
\end{eg}

% \begin{eg}
%     For $11$,

%     $\left ( \frac{11}{p} \right ) = \begin{cases}
%         \left ( \frac{p}{11} \right ) &\text{ p \equiv 1 \pmod 4 } \\
%         \left ( -\frac{p}{11} \right ) &\text{ p \equiv 3 \pmod 4 } \\
%     \end{cases}$

%     Step 2: Quadratic residue are $1, 3, 4, 5, 9$ and quadratic non-residues are $2, 6, 7, 8, 10$.

%     \vspace{1em}
    
%     Step 3: If $p \equiv 1 \pmod 4$ then $\left ( \frac{p}{11} \right ) = 1$ if $ p \equiv 1, 3, 4, 5, 9 \pmod {11}$ and if $p \equiv 3 \pmod 4$ then $\left ( \frac{p}{11} \right ) = -1$ if $p \equiv 2, 6, 7, 8, 10 \pmod {11}$

%     \vspace{1em}
    
%     Step 4: 

%     We need to solve the 10 pairs of Chinese remainder theorem problems.

%     \vspace{1em}
    
%     So $11 $ is a quadratic resiude mod p if and only if, 
%     \[
%         p \equiv \pm 1, \pm 5, \pm 7, \pm 9, \pm 19 \pmod 44
%     \]
% \end{eg}
