\documentclass[12pt]{exam}
\usepackage{amsmath,amstext,amssymb,amsthm}   
\usepackage{enumerate}
%%%%%%%%%%%%%%%%%%%%%%%%%%%%%%
%%%%%%%%%%%%%%%%%%%%%%%%%%%%%%
%%%%%%%%%%%%%%%%%%%%%%%%%%%%%%
 



\begin{document}



 \begin{center}
 {\Large Homework 5, Math 4150 }
 \\
 \end{center}


\begin{questions}
\question  Exercise Set 4.1, \#5. 
Find all incongruent solutions of each quadratic congruence below.
\begin{parts}
\item $x^2 \equiv 23 \pmod{77}$ \newline
[Hint: Consider the congruences $x^2 \equiv 23 \pmod{7}$ and $x^2 \equiv 23 \pmod{11}$
and use the Chinese Remainder Theorem].

\textbf{Solution.} We have  $x^2 \equiv 23 \pmod {11}$ and $x^2 \equiv 23 \pmod {7}$ we can reduce both of them to, 
\begin{align*}
    x^2 &\equiv 2 \pmod 7 \\
    x^2 &\equiv 1 \pmod {11} 
\end{align*}

For numbers equiv 1 modulo $11$ we know these are $1$  and $11 -1$ so we have $x$ is either $1, 11$. And modulo $7$, we know there are either 0 or 2 solutions and listing out we get $1^2 \equiv 1, 2^2 \equiv 4, 3^2 \equiv 2$. So $3$ is a solution which means that $p - 3 = 4$ is the only other solution.

\vspace{1em}

Now using CRT we have $m_{1} = 7, m_{2} = 11$ and finding inverses to $11y_{1} \equiv 1 \pmod 7$ and $7 y_{2} \equiv 1 \pmod {11}$. We have $11 \cdot 2 - 7 \cdot 3 = 1$ so the inverses are $2$ and $-3$ which gives us, 
\[
    x = a \cdot 11 \cdot 2 + b \cdot 7 \cdot -3 \pmod {77}
\]
Now our various options are $a = 3, 4$ and $b = 1, -1$ plugging in we get, 
\[
    x = 45, 10, 67, 32 \pmod {77}
\]



\item $x^2 \equiv 11 \pmod{39}$

    We can divide this into two congruencies $x^2 \equiv 11 \equiv 2 \pmod {3}$ and $x^2 \equiv 11 \pmod {13}$. The first congruency we don't have a solution. So there isn't a solution to this problem.
\item $x^2 \equiv 46 \pmod{105}$ 

    We have $x^2 \equiv 46 \equiv 1 \pmod 5$, $x^2 \equiv 46 \equiv 4 \pmod 7$ and $x^2 \equiv 46 \equiv 1 \pmod 3$.

    \vspace{1em}
    
    We have $1, -1$ for the first one $2, 5$ for the second and $1, -1$ for the third. Now we have $M = 105$ so $m_{1} = 35, m_{2} = 21, m_{3} = 15$. Our system of congruence ids $21y_{1} \equiv a \pmod 5, 15y_{2} \equiv b \pmod 7, 35 y_{3} \equiv c \pmod 3$ solutions to which are $1, 1, -1$. So our final solution would be, 
    \[
        x \equiv a (21)(1) + b(15)(1) + c(35)(-1) \pmod {105}
    \]

    Plugging in all combinations we get,
    \[
        x \equiv 16,86, 79, 44, 26, 89, 61,19 \pmod {105}
    \]
\end{parts}

\newpage
Blank page:


\newpage
\question Exercise Set 4.1, \#6.
Let $p$ be an odd prime number. Prove that the $\frac{p-1}{2}$ quadratic residues modulo $p$ are congruent to 
\begin{equation*}
1^2,2^2,3^2,\ldots,\Big(\frac{p-1}{2} \Big)^2,
\end{equation*}
modulo $p$.


\textbf{Solution.} $1^2, 2^2, \dots, \left ( \frac{p - 1}{2}\right )^2$ consist of $\frac{p - 1}{2}$ unique square numbers so they are all quadratic residues. So it is enough to show that these residues are unique mod $p$. First assume that they are not unique so we have for some $a, b < \frac{p - 1}{2}$ that $a^2 \equiv b^2 \pmod p$. So we have $p \mid a^2 - b^2$ or $p \mid (a + b)(a - b)$. This means that either $p \mid a + b$ or $p \mid a - b$. If its the latter then we have $a = b$ as $|a - b|$ can't exceed $p - 1$ so it cannot be a multiple of $p$ leaving only $0$ as a choice. . If its the former then as $1 \le a, b \le \frac{p - 1}{2}$ so $2 \le a + b \le p - 1$ and $p$ can't divide $p - 1$. Hence we get $a= b$. So we show that in our set of $\frac{p - 1}{2}$ squares of residues, all of them have to be distinct which means that all of the $\frac{p - 1}{2}$ quadratic residues are congruent  to the above list of squares.

\newpage 
\question  Exercise Set 4.1, \#7.
\newline
\textbf{Note:} In this question you are allowed to use the standard formulae for $\sum_{j=1}^{n} j^2$ and 
 $\sum_{j=1}^{n} j^4$ without proof.
\begin{parts}
\item Let $p$ be a prime number with $p>3$. Prove that the sum of the quadratic residues modulo $p$ is divisible by $p$.

    \textbf{Solution.} We have $\sum_j j^2 = \frac{n (n + 1)(2n + 1)}{6}$. We know the sum of the residues are $\sum_j^{\frac{p - 1}{2}}$ so we have the sum as, 
    \[
       \frac{1}{6} \left( \frac{p - 1}{2}  \right) \left ( \frac{p + 1}{2}\right ) \left ( p\right )
    \]
    Now we know that this is an integer (as the formula returns an integer always). Now as $p$ is a prime we know that $6$ cannot divides $p$ so that means that $6$ divides either of the other two. So it's possible to write this as $pk$ where $k = \frac{1}{6}\left ( \frac{p - 1}{2}\right )\left ( \frac{p  +1 }{2}\right )$ is an integer.


\item Let $p$ be a prime number with $p>5$. Prove that the sum of the squares of the quadratic non-residues modulo $p$ is divisible by $p$. 

    \textbf{Solution.} We know all the residues and non residues are $1, \dots, p - 1$. Out of this $1^2, \dots, \frac{p - 1}{2}^2$ are the quadratic residues. Now $\sum^{p - 1} i^2$ is the sum of squares of all residues and non residues. And we need to subtract away the squares of residues. We know the residues themselves are $1^2 \dots, \frac{(p - 1)}{2}^2$, so squares of these would be $1^{4}, \dots, \frac{p - 1}{2}^{4}$. So first we have, 
    \[
        \sum^{p - 1}_i i^2 = \frac{1}{6}(p - 1)(p)(2p - 1)
    \]

    And we have, 

    \begin{align*}
        \sum_i^{k} j^{4} &= (k)(k + 1)(2k + 1)(3k^2 + 3k - 1) \frac{1}{30}\\
        \sum_i^{\frac{p - 1}{2}} j^{4} &=  \frac{p - 1}{2} \frac{p + 1}{2} p ...
    \end{align*}

    We already see that $p$ is a factor in this, so we have $p$ divides both the sums which means $p$ also divides $\sum_i^{p - 1} i^2 - \sum_i^{\frac{p - 1}{2}}j^{4}$ which is equivalent to the sum of squares of quadratic non-residues.


\end{parts}


\newpage
Blank page:

\newpage
Blank page:

\newpage 
\question Exercise Set 4.2, \#22.
Let $p$ be a prime number with $p \equiv 1 \pmod{4}$. Prove that 
\begin{equation*}
\sum_{a=1}^{(p-1)/2} \Big( \frac{a}{p} \Big)=0.
\end{equation*}

\textbf{Solution.}
We know from Eulers criterion that, 
\[
    \left ( \frac{a}{p}\right ) \equiv a^{\frac{p - 1}{2}} \pmod p
\]

Now $p \equiv 1 \pmod 4$ so we have $p = 4k + 1$ for some $k \in \Z$ so we have $a^{\frac{p - 1}{2}} = a^{2k} = (a^2)^{k}$. Now note that $1^2, \dots, \frac{(p - 1)}{2}^2$ gets us all the residues. And also note that given a residues $a$ i.e. $x^2 \equiv a \pmod p$ then we have $x^{4} \equiv a^2 \pmod p$ or $(x^2)^2 \equiv a^2 \pmod p$ or in other words the power of a residue is also a residue. Hence we have $a^{2k}$ is also a residue for all $a \in [0, \frac{p - 1}{2}]$. Now we need to show that these residues are unique.We have the following, 
\[
    \sum_{a = 1}^{\frac{p - 1}{2}} \left ( \frac{a}{p}\right ) \equiv \sum_a^{(p - 1)  / 2} a^{(p - 1) / 2} \equiv \sum_a^{\frac{p - 1}{2}} a^{2k} \pmod p
\]

Now note that given $a \ne b$ and $a, b < \frac{p - 1}{2}$ we have $a^{2k}$ and $b^{2k}$ are distinct. So as we can write $a^{2k} = (a^{k})^2$ it means that $a^{2k}$ is a quadratic residue. Since we show that it has to be distinct we get that $\sum_a^{\frac{p - 1}{2}} a^{2k}$ is just the sum of all the quadratic residues modulo $p$ which as we showed above is equivalent to $0$ modulo $p$. So we have $\sum_a^{(p - 1) / 2} \left ( \frac{a}{p}\right ) \equiv 0 \pmod p$. But $\left ( \frac{a}{p}\right )$ can only be 1, -1. And we're summing $(p - 1 )/ 2$ of these, so they are bounded below by $\frac{-(p - 1)}{2}$ and above by $\frac{p - 1}{2}$. So the only possible value they can take such that they are equivalent to 0 modulo $p$ is $0$. Hence we complete the proof.


\newpage 
\question Exercise Set 4.2, \#24. 
\begin{parts}
\part Let $p$ be a prime number with $p \geq 7$. Prove that at least one of $2$, $5$, and $10$ is a quadratic residue modulo $p$.

\textbf{Solution.}

We known  that $\left ( \frac{ab}{p}\right ) = \left ( \frac{a}{p}\right ) \left ( \frac{b}{p}\right )$. Now assume $2$ and $5$ are not quadratic residues then we have $\left ( \frac{2}{p}\right ) = -1$ and $\left ( \frac{5}{p}\right ) = -1$. This give us $\left ( \frac{10}{p}\right ) = \left ( \frac{2}{p}\right )\left ( \frac{5}{p}\right ) = -1 \cdot -1 = 1$. Hence if $2,5$ are not residues then $10$ must be a residue. Else one of $2,5$ has to be residues. So we have shown that at least one of the three have to be residues.

\part Could exactly two of $2$, $5$, and $10$ be quadratic residues modulo $p$ in part (a)? Why or why not?

No, exactly two cannot be residues modulo $p$. We have the following, 
\[
    \left ( \frac{10}{p}\right ) = \left ( \frac{2}{p}\right ) \left ( \frac{5}{p}\right )
\]

If we have exactly two residues then exactly two of these terms are $1$ and the other have to be $-1$. However, we see that if any two of the above are $1$, then the other one has to be 1 as well. Hence we cannot have exactly two residues from the list.

\part Let $p$ be a prime number with $p \geq 7$. Prove that there at least two consecutive quadratic residues modulo $p$.
[Hint: use part (a)]. 

\textbf{Solution.}
We need residues such that $a^2 - b^2 = 1$ or that $(a + b) (a - b) = 1$. We need it such that  $a + b =u$ and $a - b = \overline {u}$. So $2a = u + \overline{u}$ and $a = \frac{u + \overline{u}}{2}$ and $b = \frac{u - \overline{u}}{2}$ which would produce us two residues $a^2$ and $b^2$ such that $a^2 - b^2 = 1$. Now as one of $\{2,5,10\}$ are residues for a given $p$ we can choose u such that $u^2$ is in this set. This prevent any trivial solutions (such as $0, 1$). Hence, we found a construction for two consequtive residues for primes greater than 7.


\end{parts}

\newpage 
\question Exercise Set 4.2, \# 26.
Prove that there are infinitely many prime numbers expressible in the form $4n+1$ where $n$ is a positive integer.
\newline
[Hint: Assume, by way of contradiction, that there are only finitely many such prime numbers $p_1,\ldots,p_r$.
Consider the positive integer $4p_1^2 p_2^2 \cdots p_r^2+1$ and use Theorem 4.6].

\textbf{Solution.} Assume there are finitely many primes say $p_{1}, \dots, p_r$. Now consider $N = 4p_{1}^2 \dots p_r^2 + 1$. We see that $N \equiv 1 \pmod 4$. Now $N$ is clearly in the form $4k + 1$ for some $k$. So any prime divisor of this must be of form $p \equiv 1 \pmod 4$. Now we have that $N = (2p_{1} \dots p_r)^2 + 1$. Let $2p_{1} \dots p_r = M$ so $M^2 \equiv -1 \pmod N$ and as $p$ is a prime divisor of $N$ we have $M^2 \equiv -1  \pmod p$ This means that $-1$ is a quadratic residue modulo $p$ which means that $p \equiv 1 \pmod 4$. However, note that we cannot have $p \in  \{p_{1}, \dots, p_r\}$ as then we have $p \mid 4p_{1}^2 \dots p_r^2 + 1$, but as $p$ divides the first (if $p \in \{p_{1}, \dots, p_r\}$) then we also have $p \mid 1$ which is not possible.  Hence, this means we found a new prime divisor other than the finitely many ones we claimed which is of the form $4k + 1$ which is a contradiction that there are only finitely many $p_{1}, \dots, p_r$. Hence, our assumption must be wrong and there are infinitely many primes of the form $4k + 1$.


\newpage 
\question Exercise Set 4.3, \#34.\
Let $p$ and $q$ be odd prime numbers with $p=q+4a$ for some $a \in \mathbb{Z}$. Prove that 
\begin{equation*}
\Big(\frac{a}{p} \Big)=\Big(\frac{a}{q} \Big).
\end{equation*}

\textbf{Solution.} First note that $a$ is not necessarily a prime. So let us first factor it as follows, 
\[
    \left ( \frac{a}{p}\right ) = \left ( \frac{2}{p}\right )^{t} \left ( \frac{p_{1}}{p}\right )^{k_{1}} \dots \left ( \frac{p_n}{p}\right )^{k_n}
\]

Now it is enough to show that $\left ( \frac{2}{p}\right ) = \left ( \frac{2}{q}\right )$ and for any $p_k$ we have $\left ( \frac{p_k}{p}\right ) = \left ( \frac{p_k}{q}\right )$.
\vspace{1em}

For the first case i.e. $2$ we have it's equal to $(-1)^{\frac{p^2 - 1}{8}}$. Now see that $(p^2 - 1) / 8 = (q^2 + 16a^2 + 8aq - 1) / 8 = (q^2 - 1) / 8 + (2a^2 + aq)$ now if $t \ge 1$ this means that $2$ is a factor of $a$ and $a$ is even meaning $2a^2 + aq$ is even. So we get $(-1)^{(p^2 - 1)} = (-1)^{(q^2 - 1) / 8}$ or, 
\[
    \left ( \frac{2}{p}\right ) = \left ( \frac{2}{q}\right )
\]


\vspace{1em}

Now for the case for an arbitrary $p_k$. We see that $p \equiv q \pmod 4a$ which means that for any $p_k \mid a$ we also have $p \equiv q \pmod p_k$. Now in addition as $p \equiv q \pmod 4$ we have, 

\[
    \left ( \frac{p_k}{p}\right ) \left ( \frac{p}{p_k}\right ) = \left ( \frac{p_k}{q}\right )\left ( \frac{p_k}{a}\right )
\]

As $p,q$ will always share the same residue modulo $4$. Now note that when looking at  $\left ( \frac{p}{a}\right )$ and $\left ( \frac{q}{a}\right )$ we see that if $p$ is a residue then as $p \equiv p - 4a\equiv q$ is also a residue and similarly if $q$ is a residue then $q \equiv q + 4a \equiv p$ is also a residue. Hence either both are residues or both are non residues. In either case their value is the exactly same say $k$. So we have, 
\[
    \left ( \frac{p_k}{p}\right ) k = \left ( \frac{p_k}{q}\right ) k
\]

Or that $\left ( \frac{p_k}{p}\right ) = \left ( \frac{p_k}{q}\right )$. Hence we get for any $p_k$ that the legedre symbol is the same modulo $p, q$ so their product must also all be the same or in other words we get, 
\[
    \left ( \frac{a}{p}\right ) = \left ( \frac{a}{q}\right )
\]


\newpage 
\question Exercise Set 4.3, \#35.
Let $p$ be an odd prime number (with $p>3$ in parts (b) and (c)). 
Prove the following statements.
\begin{parts}
\item $\big( \frac{-2}{p} \big)=1$ if and only if $p \equiv 1,3 \pmod{8}$.

    \textbf{Solution.} We can write $\left ( \frac{-2}{p}\right ) = \left ( \frac{-1}{p}\right ) \left ( \frac{2}{p}\right )$. Now the product is equal to $1$ if and only if either both are equal to 1 or both are equal to $-1$. For $\left ( \frac{2}{p}\right ) = 1$ we have  $p \equiv 1, 7 \pmod 8$  and we have $\left ( \frac{-1}{p}\right ) = 1$ if $p  \equiv 1 \pmod 4$. So we have $4 \mid p - 1$ and $8 \mid p - 1$ or $8 \mid p - 7$ to satisfy both we have $8 \mid p - 1$ so $p \equiv 1 \pmod 8$.

    \vspace{1em}

    Now if both are equal to $-1$ we have $p \equiv 3 \pmod 4$ and either $p \equiv 3, 5 \pmod 8$. But $p \equiv 3 \pmod 4$ implies $p \equiv 3 \pmod 8$ so we have $p \equiv 3 \pmod 8$ as the only condition.

    \vspace{1em}
    
    So we have either $p \equiv 1, 3 \pmod 8$
    

\item $\big( \frac{3}{p} \big)=1$ if and only if $p \equiv \pm 1 \pmod{12}$.

\textbf{Solution.} We have $\left ( \frac{3}{p}\right ) \left ( \frac{p}{3}\right ) = 1$ if and only if $p \equiv 1 \pmod 4$ else we have $\left ( \frac{3}{p}\right ) \left ( \frac{p}{3}\right ) = -1$ if and only if $p \equiv 3 \pmod 4$. Now in the first case we need $\left ( \frac{p}{3}\right ) = 1$. We know that $p$ has to be a quadratic residue so we need $p \equiv 1 \pmod 3$. So we have both $p \equiv 1 \pmod 4$ and $p \equiv 1 \pmod 3$. Using CRT we have $x \equiv 4 \cdot 1 \cdot 1 + 3 \cdot -1 \cdot 1 \equiv 1 \pmod 12$.

\vspace{1em}

Now in the second case we have $\left ( \frac{p}{3}\right ) = -1$. For this we need $p$ to be a quadratic non-residue so we have $p \equiv 2 \pmod 3$ and $p \equiv 1 \pmod 4$. So this gives us $x \equiv 4 \cdot 1 \cdot 2 + 3 \cdot -1 \cdot 3 \equiv -1 \pmod 12 $

\vspace{1em}

Both the cases give us that $p \equiv \pm 1 \pmod {12}$


\item $\big( \frac{-3}{p} \big)=1$ if and only if $p \equiv 1 \pmod{6}$.

    \textbf{Solution.}

    We can write $\left ( \frac{-3}{p}\right ) =  \left ( \frac{-1}{p}\right )\left ( \frac{3}{p}\right ) = 1$

    Now this product to equal $1$ we need both to be either 1 or both to be -1. 
    \vspace{1em}
    
    Case 1: Both are 1. So we have $\left ( \frac{-1}{p}\right ) = 1$ which means that we have $p \equiv 1 \pmod 4$. And for $\left ( \frac{3}{p}\right ) = 1$ we see that we already have $p \equiv 1 \pmod 4$ which means that $\left ( \frac{3}{p}\right ) \left ( \frac{p}{3}\right ) = 1$ so $\left ( \frac{p}{3}\right ) = 1$ which is true only if $p \equiv 1 \pmod 3$.

    \vspace{1em}
    
    Case 2: We have both as $-1$. For $\left ( \frac{-1}{p}\right ) = -1$ we need $p \equiv 3 \pmod 4$. Now this case implies that we have $\left ( \frac{3}{p}\right ) \left ( \frac{p}{3}\right ) = -1$ so we have $\left ( \frac{p}{3}\right ) = 1$ or that $p \equiv 1 \pmod 3$.

    \vspace{1em}
    
    Now note that in both cases we have $p \equiv 1 \pmod 3$ and cases modulo $4$. But as $p$ is an odd prime we know it can be either $1, 3$ modulo $4$. SO the only condition we need to care about is $p \equiv 1 \pmod 3$. Now, modulo $6$, a prime can be either $1$ or $5$. Between these two the first one implies $1$ modulo $3$. Hence we have $p \equiv 1 \pmod 6$.
    
\end{parts}


\newpage 
Blank page:


\end{questions}

\end{document} 


 
%%%% don't delete the last line!
\end{document}
