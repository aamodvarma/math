\documentclass[12pt]{exam}
\usepackage{amsmath,amstext,amssymb,amsthm}   
\usepackage{enumerate}
%%%%%%%%%%%%%%%%%%%%%%%%%%%%%%
%%%%%%%%%%%%%%%%%%%%%%%%%%%%%%
%%%%%%%%%%%%%%%%%%%%%%%%%%%%%%
 


\begin{document}


 \begin{center}
 {\Large Homework 7, Math 4150 }
 \\
 \end{center}

\begin{questions}
\question Exercise Set 6.1, \#3
Apples and oranges at a grocery store cost 14 cents and 17 cents each respectively. A customer spends $\$2.90$ for apples and oranges. How many pieces of each fruit did the customer buy? 

\textbf{Solution.}
We can represent this as the following equations taking $\$2.9$ as $290$ cents to give us, 
\[
    14x + 17y = 290
\]
Now we need to find positive solutions to $x,y$. First note that we have $(14, 17) = 1$ and we have $17 \cdot 5 + 14 \cdot (-6) = 1$. Now we can expand this solution to get, 
\[
    14 \cdot ( -1740) + 17 \cdot 1450 = 290
\]

However now note that we can represent solutions for this in the following manner, 
\begin{align*}
    x = x_{0} + (b / d) n = (-1740) + (17) \cdot n\\
    y = y_{0} - (b / d) n = (1450) - (14) \cdot n
\end{align*}

As we need both as positive we need $1450 - 14n > 0$ and $-1750 + 17n > 0$. First condition gives us, 
\begin{align*}
    1450 - 14n &> 0\\
    1450  &> 14n\\
    103.5  &> n\\
    n \le 103
\end{align*}

and the second gives us, 
\begin{align*}
    -1750 + 17n &> 0\\
     17n &> 1750\\
     n &> 102.9\\
     n &\ge 103\\
\end{align*}

So we have $103 \le n \le 103$ which means $n = 103$ and pluging this back in we get the only solution, 
\[
    x = (-1740) + 17 \cdot 103 = 11 \text{ and } y = (1450) - 14 \cdot 103 = 8
\]

So the customers buys $11$ apples and $8$ oranges.
\newpage
Blank page:
 

\newpage
\question Exercise Set 6.2, \#11(a),(d), \#12(d).
Prove or disprove the following statements:
\begin{parts}
\item The Diophantine equation $3x^2-7y^2=2$ has no integral solutions.

    \textbf{Solution.} 
    Consider the above modulo $7$. So we have the following,
    \begin{align*}
        3x^2 - 2 &= 7y^2\\
        3x^2 \equiv 2 &\pmod 7\\
    \end{align*}

    But as $(3, 7) = 1$, $3$ has an inverse modulo $7$ and is $(-2)$ so we have,
    \begin{align*}
        x^2 \equiv 2 (-2) \pmod 7\\
        x^2 \equiv 3 \pmod 7\\
    \end{align*}

    However, now note the following. Both $3, 7$ are $\equiv 3 \pmod 4$ so using quadratic reciprocity we have,
    \begin{align*}
        \left ( \frac{3}{7}\right ) = -\left ( \frac{7}{3}\right ) = - \left ( \frac{1}{3}\right ) = -1
    \end{align*}

    Hence $3$ is not a quadratic residue modulo $7$ and there is no $x$ that satisfies the equation above. Hence, the Diophantine equation does not have integral solutions.


\item The Diophantine equation $x^3-5=7y^3$ has no integral solutions. 
    \textbf{Solution.} Take the equation modulo $7$ to get, 
    \begin{align*}
        x^{3} \equiv 5 \pmod 7
    \end{align*}

    Now we know that $3$ is a primitive root of $7$ so we can rewrite the above as follows, 
    \begin{align*}
        ind_3 x^{3} \equiv ind_3 5 \pmod 6
    \end{align*}

    But we have $3^{5} \equiv 5 \pmod 7$ so $ind_3 5 = 5$ which gives us, 

    \begin{align*}
        3 ind_3 x \equiv 5 \pmod 6
    \end{align*}
    Now note that this equation does not have a solution for $ind_3 x$ as $gcd(6, 3) = 3$ does not divide $5$. Hence, the equation is not solvable.

\item The Diophantine equation $x^2+2y^2+3=8z$ has no integral solutions.

    Take the equation modulo $8$ and we have,
    \begin{align*}
        x^2 + 2y^2 + 3 \equiv 0 \pmod 8\\
        x^2 + 2y^2 \equiv 5 \pmod 8
    \end{align*}

    Now note that modulo $8$ we have the quadratic residues for $8$ are $0^2, 1^2, \dots, 7^2 = \{0, 1, 4\}$ and hence the possible values of $2y^2$ are $0, 1 \cdot 2, 4 \cdot 2 = 0$. Hence we have $x^2 \equiv 0,1,4 \pmod 8$ and $2y^2 \equiv 0, 2 \pmod 8$ and note that all combinations cannot equal $5$. Hence no solution exists.
\end{parts}

\newpage
Blank page:

\newpage
\question Exercise Set 6.3, \#16(b).
Find all solutions in positive integers to the Diophantine equation $x^2+3y^2=z^2$.
\newline 
[HINT: Parallel the proof of Theorem~6.3 that concerns primitive Pythagorean triples.]

\textbf{Solution.} First let us consider the primitive solution such that we have $(x, y, z) = 1$ which means we need either $x$ or $y$ is even. Now rewrite the equation as follows,
\begin{align*}
    3y^2 &= z^2 - x^2\\
     y^2 &= \frac{1}{3}(z + x)(z - x)\\
\end{align*}

Now note that $2 \mid y$ and as $z,x$ are odd we have $2 \mid z + x$ and $2 \mid z - x$ this gives us, 
\begin{align*}
     \left (\frac{y}{2} \right)^2 &= \frac{1}{3}\left (\frac{z + x}{2}\right ) \left(\frac{z - x}{2}\right)\\
\end{align*}

Now note that we have $\left ( \frac{ z + x}{2}, \frac{z - x}{2}\right ) = 1$ as if they had a divisor $d$ then it would divide the sum and difference of the two or we would have $d \mid z$ and $d \mid x$ which means $x,y,z$ is not the primitive solution. Hence we have them as coprime. So this means we have two cases, either $3 \mid \frac{z + x}{2}$ or $3 \mid \frac{z - x}{2}$.

\vspace{1em}
Case 1: $3 \mid \frac{z + x}{2}$. Now note that we have $\left ( \frac{1}{3} \frac{z + x}{2}, \frac{z - x}{2}\right ) = 1$. And as the product of the two is a square and they are coprime each of them themselves must be perfect squares. So there is  some $m, n$ such that we have $m^2 = \frac{1}{3} \frac{z + x}{2}$ and $n^2 = \frac{z - x}{2}$. Now we have that we have $\frac{z + x}{2} = 3m^2$ and $\frac{z - x}{2} = n^2$. So their sum is $3m^2 + n^2 = z$ and differnece is $3m^2 - n^2 = x$ and note we also have $y^2 = 4m^2n^2$ or $y = 2mn$. Hence we have the solution, 
\[
    x = 3m^2 - n^2, y = 2mn, z = 3m^2 + n^2
\]


\vspace{1em}

Case 2: $3 \mid \frac{z - x}{2}$. Now note that we have $\left ( \frac{z + x}{2}, \frac{1}{3}\frac{z - x}{2}\right ) = 1$. And as the product of the two is a square and they are coprime each of them themselves must be perfect squares. So there is  some $m, n$ such that we have $m^2 =  \frac{z + x}{2}$ and $n^2 = \frac{1}{3} \frac{z - x}{2}$. Now we have that we have $\frac{z - x}{2} = 3n^2$ and $\frac{z + x}{2} = m^2$. So their sum is $3n^2 + m^2 = z$ and differnece is $3n^2 - m^2 = x$ and note we also have $y^2 = 4m^2n^2$ or $y = 2mn$. Hence we have the solution, 
\[
    x = 3n^2 - m^2, y = 2mn, z = 3n^2 + m^2
\]


Now for both cases we have solutions of the form, 
\[
    x = 3m^2 - n^2, y = 2mn, z = 3m^2 + n^2
\]


\vspace{1em}

Now note that we have a case where $x$ is even and $y, z$ are odd. But we see that in this case consider the equation modulo $4$ we get $x^2 + 3y^2 \equiv z^2 \pmod 4$. Now if $x$ is even then $x^2$ has to be $0 \pmod 4$ and if $y^2,z^2$ are odd they have to be $\equiv 1 \pmod 4$. But this means we have $x^2 \equiv 0 \pmod 4$ and $3y^2 \equiv 3 \pmod 4$ and $z^2 \equiv 1 \pmod 4$ and putting the three together we get $x^2 + 3y^2 - z^2 \equiv 0 + 3 - 1 \equiv 2 \pmod 4$ which gives us $0\equiv 2 \pmod 4$ which is an obvious contradiction. Hence, the only possibility is the only above where $x,z$ is odd and $y$  is even.







\newpage 
Blank page:


\newpage 
\question 
Exercise Set 6.4, \#21 
Prove that the Diophantine equation $x^4-y^4=z^2$ has no solutions in non-zero integers $x,y,z$.
\newline 
[HINT: The proof of Theorem 6.4 i.e. ``Fermat Descent" might give you some ideas on how to get started].

\textbf{Solution.} First we rewrite the equation as follows, 
\[
    x^{4} = z^2 + y^{4}
\]

Now if we consider $(x^2)^2 = z^2 + (y^2)^2$ and consider the solution where $x^2 = x_{0}, z = z_{0}, y^2 = y_{0}$ don't share a common GCD with either $y^2$ or $z$ being even. Now as all the solutions $x_{0},y_{0},z_{0}$ are integer solutions, by the well ordering principle we can choose the minimal solution such that $x_{0}$ has the minimal possible value. But we found another solution $x_{1}, y_{1}, z_{1}$ such that we have $x_{1} < x_{0}$ which contradicts our assumption that $x_{0}$ is the minimal. Hence our assumption must be wrong.

\vspace{1em}

\textbf{Case 1:} $y^2$ is odd. Using the Pythagorean triplet formulation we can find $m,n$ co prime with one being even such that, 
\begin{align*}
    x^2 = x_{0} &= m^2 + n^2 \\
    y^2 = y_{0} &= m^2 - n^2\\
    z  = z_{0} &= 2mn
\end{align*}
Now multiply the first two to get, 
\[
    (xy)^2 = (m^2 + n^2)(m^2 - n^2) = m^{4} - n^{4}
\]

Or we get $m^{4} = (xy)^2 + n^4$. Now note that this is a solution to our original equation and if we take $x_{1} = m, z_{1} = xy, y_{1} = n$ then note we have $x_{1} = m \le m^2 < m^2 + n^2 = x^2 = x_{0}$. So we get $x_{1} < x_{0}$. However, note that we choose $x_{0}$ such that $x_{0}$ is the minimal solution.

\vspace{1em}

\textbf{Case 2:} Now take $y^2$ as even. So find $m, n$ such that we have, 
\begin{align*}
    x^2 = x_{0} &= m^2 + n^2\\
    y^2 = y_{0} &= 2mn\\
    z = z_{0} &= m^2 - n^2
\end{align*}

Note that we have two cases here either  $m$ is even or $n$ is even. Now further consider $x^2 = m^2 + n^2$ and consider the primitive solution for this as well (so x, m, n are coprime and either $m, n$ are even). Now note we can use the pythagorean triplet formula once again to find $a, b$ for two cases one for $m$ is even and  $n$ is even. For $m$ is even we have
\begin{align*}
    x = a^2 + b^2, m = 2ab, n = a^2 - b^2
\end{align*}

and for $n$ is even we have
\begin{align*}
    x = a^2 + b^2, m = a^2 - b^2, n = 2ab
\end{align*}

However now note from above that we have $y^2 = 2mn$. If $m$ is even then pair $2$ with $m$ and we get $2m$ and $n$ are perfect squares. So we have $2m = (2c)^2$ or $m = 2c^2$ for some $c \in Z^{+}$. Now plugging this to the case where $m$ is even we have, $m = 2ab = 2c^2$ or $c^2 = ab$ which means both $a, b$ are squares as well. And if $n$ is even then we have $y^2 = m (2n)$ and $2n$ is a square so we can write $2n = (2c)^2$ or $n = 2c^2$ which gives us $n = 2ab = 2c^2$ which means $ab = c^2$ or $a, b$ are perfect squares.
\vspace{1em}

Note that in both cases we have $a, b$ are perfect squares and either $2m$ and $n$ or $2n$ and $m$ are perfect squares. In the first case note we have $n = a^2 - b^2$ or $n + b^2 = a^2$. But as $n, a, b$ are squares take $n = z_{1}^2, a = x_{1}^2, b = y_{1}^2$ and we have, 
\[
    x_{1}^{4} = z_{1}^{2} + y_{1}^{4}
\]
However, now note that we have $x_{1} < x_{1}^2 = a < a^2 + b^2 = x$. Hence, we found a solution set $x_{1}, y_{1}, z_{1}$ such that $x_{1} < x$ which is not possible as $x$ is the smallest value by assumption. Hence, we have contradiction.

\vspace{1em}

Now for the second case we have $2n$ or $m$ are perfect. And note we have $m = a^2 - b^2$ or $a^2 = m + b^2$. As the three are squares we can take $a = x_{1}^2, m = z_{1}^2, b = y_{1}^2$ and we get $x_{1}^{4} = z_{1}^{2} + y_{1}^{4}$ and note we have $x_{1} < x_{1}^2 = a < a^2 + b^2 = x$ and we get the same contradiction as above.


\vspace{1em}

Hence, we see in both cases  $y$ being odd or even we get a contradiction. Which means our assumption that there exists  a solution must be wrong and there does not exist a solution.

\newpage 
Blank page:

\newpage
\question 
Exercise Set 6.4, \#27 
Prove that the Diophantine equation $x^2+y^2=z^3$ has infinitely many integral solutions.
[HINT: Consider $x=n^3-3n$ and $y=3n^2-1$ for $n \in \mathbb{Z}$].


\textbf{Solution}. Consider we have $x = n^{3} - 3n$ and $y = 3n^2 - 1$. Now we have,
\begin{align*}
    x^2 + y^2 &= (n^{3} - 3n)^2 + (3n^2 -1)^{2}\\
              &= (n^{6} + 9n^2 - 6n^{4}) + (9n^{4} + 1 - 6n^2)\\
              &= n^{6} + 3n^2 + 3n^{4} + 1\\
              &= (n^2)^{3} + 1^{3} + 3 (1^2 \cdot n^2) + 3 (1 \cdot (n^2)^2)\\
              &= (n^{2} + 1)^{3}
\end{align*}

So take $z = n^2 + 1$ and we have $x^2 + y^2 = z^2$ and note that for each $n \in Z$ we if we take $x = n^{3} - 3n, y = 3n^2 - 1, z = n^2 + 1$ we have infinitely many  integral solutions to the equation $x^2 + y^2 + z^2$.

\end{questions}
\end{document} 


 
%%%% don't delete the last line!
\end{document}
