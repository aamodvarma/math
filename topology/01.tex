\chapter{Preliminaries}

\section{Axiom of Choice}

\textbf{Axiom of choice.} Given a collection $\mathcal{A}$ of disjoint nonempty sets, there exists a set $C$ consisting exactly one element from each element of $\mathcal{A}$.
\begin{note}
    This means that the set $C$ is such that $C$ is contained in the union of the elements of $\mathcal{A}$ and for each $A \in \mathcal{A}$ the set $C \cap A$ has a single element.
\end{note}

\begin{lemma}[Choice function]
    Given a collection $\mathcal{B}$ of nonempty sets (not necessarily disjoint), there exists a function, 
   \[
       c: \mathcal{B} \to \bigcup_{B \in \mathcal{B}}B
   \]

   such that $c(B)$ is an element of B, for each $B \in \mathcal{B}$
\end{lemma}
\begin{note}
    the function $c$ is called a \textit{choice function} for the collection $\mathcal{B}$
\end{note}
\begin{remark}
    The difference from the A.O.C is that in this lemma the sets need not be disjoint. So we have have $\mathcal{B}$ as the powerset of any set (excluding $\phi$ that is)
\end{remark}
\begin{proof}
    For a given $B$ in $\mathcal{B}$ consider, 
    \[
        B' = \{(B, x) : x \in B\}
    \]
    Now note that $B'$ is nonempty as $B$ is nonempty. Also note that for distinct $B_{1}, B_{2}$  we have $B_{1}', B_{2}'$ are distinct as well. Now consider, 
    \[
        \mathcal{C} = \{B': B \in \mathcal{B}\}
    \]

    Note that sets in $\mathcal{B}$ are disjoint. So we can use axiom of choice there is a set $c$ consisting of exactly one element from each elements of $\mathcal{C}$. Now note we have an assignment for each element of $\mathcal{C}$ to an ordered pair $(B, x)$ and hence $c$ is a rule for a function from the collection $\mathcal{B}$ to the set $\bigcup_{B \in \mathcal{B} }B$
\end{proof}

\section{Well Ordered sets}
\begin{definition}
    A set $A$ with an order relation $<$ is said to be well-ordered if every nonempty subset of $A$ has a smallest element.
\end{definition}
\begin{eg}
    $\{1, 2\} \times \Z_+$ in the dictionary ordering, so we have, 
    \[
        (1, 1), (1, 2), (1, 3), \dots, (2, 1), (2, 2), (2, 3)
    \]

    where, 
    \[
    (a, b) < (c, d) \text{ if } a < c \text{ or } a = c \text{ and } b < d
    \]

\end{eg}
\begin{eg}
    The set of integers is not well-ordered in the usual sense. For instance the subset containing the negative integers has no smallest element. Neither is the real numbers in $[0, 1]$ either as any open subset does not have a smallest element.
\end{eg}


\begin{theorem}
    Every nonempty finite ordered set has the order type of a section $\{1, \dots, n\}$ of $\Z_+$, so it is well ordered.
\end{theorem}

\begin{proof}
    First we claim that every finite ordered set has a largest element (by induction). Secondly we can construct an order preserving bisection of $A$ with $\{1, \dots, n\}$ for some $n$ by induction as well. Trivial for $n = 1$, assume for $n - 1$, now for case $n$ we isolate the largest element, use $n - 1$ case and define $f: A \to \{1, \dots, n\}$ by setting $f(x) = f'(x)$ and $f(b) = n$ where $f' :  A - \{b\} \to \{1, \dots, n - 1\}$.
\end{proof}

\begin{theorem}
    If $A$ is a set, there exists an order relation on $A$ that is a well-ordering
\end{theorem} 
