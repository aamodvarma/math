\chapter{Topological Spaces and Continuous Functions}

\section{Topological Spaces}


\begin{definition}
    A \emph{\textbf{topology}} on a set $X$ is a collection $\mathcal{T}$ of aubsets of $X$ having the following properties, 
    \begin{enumerate}[label=(\arabic*)]
        \item $\emptyset$ and $X$ are in $\mathcal{T}$ 
      \item  The union of the elements of any subcollection of $\mathcal{T}$ is in $\mathcal{T}$
      \item The intersection of the elements of any finite subcollection of $\mathcal{T}$ is in $\mathcal{T}$
    \end{enumerate}

    A set $X$ for which a topology $\mathcal{T}$ has been specified is called a topological space.
\end{definition}
\begin{note}
    A topological space is an ordered pair $(X, \mathcal{T})$
\end{note}
\begin{remark}
    Note that properties (2) and (3) are closely related to the properties of open sets in analysis. In fact this is how we define open sets more generally.
\end{remark}
\begin{definition}
If $X$ is a topological space with topology $\mathcal{T}$ then a subset $U$ of $X$ is an \textbf{open set} of X if $U$ belongs to the collection $\mathcal{T}$.
\end{definition}
\begin{remark}
    This is a definition for open sets in the context of a topology.
\end{remark}
\begin{note}
    Using this terminology we can say that a toplogical space is a set $X$ with a collection of subsets of $X$ called open sets such that $\emptyset$ and $X$ are both open and such that arbitrary unions and finite intersections of open sets are open.
\end{note} 


\begin{eg}
    If $X$ is a set, the collection of all subsets of $X$ is a topology on $X$ called the \textbf{\emph{discrete topology}}. The collection with just $\emptyset$ and $X$ is called the \textbf{\emph{trivial topology}}
\end{eg}


\begin{eg}
    If $X$ is a set and $\mathcal{T}_f$ is the collection of all subsets $U$ of $X$ such that $X -U$ is either finite or all of $X$, then $\mathcal{T}_f$ is a topology on $X$ called the $\textbf{\emph{finite complement topology}}$. First we have $X, \emptyset$ is in $\mathcal{T}_f$, since $X - X$ and $X - \emptyset$ are finite. Now given an indexed family of elements of $\mathcal{T}_f$ i.e. $\{U_{\alpha}\}$ that satisfy this property we need to show (2) and (3) hold.We see, 
    \[
        X - \bigcup U_{\alpha} = \bigcap (X - U_{\alpha})
    \]
    This is finite as each $X - U_{\alpha}$ is finite by assumption of $U$. Similarly given $U_{1}, \dots, U_n$ we see, 
    \[
        X - \bigcap^{n}  U_t = \bigcup^{n} (X - U_i)
    \]
    again the latter is finite as union of finite sets is finite.
\end{eg}

\begin{definition}
    Given $\mathcal{T}$ and $\mathcal{T}'$ are two topologies on $X$. If $\mathcal{T} \subset \mathcal{T}'$. Then $\mathcal{T}'$ is \textbf{\emph{finer}}  than $\mathcal{T}$. If $\mathcal{T}'$  properly contains $\mathcal{T}$ then it's \textbf{\emph{strictly finer}}. We also use the terminology \textbf{\emph{coarser}} and $\textbf{\emph{ striclty coarser}}$ the other way around. $\mathcal{T}$ is \textbf{\emph{comparable}} with $\mathcal{T}'$ if either $\mathcal{T} \subset \mathcal{T}'$ or $\mathcal{T}' \subset \mathcal{T}$
\end{definition}
\begin{note}
    This means that the "finest" topology is the discrete topology and the "coarsest" is the trivial.
\end{note} 

\section{Basis for a Topology}
\begin{definition}
    If $X$ is a set, a \textbf{\emph{basis}} for a topology on $X$ is a collection $\mathcal{B}$ of subsets of $X$ (called the basis elements) such that, 
    \begin{enumerate}
        \item For each $x \in X$, there is at least one basis element $B$ containing $x$.
        \item If $x$ belongs to $B_{1} \cap B_{2}$, then there exists a basis element $B_{3}$ containing $x$ such that $B_{3} \subset B_{1} \cap B_{2}$
    \end{enumerate}
\end{definition}
\begin{remark}
    If $\mathcal{B}$ satisfies the above two condition, then the we define the \textbf{\emph{topology \mathcal{T} generated by \mathcal{B}}} as follows: A subset $U$ of $X$ is said to be open in $X$ (in other words in $\mathcal{T}$) if for each $x \in U$, there is $B \in \mathcal{B}$ such that $x \in B$ and $B \subset U$. Each basis element is itself an element in $\mathcal{T}$.
\end{remark}

\begin{eg}
    If $\mathcal{B}$ is the collection of all circular regions in the plane (interiors of circles). Then $\mathcal{B}$ satisfies both conditions of a basis. In the topology generated, a subset $U$ of the plane is open if every $x$ in $U$ lies in some circular region contained in $U$ (note this implies that the border points cannot be included).
\end{eg}
\begin{eg}
    If $X$ is any set, the collection of all one-point subsets of $X$ is a basis for the discrete topology on $X$.
\end{eg}

\vspace{1em}

\qquad We need to check that the collection $\mathcal{T}$ generated by $\mathcal{B}$ is a topology. First if $U = \emptyset$, then clearly it satisfies our condition and hence is in $\mathcal{T}$ if $U = X$ then clearly for each $x \in U $ there is a basis element $B$ (by definition of a basis) containing $X$ and trivially $B \subset X$ and hence $X \in \mathcal{T}$.

\vspace{1em}

Now take two elements $U_{1}, U_{2}$ in $\mathcal{T}$. We need to show that $U_{1} \cap U_{2}\in\mathcal{T}$.  First choose $B_{1}, B_{2}$  containing $x$ such that $B_{1} \subset U_{1}$ and $B_{2} \subset U_{2}$. The second condition gives us $B_{3}$ containing $x$ such that $B_{3} \subset B_{1} \cap B_{2}$. Now $x \in B_{3}$ and $B_{3} \subset U_{1} \cap U_{2}$ so we have $U_{1} \cap U_{2} \in \mathcal{T}$.

\section{The Order Topology}
\section{The Product Topology}
\section{The Subspace Topology}
