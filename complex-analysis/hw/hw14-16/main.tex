\documentclass[a4paper]{report}
\usepackage[utf8]{inputenc}
\usepackage[T1]{fontenc}
\usepackage{textcomp}

\usepackage{url}

% \usepackage{hyperref}
% \hypersetup{
%     colorlinks,
%     linkcolor={black},
%     citecolor={black},
%     urlcolor={blue!80!black}
% }

\usepackage{graphicx}
\usepackage{float}
\usepackage[usenames,dvipsnames]{xcolor}

% \usepackage{cmbright}

\usepackage{amsmath, amsfonts, mathtools, amsthm, amssymb}
\usepackage{mathrsfs}
\usepackage{cancel}

\newcommand\N{\ensuremath{\mathbb{N}}}
\newcommand\R{\ensuremath{\mathbb{R}}}
\newcommand\F{\ensuremath{\mathscr{F}}}
\newcommand\Z{\ensuremath{\mathbb{Z}}}
\renewcommand\O{\ensuremath{\emptyset}}
\newcommand\Q{\ensuremath{\mathbb{Q}}}
\newcommand\C{\ensuremath{\mathbb{C}}}
\let\implies\Rightarrow
\let\impliedby\Leftarrow
\let\iff\Leftrightarrow
\let\epsilon\varepsilon

% horizontal rule
\newcommand\hr{
    \noindent\rule[0.5ex]{\linewidth}{0.5pt}
}

\usepackage{tikz}
\usepackage{tikz-cd}

% theorems
\usepackage{thmtools}
\usepackage[framemethod=TikZ]{mdframed}
\mdfsetup{skipabove=1em,skipbelow=0em, innertopmargin=5pt, innerbottommargin=6pt}

\theoremstyle{definition}

\makeatletter

\declaretheoremstyle[headfont=\bfseries\sffamily, bodyfont=\normalfont, mdframed={ nobreak } ]{thmgreenbox}
\declaretheoremstyle[headfont=\bfseries\sffamily, bodyfont=\normalfont, mdframed={ nobreak } ]{thmredbox}
\declaretheoremstyle[headfont=\bfseries\sffamily, bodyfont=\normalfont]{thmbluebox}
\declaretheoremstyle[headfont=\bfseries\sffamily, bodyfont=\normalfont]{thmblueline}
\declaretheoremstyle[headfont=\bfseries\sffamily, bodyfont=\normalfont, numbered=no, mdframed={ rightline=false, topline=false, bottomline=false, }, qed=\qedsymbol ]{thmproofbox}
\declaretheoremstyle[headfont=\bfseries\sffamily, bodyfont=\normalfont, numbered=no, mdframed={ nobreak, rightline=false, topline=false, bottomline=false } ]{thmexplanationbox}


\declaretheorem[numberwithin=chapter, style=thmgreenbox, name=Definition]{definition}
\declaretheorem[sibling=definition, style=thmredbox, name=Corollary]{corollary}
\declaretheorem[sibling=definition, style=thmredbox, name=Proposition]{prop}
\declaretheorem[sibling=definition, style=thmredbox, name=Theorem]{theorem}
\declaretheorem[sibling=definition, style=thmredbox, name=Lemma]{lemma}



\declaretheorem[numbered=no, style=thmexplanationbox, name=Proof]{explanation}
\declaretheorem[numbered=no, style=thmproofbox, name=Proof]{replacementproof}
\declaretheorem[style=thmbluebox,  numbered=no, name=Exercise]{ex}
\declaretheorem[style=thmbluebox,  numbered=no, name=Example]{eg}
\declaretheorem[style=thmblueline, numbered=no, name=Remark]{remark}
\declaretheorem[style=thmblueline, numbered=no, name=Note]{note}

\renewenvironment{proof}[1][\proofname]{\begin{replacementproof}}{\end{replacementproof}}

\AtEndEnvironment{eg}{\null\hfill$\diamond$}%

\newtheorem*{uovt}{UOVT}
\newtheorem*{notation}{Notation}
\newtheorem*{previouslyseen}{As previously seen}
\newtheorem*{problem}{Problem}
\newtheorem*{observe}{Observe}
\newtheorem*{property}{Property}
\newtheorem*{intuition}{Intuition}


\usepackage{etoolbox}
\AtEndEnvironment{vb}{\null\hfill$\diamond$}%
\AtEndEnvironment{intermezzo}{\null\hfill$\diamond$}%




% http://tex.stackexchange.com/questions/22119/how-can-i-change-the-spacing-before-theorems-with-amsthm
% \def\thm@space@setup{%
%   \thm@preskip=\parskip \thm@postskip=0pt
% }

\usepackage{xifthen}

\def\testdateparts#1{\dateparts#1\relax}
\def\dateparts#1 #2 #3 #4 #5\relax{
    \marginpar{\small\textsf{\mbox{#1 #2 #3 #5}}}
}

\def\@lesson{}%
\newcommand{\lesson}[3]{
    \ifthenelse{\isempty{#3}}{%
        \def\@lesson{Lecture #1}%
    }{%
        \def\@lesson{Lecture #1: #3}%
    }%
    \subsection*{\@lesson}
    \testdateparts{#2}
}

% fancy headers
\usepackage{fancyhdr}
\pagestyle{fancy}

% \fancyhead[LE,RO]{Gilles Castel}
\fancyhead[RO,LE]{\@lesson}
\fancyhead[RE,LO]{}
\fancyfoot[LE,RO]{\thepage}
\fancyfoot[C]{\leftmark}
\renewcommand{\headrulewidth}{0pt}

\makeatother

% figure support (https://castel.dev/post/lecture-notes-2)
\usepackage{import}
\usepackage{xifthen}
\pdfminorversion=7
\usepackage{pdfpages}
\usepackage{transparent}
\newcommand{\incfig}[1]{%
    \def\svgwidth{\columnwidth}
    \import{./figures/}{#1.pdf_tex}
}

% %http://tex.stackexchange.com/questions/76273/multiple-pdfs-with-page-group-included-in-a-single-page-warning
\pdfsuppresswarningpagegroup=1

\author{Aamod Varma}
\setlength{\parindent}{0pt}


\title{MATH 4320 HW14-16}
\author{Aamod Varma}
\begin{document}
\maketitle

\subsection*{Problem 2}
Let, 
$$ f(z) = \frac{z^{-1/ 2}}{z^2 + 1} = \frac{e^{(-1 /2) \log z}}{z^2 + 1} $$ 

Now consider the indented contour and we have, 
$$ \int_{L_1} f(x) dx +  \int_{L_2} f(x) dx + \int_{C_\rho} f(z) dz + \int_{C_R}f(z) dz = \int_C f(z)dz$$ 
Rearranging we have, 

$$ \int_{L_1} f(x) dx +  \int_{L_2} f(x) dx =  \int_C f(z)dz- \int_{C_\rho} f(z) dz - \int_{C_R}f(z) dz $$ 

First we calculate $\int_C f(z) dz$. Within our contour the only singularity is when  $z = i$. So the integral is equal to,  
$$ \int_C f(z) = 2\pi i Res_{z = i} f(z) $$ 

Now $$Res_{z = i} f(z) = \frac{e^{-1/ 2\log(i)}}{2i} = \frac{e^{-1 /2 (i\pi / 2) }}{2i}$$
$$ = \frac{e^{-i\pi / 4}}{2i}  = (\frac{1}{\sqrt{2}} - \frac{i}{\sqrt{2}})\frac{1}{2i} $$ 

This gives us, 
$$ \int_C f(z) = 2\pi i \cdot  (\frac{1}{\sqrt{2}} - \frac{i}{\sqrt{2}})\frac{1}{2i} $$ 

$$ =   \frac{\pi}{\sqrt{2}} - \frac{\pi i}{\sqrt{2}} = \frac{\pi}{\sqrt{2}} (1 - i) $$ 


Now let us look at, 
$$ \int_{L_1} f(x) dx + \int_{L_2} f(x) dx $$ 

We can write these integrals using the parameterization $z = re^{i 0}, \rho \le r \le R$ and $z = re^{i \pi}, \rho \le r \le R$ 
\begin{align*}
&= \int_{\rho}^{R} \frac{e^{-1 / 2\log(r)}}{r^2 + 1}  + \int_{\rho}^{R} \frac{e^{-1 / 2\log(re^{\pi})}}{r^2 + 1} \\
&= \int_{\rho}^{R} \frac{e^{-1 / 2\log(r)}}{r^2 + 1}  + \int_{\rho}^{R} \frac{e^{-1 /2 \log(r)} e^{-i\pi/2 }}{{r^2 + 1} }\\
&= (1 + e^{-i\pi /2})\int_{\rho}^{R} \frac{e^{-1 / 2\log(r)}}{r^2 + 1}  \\
&= (1 - i)\int_{\rho}^{R} \frac{e^{-1 / 2\log(r)}}{r^2 + 1}  \\
\end{align*}

So we have, 
$$ &= (1 - i)\int_{\rho}^{R} \frac{e^{-1 / 2\log(r)}}{r^2 + 1} = \frac{\pi}{\sqrt{2}}(1 - i) - \int_{C_\rho} f(z) dz - \int_{C_R} f(z) dz$$


Now we can bound $f(z)$ as follows because we can say $|\sqrt{z} | \ge | \sqrt{\rho}|$ and  $|z^2 + 1| \ge |1 - \rho^2|$ so, 
$$ f(z) \le \frac{1}{\sqrt{\rho} (1 - \rho^2)} $$ 

Or, 
$$ \int_{C_\rho} f(z)\le \frac{2\pi \rho}{\sqrt{\rho} (1 - \rho^2)} = \frac{2\pi \sqrt{\rho}}{1 - \rho^2} $$ 

Now as $\rho \rightarrow 0$ we have $1 - \rho^2$ goes to $1$ while the numerator vanishes to zero. Hence we can say that, 
$$ \lim_{\rho \to 0} \int_{C_\rho} f(z) = 0 $$ 

Similarly we have, 
$$ \int_{C_R} f(z) \le \frac{2 \pi \sqrt{R}}{R^2 - 1} $$ 

As the power of the denominator in terms of $R$ is higher we have, 
$$ \lim_{R \to \infty} f(z) dz = 0 $$ 

This gives us, 
$$ (1 - i)\int_{0}^{\infty} \frac{e^{-1 / 2\log(r)}}{r^2 + 1} = \frac{\pi}{\sqrt{2}}(1 - i) $$
$$ \int_{0}^{\infty} \frac{e^{-1 / 2\log(r)}}{r^2 + 1} = \frac{\pi}{\sqrt{2}} $$



\subsection{Problem 1}
We have, 
$$ \int_0^{2\pi} \frac{d\theta}{5 + 4 \sin \theta} $$ 

Let $z = e^{i\theta}$ which gives  us, $\frac{dz}{iz} = d\theta$ and $\sin \theta = \frac{z - z^{-1}}{2i}$. So we can write, 
\begin{align*}
\int_C \frac{dz}{iz (5 + 4\frac{z - z^{-1}}{2i})}\\
&= \int_C \frac{dz}{z (5i + 2 (z - z^{-1}))}\\
&= \int_C \frac{dz}{5zi + 2 z^2 - 2}\\
&= \int_C \frac{dz}{2(z + \frac{i}{2}) (z + 2i)}\\
\end{align*}

Taking our unit circle we can see that $z = -\frac{i}{2}$ is inside our contour hence the integral evaluates to, 
$$ 2\pi i \frac{1}{2 (2i - i / 2)} $$ 
$$ = \pi i \frac{2}{3i} $$ 
$$ = \frac{2\pi}{3}$$


\subsection*{Problem 3}
We have, 
$$ \int_0^{2\pi} \frac{\cos^2 3\theta d\theta}{5 - 4\cos 2\theta} $$ 


We can write $\cos 3\theta = \frac{e^{3i\theta} + e^{-3i\theta}}{2}$. If we take $e^{i\theta} = z$ then we get, 
\begin{align*}
    \cos(3\theta) &= \frac{z^{3} + z^{-3}}{2}\\
    \cos^2(3\theta) &= \frac{(z^{3} + z^{-3})}{2}^2\\
                    &= \frac{z^{6} + z^{-6} + 1 }{2}\\
                    &= \frac{1 + \cos(6\theta)}{2}
\end{align*}

We know that $\cos(6\theta) = Re (e^{6i\theta})$. So we can rewrite our integral as, 
$$ \frac{1}{2}\int_0^{2\pi} \frac{1}{5 - 4 \cos 2\theta} d\theta + \frac{1}{2}Re \int_0^{2\pi} \frac{e^{6i\theta}}{5 - 4 \cos 2\theta} d\theta $$ 

Taking $e^{i\theta} = z$ we can write the first integral as,
\begin{align*}
    \int_{C } \frac{i}{2}      \frac{z}{(z^2 - 2)(2z^2 - 1)} dz\\
\end{align*}
Using residue the value of the integral is, 
$$ 2\pi i Res_{z = \frac{1}{\sqrt{2}}, \frac{-1}{\sqrt{2}}} \frac{iz}{2(z^2 - 2)(2z^2 - 1)} $$ 
which is, 
$$ = \frac{\pi}{3}$$

Similarly we have, 
$$ Re( 2\pi i Res_{z = \frac{1}{\sqrt{2}},-\frac{1}{\sqrt{2}}}  \frac{i}{2} \frac{z^{7}}{(z^2 - 2)(2z^2 - 1)} )$$ 

Which will equal to, 
$$ \frac{\pi}{24} $$ 

So our integral is, 
\begin{align*}
&= \frac{\pi}{3} + \frac{\pi}{24}\\
&= \frac{9\pi}{24}\\
&= \frac{3\pi}{8}
\end{align*}


\subsection*{Problem 1}
(a). We have $f(z) = z^2$ which has two zeroes inside the unit circle and no poles, hence we have, 
$$ 2\pi (Z - P) = 2\pi (2 - 0) = 4\pi $$ 

(b). We have $f(z) = \frac{1}{z^2}$. Inside the unit circle we have no zeroes but two poles because $z^2$ has two zeroes and its in the denominator. Hence we have, 
$$ 2\pi (Z - P) = 2\pi (0 - 2) = -4\pi $$ 

(c). We have $f(z) = (2z - 1)^{7} / z^{3}$. The numerator is a polynomial of degree 7 hence it has 7 zeroes. And the denominator is of degree 3 hence it has 3 zeroes corresponding to 3 poles of the function. Hence we have, 
$$ 2\pi (Z - P) = 2\pi (7 - 3) = 2\pi (4) = 8\pi $$ 

\subsection*{Problem 8}
We have, 
$$ 2z^{5} - 6z^2 + z + 1 = 0$$

First within the unit circle if we take $f(z) = -6z^2 + z + 1$ and $g(z) = 2z^{5}$. Then we have, 
$$ |g(z)| \le |f(z)| $$ as 
$$ |6z^2| + |z| + |1| \ge |2z^{5}| \text{ for $z < 1$ } $$ 

Now considering $z < 2 $, we take $f(z) = 2z^{5} + 1$ and $g(z) = -6z^2 + z$. So we have on $|z| = 2$
\begin{align*}
    -6z^2 + z + 1 &\le |6 \cdor 4 |  + |2| + |1|\\
              &= 24 + 2 + 1\\
              &\le 32\\
              &= 2^{5}\\
              &= z^{5}\\
              &\le z^{5}
\end{align*}

So we $z^{5}$ denominating over the circle, hence the sum will have $5$ zeroes in the circle $z < 2$. So in the annulus we have $5 - 2$ zeroes which is $3$.


\subsection*{Problem 3}
Our transformation should rotate by $\frac{\pi}{2}$ anti-clockwise and shift it to the right by a unit of 1.

Firstly rotating by $\frac{\pi}{2}$ is equivalent to multiplying by $e^{i \frac{\pi}{2}}$. So we have $z e^{i\frac{\pi}{2}} = zi$

And shifting by a unit of one in the positive x axis is adding 1 so we have $f(z) = zi + 1$


 \subsection*{Problem 5}
We have the domain $x > 1$ and $y > 0$. First we know that any line $x = c$ is transformed into the circle, 
$$ (u - \frac{1}{2c})^2 + v^2 = (\frac{1}{2c})^2 $$ so for any line $x > 1$ we have, 
$$ (u - \frac{1}{2})^2 + v^2 < \frac{1}{4} $$ 

And because we have $y > 0$ we have $C > 0$ which implies $-C < 0$ or that $v < 0$

 \subsection*{Problem 11}
 We can look at it as $w = \frac{1}{z} = z^{-1} = e^{-i\theta}$.
 So as $\theta$ increases we have $arg(w)$ decreasing hence the orientation is opposite or negative.

 \subsection*{Problem 5}
 First we show the boundary of the strip is mapped in a one to one manner onto the real axis in the w plane.

 We have, 
 $$ u = \sin(x) \cosh(y) \qquad v = \cos(x)\sinh(y) $$ 

 So we have our first boundary as $x = \frac{-\pi}{2}$ which gives us, 
 $$ u = -\cosh(y), v = 0$$ restricting $y $ to be non-negative we have a point $(\frac{-\pi}{2},y)$ mapped to $(-\cosh(y), 0)$. This means that as $y$ increase along the boundary we have the image moving towards the left from $D'$ towards $E'$. Now points $(x,0)$ on the horizontal segment will have an image, 
 $$ u = \sin(x), v = 0 $$ 

 But as $-\frac{\pi}{2} \le x \le 0$ we have $\sin(x)$ goes from $-1$ to $0$. Which would be  $D'$ to $C'$.

 Now each point on the interior of our domain will lie on the vertical lines $x = c_1,y >0$. So we have, 
 $$ u = \sin(c_1)\cosh(y), v = \cos(c_1)\sinh(y) $$ 

 Now because $y$ is always positive  but $x$ is always negative we have $u$ is negative and $v$ is positive. In other words we have, 
 $$ \frac{u^2}{\cosh^2(y)}  + \frac{v^2}{\sinh^2(y)} = 0$$ 

 Such that it as $y$ increases it moves to the left of the hyperbola.


\subsection*{Problem 4}
We know that the function $f(z) = \sin(z)$ maps the semi infinite strip onto the first quadrant. This can be seen as we can write, 
$$ u = \sin(x) \cosh(y), v = \cos(x)\sinh(y) $$

And because $0 \le x \le \frac{\pi}{2}$ and $y \ge 0$. Both  $u$ and $v$ is always greater than zero.

Now this means that any $Z = \sin(z)$ for any $z$ in our domain is such that $arg(Z) \le \frac{\pi}{2}$.

Now considering the function $F_0(z) = z^{1 /2}$ we know that this will map any $z $ to a point where $arg(z) = 2 arg(f(z))$. Hence we know that $arg(F_0(Z)) \le \frac{\pi}{4}$ which is the first octane in the image.


\subsection*{Problem 3}
We know that under the mapping $w = 1 /z$ a domain in the $x,y$ plane in the form, 
$$ A(x^2 + y^2) + Bx + Cy + D = 0 $$  will be mapped to, 
$$ D(u^2 + v^2) + Bu - Cv + A = 0 $$ 

So our line $y = x - 1$ is such that $A = 0, B = 1, C = -1, D = -1$ which is mapped to, 
$$ -1(u^2 + v^2) + u + v = 0 $$ or, 
$$ u^2 - u + v^2 - v = 0 $$ 

Similarly we have $y = 0$ which is when $A = 0, B = 0, C = 1, D = 0$ which gets mapped to $v = 0$

Now at  $z_0 = 1$ which is at the point $(1,0)$, the angle is $\frac{\pi}{4}$. 

Now in the image we have our first line mapping to  $(u,v) = (1,0)$ in the circle and the second line mapping to $(0,0)$ in the line $v = 0$. The angle between these two points are  $\frac{\pi}{4}$ as well.

Hence the angle is preserved and we verified the conformality of the mapping at $z = 1$.

            
\end{document}

