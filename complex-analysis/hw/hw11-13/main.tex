\documentclass[a4paper]{report}
\input{preamble.tex}
\title{MATH 4320 HW11-13}
\author{Aamod Varma}
\begin{document}
\maketitle

\subsection*{Problem 1}
(a). We have the expansion of $\frac{1}{z + z^2}$ as, 
$$ \sum_{n=0}^{\infty} (-1)^{n} z^{n - 1} $$ 
$$ = \frac{1}{z} + \sum_{n=0}^{\infty} (-1)^{n+1} z^{n} $$ 

So we have our residue as $1$ 

(b). We have expansion of $z \cos(\frac{1}{z})$ as, 
$$ z \sum_{n=0}^{\infty} (-1)^{n}\frac{(\frac{1}{z})^{2n}}{2n!} = \sum_{n=0}^{\infty} (-1)^{n} \frac{1}{z^{2n - 1} 2n!} $$ 

We have power of $z$ as 1 when $n = 1$ so coefficient is $\frac{-1^{1}}{2!} = -\frac{1}{2}$ 

So the residue is $-\frac{1}{2}$ 

(c). We have $\frac{z - \sin z}{z}$. We can first rewrite this as $1 - \frac{\sin z}{z}$. The expansion of which is, 
$$ 1 - \sum_{n=0}^{\infty} (-1)^{n} \frac{z^{2n}}{(2n + 1)!} $$ 

We see for all values of $n$ the highest power of $z$ is greater than equal to $0$. Hence the residue term is $0$.

(d). We have $\frac{\cot z}{z^{4}}$
We can write this as, 
$$ \frac{\cos z}{z^{4} \sin z} $$ 
\begin{align*}
    &= \frac{1}{z^{4}} \frac{1 - \frac{z^{2}}{2!} + \dots}{z - \frac{z^{3}}{3!} + \dots}\\
    &=\frac{1}{z^{5}} \frac{1 - \frac{z^{2}}{2!} + \dots}{1 - \frac{z^{2}}{3!} + \dots}
\end{align*}
If we have $w = \frac{z^{2}}{3!}$ 
$$
    =\frac{1}{z^{5}} \frac{1 - \frac{z^{2}}{2!} + \dots}{1 - w}
    $$


    And as $|w| < 1$ we have, 
    $$=\frac{1}{z^{5}} (1 - \frac{z^{2}}{2!} + \dots )(1 + w + w^2 +\dots)$$


    So the coefficient of the $\frac{1}{z}$ term would be $\frac{1}{(3!)^2} - \frac{1}{5!} + \frac{1}{4!} - \frac{1}{2!3!}$ 
    $$ = -\frac{1}{45} $$ 



    (e). We have $\frac{\sinh z}{z^{4}(1 - z^2)}$ 

    We can expand $\sinh z$ as, 
    $$ \sum_{n=0}^{\infty} \frac{z^{2n + 1}}{(2n + 1)!} = (z + \frac{z^{3}}{3!} + \dots) $$ 

    And we can expand $\frac{1}{z^{4}(1 - z^2)}$ as, 
$$ \frac{1}{z^{4}} \sum_{n=0}^{\infty} z^{2n}=  \sum_{n=0}^{\infty} z^{2n - 4} $$ 
$$  = \frac{1}{z^{4}} + \frac{1}{z^2} + \dots$$ 

The product of both will be, 
$$ \bigg ( z + \frac{z^{3}}{3!} + \dots\bigg )  \bigg ( \frac{1}{z^{4}} + \frac{1}{z^2} + \dots\bigg )$$ 

So the coefficient of $\frac{1}{z}$ in this product is, 
$$ 1 + \frac{1}{3!} = 1 + \frac{1}{6} = \frac{7}{6} $$ 


\subsection*{Problem 3}
We have, 
$$ \int_C \frac{4z - 5}{z (z - 1)} $$ 

Now using residue at infinity we know this integral is, 
$$ = 2\pi i Res_{z = 0}(\frac{1}{z^2}f(\frac{1}{z})) $$ 

\begin{align*}
    Res_{z = 0} \bigg ( \frac{1}{z^2}f\bigg(\frac{1}{z}\bigg)\bigg)&= \frac{1}{z^2} \frac{ \frac{4}{z} - 5}{\frac{1}{z}(\frac{1}{z}-1)}\\
                                                                   &= \frac{4}{z(1-z)} - \frac{5}{1-z}\\
                                                                   &= 4 \sum_{n=0}^{\infty} z^{n-1} - 5 \sum_{n=0}^{\infty} z^{n}\\
\end{align*}

So coefficient of the $\frac{1}{z}$ term is when $n = 0$ where we have, 
$$ \frac{4}{z} $$ which is $4$.

Hence our integral evaluates to $2\pi i \cdot  4 = 8\pi i$


\subsection*{Problem 6}
We have $f$ is analytic throughout the finite plane except for a finite number of singular points. So consider a contour $C$ that includes all our finite number of singular points. So we know the integral around this contour is,
$$ \frac{1}{2\pi i} \int_C f(z) dz=  Res_{z=z_1} + \dots + Res_{z = z_n} $$ 

Now because there are no singular points outside this contour we also know that, 
$$ \frac{1}{2\pi i}\int_C f(z) = - Res_{z=\infty} $$ 


So Putting the two together we have, 

$$  Res_{z=z_1} + \dots + Res_{z = z_n}+ Res_{z=\infty}  = 0$$ 



\subsection*{Problem 2}
(a).  We have, 
$$ \frac{1 - \cosh z}{z^{3}} $$  whose series expansion is, 
$$ \frac{1}{z^{3}} - \sum_{n=0}^{\infty} \frac{z^{2n - 3}}{(2n)!} $$ 
$$ =\frac{1}{z^{3}} - \frac{1}{z^{3}} - \sum_{n=1}^{\infty} \frac{z^{2n - 3}}{(2n)!} $$ 
$$ = -\sum_{n=0}^{\infty} \frac{z^{2n - 1}}{(2n + 2)!} $$ 
$$ = -\frac{1}{2z} - \sum_{n=0}^{\infty} \frac{z^{2n + 1}}{(2n + 4)!} $$ 


So we see we have a pole of order 1 and residue is $B = -\frac{1}{2}$ 

(b). We have,  
$$ \frac{1 - e^{2z}}{z^{4}} $$ 
Expansion is, 
\begin{align*}
 &=\frac{1}{z^{4}} - \frac{1}{z^{4}} \sum_{n=0}^{\infty} 2^{n}z^{n} \frac{1}{n!}\\
 &= \frac{1}{z^{4}} - \sum_{n=0}^{\infty} 2^{n}z^{n - 4}\frac{1}{n!}\\
 &= \frac{1}{z^{4}} - \frac{1}{z^{4}} - \sum_{n=1}^{\infty} 2^{n}z^{n - 4}\frac{1}{n!}\\
 &= -\sum_{n=0}^{\infty} 2^{n + 1}z^{n - 3}\frac{1}{(n+1)!}
\end{align*}

So our pole is of order 3 as the highest power of $\frac{1}{z}$ is $3$ when $n = 0$. And  coefficient of $\frac{1}{z}$ is when $n = 2$ where we have, 
$$ -2^{3}\frac{1}{z}\frac{1}{3!} = -\frac{4}{3} \frac{1}{z} $$ 

So residue is $-\frac{4}{3}$ 


(c). We have $$\frac{e^{2z}}{(z - 1)^2}$$

Expansion around $z = 1$ is, 
\begin{align*}
    e^{2(z-1 + 1} \frac{1}{(z-1)^2} &= e^2e^{2(z - 1)} \frac{1}{(z -1)^2}\\
                                    &= e^2 \sum_{n=0}^{\infty} 2^{n}(z-1)^{n-2} \frac{1}{n!}
\end{align*}

So when $n = 0$ we have highest power of $\frac{1}{z}$ as $m= 2$. Hence pole is of order $2$. And when $n = 1$ we have coefficient as  $2e^2$ which is our residue.

\subsection*{Problem 2}

(a). We have an isolated singular point at $ = -1$. So our residue is 
$$-1^{\frac{1}{4}} = e^{\frac{\pi}{4}} = \cos(\frac{\pi}{4}) + i \sin(\frac{\pi}{4})$$ 
$$ = \frac{1 + i}{\sqrt{2}} $$ 

(b). We have, 
$$ \frac{Log(z)}{(z^2 + 1)^2} $$ 

which can be written as, 
$$ \frac{Log(z)}{(z + i)^2(z - i)^2} $$ 

If  $\phi(z) = \frac{Log(z)}{(z + i)^2}$ we have, 
$$  \frac{\phi(z)}{(z - i)^2}$$ 

As $z =i$ is an isolated singular point we have, 
$$ \frac{\phi^{2 - 1}(z)}{(2 - 1)!} $$ 

$$\phi'(z) = \frac{(z + i)^2 \frac{1}{z} - Log(z) 2(z + i)}{(z + i)^4}$$

And $$\phi'(i) =( -\frac{4}{i} - i\frac{\pi}{2}4i  ) \frac{1}{16}$$

$$ =( 4i + 2\pi) \frac{1}{16} $$ 
$$ = \frac{2i + \pi}{8} $$ 


(c). We have $$\frac{z^{\frac{1}{2}}}{(z^2 + 1)^2}$$


We can write this as, 
$$ \frac{z^{\frac{1}{2}}}{(z + i)^2(z - i)^2} $$ 

We have, 
$$ \phi(z) = \frac{z^{1 /2}}{(z + i)^2} $$ 

So $$\phi'(z) = \frac{(z + i)^2 \frac{z^{-1/ 2}}{2} - z^{1 /2} 2(z + i)}{(z + i)^{4}} $$



So 
\begin{align*}
    \phi'(i) &= (-\frac{2}{i^{1/ 2}} - i^{1 /2}4i) ( \frac{1}{16})\\
             &= (i2i^{1 /2} - i^{1 /2}4i) ( \frac{1}{16})\\
             &= (i^{3 /2} - i^{3 /2}2) ( \frac{1}{8})\\
             &= (i^{3 /2} - i^{3 /2}2) ( \frac{1}{8})\\
             &= -i^{3 /2} ( \frac{1}{8})\\
             &= (\frac{1}{\sqrt{2}}  - \frac{i}{\sqrt{2}})( \frac{1}{8})\\
             &= \frac{1 - i}{8\sqrt{2}}
\end{align*}

\subsection*{Problem 4}
We need to find, 
$$ \int_C \frac{3z^{3} + 2}{(z - 1)(z^2 + 9)} \: dz $$ 

(a). Inside our contour $|z - 2| \le 2$ we have only one singularity. Hence the integral will evaluate  $2\pi i Res_{z = 1} f(z)$. 

So we have $\phi(z) = \frac{3^{3} + 2}{(z^2 + 9)}$ 

Our pole is of order $1$ so we have, 
$$ \phi(1) = \frac{5}{10} $$ 

And our integral is $2\pi i \frac{1}{2} =  \pi i$


(b). Inside our contour $|z|= 4$ we have three singularities hence the integral is sum of all three residues at that point. So we have, 
$$2\pi i( Res_{1}f(z) +  Res_{3i}f(z) +  Res_{-3i}f(z)  )$$ 

$$ Res_1f(z) = \frac{1}{2} $$ 

$$ Res_{3i}f(z) =  \frac{3(3i)^{3} + 2}{(3i - 1)(6i)}$$ 

$$ Res_{-3i}f(z) =  \frac{-3(3i)^{3} + 2}{(3i + 1)(6i)}$$ 

The sum times $2\pi i$ evaluates to, 
$$ 6\pi i $$ 
$$  $$ 

\subsection*{Problem 7}
(a). 
$$ f(z) = \frac{(3z + 2)^2}{z(z - 1)(2z + 5)} $$ 

So  using residue at infinity it is enough to find, 
$$ 2\pi i Res_{z = 0} \frac{(\frac{3}{z} + 2)^2}{\frac{1}{z} (\frac{1}{z} - 1)(2\frac{1}{z}+5)} $$ 

Which is, 

$$ 2\pi i Res_{z = 0} \frac{(3 + 2z)^2}{z(1 -z)(2+5z)}$$ 


$$ = 2\pi i (\frac{9}{2}) = 9\pi i $$ 


(b). 
We have,  
$$ f(z) = \frac{z^{3}e^{1/ z}}{1 + z^{3}} $$ 

Our integral would be equivalent to, 
$$ 2\pi i Res_{z = 0} (\frac{1}{z^2}f(\frac{1}{z})) $$ 
$$&= 2\pi i Res_{z = 0}(\frac{e^{z}}{z^2(z^{3} + 1)})$$

We have $\phi(z) = \frac{e^{z}}{z^{3} + 1}$ and $m = 1$. So we have, 
$$ \phi'(0) = 1 $$ 

Hence our integral is $2\pi i \cdot 1 = 2\pi i$


\subsection*{Problem 3}
(a). We have $Res_{z =  \frac{\pi i}{2}}  \frac{\sinh z}{z^2 \cosh z}$


Using theorem we have $Res_{z =  \frac{\pi i}{2}}  \frac{\sinh z}{z^2 \cosh z} = \frac{\sinh \pi i / 2}{(\pi i / 2)^2 (-\sinh (\pi i /2) )+ \cosh (\pi i / 2) 2(\pi i / 2)}$



$$ = -\frac{4}{\pi^2} $$ 


(b). We have $Res_{z = \pi i} \frac{e^{zt}}{\sinh z} + Res_{z = -\pi i} \frac{e^{zt}}{\sinh z}$ 

Using theorem,
$$ Res_{z = \pi i} \frac{e^{zt}}{\sinh z} = \frac{e^{(\pi i)t}}{\cosh (\pi i)}  = \frac{\cos(\pi t) + i \sin(\pi t)}{-1}$$ 
$$ Res_{z = -\pi i} \frac{e^{zt}}{\sinh z} = \frac{e^{(-\pi i)t}}{\cosh (-\pi i)}  = \frac{\cos(-\pi t) + i \sin(-\pi t)}{-1}$$ 

So their sum is, 
$$ -2 \cos(\pi t) $$ 
\subsection*{Problem 6}

First we need to find, 
$$ \int_{C_N} \frac{dz}{z^2 \sin z} $$ 

We know this integral would be, 
$$ 2\pi i \sum_{n=1}^{K} Res_{z = z_n} f(z) $$ 

Where $z_n$ are the singularities of our function $f$ within our domain. So we have $f(z) = \frac{1}{z^2 \sin z}$ 

So our singularities are  when $z = 0, z = \pm \pi , z = \pm 3\pi ,\dots$. So our integral would be,  

Or in other words  we have $z = 0, z = (n)\pi$ for $n \in \N$
$$ 2\pi i \sum_{n=1}^{N} Res_{z = z_n} f(z) $$ 


If $p(z) = 1$ and $q(z) = z^2 \sin z$ such that $f(z) = p(z) / q(z)$
We know if $q(z) \ne 0$ residue would be, 
$$ \frac{p(z_n)}{q'(z_n)} = \frac{1}{(z_n^2 \cos z_n +2z_n \sin z_n )} $$ 

First if $z_n = 0$ we can find residue using the Taylor expansion,

We have, 
\begin{align*}
    \frac{1}{z^2 \sin z} &= \frac{1}{z^2 (z - \frac{z^{3}}{3!} + \frac{z^{5}}{5!} - \dots)}\\
                         &= \frac{1}{z^{3}(1 - \frac{z^2}{3!} + \frac{z^{4}}{5!} - \dots)}\\
                         &= \frac{1}{z^{3}(1 - (\frac{z^2}{3!} - \frac{z^{4}}{5!} - \dots))}\\
\end{align*}
If $w =  (\frac{z^2}{3!} - \frac{z^{4}}{5!} - \dots)$ we have
  $$&= \frac{1}{z^{3}(1 - w)}$$

  And for $w \le 1$ we have,  
  \begin{align*}
      \frac{1}{z^{3}(1 - w)} &= \frac{1}{z^{3}} \sum_{n=0}^{\infty} w^{n}\\
                            &= \frac{1}{z^{3}} \sum_{n=0}^{\infty} (\frac{z^2}{3!} - \frac{z^{4}}{5!} + \dots)^{n}
  \end{align*}

  We need the coefficient for $\frac{1}{z}$. In our case the that only happens in the first element of the sequence on the right so we have which is, 
  $$ \frac{1}{z^{3}} + (\frac{z^2}{3!} + \dots) $$ 
  $$ \frac{1}{3!z} + \dots $$ 

  So we have our residue as $\frac{1}{3!} = \frac{1}{6}$ 

  Now as for all the other singularities we see that $q'(z_n) \ne 0$ we have,  
  $$ \frac{p(z_n)}{q'(z_n)} = \frac{1}{z_n^2 \cos z_n + 2z_n \sin z_n} $$ 

  We have our singularities as, 
  $$ n\pi  \text { for $n \in \N$ }$$  

  So we have 
  \begin{align*}
      Res_{z = z_n} &= \frac{1}{(n)^2 \pi^2 \cos n \pi + 0 \cdot 2z_n }\\
                    &=  \frac{1}{n^2 \pi^2 \cos n \pi }\\
                    &=  \frac{1}{n^2 \pi^2 (-1)^{n} }\\
                    &=  \frac{(-1)^{n}}{n^2 \pi^2 }\\
  \end{align*}

  Now because $\frac{(-1)^{n}}{n^2\pi^2}$ is an even function, we have $f(-n) = f(n)$. Hence,  
  $$ \sum_{n= -N}^{N} \frac{(-1)^{n}}{n^2\pi^2} = 2 \sum_{n=1}^{N} \frac{(-1)^{n}}{n^2\pi^2} $$

  So the sum of all our residues is, 
  $$ \frac{1}{6}  + 2 \sum_{n=1}^{N} \frac{(-1)^{n}}{n^2\pi^2}$$ 

  So our integral is, 
  $$ \int_{C_N} \frac{dz}{z^2 \sin z} = 2\pi i\bigg [\frac{1}{6}  + 2 \sum_{n=1}^{N} \frac{(-1)^{n}}{n^2\pi^2} \bigg ] $$ 



  Now we are given that the integral goes to $0$ as $N \rightarrow \infty$ this means that, 
  \begin{align*}
      2 \sum_{n=1}^{\infty} \frac{(-1)^{n}}{n^2 \pi^2}  &= - \frac{1}{6}\\
      \sum_{n=1}^{\infty} \frac{(-1)^{n}}{n^2 \pi^2}  &= - \frac{1}{12}\\
      \sum_{n=1}^{\infty} \frac{(-1)^{n}}{n^2 }  &= - \frac{\pi^2}{12}\\
      \sum_{n=1}^{\infty} \frac{(-1)^{n + 1}}{n^2 }  &= \frac{\pi^2}{12}\\
  \end{align*}







\subsection*{Problem 5}
We have, 
$$ \int_0^{\infty} \frac{x^2 dx}{(x^2 + 1)(x^2 + 4)} $$ 

First we have the function is even which means that $f(-x) = f(x)$ hence,  
$$ \int_{-\infty}^{\infty} f(x) dx= 2\int_{0}^{\infty}f(x) dx $$ 

First consider the complex valued function, 
$$ f(z) = \frac{z^2 }{(z^2 + 1)(z^2 + 4)}$$ 
Consider the positively oriented semicircle and we have, 
$$ \int_{C} f(z) dz = \int_{-R}^{R} f(x) dx + \int_{C_R} f(z) dz $$ 


So first integral we see our singularities which are $z = i$ and $z = 2i$. So our integral is  the sum of residues at these two points.

$$ f(z) = \frac{z^2}{(z + i)(z - i)(z + 2i)(z - 2i)} $$ 

So $Res_{z = i} f(z) = \frac{i^2}{2i(3i(-i))} = -\frac{1}{6i}$ 

And $Res_{z = 2i} f(z) = \frac{4i^2}{(3i)(i)(4i)} = -\frac{4}{-12i} = \frac{1}{3i}$ 

Sum is $\frac{1}{6i}$ and integral is $2\pi i \frac{1}{6i} = \frac{\pi}{3}$ 

Now we can also show the, 
$$ \int_{C_R} f(z) dz $$  goes to $0$ as we can bound $|f(z)| \le \frac{R^2}{(R^2 -1)(R^2 - 4)}$ 

So we see the power of the denominator is greater than numerator hence  as $R \rightarrow \infty$ the integral goes to zero.

Hence  we have, 
$$ \int_C f(z) dz = \int_{-\infty}^{\infty} f(x) dx  = \frac{\pi}{3} $$ 

$$ 2 \int_0^{\infty} = \frac{\pi}{3} $$ 
$$ \int_0^{\infty} = \frac{\pi}{6} $$ 
\subsection*{Problem 4}

$$ \int_{{-\infty}}^{{\infty}} {\frac{x \sin (ax)}{x^{4} + 4}} \: d{x} {} $$ 

We can take $$f(z) = \frac{ze^{iaz}}{z^{4} + 4}$$ 

If we consider the positively oriented semicircle from $-R$ to $R$ we have, 
$$ \int_C f(z) dz = \int_{C_R} f(z) dz + \int_{{-\infty}}^{{\infty}} {f(x)} \: d{x} {} $$ 



Our singularities is when $z^{4}= -4$. So we have,  
\begin{align*}
    z^{4} &= -4  \\
    z^{4}&= 4e^{-(\pi + 2n\pi) i}\\
    z&= (4e^{-(\pi + 2n\pi) i})^{1 / 4}\\
    z&= (4e^{-(\pi /4 + \pi n/2) i})\\
\end{align*}

So the singularities within our contour are when $n = 1, 2$ which are, 
$$ z = i + 1, z = i - 1 $$  



Now using theorem the residues are as follows, 
$$ Res_{z = i + 1} f(z) = \frac{p(i + 1)}{q'(i + 1)}  = \frac{(i + 1)e^{ia(i + 1)}}{4(i + 1)^{3}}$$ 
\begin{align*}
    \frac{(i + 1)e^{ia(i + 1)}}{4(i + 1)^{3}} &=\frac{ e^{ia(i + 1)}}{4(i + 1)^2}\\
\frac{ e^{ia }e^{- a}}{8i}\\
\end{align*}

Similarly we have, 
$$ Res_{z = i - 1} f(z) = \frac{p(i - 1)}{q'(i - 1)}  = \frac{(i - 1)e^{ia(i - 1)}}{4(i - 1)^{3}}$$ 
\begin{align*}
    \frac{(i - 1)e^{ia(i - 1)}}{4(i - 1)^{3}} &=\frac{ e^{ia(i - 1)}}{4(i - 1)^2}\\
\frac{ e^{-ia }e^{- a}}{-8i}\\
\end{align*}

So sum of residues is 

\begin{align*}
&= \frac{ e^{ia }e^{- a}}{8i} + \frac{ e^{-ia }e^{- a}}{-8i}\\
&= \frac{e^{-a}}{8i} (e^{ia} - e^{-ia})\\
&= \frac{e^{-a}}{8i}  (\cos(a) + i \sin(a) - \cos(a)+ i \sin(a))\\
&= \frac{e^{-a}}{8i}  (2 i \sin(a) )\\
&= \frac{e^{-a}}{4}   \sin(a) \\
\end{align*}
So our integral is
\begin{align*}
    &= 2\pi i\frac{e^{-a}}{4}   \sin(a)\\
    &= \frac{\pi i e^{-a} \sin(a)}{2}\\
\end{align*}

And we have, 
$$ Im (\int_C f(z) dz ) = \frac{2\pie^{-a}\sin(a)}{2} $$ 



So we have, 
$$  Im (\int_C f(z) dz )  = \int_{{-R}}^{{R}} {f(x)} \: d{x}  + \int_{C_R}f(z) dz$$ 

We know the right integral goes to zero when $R$ goes to $\infty$ as we can write our function as, 
$$ f(z) = \frac{ze^{iaz}}{z^{4} + 4} = \phi(z)e^{iaz} $$ 

And using Jordan lemma we have, 
$$ \lim_{R \rightarrow \infty }\int_{C_R} \phi(z) e^{iaz} dz  = 0$$ 

So we get, 
$$ \int_{{-\infty}}^{{\infty}} {\frac{x\sin(ax)}{x^{4} + 4}} \: d{x} = \frac{\pi}{2}e^{-a}\sin(a) $$



\subsection*{Problem 9}

We have, 
$$ \int_{{-\infty}}^{{\infty}} {\frac{x \sin x}{x^2 + 2x + 2}} \: d{x} {} $$

First lets construct  our function as, 
$$ f(z) = \frac{z e^{iz}}{z^2 + 2z + 2} = \frac{z e^{iz}}{(z+1)^2 + 1}$$ 

Considering the positively oriented contour lying in the upper half plane we have, 
$$ \int_{C} f(z) dz = \int_{{-\infty}}^{{\infty}} {f(z)} \: d{z}  + \int_{C_R} f(z) dz $$ 

First to find $ \int_{C} f(z) dz$ we look at the singularities within our domain. We have, 
$$ (z+1)^2 = -1  = e^{(-\pi + 2n\pi)i}$$ 
$$ z+1 = e^{(-\frac{\pi}{2} + n\pi)i}$$ 
$$ z  = e^{(-\frac{\pi}{2} + n\pi)i} - 1$$ 

So the only singularity in our domain is when $n = 1$ and we have $z = i - 1$.

So we have our residue as,  
\begin{align*}
    Res_{z = i -1} f(z) = \frac{p(i - 1)}{q'(i - 1)} &= \frac{ze^{iz}}{2(z + 1)}\\
                                                     &= \frac{(i - 1)e^{i(i - 1)}}{2i}\\
                                                     &= \frac{(i - 1)e^{-1 - i}}{2i}\\
                                                     &= \frac{(i - 1)e^{-1 }e^{- i}}{2i}\\
                                                     &= \frac{(i - 1)e^{-1 }(\cos(1) - i\sin(1))}{2i}\\
                                                     &= \frac{e^{-1 }(i\cos(1) + \sin(1) - \cos(1) + i \sin(1))}{2i}\\
                                                     &= \frac{e^{-1}}{2i} (i(\cos 1 + \sin 1) + \frac{e^{-1}}{2i}(\sin 1 - \cos 1)
\end{align*}

So our integral would be $2\pi i Res_{z = i - 1} f(z)$,  
$$ = i\pi e^{-1} (\cos 1 + \sin 1) + \pi e^{-1}(\sin 1 - \cos 1) $$ 


Now because we only need the imaginary part as $Im f(z) = f(x)$ we have, 
$$Im \int_C f(z) dz = \int_{{-R}}^{{R}} {f(x)} \: d{x}  + Im \int_{C_R} f(z) dz $$ 
$$ \pi e^{-1}(\cos 1 + \sin 1) = \int_{{-R}}^{{R}} {f(x)} \: d{x}  + Im \int_{C_R} f(z) dz $$ 

However we know that j
$$ \lim_{R \to \infty} \int_{C_R} f(z) dz   = 0$$ 
as we can write $f(z) = \phi(z) e^{iz}$ so using jordans lemma we have the integral goes to zero.

Hence we get as $R \rightarrow \infty$, 
$$ \int_{{-\infty}}^{{\infty}} {f(x)} \: d{x} = \pi e^{-1} (\cos 1 + \sin 1)  $$ 

\end{document}

