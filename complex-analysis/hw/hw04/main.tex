\documentclass[a4paper]{report}
\input{preamble.tex}
\title{MATH 4320 HW04}
\author{Aamod Varma}
\begin{document}
\maketitle

\subsection*{Problem 3}
(a).  We can write \[
P(z) = a_0 + a_1z + \dots + a_nz^n 
.\]
Using results from section 20 we know that for $f(z),g(z)$\[
    (f(z) + g(z))' = f'(z) + g'(z)
.\] 

Using this idea we can write the polynomial as, \[
    P'(z) = (a_0+ a_1z + \dots)' + (a_nz^n)'
.\] 
Similary we can apply this for each functino as follows,\[
    P'(z) = \frac{d}{dx}(a_0) + \frac{d}{dx}(a_1z) + \dots \frac{d}{dx}(a_nz^n)
.\] 
We know that $\frac{d}{dx}z^n = nz^{n - 1}$ 

So we can write, \[
    P'(z) = 0 + a_1 + 2a_2z + \dots + na_nz^{n-1}
.\] 

(b). We need to find the coefficients, $a_0,a_1,\dots,a_n$. To do this we need to remove the $z$ term that is multiplied it and set the polynomial to  $0$ to isolate our coefficient.

So to find any $a_n$  for $n \ge 1$ we first derive $P(z), n$  times.
\[
    P^{n}(z) = (n)(n-1)(n-2)\dots(1)a_nz^0 + (n+1)(n)\dots(2)a_{n+1}z^1 + \dots
.\] \[
P^{n} (z) = (n!)a_n + \frac{(n+1)!}{1!}a_{n+1}z^1 + \frac{(n+2)!}{2!}a_{n + 2}z^2 + \dots
.\] 
\[
P^n (0) = (n!)a_n + 0 + \dots + 0
.\] 
\[
a_n = \frac{P^n(0)}{n!}
.\] 

For $n = 0$ it is obvious that we can plug in  $z = 0$ to get,  \[
a_0 = P(0)
.\] 


\subsection*{Problem 1}
(a). $f(z) = \bar z$;

We need to satisfy the Cauchy-Remann equations for $f'(z)$ to be defined, so, \[
    u_x = v_y, u_y = -v_x
.\] 
We have, $z = x + iy$ and $\bar z = x - iy$. So, $u(x,y) = x, v(x,y) = -y$

We can write,  \[
    u_x = 1, u_y = 0
.\]
\[
    v_x = 0, v_y = -1
.\] 

So we can see that, $u_x \neq v_y$ which means that our derivate $f'(z)$ cannot exist.

(b). $f(z) = z - \bar z$

We have, $z = x + iy$ so  $f(z) = (x + iy) - (x - iy) = 2iy$

We can write,  $u(x,y) = 0, v(x,y) =  2y$
So, 
 \[
    u_x = 0, u_y = 0
.\] 
\[
v_x = 0, v_y = 2
.\] 
We can see that, $u_x \neq v_y$, so the derivative, $f'(z)$ cannot exist.


(c). $f(z) = 2x + ixy^2$

We have, $f(z) = 2x + ixy^2$

We can write, $u(x,y) = 2x, v(x,y) = xy^2$

So, \[
u_x = 2, u_y = 0
.\]
\[
v_x = y^2, v_y = 2xy
.\] 
We see that for this to be satisfied we need, \[
    2 = 2xy \text{ and } y^2 = 0
.\] 

But if $y^2 = 0$ then $y = 0$ So  $2xy = 0 \neq 2$

So for no values of $x,y$ will the equation be satisfied. Hence our derivative,  $f'(z)$ cannot exist.

(d). $f(z) = e^x e^{-iy}$

If $z = x + iy$ and  $f(z) = e^x e^{-iy}$ we can write, \[
f(z) = e^x (\cos(y) - i\sin(y)) = e^x\cos(y) - ie^x\sin(y)
.\] 

So, $u(x,y) = e^x\cos(y), v(x,y) = -e^x\sin(y)$

We have, \[
u_x = e^x\cos(y), u_y = -e^x\sin(y)
.\] 
\[
v_x = -e^x\sin(y), v_y = -e^x\cos(y)
.\] 

So we need, $e^x\cos(y) = -e^x\cos(y)$ and $e^x\sin(y) = e^x\sin(y)$

As $e^x \neq 0$ we have, $cos(y) = 0$ and  $sin(y) = 0$

However there exist no value  $y$ to satisfy both these conditions, hence our derivative  $f'(z)$ cannot exist.


\subsection*{Problem 8}
(a).

\[
    \frac{\partial  F}{\partial \bar z} =    \frac{\partial  F}{\partial x}    \frac{\partial  x}{\partial \bar z}   +    \frac{\partial  F}{\partial y}  \frac{\partial  y}{\partial \bar z}
.\] 
We know, \[
\frac{\partial  x}{\partial \bar z}  = \frac{\partial }{\partial \bar z}  (\frac{z + \bar z}{2}) = \frac{1}{2}
.\] 
Similarly, 
\[
\frac{\partial  y}{\partial \bar z}  = \frac{\partial }{\partial \bar z}  (\frac{z - \bar z}{2i}) = \frac{-1}{2i} = \frac{i}{2}
\]

So we can write, \[
    \frac{\partial  F}{\partial \bar z} =    \frac{\partial  F}{\partial x}   \frac{1}{2}   +    \frac{\partial  F}{\partial y}  \frac{i}{2} = \frac{1}{2}\bigg(   \frac{\partial  F}{\partial x} + i  \frac{\partial  F}{\partial y} \bigg)
\]

(b). \[
    \frac{\partial  f}{\partial \bar z} = \frac{1}{2}\bigg(   \frac{\partial  f}{\partial x} + i  \frac{\partial  f}{\partial y} \bigg)
.\] 
We are give that, $f(z) = u(x,y) + iv(x,y)$

So, \[
    \frac{\partial  f}{\partial \bar z} = \frac{1}{2}\bigg(   \frac{\partial}{\partial x}(u(x,y) + iv(x,y)) + i  \frac{\partial  }{\partial y} (u(x,y) + iv(x,y))\bigg)
.\] 
\[
= \frac{1}{2}\bigg(   u_x + iv_y + i(u_y + iv_y))\bigg)
.\] 

\[
= \frac{1}{2}\bigg(   u_x + iv_y + iu_y - v_y\bigg)
.\] 
\[
= \frac{1}{2}\bigg((u_x - v_y)+ i(u_y + v_x)\bigg)
.\] 

Now if $f(z)$ satisfies the Cauchy-Riemann equations then we know that  $u_x = v_y$ and  $u_y = -v_x$ and,  \[
\frac{\partial f}{\partial \bar z} = \frac{1}{2}(0) = 0
.\] 


\subsection*{Problem 2}
(a). $f(z) = xy + iy$

Using the Cauchy-Riemann equations we see, \[
u_x = y, u_y = x
.\] 
\[
v_x = 0, u_y = 1
.\] 

So we have, $y = 1$ and $x = 0$

Which means that the function has a derivate at only  $y = 1, x= 0$. However for a functino to be analytic at a point it has to have a derivative at everypoint in some neighborhood of the point. But in this case it only has it at the point  $(0,1)$ and nowhere else. So it is not analytic anywhere.


(b). $f(z) = 2xy + i(x^2 - y^2)$
We have, \[
u_x = 2y, u_y = 2x
.\] 
\[
v_x = 2x, v_y = -2y
.\] 

So we need, $2y = -2y$ and $2x = -2x$

The only values that satisfy this are  $x = 0, y = 0$

So the functino is not analytic anywhere because it has a derivative only at  $(0,0)$ and nowhere else.

(c).  $f(z) = e^ye^{ix}$

$$f(z) = e^y(\cos(x) + i\sin(x))$$


So, \[
u_x = e^y(-\sin(x)), u_y = e^y\cos(x)
.\] 
\[
v_x = e^y(\cos(x)), v_y = e^y \sin(x)
.\] 

So we have, $u_x = v_y$ and $u_y = -v_x$

So,  \[
    2e^ysin(x) = 0 \text{ which means } sin(x) = 0
.\] 
and, \[
    2e^ycos(x) = 0 \text { which mean } cos(x) = 0
.\] 

So our values of $x,y$ for which the eqations hold are, $x = n\pi, n \in Z$ and $y = \frac{\pi}{2} + n\pi$ for $n \in Z$. So we have a derivative only exactly at these point. However the derivate doesn't exist in any neighborhood of these points which means the functino isn't analytic.

\subsection*{Problem 6}
We have, \[
g(z) = ln r + i \theta \quad (r > 0, 0 < \theta < 2\pi)
.\] 

The cauchy-reimann equation for polar coordinates tell us that, \[
    ru_r = v_{\theta}, u_{\theta} = -r v_r
.\] 

$$u_r = \frac{1}{r}, u_\theta = 0$$
$$v_r = 0, v_\theta = 1$$


So we have, $r \frac{1}{r} = 1, 1 = 1$ and $0 = 0$ 

Our functino satifies the cauchy-reimann equations at all points.

It's derivative in the domain is, $f'(z) = e^{-i\theta}(\frac{1}{r}) = \frac{1}{re^{i\theta}} = \frac{1}{z}$ 

We have $G(z) = g(z^2+ 1)$

This is a composition of functions where $f(z) = z^2 + 1$ then $G(z) = g(f(z))$

For this to be the case we need to show that both the functinos are analytic so that we can use the chain rule to differentiate them.

We first know that $f(z) = z^2 + 1$ is analytic everywhere as it satisfies the Cuachy-Remiann equations.

In the quadrant $x> 0, y > 0$ we have $Im(z^2 + 1) = 2xy > 0$. If this is the case then we know $rsin(\theta)$ has to be positive which means $ 0 < \theta < \pi < 2\pi$. Our functions are analytic here. 

So we can use the chain rule to differnetiate as follows, \[
G'(z) = g'(z^2 + 1) f'(z) = \frac{1}{z^2 + 1} f'(z)  = \frac{2z}{z^2 + 1}
.\] 

                                        

\end{document}
