\documentclass[a4paper]{report}
\usepackage[utf8]{inputenc}
\usepackage[T1]{fontenc}
\usepackage{textcomp}

\usepackage{url}

% \usepackage{hyperref}
% \hypersetup{
%     colorlinks,
%     linkcolor={black},
%     citecolor={black},
%     urlcolor={blue!80!black}
% }

\usepackage{graphicx}
\usepackage{float}
\usepackage[usenames,dvipsnames]{xcolor}

% \usepackage{cmbright}

\usepackage{amsmath, amsfonts, mathtools, amsthm, amssymb}
\usepackage{mathrsfs}
\usepackage{cancel}

\newcommand\N{\ensuremath{\mathbb{N}}}
\newcommand\R{\ensuremath{\mathbb{R}}}
\newcommand\F{\ensuremath{\mathscr{F}}}
\newcommand\Z{\ensuremath{\mathbb{Z}}}
\renewcommand\O{\ensuremath{\emptyset}}
\newcommand\Q{\ensuremath{\mathbb{Q}}}
\newcommand\C{\ensuremath{\mathbb{C}}}
\let\implies\Rightarrow
\let\impliedby\Leftarrow
\let\iff\Leftrightarrow
\let\epsilon\varepsilon

% horizontal rule
\newcommand\hr{
    \noindent\rule[0.5ex]{\linewidth}{0.5pt}
}

\usepackage{tikz}
\usepackage{tikz-cd}

% theorems
\usepackage{thmtools}
\usepackage[framemethod=TikZ]{mdframed}
\mdfsetup{skipabove=1em,skipbelow=0em, innertopmargin=5pt, innerbottommargin=6pt}

\theoremstyle{definition}

\makeatletter

\declaretheoremstyle[headfont=\bfseries\sffamily, bodyfont=\normalfont, mdframed={ nobreak } ]{thmgreenbox}
\declaretheoremstyle[headfont=\bfseries\sffamily, bodyfont=\normalfont, mdframed={ nobreak } ]{thmredbox}
\declaretheoremstyle[headfont=\bfseries\sffamily, bodyfont=\normalfont]{thmbluebox}
\declaretheoremstyle[headfont=\bfseries\sffamily, bodyfont=\normalfont]{thmblueline}
\declaretheoremstyle[headfont=\bfseries\sffamily, bodyfont=\normalfont, numbered=no, mdframed={ rightline=false, topline=false, bottomline=false, }, qed=\qedsymbol ]{thmproofbox}
\declaretheoremstyle[headfont=\bfseries\sffamily, bodyfont=\normalfont, numbered=no, mdframed={ nobreak, rightline=false, topline=false, bottomline=false } ]{thmexplanationbox}


\declaretheorem[numberwithin=chapter, style=thmgreenbox, name=Definition]{definition}
\declaretheorem[sibling=definition, style=thmredbox, name=Corollary]{corollary}
\declaretheorem[sibling=definition, style=thmredbox, name=Proposition]{prop}
\declaretheorem[sibling=definition, style=thmredbox, name=Theorem]{theorem}
\declaretheorem[sibling=definition, style=thmredbox, name=Lemma]{lemma}



\declaretheorem[numbered=no, style=thmexplanationbox, name=Proof]{explanation}
\declaretheorem[numbered=no, style=thmproofbox, name=Proof]{replacementproof}
\declaretheorem[style=thmbluebox,  numbered=no, name=Exercise]{ex}
\declaretheorem[style=thmbluebox,  numbered=no, name=Example]{eg}
\declaretheorem[style=thmblueline, numbered=no, name=Remark]{remark}
\declaretheorem[style=thmblueline, numbered=no, name=Note]{note}

\renewenvironment{proof}[1][\proofname]{\begin{replacementproof}}{\end{replacementproof}}

\AtEndEnvironment{eg}{\null\hfill$\diamond$}%

\newtheorem*{uovt}{UOVT}
\newtheorem*{notation}{Notation}
\newtheorem*{previouslyseen}{As previously seen}
\newtheorem*{problem}{Problem}
\newtheorem*{observe}{Observe}
\newtheorem*{property}{Property}
\newtheorem*{intuition}{Intuition}


\usepackage{etoolbox}
\AtEndEnvironment{vb}{\null\hfill$\diamond$}%
\AtEndEnvironment{intermezzo}{\null\hfill$\diamond$}%




% http://tex.stackexchange.com/questions/22119/how-can-i-change-the-spacing-before-theorems-with-amsthm
% \def\thm@space@setup{%
%   \thm@preskip=\parskip \thm@postskip=0pt
% }

\usepackage{xifthen}

\def\testdateparts#1{\dateparts#1\relax}
\def\dateparts#1 #2 #3 #4 #5\relax{
    \marginpar{\small\textsf{\mbox{#1 #2 #3 #5}}}
}

\def\@lesson{}%
\newcommand{\lesson}[3]{
    \ifthenelse{\isempty{#3}}{%
        \def\@lesson{Lecture #1}%
    }{%
        \def\@lesson{Lecture #1: #3}%
    }%
    \subsection*{\@lesson}
    \testdateparts{#2}
}

% fancy headers
\usepackage{fancyhdr}
\pagestyle{fancy}

% \fancyhead[LE,RO]{Gilles Castel}
\fancyhead[RO,LE]{\@lesson}
\fancyhead[RE,LO]{}
\fancyfoot[LE,RO]{\thepage}
\fancyfoot[C]{\leftmark}
\renewcommand{\headrulewidth}{0pt}

\makeatother

% figure support (https://castel.dev/post/lecture-notes-2)
\usepackage{import}
\usepackage{xifthen}
\pdfminorversion=7
\usepackage{pdfpages}
\usepackage{transparent}
\newcommand{\incfig}[1]{%
    \def\svgwidth{\columnwidth}
    \import{./figures/}{#1.pdf_tex}
}

% %http://tex.stackexchange.com/questions/76273/multiple-pdfs-with-page-group-included-in-a-single-page-warning
\pdfsuppresswarningpagegroup=1

\author{Aamod Varma}
\setlength{\parindent}{0pt}


\title{MATH 4320 HW09-10}
\author{Aamod Varma}
\begin{document}
\maketitle

\subsection*{Problem 2}
(a). We have $C$ is the circle $|z - i|  = 2$ in the positive sense. We have our integral, 
$$ \int_C \frac{1}{z^2 + 4}dz $$ 

First we can rewrite this as,
$$ \int_C \frac{1}{(z + 2i)(z - 2i)} $$ 

So we see that the singularity is it $-2i$ and $2i$. However $z = -2i$ lies outside our contour $C$. So let $f(z) = \frac{1}{z + 2i}$ and we write it is, 
$$ \int_C \frac{f(z)}{z - 2i} $$ 

Using theorem we know this is equivalent to $2\pi i f(z')$ where $z'$ is the singularity point which is at $z = 2i$ in this case. So we have, 
$$ = 2\pi i \frac{1}{2i + 2i} $$ 
$$ = 2\pi i \frac{1}{4i} = \frac{\pi}{2} $$ 


(b). We have $\frac{1}{(z^2+4)^2}$. Let us rewrite this as, 
$$ \frac{1}{((z + 2i)(z - 2i))^2} = \frac{1}{(z + 2i)^2(z - 2i)^2}  $$ 

We already know that $-2i$ lies outside our contour so let $f(z) = \frac{1}{(z + 2i)^2}$. And we get, 
$$ \int_C \frac{f(z)}{(z - 2i)^2} $$ 

We have, 
$$ f^{(n)}z = \frac{n!}{2\pi i} \int_C \frac{f(z)}{(z - z')^{n + 1}}dz $$ 

So in our case we have $n = 1$ so, 
$$ f'(z) = \frac{1}{2\pi i} \int_C \frac{f(z)}{(z - 2i)^2} $$ 

So our integral is, 
$$ f'(z) 2 \pi i \text{ where $z = 2i$ } $$

We know $f(z) = \frac{1}{(z + 2i)^2}$ so $f'(z) = -\frac{2}{(z + 2i)^{3}}$, so our integral is, 
$$ -\frac{2}{(z + 2i)^3} 2\pi i = -\frac{2}{(4i)^{3}} 2 \pi i $$ 
$$ = -4\pi i \frac{1}{-64i} $$ 
$$ = \frac{\pi}{16} $$ 
\subsection*{Problem 4}
Consider the case when $z$ is inside the the contour. This means that there is a singularity at $s = z$. We nkow using the cauchy goursat extension that, 
$$ f^{n}(z) = \frac{n!}{2\pi i}\int_C \frac{f(s)}{(s - z)^{n + 1}} ds$$ 

We see that our term with singuarity is in the denominator and hence we can take $f(s) = s^{3} + 2s$. So let us first rewrite our itnegral as, 
$$ \int_C \frac{f(s)}{(s - z)^{3}}ds $$ 
In our case we have $n = 2$ so we have,  
$$ f''(s) = \frac{2!}{2\pi i} g(z) $$

We have $f(s) = s^{3} + 2s$ so $f'(s) = 3s^2 + 2$ and $f''(s) = 6s$.

So,  
$$ 6z = \frac{2}{2\pi i}g(z) $$
$$ g(z) = 6\pi iz $$ 

Now when $z$ is outside the contour we see that our functino is all analytic inside our contour. So as the contour is closed we know that the integral will be zero.

\subsection*{Problem 6}
We need to show the functino is analytic at each point $z$ interior to C which means that we need to show the existence of the derivative at any neighborhood of each of the points in our contour.

We have, 
$$ g(z) = \frac{1}{2 \pi i} \int_C \frac{f(s)}{s - z} ds $$ 


Using the definitino of the derivative we have, 
\begin{align*}
    g'(z) &= \lim_{h \to 0} \frac{g(z+h) - g(z)}{h}\\
          &= \lim_{h \to 0}\frac{1}{2\pi i h} \int_{{C}}^{{}} {\frac{f(s)}{s - (z+h)} + \frac{f(s)}{(s - z)}} \: d{s} {}\\
          &= \lim_{h \to 0}\frac{1}{2\pi i} \int_C \frac{f(s)}{(s-z)(s - (z+h))} \: ds\\
          &= \lim_{h \to 0}\frac{1}{2\pi i} \int_C \frac{f(s)}{(s-z)^2}ds +\int_C \frac{hf(s)}{(s-z-h)(s-z)^2} \: ds\\
\end{align*}

The right hand integral goes to zero as $h \rightarrow 0$

So we have, 
$$ g'(z) = \frac{1}{2\pi i} \int_C \frac{f(s) \: ds}{(s-z)^2} $$ 

At all points $z$ within our contour

\subsection*{Problem 1}
We have $f(z)$ is entire and we have $u(x,y) \le u_0$ for all $(x,y)$. We need to show  $u(x,y)$ is constant throughout the plane.

We have $g(z) = e^{f(z)}$. We write $f(z) = u + iv$ where both u and v are functinos on  $x$ and $y$. So we get, 
$$ |g(z)| = |e^{u + iv}| = |e^{u}e^{iv}| = |e^{u}| |e^{iv}| $$ We know that $|e^{iv}| = 1$ so we get, 
$$ |e^{u}||e^{iv}| = |e^{u}| \cdot 1  \le |e^{u_0}|$$ 

So we've shown that the functino $g(z)$ is bounded. Now because it is entire then it must be constant according to Liouvillie's theorem. For that to be true we need $f(z)$ to be constant hence $u(x,y) = Re(f(z))$  must be constant.

\subsection*{Problem 6}

% \begin{figure}[ht]    \centering    \incfig{prob6}    \caption{prob6}    \label{fig:prob6}\end{figure}

Our functino is $f(z) = e^{z} = e^{x + iy} = e^{x}e^{iy} = e^{x}(\cos(y) + i \sin(y))$ 

Which means the funcitno we wnat to analyse that is $u(x,y) = Re(f(z))$ is  
$$ u(x,y) = e^{x}\cos(y)$$

We know that the maximum value of $e^{x}$ in our domain is at $x = 1$ and the maximum value of $\cos(y)$ in our domain is at $y = 0$ as $\cos(0) = 1$. Hence the max value of our funcitno $e^{x}$ is at $z = 1$. 

The minimum value of  $e^{x}$ is equal to $e$ at  $x = 0$  and of $\cos(y)$  is when $y = \pi$ where  $\cos(y) = -1$. Hence the minmum value of  $u$ will be when  we have the max value of $e^{x}$ and the min value of $\cos(y)$ which is at $1 + \pi i$


\subsection*{Problem 8}
(a). We have $(z- z_0) (z^{k-1} + z^{k-2}z_0 + \dots + z(z_0)^{k - 2} + (z_0)^{k-1}$. Now let us expand this as follows, 
\begin{align*}
    (z^{k} + z^{k-1}z_0  + \dots + z^2z_0^{k-2} + zz_0^{k-1}) - (z_0z^{k-1} + z^{k-2}z_0^2 + \dots + zz_0^{k-1} + z_0^{k})
\end{align*}
We see that the middle terms cancel each other out leaving ounly the outer terms.
$$ = z^{k} - z_0^{k} $$ 

(b). Now using this factorization We have, 
$$ P(z) = a_0 + a_1z + \dots + a_nz^{n} $$  and 
$$ P(z_0) = a_0 + a_1z_0 + \dots + a_nz_0^{n} $$ 

So, 
$$ P(z) - P(z_0) = a_1(z - z_0) + a_2(z^2  - z_0^2) + \dots + a_n(z^{n} - z_0^{n}) $$ 

Using our facorizatino from above we have, 
\begin{align*}
    P(z) - P(z_0) &=   a_1(z - z_0) + a_2(z-z_0)Q_2(z) + \dots + a_n(z - z_0)Q_n(z) \\
                  &= (z - z_0)(a_1Q_1(z) + \dots + a_nQ_n(z)) \\
                  &=(z-z_0)Q(z) 
\end{align*}


\subsection*{Problem 1}
Using definitnio we need to show that for any choice of $\epsilon$ we can find a $n_0$ such that $\forall n > n_0$, 
$$ \bigg (\frac{1}{n^2} + i \bigg)  - i < \epsilon$$ 

We have $\frac{1}{n^2} < \epsilon$ and, 
$$ \frac{1}{\epsilon} < n^2 $$ 
$$ n > \frac{1}{\sqrt{\epsilon}} $$ 

So for any $n_0 = n > \frac{1}{\sqrt{\epsilon}}$ we have  $|\frac{1}{n^2} + i - i| < \epsilon$ which makes $i$ the limit of the sequence.


\subsection*{Problem 3}
We have $\lim_{n \to \infty} z_n =z$. Using the definitino we know that, for a given $\epsilon$ $,\exists n_0$ such that $\forall n > n_0$
$$ |z_n - z| < \epsilon $$  

Now we know that $||z_n| - |z|| \le |z_n - z|$ which  means that,  
$$ ||z_n| - |z|| <  \epsilon $$ and there exists $n_0$ for this epsilon such that it is true  $\forall n > n_0$. Hence we can say that, 
$$ \lim_{n \to \infty} |z_n| = |z| $$ 


\subsection*{Problem 7}
We have  $\sum_{n=1}^{\infty} z_n = S$. Let  $c$ be a complex number $x + iy$ and then we have, 
\begin{align*}
    \sum_{n=1}^{\infty} cz_n = \sum_{n=1}^{\infty} (x+iy)z_n &= \sum_{n=1}^{\infty} xz_n + i \sum_{n=1}^{\infty} yz_n\\
&=  x\sum_{n=1}^{\infty} z_n + i y\sum_{n=1}^{\infty} z_n\\
&= xS + iyS \\
&= S(x+iy) =cS \\
\end{align*}


\subsection*{Problem 2}
We need to find the taylor series of $e^{z}$ we know  
$$ f(z) = \sum_{n=0}^{\infty} f^{(n)}(z_0)\frac{1}{n!} (z-z_0)^{n} $$ 

(a). In our case we have $f = e^{z}$ and $z_0 = 1$

We also know that $f^{(n)}z_0 = e^{z_0} = e$  for any value of $n$ as $f'(z) = e^{z}$. Hence we have, 
$$ e^{z} = e \sum_{n=0}^{\infty} \frac{(z-1)^{n}}{n!} $$ 

(b). We know that $$e^{z} = \sum_{n=z}^{\infty} \frac{z^{n}}{n!}$$ Now let us replace $z$ with $z - 1$ and we get, 
$$ e^{z - 1} = \sum_{n=0}^{\infty} \frac{(z-1)^{n}}{n!} $$ 
$$ \frac{e^{z}}{e} =  \sum_{n=0}^{\infty} \frac{(z-1)^{n}}{n!} $$ 
$$ e^{z} = e \sum_{n=0}^{\infty} \frac{(z-1)^{n}}{n!} $$ 

\subsection*{Problem 3}
We have, 
$$ f(z) = \frac{z}{z^{4} + 4} = \frac{z}{4} \cdot \frac{1}{1 + \frac{z^{4}}{4}} $$ 

We know that $$\frac{1}{1 - z} = \sum_{n=0}^{\infty} z^{n}$$

Let us replace $z $ with $-\frac{z^{4}}{4}$ and we get, 
\begin{align*}
    \frac{1}{1 + \frac{z^{4}}{4}} &= \sum_{n=0}^{\infty} (\frac{-z^{4}}{4})^{n}\\
                                  &= \sum_{n=0}^{\infty} (\frac{-z^{4n}}{4^{n}})\\
    \frac{z}{4} \cdot \frac{1}{1 + \frac{z^{4}}{4}} &= \sum_{n=0}^{\infty} (-1)^{n} \frac{z^{4n + 1}}{2^{2n + 2}}
\end{align*}



\subsection*{Problem 10}
(a). First we know that, 
$$ \sinh z = \sum_{n=0}^{\infty} \frac{z^{2n + 1}}{(2n+1)!} $$  which means that
$$ \frac{\sinh z}{z^2} = \sum_{n=0}^{\infty} \frac{z^{2n - 1}}{(2n+1)!} $$ 

Let us take the first term out so we get, 
\begin{align*}
     &= \frac{1}{z} + \sum_{n=1}^{\infty} \frac{z^{2n - 1}}{(2n+1)!}\\
     &= \frac{1}{z} + \sum_{n=0}^{\infty} \frac{z^{2(n + 1) - 1}}{(2(n + 1) + 1 )!}\\
     &= \frac{1}{z} + \sum_{n=0}^{\infty} \frac{z^{2n + 1}}{(2n +3 )!}\\
\end{align*}

(b). We know, 
$$\sin(z) = \sum_{n=0}^{\infty} (-1)^{n} \frac{z^{2n + 1}}{(2n+1)!}$$
So, 
\begin{align*}
    \sin(z^2) &= \sum_{n=0}^{\infty} (-1)^{n} \frac{z^{4n + 2}}{(2n+1)!}\\
    \frac{\sin(z^2)}{z^{4}} &= \sum_{n=0}^{\infty} (-1)^{n} \frac{z^{4n - 2}}{(2n+1)!}\\
                            &= \frac{1}{z^2} - \frac{z^2}{3!} + \frac{z^{6}}{5!} + \dots
\end{align*}
\subsection*{Problem 2}
\begin{align*}
    f(z) &= \frac{1}{1 + z} = \frac{1}{z} \cdot \frac{1}{1 + 1 /z}\\
\end{align*}

We know, 
$$ \frac{1}{1 - z} =  \sum_{n=0}^{\infty} z^{n}$$ 

So we have, 
$$ \frac{1}{1+ 1 /z} = \sum_{n=0}^{\infty} (\frac{-1}{z})^{n}  =\sum_{n=0}^{\infty}  (-1)^{n} \frac{1}{z^{n}}$$ 
\begin{align*}
    f(z) &= \sum_{n=0}^{\infty} (-1)^{n} \frac{1}{z^{n+1}}\\
         &= \sum_{n=1}^{\infty} (-1)^{n+1} \frac{1}{z^{n}}
\end{align*}
\subsection*{Problem 4}

(1). 
$$ \frac{1}{1-z} = \sum_{n=0}^{\infty} z^{n} $$ 
$$  \frac{1}{z^2(1-z)} = \sum_{n=0}^{\infty} z^{n-2} = \frac{1}{z^2} + \frac{1}{z} + \sum_{n=0}^{\infty} z^{n}$$ 

This would be useful in $ 0 < |z| < 1$


(2). 
We have, 
$$ f(z) = \frac{1}{z^2(1- z)} $$  we can rewrite this as, 
$$ f(z) = \frac{1 / z^{3}}{1 / z - 1} = - \frac{1 / z^{3}}{1 - 1/z}  $$ 

We have $\frac{1}{1 - z} = \sum_{n=0}^{\infty} z^{n}$ so, 
$$ \frac{1}{1 - 1 /z} = \sum_{n=0}^{\infty} \frac{1}{z^{n}} $$ 
$$ -\frac{1 / z^{3}}{1 - 1 /z} = -\sum_{n=0}^{\infty} \frac{1}{z^{n + 3}} $$ 
$$ = - \sum_{n=3}^{\infty} \frac{1}{z^{n}} $$ 

which would be valid at $1 < |z| < \infty$
\subsection*{Problem 5}

1. $D_1$
We have, 
$$ \frac{1}{2 - z} = \frac{1}{2(1 - \frac{z}{2})} = \sum_{n=0}^{\infty} \frac{z^{n}}{2^{n + 1}} $$ 

And, 
$$ \frac{1}{z - 1} = - \sum_{n=0}^{\infty} z^{n} $$ 

So, 
$$ f(z) = \sum_{n=0}^{\infty} (2^{-n-1} - 1) z^{n}$$ 

2. $D_2$ 
We have, 
\begin{align*}
    \frac{1}{z-1} = \frac{1}{z(1 - 1 /z)} &= \sum_{n=0}^{\infty} \frac{1}{z^{n+1}}\\
                                          &= \sum_{n=1}^{\infty} \frac{1}{z^{n}}\\
\end{align*}

And, 
$$ \frac{1}{2-z} = \sum_{n=0}^{\infty} \frac{z^{n}}{2^{n+1}} $$ 

So, 
$$ f(z) = \sum_{n=0}^{\infty} \frac{z^{n}}{2^{n+1}} + \sum_{n=1}^{\infty} \frac{1}{z^{n}} $$ 

3. $D_3$ 

We have, 
$$ \frac{1}{z-1} = \frac{1}{z(1 - 1 /z)} = \sum_{n=0}^{\infty} \frac{1}{z^{n + 1}} $$ 
$$ = \sum_{n=1}^{\infty}\frac{1}{z^{n}} $$ 

Similarl, 
$$ \frac{1}{2-z} = \frac{1}{z(2 /z - 1)} = -\frac{1}{z(1 - 2 /z)}$$ 
$$ = - \sum_{n=1}^{\infty} \frac{2^{n - 1}}{z^{n}} $$ 

So we get, 
$$ f(z) = \sum_{n=1}^{\infty} \frac{1 - 2^{n - 1}}{z^{n}} $$ 


\subsection*{Problem 2}
We substitute $z$ with $\frac{1}{1 - z}$ and we have, 
$$ \frac{1}{(1 - (\frac{1}{1 - z}))^2} = \sum_{n=0}^{\infty} \frac{n+1}{(1-z)^{n}} $$ 
$$ \frac{(1-z)^2}{z^2} = \sum_{n=0}^{\infty} \frac{n+1}{(1-z)^{n}} $$ 
\begin{align*}
    \frac{1}{z^2} &= \sum_{n=0}^{\infty} \frac{n+1}{(1-z)^{n+2}}\\
                  &= \sum_{n=2}^{\infty} \frac{(-1)^{n}(n-1)}{(z - 1)^{n}}
\end{align*}


\end{document}



