\documentclass[a4paper]{report}
\usepackage[utf8]{inputenc}
\usepackage[T1]{fontenc}
\usepackage{textcomp}

\usepackage{url}

% \usepackage{hyperref}
% \hypersetup{
%     colorlinks,
%     linkcolor={black},
%     citecolor={black},
%     urlcolor={blue!80!black}
% }

\usepackage{graphicx}
\usepackage{float}
\usepackage[usenames,dvipsnames]{xcolor}

% \usepackage{cmbright}

\usepackage{amsmath, amsfonts, mathtools, amsthm, amssymb}
\usepackage{mathrsfs}
\usepackage{cancel}

\newcommand\N{\ensuremath{\mathbb{N}}}
\newcommand\R{\ensuremath{\mathbb{R}}}
\newcommand\F{\ensuremath{\mathscr{F}}}
\newcommand\Z{\ensuremath{\mathbb{Z}}}
\renewcommand\O{\ensuremath{\emptyset}}
\newcommand\Q{\ensuremath{\mathbb{Q}}}
\newcommand\C{\ensuremath{\mathbb{C}}}
\let\implies\Rightarrow
\let\impliedby\Leftarrow
\let\iff\Leftrightarrow
\let\epsilon\varepsilon

% horizontal rule
\newcommand\hr{
    \noindent\rule[0.5ex]{\linewidth}{0.5pt}
}

\usepackage{tikz}
\usepackage{tikz-cd}

% theorems
\usepackage{thmtools}
\usepackage[framemethod=TikZ]{mdframed}
\mdfsetup{skipabove=1em,skipbelow=0em, innertopmargin=5pt, innerbottommargin=6pt}

\theoremstyle{definition}

\makeatletter

\declaretheoremstyle[headfont=\bfseries\sffamily, bodyfont=\normalfont, mdframed={ nobreak } ]{thmgreenbox}
\declaretheoremstyle[headfont=\bfseries\sffamily, bodyfont=\normalfont, mdframed={ nobreak } ]{thmredbox}
\declaretheoremstyle[headfont=\bfseries\sffamily, bodyfont=\normalfont]{thmbluebox}
\declaretheoremstyle[headfont=\bfseries\sffamily, bodyfont=\normalfont]{thmblueline}
\declaretheoremstyle[headfont=\bfseries\sffamily, bodyfont=\normalfont, numbered=no, mdframed={ rightline=false, topline=false, bottomline=false, }, qed=\qedsymbol ]{thmproofbox}
\declaretheoremstyle[headfont=\bfseries\sffamily, bodyfont=\normalfont, numbered=no, mdframed={ nobreak, rightline=false, topline=false, bottomline=false } ]{thmexplanationbox}


\declaretheorem[numberwithin=chapter, style=thmgreenbox, name=Definition]{definition}
\declaretheorem[sibling=definition, style=thmredbox, name=Corollary]{corollary}
\declaretheorem[sibling=definition, style=thmredbox, name=Proposition]{prop}
\declaretheorem[sibling=definition, style=thmredbox, name=Theorem]{theorem}
\declaretheorem[sibling=definition, style=thmredbox, name=Lemma]{lemma}



\declaretheorem[numbered=no, style=thmexplanationbox, name=Proof]{explanation}
\declaretheorem[numbered=no, style=thmproofbox, name=Proof]{replacementproof}
\declaretheorem[style=thmbluebox,  numbered=no, name=Exercise]{ex}
\declaretheorem[style=thmbluebox,  numbered=no, name=Example]{eg}
\declaretheorem[style=thmblueline, numbered=no, name=Remark]{remark}
\declaretheorem[style=thmblueline, numbered=no, name=Note]{note}

\renewenvironment{proof}[1][\proofname]{\begin{replacementproof}}{\end{replacementproof}}

\AtEndEnvironment{eg}{\null\hfill$\diamond$}%

\newtheorem*{uovt}{UOVT}
\newtheorem*{notation}{Notation}
\newtheorem*{previouslyseen}{As previously seen}
\newtheorem*{problem}{Problem}
\newtheorem*{observe}{Observe}
\newtheorem*{property}{Property}
\newtheorem*{intuition}{Intuition}


\usepackage{etoolbox}
\AtEndEnvironment{vb}{\null\hfill$\diamond$}%
\AtEndEnvironment{intermezzo}{\null\hfill$\diamond$}%




% http://tex.stackexchange.com/questions/22119/how-can-i-change-the-spacing-before-theorems-with-amsthm
% \def\thm@space@setup{%
%   \thm@preskip=\parskip \thm@postskip=0pt
% }

\usepackage{xifthen}

\def\testdateparts#1{\dateparts#1\relax}
\def\dateparts#1 #2 #3 #4 #5\relax{
    \marginpar{\small\textsf{\mbox{#1 #2 #3 #5}}}
}

\def\@lesson{}%
\newcommand{\lesson}[3]{
    \ifthenelse{\isempty{#3}}{%
        \def\@lesson{Lecture #1}%
    }{%
        \def\@lesson{Lecture #1: #3}%
    }%
    \subsection*{\@lesson}
    \testdateparts{#2}
}

% fancy headers
\usepackage{fancyhdr}
\pagestyle{fancy}

% \fancyhead[LE,RO]{Gilles Castel}
\fancyhead[RO,LE]{\@lesson}
\fancyhead[RE,LO]{}
\fancyfoot[LE,RO]{\thepage}
\fancyfoot[C]{\leftmark}
\renewcommand{\headrulewidth}{0pt}

\makeatother

% figure support (https://castel.dev/post/lecture-notes-2)
\usepackage{import}
\usepackage{xifthen}
\pdfminorversion=7
\usepackage{pdfpages}
\usepackage{transparent}
\newcommand{\incfig}[1]{%
    \def\svgwidth{\columnwidth}
    \import{./figures/}{#1.pdf_tex}
}

% %http://tex.stackexchange.com/questions/76273/multiple-pdfs-with-page-group-included-in-a-single-page-warning
\pdfsuppresswarningpagegroup=1

\author{Aamod Varma}
\setlength{\parindent}{0pt}


\title{MATH 4320 HW03-4}
\author{Aamod Varma}
\begin{document}
\maketitle


\subsection*{Problem 3}
(a). $lim_{z \rightarrow z_o} \frac{1}{z^n}(z_0 \neq 0)$

We know, $$lim_{z \rightarrow z_o} \frac{1}{z^n} = \frac{lim_{z \rightarrow z_0}1} {lim_{z \rightarrow z_0}{z^n}} = \frac{1}{z_0^n}$$

(b). $lim_{z \rightarrow i} \frac{iz^3 - 1}{z + i}$


We know, $$lim_{z \rightarrow i} \frac{iz^3 - 1}{z + i} = \frac{lim_{z \rightarrow z_0}iz^3 - 1} {lim_{z \rightarrow z_0}{z + i}} = \frac{i^4 - 1}{2i} = \frac{0}{2i} = 0$$


(c). $lim_{z \rightarrow z_0} \frac{P(z)}{Q(z)}$

We know, $$lim_{z \rightarrow z_0} \frac{P(z)}{Q(z)} = \frac{lim_{z \rightarrow z_0}P(z)} {lim_{z \rightarrow z_0}{Q(z)}} = \frac{P(z_0)}{Q(z_0)}$$




\subsection*{Problem 7}
Using the definition of limits we know that, $\forall \epsilon > 0, \exists \delta$ such that. \[
    |f(z) - w_0| < \epsilon \text{ whenever } 0 < z - z_0 < \delta
.\] 
We know from the triangle inequality that, \[
|a - b| \ge ||a| - |b||
.\] 
Using this we can say, \[
||f(z)| - |w_0|| \le |f(z) - w_0|
.\] 

So we have, \[
||f(z)| - |w_0|| < \epsilon
.\] 

Now using the definition of limits once again, we get $\forall \epsilon > 0, \exists \delta$ the same as before, such that  \[
    ||f(z)| - |w_0|| < \epsilon \text{ whenever } 0 < |z - z_0| < \delta
.\] 
This shows that we can write, \[
\lim_{z \to z_0} |f(z)| = |w_0|
.\] 

\subsection*{Problem 10}
(a). $\lim_{z \to \infty} \frac{4z^2}{(z-1)^2} = 4$ 

We know that, \[
\lim_{z \to 0} f\bigg (\frac{1}{z}\bigg ) = w_0
.\] 
then, \[
\lim_{z \to \infty} f(z) = w_0
.\] 

So we can say, \[
    \lim_{z \to \infty} \frac{4z^2}{(z-1)^2} = \lim_{z \to 0} \frac{\frac{4}{z^2}}{(\frac{1}{z} - 1)^2}
.\]
\[
    = \lim_{z \to 0} \frac{4}{(1 - z)^2}
.\] 
\[
    = \frac{4}{1-0} = 4
.\] 

(b). $\lim_{z \to 1} \frac{1}{(z-1)^3} = \infty$

Using theorem we know that, \[
    \lim_{z \to z_0} f(z) = \infty \text{ if } \lim_{z \to z_0} \frac{1}{f(z)} = 0
.\] 
\[
    \lim_{z \to 1} \frac{(z-1)^3}{1} = \frac{(1-1)^3}{1} = 0
.\] 
So using the theorem we get, \[
 \lim_{z \to 1} \frac{1}{(z-1)^3} = \infty
.\] 

(c). $\lim_{z \to \infty} \frac{z^2 + 1}{z-1} = \infty$
Using theorem we know that, \[
    \lim_{z \to \infty} f(z) = \infty \text{ if } \lim_{z \to 0} \frac{1}{f(\frac{1}{z})} = 0
.\] 

Using this, \[
    \lim_{z \to 0} \frac{\frac{1}{z} - 1}{\frac{1}{z^2} + 1} = \frac{1 - 1}{1 + 1} = 0
.\]         

So, \[
 \lim_{z \to \infty} \frac{z^2 + 1}{z-1} = \infty
.\] 



\subsection*{Problem 3}
(a).  We can write \[
P(z) = a_0 + a_1z + \dots + a_nz^n 
.\]
Using results from section 20 we know that for $f(z),g(z)$\[
    (f(z) + g(z))' = f'(z) + g'(z)
.\] 

Using this idea we can write the polynomial as, \[
    P'(z) = (a_0+ a_1z + \dots)' + (a_nz^n)'
.\] 
Similary we can apply this for each functino as follows,\[
    P'(z) = \frac{d}{dx}(a_0) + \frac{d}{dx}(a_1z) + \dots \frac{d}{dx}(a_nz^n)
.\] 
We know that $\frac{d}{dx}z^n = nz^{n - 1}$ 

So we can write, \[
    P'(z) = 0 + a_1 + 2a_2z + \dots + na_nz^{n-1}
.\] 

(b). We need to find the coefficients, $a_0,a_1,\dots,a_n$. To do this we need to remove the $z$ term that is multiplied it and set the polynomial to  $0$ to isolate our coefficient.

So to find any $a_n$  for $n \ge 1$ we first derive $P(z), n$  times.
\[
    P^{n}(z) = (n)(n-1)(n-2)\dots(1)a_nz^0 + (n+1)(n)\dots(2)a_{n+1}z^1 + \dots
.\] \[
P^{n} (z) = (n!)a_n + \frac{(n+1)!}{1!}a_{n+1}z^1 + \frac{(n+2)!}{2!}a_{n + 2}z^2 + \dots
.\] 
\[
P^n (0) = (n!)a_n + 0 + \dots + 0
.\] 
\[
a_n = \frac{P^n(0)}{n!}
.\] 

For $n = 0$ it is obvious that we can plug in  $z = 0$ to get,  \[
a_0 = P(0)
.\] 


\subsection*{Problem 1}
(a). $f(z) = \bar z$;

We need to satisfy the Cauchy-Remann equations for $f'(z)$ to be defined, so, \[
    u_x = v_y, u_y = -v_x
.\] 
We have, $z = x + iy$ and $\bar z = x - iy$. So, $u(x,y) = x, v(x,y) = -y$

We can write,  \[
    u_x = 1, u_y = 0
.\]
\[
    v_x = 0, v_y = -1
.\] 

So we can see that, $u_x \neq v_y$ which means that our derivate $f'(z)$ cannot exist.

(b). $f(z) = z - \bar z$

We have, $z = x + iy$ so  $f(z) = (x + iy) - (x - iy) = 2iy$

We can write,  $u(x,y) = 0, v(x,y) =  2y$
So, 
 \[
    u_x = 0, u_y = 0
.\] 
\[
v_x = 0, v_y = 2
.\] 
We can see that, $u_x \neq v_y$, so the derivative, $f'(z)$ cannot exist.


(c). $f(z) = 2x + ixy^2$

We have, $f(z) = 2x + ixy^2$

We can write, $u(x,y) = 2x, v(x,y) = xy^2$

So, \[
u_x = 2, u_y = 0
.\]
\[
v_x = y^2, v_y = 2xy
.\] 
We see that for this to be satisfied we need, \[
    2 = 2xy \text{ and } y^2 = 0
.\] 

But if $y^2 = 0$ then $y = 0$ So  $2xy = 0 \neq 2$

So for no values of $x,y$ will the equation be satisfied. Hence our derivative,  $f'(z)$ cannot exist.

(d). $f(z) = e^x e^{-iy}$

If $z = x + iy$ and  $f(z) = e^x e^{-iy}$ we can write, \[
f(z) = e^x (\cos(y) - i\sin(y)) = e^x\cos(y) - ie^x\sin(y)
.\] 

So, $u(x,y) = e^x\cos(y), v(x,y) = -e^x\sin(y)$

We have, \[
u_x = e^x\cos(y), u_y = -e^x\sin(y)
.\] 
\[
v_x = -e^x\sin(y), v_y = -e^x\cos(y)
.\] 

So we need, $e^x\cos(y) = -e^x\cos(y)$ and $e^x\sin(y) = e^x\sin(y)$

As $e^x \neq 0$ we have, $cos(y) = 0$ and  $sin(y) = 0$

However there exist no value  $y$ to satisfy both these conditions, hence our derivative  $f'(z)$ cannot exist.


\subsection*{Problem 8}
(a).

\[
    \frac{\partial  F}{\partial \bar z} =    \frac{\partial  F}{\partial x}    \frac{\partial  x}{\partial \bar z}   +    \frac{\partial  F}{\partial y}  \frac{\partial  y}{\partial \bar z}
.\] 
We know, \[
\frac{\partial  x}{\partial \bar z}  = \frac{\partial }{\partial \bar z}  (\frac{z + \bar z}{2}) = \frac{1}{2}
.\] 
Similarly, 
\[
\frac{\partial  y}{\partial \bar z}  = \frac{\partial }{\partial \bar z}  (\frac{z - \bar z}{2i}) = \frac{-1}{2i} = \frac{i}{2}
\]

So we can write, \[
    \frac{\partial  F}{\partial \bar z} =    \frac{\partial  F}{\partial x}   \frac{1}{2}   +    \frac{\partial  F}{\partial y}  \frac{i}{2} = \frac{1}{2}\bigg(   \frac{\partial  F}{\partial x} + i  \frac{\partial  F}{\partial y} \bigg)
\]

(b). \[
    \frac{\partial  f}{\partial \bar z} = \frac{1}{2}\bigg(   \frac{\partial  f}{\partial x} + i  \frac{\partial  f}{\partial y} \bigg)
.\] 
We are give that, $f(z) = u(x,y) + iv(x,y)$

So, \[
    \frac{\partial  f}{\partial \bar z} = \frac{1}{2}\bigg(   \frac{\partial}{\partial x}(u(x,y) + iv(x,y)) + i  \frac{\partial  }{\partial y} (u(x,y) + iv(x,y))\bigg)
.\] 
\[
= \frac{1}{2}\bigg(   u_x + iv_y + i(u_y + iv_y))\bigg)
.\] 

\[
= \frac{1}{2}\bigg(   u_x + iv_y + iu_y - v_y\bigg)
.\] 
\[
= \frac{1}{2}\bigg((u_x - v_y)+ i(u_y + v_x)\bigg)
.\] 

Now if $f(z)$ satisfies the Cauchy-Riemann equations then we know that  $u_x = v_y$ and  $u_y = -v_x$ and,  \[
\frac{\partial f}{\partial \bar z} = \frac{1}{2}(0) = 0
.\] 


\subsection*{Problem 2}
(a). $f(z) = xy + iy$

Using the Cauchy-Riemann equations we see, \[
u_x = y, u_y = x
.\] 
\[
v_x = 0, u_y = 1
.\] 

So we have, $y = 1$ and $x = 0$

Which means that the function has a derivate at only  $y = 1, x= 0$. However for a functino to be analytic at a point it has to have a derivative at everypoint in some neighborhood of the point. But in this case it only has it at the point  $(0,1)$ and nowhere else. So it is not analytic anywhere.


(b). $f(z) = 2xy + i(x^2 - y^2)$
We have, \[
u_x = 2y, u_y = 2x
.\] 
\[
v_x = 2x, v_y = -2y
.\] 

So we need, $2y = -2y$ and $2x = -2x$

The only values that satisfy this are  $x = 0, y = 0$

So the functino is not analytic anywhere because it has a derivative only at  $(0,0)$ and nowhere else.

(c).  $f(z) = e^ye^{ix}$

$$f(z) = e^y(\cos(x) + i\sin(x))$$


So, \[
u_x = e^y(-\sin(x)), u_y = e^y\cos(x)
.\] 
\[
v_x = e^y(\cos(x)), v_y = e^y \sin(x)
.\] 

So we have, $u_x = v_y$ and $u_y = -v_x$

So,  \[
    2e^ysin(x) = 0 \text{ which means } sin(x) = 0
.\] 
and, \[
    2e^ycos(x) = 0 \text { which mean } cos(x) = 0
.\] 

So our values of $x,y$ for which the eqations hold are, $x = n\pi, n \in Z$ and $y = \frac{\pi}{2} + n\pi$ for $n \in Z$. So we have a derivative only exactly at these point. However the derivate doesn't exist in any neighborhood of these points which means the functino isn't analytic.

\subsection*{Problem 6}
We have, \[
g(z) = ln r + i \theta \quad (r > 0, 0 < \theta < 2\pi)
.\] 

The cauchy-reimann equation for polar coordinates tell us that, \[
    ru_r = v_{\theta}, u_{\theta} = -r v_r
.\] 

$$u_r = \frac{1}{r}, u_\theta = 0$$
$$v_r = 0, v_\theta = 1$$


So we have, $r \frac{1}{r} = 1, 1 = 1$ and $0 = 0$ 

Our functino satifies the cauchy-reimann equations at all points.

It's derivative in the domain is, $f'(z) = e^{-i\theta}(\frac{1}{r}) = \frac{1}{re^{i\theta}} = \frac{1}{z}$ 

We have $G(z) = g(z^2+ 1)$

This is a composition of functions where $f(z) = z^2 + 1$ then $G(z) = g(f(z))$

For this to be the case we need to show that both the functinos are analytic so that we can use the chain rule to differentiate them.

We first know that $f(z) = z^2 + 1$ is analytic everywhere as it satisfies the Cuachy-Remiann equations.

In the quadrant $x> 0, y > 0$ we have $Im(z^2 + 1) = 2xy > 0$. If this is the case then we know $rsin(\theta)$ has to be positive which means $ 0 < \theta < \pi < 2\pi$. Our functions are analytic here. 

So we can use the chain rule to differnetiate as follows, \[
G'(z) = g'(z^2 + 1) f'(z) = \frac{1}{z^2 + 1} f'(z)  = \frac{2z}{z^2 + 1}
.\] 


\section*{Week 5}
\subsection*{1}
We know from the Cauchy-Riemann equations in polar coordinates that $$ru_r = v_{\theta} \text{ and }  r v_r = -u_{\theta}$$

So we have, $u_r = \frac{v_\theta}{r}$ which means, 
$$ u_{rr} = \frac{1}{r}v_{\theta r} + -v_{\theta}\frac{1}{r^2}  $$ 
and, $$ r^2u_{rr} = r v_{\theta r} -v_{\theta} $$ 

We also know that $ru_r = v_\theta$

And as $u_\theta = -rv_r$ $$ u_{\theta\theta} = -rv_{r\theta}$$ 

So we get, $$ r^2u_{rr} + ru_r + u_{\theta\theta} = rv_{\theta r} - v_\theta + v_\theta -rv_{r \theta} = 0 $$ 

Similarly taking the second order partial derivatives of $v(r,\theta)$.

We first know that $v_r = -\frac{u_\theta}{r}$ 

So we write, $$ v_{rr} =  -\frac{1}{r} u_{\theta r} + u_\theta \frac{1}{r^2}$$ 
$$ r^2v_{rr} = -ru_{\theta r} + u_\theta $$ 

We also know $v_r = -u_\theta \frac{1}{r}$ so $$ rv_r = -u \theta $$ 

Laslty we can write, $$ v_{\theta\theta} = \frac{d}{d\theta} r u_r = r u_{r\theta} $$ 

So putting it all together we get, $$ r^2v_{rr} + rv_r + v_{\theta\theta} = 0 $$ 

\subsection*{Problem 1}
(a). $\exp(2 \pm 3\pi i) = -e^2$ 

We know $e^z = e^x e^{iy}$ so we get, $$ =e^2 e^{3n\pi i} $$ $n = \pm 1$ 

We can write $e^{3n\pi i} = -1$ as  $\sin(3\pi) = -1$ and  $\cos(3\pi) = 0$

So we have  $e^z = -e^2$

(b). $\exp(\frac{1}{2} + \frac{\pi i}{4}) = \sqrt{\frac{e}{2}} (1 + i)$ 

We write, $$ = e^{\frac{1}{2}} e^{i\frac{\pi}{4}} $$ 
$$ = \sqrt{e} (\cos(\frac{\pi}{4}) + i \sin(\frac{\pi}{4})) $$ 
$$ = \sqrt{e}\bigg (\frac{1}{\sqrt{2}} + i \frac{1}{\sqrt{2}}\bigg ) $$ 
$$  = \sqrt{\frac{e}{2}}(1 + i) $$ 


(c). $\exp(z + \pi i) = -\exp (z)$
$$  = e^z e^{i\pi}$$
$$ = e^z (-1) $$  as $e^{i\pi} = -1$

So we have, $$ = -e^z$$

\subsection*{Problem 6}

$$|e^{z^2}| \le e^{|z|^2}$$

We can write $z^2 = x^2 - y^2 + 2xy$ and $|z|^2 = x^2 + y^2$ 

So we have, $$ |e^{z^2}| = |e^{x^2 - y^2} e^{2xyi}| $$ 
$$ = |e^{x^2 - y^2} (\cos 2xy + i \sin 2xy)|  = e^{x^2 - y^2}$$ 

We can write, $$ e^{|z|^2} = e^{x^2 + y^2} $$ 

So now it is trivial to see why $e^{x^2 - y^2} \leq e^{x^2 + y^2}$

\subsection*{Problem 13}
 Consider the funciton $$ F(z) = U(x,y) + i V(x,y)$$ 

 To show that $U(x,y)$ and  $V(x,y)$ are harmonic it is enough to show that  $F(z)$ is analytic in domain $D$.

 To do this we need to show that it satisfies the Cauchy-Riemann equations. 

 So we need to show that, $U_x = V_y$ and  $U_y = -V_x$ 

 $$ U_x = e^u (-\sin v) v_x + (\cos v) e^u u_x $$ 
 $$ V_y = e^u (\cos v) v_y + (\sin v) e^u u_y $$ 

 As $f(z)$ is analytic we know $u_x = v_y$ and  $u_y = -v_x$
So using this we have, $$ U_x = e^u u_y (\sin v) + e^u u_x (\cos v) $$ 
$$ V_y = e^u u_x (\cos v) + e^u u_y (\sin v)$$

Now we can easily see that $U_x = V_y$

Similarly,  $$ U_y = e^u (-\sin v) v_y + e^u u_y (\cos v) $$ 
$$ V_x = e^u \cos(v) v_x + \sin(v) e^u u_x $$ 

Using our equation we get, $$ U_y = - e^u u_x(\sin v)  -e^u v_x (\cos v)$$  
$$ V_x = e^u v_x \cos (v) + e^u u_x \sin(v) $$ 

It is easy to see that $U_y = -V_x$

This hows us that our function $F(z) = U(x,y) + iV(x,y)$ is analytic in some domain  $D$.

Using theorem we know that this implies that  $U(x,y)$ and  $V(x,y)$ are harmonic in the domain  $D$.

 \subsection*{Problem 3}
 $$Log(i^3) \ne 3 Log(i)$$

 We know $Log(z) = \ln r + i \theta$

 Here, $i^3 = e^{\frac{3\pi}{2} i}$ so we have, 
 $$ Log(i^3) = \ln|1| + i {\frac{-\pi}{2}} = i \frac{-\pi}{2} \text{ as $-\pi \le \theta \le \pi$ }$$ 

 Now $3 Log(i) = 3 Log(e^{i\frac{\pi}{2}}) = 3 (i\frac{\pi}{2}) = i \frac{3\pi}{2}$ 

 So we see that in the principal branch $3 Log(i)$ lies outside the branch where  $Log(i^3)$ is defined. Which means in this branch it is not the case that $3 Log(i) = Log(i^3)$


 \subsection*{Problem 8}
 To find roots of $log(z) = i \frac{\pi}{2}$ 

 
 $$ z = e^{log(z)} = e^{i \frac{\pi}{2}} = i$$ 

 \subsection*{Problem 3}
 To show  $ \log(\frac{z_1}{z_2})  $ is not valid when we use $Log$

 Let $z_1 = re^{i\theta}$   and $z_2 = \rho e^{i \phi}$

 Then  
 $$ Log\bigg(\frac{z_1}{z_2}\bigg) = Log \bigg (\frac{r}{\rho} e^{i(\theta - \phi)}\bigg) =  ln\bigg(\frac{r}{\rho}\bigg) + i(\theta - \phi)$$ 

 However take a value of $z_1,z_2$ such that $\theta = \pi$ and $\phi = -\pi$ and $r = \rho = 1$ we have,  
 $$ Log(\frac{z_1}{z_2}) = 2\pi i$$ 

 However we know that $Log$ is only defined in the principal branch  $[-\pi, \pi]$. And even though both our values  $z_1, z_2$ was defined in the principal branch, our resultalt value $i_2\pi$ lies outside where $Log$ is defined.

 Hence  when we switch to $Log$ it is not always valid.


\subsection*{Problem 8}
(a). $z^{c_1} z^{c_2} = z^{c_1 + c_2}$

We know $z^{c_1} = e^{c_1 log(z)}$

So, 
$$ z^{c_1} z^{c_2} = e^{c_1 log(z)} e^{c_2 log(z)} = e^{c_1 log(z) + c_2log(z)}$$ 
$$ = e^{log(z)(c_1 + c_2)} = z^{c_1 + c_2} $$ 

(b). $\frac{z^{c_1}}{z^{c_2}} = z^{c_1- c_2}$ 

We know $z^{c_1} = e^{c_1 \log z}$

So, 
$$ \frac{z^{c_1}}{z^{c_2}} = e^{c_1 \log z - c_2 \log z}  = e^{\log z(c_1 - c_2)}$$ 
$$ = z^{c_1 - c_2} $$ 


(c). $(z^{c})^n = z^{cn}$


We know $z^{c} = e^{c \log z}$
$$(z^{c})^n = e^{(c \log z)n} = e^{(cn \log z)} = z^{cn}$$


\subsection*{Problem 2}
(a). Expression 4 tells us that, 
$$ e^{iz} = \cos(z) + i\sin(z) $$ 


We have to find $e^{iz_1}e^{iz_2}$

Using the expression above we get, 
$$ = (\cos(z_1) + i\sin(z_1))(\cos(z_2) + i\sin(z_2))$$ 

$$ = \cos{z_1}\cos{z_2} + i(\sin(z_2)\cos(z_1)) + i(\sin(z_1)\cos(z_2)) - \sin(z_1)\sin(z_2) $$ 
$$ = \cos{z_1}\cos{z_2}  - \sin(z_1)\sin(z_2) + i(\sin(z_2)\cos(z_1) + \sin(z_1)\cos(z_2)) $$ 


Now to expand, $e^{-iz_1}e^{-iz_2}$


$$ = e^{i(-z_1)}e^{i(-z_2)} $$
$$ = \cos(-z_1)\cos(-z_2)  - \sin(-z_1)\sin(-z_2) + i(\sin(-z_2)\cos(-z_1) + \sin(-z_1)\cos(-z_2)) $$ 
$$ = \cos(z_1)\cos(z_2)  - (-\sin(z_1)\times -\sin(z_2)) + i(-\sin(z_2)\cos(z_1) + -\sin(z_1)\cos(z_2)) $$ 

$$ = \cos(z_1)\cos(z_2)  - \sin(z_1)\sin(z_2)) - i(\sin(z_2)\cos(z_1) + \sin(z_1)\cos(z_2)) $$ 

\subsection*{Problem 5}
(a).

We know from theorem that $$\sin^2(z) + \cos^2(z) = 1$$

Dividing both sides by $\cos^2(z)$ we get, 
$$ \frac{\sin^2(z)}{\cos^2(z)} + 1 = \frac{1}{\sin^2(z)} $$ 

So we have, 
$$ 1 + \tan^2(z) = \csc^2(z) $$ 

(b)
We know from theorem that $$\sin^2(z) + \cos^2(z) = 1$$

Dividing both sides by $\sin^2(z)$ we get, 
$$ 1 + \frac{\cos^2(z)}{\sin^2(z)} = \frac{1}{\cos^2(z)} $$ 

So we have, 
$$ 1 + \cot^2(z) = \sec^2(z) $$ 


\subsection*{Problem 9}

(a). We know that $|\sin z|^2 = \sin^2 x + \sinh^2 y$

We see that, 
$$ \sinh^2 y \le |\sin z|^2 \implies |\sinh y| \le |\sin z|$$ 


To show the second part of the inequality we see that, 
$$ |\sin z|^2  = |\sin(x + iy)|^2$$
$$ =|\sin x \cos(iy) + \cos x \sin(iy) ^2$$ 
$$ =|\sin x \cosh(y) + i(\cos x \sinh(y) ^2$$ 
$$ =|\sin^2 x \cosh^2(y) + (\cos^2 x \sinh^2(y) $$ 
$$ \le|\sin^2 x \cosh^2(y) + (\cos^2 x \cosh^2(y) $$ 
$$ \cosh^2 y$$
as $\sinh^2 y \le \cosh^2 y$

So we have, $|\sin z | \le \cosh y$

So, 
$$ |\sinh y| \le |\sin z| \le \cosh y $$ 

(b). Similar to the previous one we know that $|\cos z|^2 = \cos^2 x + \sinh^2 y$

We get, 
$$ \sinh^2 y \le |\cos z|^2 \implies |\sinh y| \le |\cos z| $$


Using the same process as (a) we get, $|\cos z | \le \cosh y$

So,  
$$ |\sinh y| \le |\cos z| \le \cosh y $$ 




\end{document}
