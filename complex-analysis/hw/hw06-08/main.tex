\documentclass[a4paper]{report}
\input{preamble.tex}
\title{MATH 4320 HW06-8}
\author{Aamod Varma}
\begin{document}
\maketitle


\subsection*{Problem 2}

(a). We know $\sinh(z) =  \frac{e^{z}-e^{-z}}{2}$ and $\cosh(z) =\frac{e^{z}+e^{-z}}{2} $

So, $$2\sinh(z)\cosh(z) = 2 \frac{e^{z}-e^{-z}}{2}\frac{e^{z}+e^{-z}}{2} $$
$$ =  2\frac{e^{2z} + 1 - 1 - e^{-z}}{4}$$ 
$$ = \frac{e^{2z} - e^{-2z}}{2}$$ 
$$ = \sinh(2z) $$ 



(b). $\sin(2z) = 2\sin(z)\cos(z)$

We know that  $-i\sinh(iz) = \sin(z)$ and  $\cosh(iz) = \cos(z)$. Let $iz = z'$ then,  
$$ \sinh(z') = \frac{-\sin(\frac{z'}{i})}{i} $$  
$$ \cosh(z') = \cos(\frac{z'}{i}) $$  

So, $$2\sinh(z')\cosh(z') = -2\frac{\sin(\frac{z'}{i})}{i}\cos(\frac{z'}{i}) $$

$$= -2\sin\frac{\frac{2z'}{i}}{i}$$
$$= \sinh(2z')$$

\subsection*{Problem 6}
(a). $|\cosh z|^2 = \sinh^2 x + \cos^2 y$

This means that  
$$ \sinh^2 x \le |\cosh z|^2 $$ 
$$ |\sinh x| \le |\cosh z| $$ 


Now we need to show that $|\cosh z| \le |\cosh x|$. We know that, 
$$ |\cosh z| = |\cosh x \cos y + i \sinh x \sin y| $$ 
$$ \cosh^2 z = \cosh^2 x \cos^2 y -  \sinh^2 x \sin^2 y $$ 
$$ \cosh^2 z  + \sinh^2 x \sin^2 y  = \cosh^2 x \cos^2 y $$ 
So, 
$$ \cosh^2 y \le \cosh^2x \cos^2 y $$ 

But we know that $\cos^2y \le 1$ so, 
$$ \cos^2 y \le \cosh^2 y $$  or 
$$ |\cosh z | \le |\cosh x| $$ 


So we've shown that 
$$ |\sinh x| \le |\cosh z| \le |\cosh x| $$ 


(b). $|\sinh x| \le |\cosh z| \le \cosh x$


\subsection*{Problem 14}
We have $\cosh^2 x - \sinh^2 x = 1 $

(a). 
\begin{align*}
    \cosh^2 z - \sinh^2 z &= (\cosh x \cos y + i \sinh x \sin y)^2 - (\sinh x \cos y + i \cosh x \sin y)^2\\
                          &= (\cosh^2 x \cos^2 y - \sinh^2 x \sin^2 y + 2i\cosh x \cos y \sinh x \sin y)\\
                          &-(\sinh^2 x \cos^2 y - \cosh^2 x \sin^2 y + 2i\cosh x \cos y \sinh x \sin y)\\
                          &=(\cosh^2 x \cos^2 y - \sinh^2 x \sin^2 y) - (\sinh^2 x \cos^2 y - \cosh^2 x \sin^2 y)\\
                          &=(\cos^2y (\cosh^2 x - \sinh^2 x)) + (\sin^2y (\cosh^2 x - \sinh^2 x))\\
                          &=(\cos^2y) + (\sin^2 y)\\
                          &=1
\end{align*}


(b). We have $\sinh x + \cosh x = e^x$


\begin{align*}
    \sinh z + \cosh z &= \sinh x \cos y + i\cosh x \sin y +  \cosh x \cos y + i\sinh x \sin y\\
                      &= \cos y(\sinh x + \cosh x) + i \sin y(\sinh x + \cosh x)\\
                      &= \cos y(e^x) + i\siny(e^{x})\\
                      &= e^{x}(\cos y + i \sin y)\\
                      &= e^{x}e^{iy} =  e^{x + iy}\\
                      &= e^{z}
\end{align*}


\subsection*{Problem 2}

\begin{align*}
    \sin z &= 2\\
    z &= \sin^{-1}(2)\\
      &= -i \log [2i + (1- 4)^{\frac{1}{2}}]\\
      &= -i \log [i(2 + \sqrt{3})]\\
      &= -i (\ln [2 + \sqrt{3}] + i(\frac{\pi}{2} + 2n\pi))\\
      &=  (\frac{\pi}{2} + 2n\pi) -i\ln [2 + \sqrt{3}]\\
\end{align*}

            

\subsection*{Problem 2}
(a). $\int_{{0}}^{{1}} {1+it}^2 \: dt $
\begin{align*}
&=\int_{{0}}^{{1}} {1 - t^2 + 2it} \: d{t} {}\\
&=(t - \frac{t^{3}}{3}+it^2)]_0^1\\
&=(1 - \frac{1}{3} + i) - (0)\\
&=\frac{2}{3}+ i
\end{align*}

(b). $\int_{{1}}^{{2}} {\frac{1}{t}-i}^2 \: d{t} {}$
\begin{align*}
    &=\int_{{1}}^{{2}} {\frac{1}{t^2} - 1 - \frac{2i}{t}} \: d{t} {}\\
    &=\frac{-1}{t} - t - 2i \ln(t)]_1^2\\
    &=(-\frac{1}{2} - 2 - 2i\ln(2)) - (-1 -1)\\
    &=(-\frac{1}{2}-2-i\ln(4) + 2)\\
    &=-\frac{1}{2}-i\ln4
\end{align*}

(c). $\int_{{0}}^{{\frac{\pi}{6}}} {e^{i2t}} \: d{t}$


\begin{align*}
    &=\int_{{0}}^{{\frac{\pi}{6}}} {e^{2it}} \: d{t} {}\\
    &=\frac{e^{2it}}{2i}]_0^{\frac{\pi}{6}}\\
    &=(e^{i\frac{\pi}{3}} \frac{1}{2i}) - (\frac{1}{2i})\\
    &=\frac{1}{2i} (e^{i\frac{\pi}{3}} - 1)\\
    &=\frac{1}{2i}(\cos(\frac{\pi}{3}) + i\sin(\frac{\pi}{3}) - 1)\\
    &=\frac{1}{2i}(\frac{1}{2} + i\frac{\sqrt{3}}{2} - 1)\\
    &=\frac{1}{2i}(-\frac{1}{2} + i\frac{\sqrt{3}}{2})\\
    &=\frac{\sqrt(3)}{4} + \frac{i}{4}
\end{align*}

(d). $\int_{{0}}^{{\infty}} {e^{-zt}} \: d{t} {}$
\begin{align*}
    &=\int_{{0}}^{{\infty}} {e^{-zt}} \: d{t} {}\\
    &=\frac{e^{-zt}}{-z}]_0^{\infty}\\
    &=(\frac{-1}{z}) (\frac{1}{e^{z\infty}} - \frac{1}{e^{0}})\\
    &=(-\frac{1}{z})(-1)=\frac{1}{z}
\end{align*}


\subsection*{Problem 3}

\begin{align*}
&\int_{{0}}^{{2\pi}} {e^{im\theta}e^{-in\theta}} \: d{\theta} {}\\
&=\int_{{0}}^{{2\pi}} {e^{\theta(im - in)}} \: d{\theta} {}\\
&=\frac{e^{i\theta(m - n)}}{i(m - n)}]_0^{2\pi}\\
&=\frac{1}{i(m-n)} (e^{i2\pi(m-n)} - e^{0})
\end{align*}

We know that $e^{0 + 2n\pi} = e^{0} = 1$

So if $m \ne n$ we have, 
$$ = \frac{1}{i(m-n)} (1 - 1) = 0 $$ 

If $m = n$ then we have, 
$$ \int_{{0}}^{{2\pi}} {e^{i\theta(0)}} \: d{\theta} {} $$ 
$$ =2\pi $$ 


\subsection*{Problem 2}
We have $C : |z| = 2$ where $Re(z)$ is positive we have, 
$$ z= z(\theta) = 2e^{i\theta}, \bigg ( -\frac{\pi}{2} \le \theta \le \frac{\pi}{2} \bigg) $$  and 
$$ z = Z(y) = \sqrt{4 - y^2} + iy, (-2 \le y \le 2) $$ 

We need to show that $Z(y) = z[\phi(y)]$ where, 
$$ \phi(y) = \arctan \frac{y}{\sqrt{4 - y^2}}, \bigg( \frac{\pi}{2} < \arctan t < \frac{\pi}{2}\bigg)$$ 


We are given that $\theta = \phi(y) = \arctan \frac{y}{\sqrt{4 - y^2}}$. We can write this as, 
$$ \tan \theta  = \frac{y}{\sqrt{4 - y^2}}$$ 
Using the property $sec^2 \theta - tan^2 \theta = 1$ we can say that, 
$$ \sec \theta = \frac{2}{\sqrt{4 - y^2}} $$  or that, 
$$ \cos \theta = \frac{\sqrt{4 - y^2}}{2}\text{ cos is always positive in this region} $$  and similarly, 
$$ \sin \theta = \frac{y}{2} \text{ here y goes from -2 to 2 }$$ 

So we have, 
\begin{align*}
    z &= 2e^{i\theta}\\
      &= 2(\cos \theta + i \sin \theta)\\
      &= 2(\frac{\sqrt{4 - y^2}}{2} + i \frac{y}{2})\\
      &= \sqrt{4 - y^2} + iy, \bigg( -2 \le y \le 2 \bigg)
\end{align*}

We have, 
$$ \tan \phi(y) = \frac{y}{\sqrt{4 - y^2}}  $$ 

$$ \frac{d}{dy} \tan \phi(y) = \sec^2(\phi) \frac{d}{dy}\phi = \frac{\sqrt{4-y^2} + \frac{y^2}{\sqrt{4-y^2}}}{4-y^2} $$ 
$$\phi'(y) = \frac{1}{sec^2\phi} \frac{4}{\sqrt{4 - y^2}} > 0 $$ 

As both the terms are greater than zero.



\subsection*{Problem 6}
(a). The arc intersects the real axis when $y(x) = 0$. So when $0 < x \le 1$ we have, 
$$ y(x) = x^{3}\sin(\pi / x) $$ 
We need this to be equal to zero.

So either $x^{3} = 0$ or $\sin(\pi /x) = 0$. However $x^{3} = 0 \implies x = 0$ however $x \ne 0$ so  $\sin(\pi /x) = 0$. We know that  $\sin(\theta) = 0$ when $\theta = n\pi, n = 0,1,2...$ 
So we have $n\pi  = \frac{\pi}{x}$, 
$$ x = \frac{\pi}{n\pi} = \frac{1}{n}$$ 

When $y(x) = 0$ we knwo that $z = x$ so when $z = \frac{1}{n}$ we have $z = x + 0 = x$ where $n = 1,2,..$


(b). For  $C$ to be a smooth arc we need to show that it is continous over the domain $[0,1]$. Or that  $y'(x)$ is defined and exists in this region.

$$ y(x) = x^{3}\sin(\frac{\pi}{x}) ,  0 < x \le 1$$ 
$$ y'(x) = 3x^2(\sin\frac{\pi}{x})  - x \cos(\frac{\pi}{x})\pi$$ 

So we know that $y'(x)$ exists and is continuous and non-zero when $x \in (0,1)$ and is $0$ when $x = 1$. 


Now we need to show continuity of $y$ at $x = 0$. Or that,  
$$ \lim_{x \to 0} y(x) = y(0) = 0 $$ 

Using the epsion-delta definition we need to show that, $\forall \epsilon, \exists  \delta$ s.t.
$$ |x^{3}\sin(\frac{\pi}{x}) - 0| < \epsilon \text{ for some } | x - 0| < \delta $$ 

We know that $|x^3 \sin(\pi /x)| \le |x^{3}|$ and we can make $x$ arbitrarily small such that $|x^{3}| < \epsilon$

So if we choose  $\delta = \epsilon^{\frac{1}{3}}$ we have $|x^{3}| < \epsilon, \forall \epsilon$ which means that 
$$  |x^{3}\sin(\pi / x) | < \epsilon, \forall \epsilon$$ 

This shows that $y$ is cont. at $x=0$.

Now we need to show that  $y'(x)$ exists and is equal to 0. Or that, 

Similarly we can show that $\lim_{x \to 0} y'(x) = 0$  by taking $\delta = (\frac{e}{3})^{\frac{1}{2}}$ as we can bound $|3x^2(\sin \pi /x) - x \cos(\pi / x)\pi| < 3x^2$ because $x\cos(\frac{\pi}{ x}) \pi$ is always positive as $x$ tends to 0 from the positive real side.



\subsection*{Problem 2}
We need to find $\int_C f(z) dz$

(a). $f(z) = z - 1$ where  $z = 1 + e^{i\theta}, (\pi \le \theta \le 2\pi)$

So $dz = ie^{i\theta} d\theta$ and $f(z) dz = e^{i\theta} ie^{i\theta}d\theta$ 

We get, 
$$ \int_{{\pi}}^{{2\pi}} {ie^{2i\theta}} \: d{\theta} {} $$ 
$$= \frac{1}{2}[e^{2i\theta}]_{\pi}^{2\pi} $$ 
$$ = \frac{1}{2}0 = 0 $$ 


(b). $z = x, (0 \le x \le 2)$. We have $dz = dx$ so,  
$$ \int_0^2 x - 1 dx $$ 
$$  = [\frac{x^2}{2} - x]_0^2 =  (0 - 0) = 0$$ 


\subsection*{Problem 6}
So we have $C: $ semicircle $z = e^{i\theta}$ and $f(z)$ is the principal branch $e^{i Log z}$ 
$dz = ie^{i\theta}d\theta$ so we get, 
$$ \int_{{0}}^{{\pi}} {ie^{i(Log (e^{i\theta}) + \theta)}} \: d{\theta} {} $$ 
$$ = \int_{{0}}^{{\pi}} {ie^{i(i\theta + \theta)}} \: d{\theta} {} $$ 
$$ = \int_{{0}}^{{\pi}} {ie^{\theta(i -1)}} \: d{\theta} {} $$ 
$$ = \frac{ie^{\theta(i -1)}}{i-1}]_0^\pi$$ 
$$ =-\frac{1}{2} (i - 1) (e^{\pi i - \pi} - 1) $$ 
$$  = -\frac{1}{2} (1 - i) (e^{-\pi} + 1) $$ 

\subsection*{Problem 11}
(a). $z = 2e^{i\theta}$ so $dz = 2ie^{i\theta}d\theta$. Given $f(z) = \bar z$

If  $z = 2e^{i\theta}$ then $\bar z = 2e^{-i\theta}$. So we have, 
$$ \int_{{C}}^{{}} {\bar z} \: d{z} {} $$ 
$$ = \int_{{-\frac{\pi}{2}}}^{{\frac{\pi}{2}}} {2e^{-i\theta}2ie^{i\theta}} \: d{\theta} {} $$ 
$$ = \int_{{-\frac{\pi}{2}}}^{{\frac{\pi}{2}}} {e^{-i\theta}4ie^{i\theta}} \: d{\theta} {} $$ 
$$ =[ 2i\theta]_{-\frac{\pi}{2}}^{\frac{\pi}{2}} $$ 
$$ = 4\pi i $$ 

(b). $z = \sqrt{4 - y^2} + iy$ So $\bar z = \sqrt{4 - y^2} - iy$ and 
$$ dz = (-\frac{y}{\sqrt{4-y^2}} + i) dy $$ 

So we get, 
$$ \int_{{-2}}^{{2}} {(\sqrt{4 - y^2} - iy)(-\frac{y}{\sqrt{4 - y^2}} +i)} \: d{y} {} $$ 
$$ \int_{{-2}}^{{2}} {-y + i\sqrt{4- y^2} + \frac{iy^2}{\sqrt{4 - y^2}} + y} \: d{y} {} $$ 
$$ \int_{{-2}}^{{2}} { i\sqrt{4- y^2} + \frac{iy^2}{\sqrt{4 - y^2}} } \: d{y} {} $$ 
$$ \int_{{-2}}^{{2}} { \frac{4i}{\sqrt{4 - y^2}} } \: d{y} {} $$ 

Taking $y = 2\sin(\theta)$ and parameterizing it from $-\frac{\pi}{2} \le \theta \le \frac{\pi}{2} $ and taking $dy = 2\cos(\theta)$ we have, 
$$ \int_{{-\frac{\pi}{2}}}^{{\frac{\pi}{2}}} {\frac{4}{2\cos\theta}} \: d{\theta} {} $$ 
$$ \int_{{-\frac{\pi}{2}}}^{{\frac{\pi}{2}}} {\frac{4i}{2\cos\theta} 2\cos \theta } \: d{\theta} {} $$ 
$$ \int_{{-\frac{\pi}{2}}}^{{\frac{\pi}{2}}} {4i} \: d{\theta} {} $$ 
$$ = 4\pi i $$ 


\subsection*{Problem 1}
(a). We know that, 
$$ \bigg | \int_C \frac{z + 4}{z^{3} - 1} dz\bigg | \le \int_C \bigg | \frac{z + 4}{z^{3} - 1}\bigg | dz \le \int_C | M | dz$$ 
Where $M$ bounds our function.

We know that $ |z + 4| \le |z| + |4| = 6$ and  $||z^{3}| - |1|| \le |z^{3} - 1| < |z^{3}| + |1| $. So $7 \le |z^{3} - 1| \le 9$. So we can write, 
$$ \bigg | \frac{z + 4}{z^{3} - 1} \bigg| \le \frac{6}{7} $$ 

Which means that the integral is bounded by, 
$$ \int_C |M| = \int_C \bigg|\frac{6}{7}\bigg|  = \frac{6\pi}{7}$$ 



(b). We know that, 
$$ \bigg | \int \frac{dz}{z^2 - 1} \bigg | \le  \int\bigg | \frac{1}{z^2 - 1} \bigg |dz \le \int_C |M| dz$$ 

Where $M$ bounds our function.

We know that $ ||z^2| - |1||\le z^2 - 1$ so, $4 - 1 \le z^2 - 1$ which means that 
$$ \frac{1}{z^2 - 1} \le \frac{1}{3} $$ 

Where $|M| = \frac{1}{3}$ so we have, our integral is bounded above by, 
$$ \int_C \bigg |\frac{1}{3} \bigg | = \frac{\pi}{3} $$


\subsection*{Problem 4}

$$ \bigg | \int_C \frac{2z^2 - 1}{z^{4} + 5z^2 + 4} dz\bigg | \le \int_C \bigg | \frac{2z^2 - 1}{z^{4} + 5z^2 + 4}\bigg | dz \le \int_C | M | dz$$ 

Where $M$ is the upper bouned for our function.

First we know that $|2z^2 - 1|\le  |2z^2| + |1| = 2R^2 + 1$ and that 
$$ ||z^{4} + 5z^2| - |4||  \le z^{4} + 5z^2 + 4$$ 
$$  |||z^{4}| - |5z^2|| - |4|| \le z^{4} + 5z^2  + 4$$ 
$$ R^{4} - 5^{R^2} - 4 \le z^{4}+ 5z^2 + 4 $$ 
$$ (R^2 - 1)(R^2 - 4) \le z^{4}+ 5z^2 + 4 $$ 

So we have $|M| = \frac{2R^2 + 1}{(R^2 - 1)(R^2 - 4)}$ 

And as $\int_C dz = \pi R$ we can upperbound our integerarl by, 
$$ \frac{\pi R (2R^2 + 1)}{(R^2 - 1)(R^2 - 4)} $$ 


\subsection*{Problem 4}
Our contour is any integral that extends from $z = -3$ to $z = 3$. Which  means that $r = 3$ and $\pi \le \theta \le 2\pi$
We have 
\begin{align*}
    z &= 3e^{i\theta}\\
    f(z) &= z^{\frac{1}{2}} = \sqrt{3} e^{i\frac{\theta}{2}}\\
    dz &= 3ie^{i\theta}
\end{align*}

So, 
\begin{align*}
&\int_{\pi}^{2\pi} \frac{3}{2}\sqrt{3}i e^{i\frac{\theta}{2}}e^{i\theta} \: d\theta\\
&= 2 \sqrt{3} e^{i\frac{3\theta}{2}}]_{\pi}^{2\pi}\\
&= 2\sqrt{3} ((-1 + 0 -)( 0 - i))\\
&= 2\sqrt{3}(-1 + i)
\end{align*}

If we had gone around $C_2-C_1$ our bounds would have been $0$ to $2\pi$ which would result in a value of $-4\sqrt{3}$



\subsection*{Problem 2}
(a). All we have to show is that $f(z)$ is analytic throughotut the region between $C_1$ and $C_2$. 

We know that $\frac{1}{3z^2+1}$ is analytic except at $z = i\frac{1}{\sqrt{3}}$. We know that this lies outside the region (inside the square). So now using the priciple of deformation we know that the integral is same for $C$ and $C_2$.


(b). We have $\sin(\frac{z}{2}) = 0$ so $\frac{z}{2} = n\pi$ or $z = 2n\pi$. For $n = 0$ it lies outside our region. If $n > 0, z$ lies outside our contour as well. So we are able to deform $C_2$ to $C_1$. 

(c). We have singularity only when  $z = 0$ which is outisde our region. Hence we can deform our regino and preserve the integral.

\subsection*{Problem 5}
We are given that a function $f$ is entire. We see that the contour $C_3$ and $C_1$ are positive oriented and define a simple closed curve. According to Cauchy-Goursat theorem we know that, 
$$ \int_{C_2+C_3}f(z) =0 $$  if $f$ is entire.

So, $$\int_{C_2} f(z) dz+ \int_{C_3} f(z) dz= 0$$
$$\int_{C_2} f(z) dz =-  \int_{C_3} f(z) dz$$

Similarly we see that $C_1$ is negatively oriented while $C_3$ is positive oriented. So the contour $C_3 - C_2$ is a closed contour positive ortiented. Which means that, 

$$ \int_{C_3-C_2}f(z) =0 $$
$$\int_{C_3} f(z) dz- \int_{C_1} f(z) dz= 0$$
$$\int_{C_3} f(z) dz = \int_{C_1} f(z) dz$$


So adding both sides we get, 
$$ \int_{C_1} f(z) dz + \int_{C_2} f(z) dz= 0 $$ or, 
$$ \int_{C_1 + C_2} f(z) dz  = 0 $$
$$ \int_{C} f(z) dz  = 0 $$

\end{document}



