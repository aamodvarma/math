\chapter{Residues And Poles}
\section{Isolated Singular Points}

A singular point $z_0$ is isolated if $\exists ,0 < | z- z_0| < \epsilon$  of $z_0$ where $f$ is analytic.

\begin{eg}
   
   $$ f(z) = \frac{z - 1}{z^{5}(z^{2}+9)} $$ 


   We have $z = 0$ and $z = \pm 3i$. 
\end{eg}
\begin{eg}
   $$F(z) = Log(z) = \ln(r) + i \theta, -\pi < \theta < \pi$$

   Here $r = -1$ is always going to be a singular point. So we can't draw an epsilon neighborhood around a singular points that is isolated.
\end{eg}

\begin{eg}
   $$ f(z) = \frac{1}{\sin(\pi /2)} $$ 
   We have $z_n = \frac{1}{n}$ is a singular point. We are able to take $\epsilon = \frac{1}{2}\big|\frac{1}{n+1} - \frac{1}{n} \big |$

However $z = 0$ is not an isolated singular point because for any $\epsilon$ we can choose $N$ such that $|\frac{1}{N}| < \epsilon$
\end{eg}

\begin{eg}
   Isolated singular point at $z_0 = \infty$ is when $R_1 < |z| < \infty$
\end{eg}

\section{Residues}
When $z_0$ is an isolated singular point of $f$ there is a positive number $R$ such that $f$ is analytic at each point $z$ for which $0 < |z - z_0| < R$. So $f(z)$ has a Laurent series. 
$$ f(z) = \sum_{n=0}^{\infty} a_n (z - z_0)^{n} + \sum_{n=1}^{\infty} \frac{b_n}{(z - z_0)^{n}} $$ 

Where, 
$$ b_n = \frac{1}{2\pi i}\int_C \frac{f(z)}{(z - z_0)^{-n + 1}} dz $$ 

So, 
$$ b_1 = \frac{1}{2 \pi i} \int_C f(z) dz $$ 

This means that the contour integral of $f(z)$ around $C$ is, 
$$ \int_C f(z) dz = 2\pi i b_1$$ 

\begin{eg}
   
   $$ \int_C \frac{e^{8}-1}{z^{4}} dz\text{\qquad $C: |z| = 1$} $$ 

   First we take the Laurent series around $z = 0$. The Laurent series expansion of the function is, 
$$ \frac{1}{z^{4}}(\sum_{n=0}^{\infty} \frac{z^{n}}{n!} - 1) = \frac{1}{z^{4}} + \dots + \frac{1}{3! z} + \dots - 1 $$ 

So we see that our $b_1$ term is equal to $\frac{1}{3!}$. Hence we have, 
$$ \int_C f(z) = 2\pi i \frac{1}{3!} =  \frac{\pi i}{3} $$ 
\end{eg}

\begin{eg}
   $$\int_C \frac{dz}{z(z - 2)^{5}} \text{\qquad $C: |z - 2| = 1$}$$

   First we see that $R$ is around the singular point $2$ hence we need the expansion around $2$. We have, 
   $$ \frac{1}{z(z-2)^{5}} = \frac{1}{2(z - 2)^{5}} \sum_{n=0}^{\infty} (-1)^{n} \frac{(z - 2)^{n}}{2^{n}} $$ 

   Here $b_1$ is when $n = 4$.
\end{eg}

\section{Cauchy Residue Theorem}


$$ \int_C f(z) dz = 2 \pi i \sum_{k = 1}^{n} Res_{z - z_k} f(z) $$ 

\section{Residue at Infinity}
Let $f$ be analytic $R_1 < |z| < \infty$. Such that the point at $\infty$ is said to be an isolated point. Let $C_0 : |z| = R_0$ in the clockwise direction.

So, 
$$ \int_{C_0} f(z) dz = 2\pi i Res_{z = \infty}(f(z))$$ 



\section{Three types of Isolated singular points}
\begin{enumerate}
   \item \textbf{Removable Singular Points:} When every $b_n$ is zero so that, 
   $$ f(z) = \sum_{n=0}^{\infty} a_n(z- z_0)^{n} $$ 
   Here $z_0$ is known as a removable singular point. So the residue of a removable singular point is always $0$.
   \item \textbf{Essential Singular Points:} If an infinite number of  $b_n$ are non-zero. Then $z_0$ is a essential singular point.
   \item \textbf{Poles of Order m:} If the principal part of $f$ at $z_0$ contains at least one nonzero term but the number of terms is finite.  So  $\exists m$ such that $b_{m+1} = \dots = 0$
\end{enumerate}


\section{Residue at Poles}
\begin{theorem}
   If $z_0$ is an isolated singular point of $f$ then the following are equivalent, 
   \begin{enumerate}
      \item $z_0$ is a pole of order $m$ of $f$ 
      \item $f(z) $ can be written as, 
      $$ f(z) = \frac{\phi(z)}{(z - z_0)^{m}} $$ 
      where $\phi(z)$ is analytic and nonzero at $z_0$
   \end{enumerate}
\end{theorem}
\begin{proof}
   We can write $f(z)$ as, 
   \begin{align*}
      f(z) &= \sum_{n=0}^{\infty} a_n(z - z_0)^{n} + \frac{b_1}{(z - z_0)} + \dots + \frac{b_m}{(z - z_0)^{m}}\\
           &= \frac{1}{(z - z_0)^{m}} \sum_{n=0}^{\infty} a_n(z - z_0)^{n+m} + b_1(z - z_0)^{m-1} + \dots
   \end{align*}

   We see that the summation that we have is analytic as it is just a Taylor expansion (hence a polynomial). So we have, 
   $$ \phi(x) = \begin{cases} f(z) (z - z_0)^{m} &  z \ne z_0 \\ b_m &  z = z_0 \end{cases}  $$ 


\end{proof}

\begin{theorem}
   If $(a)$ and $(b)$ are true then, 
   $$ Res_{z = z_0} f(z) = \phi(z_0)\text{ if $m = 1$ }$$
   $$ Res_{z = z_0} f(z) = \frac{\phi^{(m+1)}(z_0)}{(m-1)!} \text{ if $m = 2,3,\dots$ }$$
\end{theorem}

\begin{proof}
   If $m = 1$ we have, 
   $$ f(z) = \sum_{n=0}^{\infty} a_n(z - z_0)^{n} + \frac{b_1}{(z - z_0)} $$ 
   So 
   $$ (z - z_0)f(z) =\phi(z) = b_1 \text{ if $z = z_0$ } $$


   Now if $m \ne 1$ we have, 
   $$ f(z) = \frac{1}{(z - z_0)^{m}} \bigg ( \sum_{n=0}^{\infty} a_n(z - z_0)^{n + m} + b_1(z - z_0)^{m - 1} + \dots + b_m \bigg) $$ 
   We can take the derivative $m - 1$ times to isolate $b_1$ which gets us, 
   \begin{align*}
      \phi^{m-1}(z) &= \dots + (m-1)! b_1 \\
      \phi^{m-1}(z_0) &=0 + (m-1)! b_1  
   \end{align*}

   So we have, 
   $$ b_1 = \frac{\phi^{m -1}(z)}{(m - 1)!} $$ 
\end{proof}


\begin{eg}
   
   $$ f(z) = \frac{z + 4}{z^2 + 1} = \frac{z + 4}{(z - i)(z + i)} $$ 

   So we have, 
   $$ \phi(z) = \frac{z + 4}{(z + i)} $$ 

   Which means our residue, 
   $$ Res_{z = i}f(z) = \phi(i) $$ 

\end{eg}



\begin{eg}
   
   $$ f(z) = \frac{1 - \cos(z)}{z^{3}}$$


\end{eg}


\section{Zeroes of Analytic Functions}
\begin{theorem}
   If $f$ is analytic at $z_0$ then the following are equivalent, 
   \begin{enumerate}
      \item $f$ has a zero of order $m$ at $z_0$
      \item $f(z) = (z - z_0)^{m}g(z)$, $g(z)$ is analytic, $g(z_0) \ne 0$
   \end{enumerate}
\end{theorem}

\begin{proof}
   $(a) \implies (b)$

   We have  $$f(z) = \sum_{n=0}^{\infty} a_n(z - z_0)^{n} = a_0 + a_1(z - z_0) + \dots$$

   Because $z = z_0$ is a zero we have, 
   $$ f(z_0) = 0 $$ 
\end{proof}


\begin{theorem}
   Suppose, 
   \begin{enumerate}
      \item $f$ is  analytic at $z_0$ 
      \item $f(z) = 0$ at each point $z$ of a domain $D$ or segment $L $ containing $z_0$ 
         
         Then $f(z) \equiv 0$ in the neighborhood
   \end{enumerate}
\end{theorem}

\begin{proof}
\end{proof}

