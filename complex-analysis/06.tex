
\begin{property}
   $\log z_1z_2 = \log z_1 + \log z_2$
\end{property}
\begin{proof}
   
   $$ \log z_1z_2 = \ln(r_1r_2) + i(\theta_1 + \theta_2 + 2n\pi) $$ 
   $$= \log z_1z_2 = \ln(r_1)+ \ln(r_2) + i(\theta_1 + \theta_2 + 2n\pi) $$ 
   $$= \log z_1z_2 = \ln(r_1) + i(\theta_1 + 2n\pi) +  \ln(r_2) + i(\theta_2 + 2n\pi) $$ 
   $$= \log z_1z_2 = \ln(r_1) + i(\theta_1 + 2n\pi) +  \ln(r_2) + i(\theta_2 + 2n\pi) $$ 
   $$= \log z_1z_2 = \log z_1 + \log z_2 $$ 
\end{proof}

\begin{property}
   $\log |z_1z_2| = \log |z_1| + \log |z_2|$
\end{property}
\begin{property}
   $\log (\frac{z_1}{z_2)} = \log (z_1) - \log(z_2)$
\end{property}

\begin{property}
   $z^n = e^{n \log(z)}$
\end{property}

\section{Power Function}
We have a complex number $c$ and we have $f(z) = z^c$. By definition we have $z^c = e^{c\log z}$

The derivative is $\frac{d}{dz} f(z) = \frac{d}{dz} (z^c)$
$$ \frac{d}{dz} e^{c \log z} = e^{c \log z} \frac{d}{dz} c \log z = e^{c\log z}  \frac{c}{z} $$ 


But we can write $ \frac{e^{c \log z} c}{e^{\log z}} = ce^{(c -1)\log z} = cz^{c-1}$. The principal value of $z^{c} = e^{cLog z}$


If the function is $f(z) = c^z$ then we have 
$$ \frac{d}{dz} c^z = \frac{d}{z} e^{z \log c} = e^{z \log c} \frac{d}{dz} z \log c = e^{z \log c} \log c = c^{z}\log c$$ 

\section{Trignometric Function}

We know that $e^{i\theta} = \cos \theta + i \sin \theta$ and $ e^{-i \theta} = \cos \theta - i \sin \theta$. So we can write, 
$$  \cos \theta = \frac{e^{i\theta} + e^{-i \theta}}{2} $$  and, 
$$\sin \theta = \frac{e^{i\theta} - e^{-i \theta}}{2i}$$  

We have $\frac{d}{dz} \sin z = \cos z$ and $\frac{d}{dz} \cos z = -\sin z$
 
\begin{property}
   $\sin(-z) = -\sin(z)$ and $\cos(-z) = \cos(z)$
\end{property}
\begin{property}
   $\sin(z_1 + z_2) = \sin z_1 \cos z_2 + \cos z_1 \sin z_2 $
\end{property}
\begin{property}
   $\sin(2z) = 2 \sin(z)\cos(z)$
\end{property}
\begin{property}
   $\sin(z + \frac{\pi}{2}) = \cos(z)$
\end{property}

% In the real plane $\sin(x)$ and  $\cos(x)$ are bounded by $1$.  

Consider the hyperbolic sin and cos functions,$$\sinh z = \frac{e^{z} - e^{-z}}{2}, \cosh z = \frac{e^{z} + e^{-z}}{2}$$

We can write $\sin z = \sin(x + iy)$. Now expanding this we get,  
$$ \sin(x)\cos(iy) + \cos(x)\sin(iy) = \sin(x) \cosh(y) + i\cos x \sinh(y)$$ 
And we have, 
$$ |\sin z|^2 = \sin^2 x + \sinh^2 y $$ 
$$ |\cos z|^2 = \cos^2 x + \cosh^2 y $$ 




\section{Inverse Trignometric Functions}
The function is $w = f(z) = \sin^{-1}z$. So we have $$\sin(w) = z = \frac{e^{iw} - e^{-iw}}{2}$$

We know $2iz = (e^{iw} -  e^{-iw}) \times  e^{iw}$, 
$$ 2iz e^{iw}= e^{2iw} - e^{0} $$ 
$$ e^{iw}^2 - 2ize^{iw} - 1 = 0 $$

Solving this we get, 
$$ e^{iw} = iz \pm (1-z^2)^{\frac{1}{2}} $$ 




