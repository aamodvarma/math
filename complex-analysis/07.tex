\chapter{Integrals}

Consider $f(z) = f(x + iy) = u(x,y) + iv(x,y)$. We can write this as, 
$$ w(t) = u(t) + iv(t) $$ 
$$ w'(t) = u'(t) + iv'(t)  $$ 


\begin{eg}
   $\frac{d}{dt} (w(t))^2 = \frac{d}{dt} (u + iv)^2$
   \begin{align*}
    &= \frac{d}{dt}(u^2 - v^2 + 2uvi) \\
    &= 2uu' - 2vv' + i(2u'v + 2uv')  \\
    &= 2(u + iv)(u' + iv')  \\
    &= 2w(t) w'(t)  
   \end{align*}
\end{eg}

\section{Definite Integrals}
The integral of $w(t)$ with respect to  $t$ is, 
$$ \int_a^b w(t)dt = \int_a^b u(t) dt + i\int_a^b v(t) dt $$ 

\begin{ex}
   Find $c$ such that,
   $$ \int_a^b w(t) dt = w(c) (b-a) \text{ where } w(t) = e^{it}, a= 0, b = 2\pi$$ 

   \begin{solution}
      We have, 
      $$ \int_0^{2\pi} e^{it}  = \int_0^{2\pi} (\cos(t) + i\sin(t) = [\sin(t) - i \cos(t)]_0^{2\pi} = 0$$ 

      Generally for arbitrary $a$ and $b$ we can show that, $\dots$
   \end{solution}
\end{ex}

\begin{remark}
   In this case $t$ is moving from $0$ to $2\pi$. But because we are in the complex plane it represents a loop.
\end{remark}


\begin{remark}
   Note that the mean value theorem for integrals does not carry over for complex integrals.
\end{remark}
The mean value theorem for integrals in the real sense means that $\exists c $ such that, 
$$ \int_a^b w(t) dt = w(c) (b-a) $$ 

However take $w(t) = e^{it}$ and choose $b = 2\pi$ and $a = 0$. The right hand side will always be non-zero while the left hand size will be 0.





\section{Contour}
In the real case we have functions that are defined on intervals of the real line. However complex inputs lie in the d plane so instead of an interval we define it using curves.
\begin{definition}[contour]
We have $z(t) = x(t) = iy(t)$ is a contour if, 

(1) C is simple arc or Jordan arc, it does not cross itself.
$$ z(t_1) \ne z(t_2), t_1\net_2 $$ 

(2) $z(a) = z(b); $ C simple closed curve.

It is positively oriented if the direction is anticlockwise
\end{definition}

\begin{eg}
   $ x = \begin{cases}x + ix, 0 \le x \le 1 \\ x + i, 1 \le x \le 2\end{cases} $
\end{eg}

\begin{eg}
   $z = re^{i\theta}, 0 \le \theta \le 2\pi$
\end{eg}

\begin{eg}
   $z = re^{i3\theta}, 0 \le \theta \le 2\pi$

   Not a simple arc 
\end{eg}

\begin{eg}
   $\int_C w(z) dz = \int_{C_1} f[z(x)]z'(x) dx + \int_{C_2}f[z(x)] z'(x)dx$

   Here $C$ is the contour from example (1).
\end{eg}


\begin{definition}[differentiable arc]
If $z'(t) = x'(t) + y'(t)i$ which is continuous on  $a \le t \le b$ then,  $C:z(t) \text{ is a differential arc}$ and 

\begin{align*}
   L &= \int_a^b |z'(t)|dt = \int_a^b \sqrt{|x'(t)|^2 + |y'(t)|^2} \text{ length.}\\
     &= \int_a^b |z'(t)|dt = \int_{\alpha}^{\beta} |z'(\phi(\tau))| \phi'(\tau)d\tau\\
   T &= \frac{z'(t)}{|z'(t)|} \text{ tangent vector}\\
\end{align*}

\end{definition}
\begin{definition}[smooth arc]
   An arc is smooth if its derivative $z'(t)$ is cont. on the closed interval $a\le t \le b$ and non-zero throughout the open interval $a < t < b$
\end{definition}
   

\section{Contour Integral}
Consider the integer, 
$$ \int_C f(z) dz \text{ or } \int_{z_1}^{z_2} f(z) dz $$ 

We can parametric in terms of $t$ as, 
$$ \int_C f(z)dz = \int_a^b f(z(t))z'(t) dt $$ 






