\section{Harmonic Function}

\begin{definition}[Harmonic function]
   A real valued functino of $H(x,y)$ is said to be harmonic if in a given domain of the $x,y$ plane, it has a continuous partial derivative of the first and second order ($H_x,H_y,H_{xx},H_{yy},H_{xy}$) and satisfies,  \[
      H_{xx}(x,y)  + H_{yy}(x,y) = 0 \text{ Laplace equation}
   .\] 

\end{definition}     


\begin{theorem}
   If $f = u(x,y) + i(x,y)$ is analytic in a domain $D$, then $u,v$ are harmonic in D
\end{theorem}


\begin{theorem}
   If $f'(z) = 0$ everywhere in $D$ then $f(z)$ is a constant in $D$.
\end{theorem}
\begin{proof}
   Consider $f(z) = u(x,y) + iv(x,y)$ given that $$f'(z) = u_x + iv_x = 0$$

   Using Cauchy-Reimann equation we have, $u_y,v_y = 0$. So all of the first order derivative s are equal to 0 in $D$.

   $U(x,y)$ is constant along any line L,  extending from $p$ to $p'$. Let the vector from $p$ to $p'$ be  $u$. So we have,  
   $$ \frac{du}{ds} = (\text{grad } u) u $$ 
   $$\text{grad } u = u_xi+ u_yj = 0 $$ 

   So u is a constant (a) on L. Similarly for $v = b$ 
   $$ f(z) = a + bi $$ 

\end{proof}

\begin{lemma}
   Suppose, 

   (a). $f(z)$ is analytic throughout $D$

   (b).  $f(z) = 0$ at each point at the domain or line segment containing $D$

   Then  $f(z) \equiv 0$ in $D$
\end{lemma}


\chapter{Elementary Functions}
\section{Exponential Function}
The exponential function is $e^z$. But we can write this as 
$$e^z = e^{x + iy} = e^x e^{iy} = e^x(cos(y) + isin(y))$$

We can also write, $$e^z = \rho e^{i \phi} \text{ where } \rho = |e^x| \text{ and } \phi = y$$ 

For a function, $e^{z_1}e^{z_2}$ we can write, 
$$e^{z_1}e^{z_2} = e^{x_1 + iy_1}e^{x_2+iy_2}$$
\[
=e^{x_1 + x_2} e^{i(y_1+y_2)}
.\] 
\[
   =e^{z_1+z_2}
.\] 

% Similarly we can write, $\frac{e^{z_1}}{e^{z_2} = e^{z_1-z_2}$s


The derivative if $e^z$ is an entire function

$$\frac{d}{dx} e^z = e^z \text{which is an entire function.}$$

$$e^{z + 2\pini} = e^z + e^{2\pini} = e^z$$

\section{Log Function}
The log function is $f(z) = log(z) = w = u + iv$. We know \[
   e^w = z = e^{u + iv} = e^u e^{iv}
.\] 
We see that $r = e^u$ and $\theta = v + 2n\pi$

$$r = e^u \implies ln(r) = u$$
Similarly, \[
\theta = v + 2n\pi
.\] 

So we have, \[
f(z) = \log(z) = \ln|z| + i\arg(z)
.\] 
and the principal direction is, \[
f(z) = \log(z) = \ln|z| + i\theta, \quad -\pi < \theta < \pi
.\] 



Some properties are, 

   $(1). e^{\log z} = z, (z \neq 0)$

   $(2). |e^z| = e^x$
   
   $(3). \log(e^z) = \ln|e^z| + i\arg(e^z)$
\[
= \ln|e^x| + i(y + 2n\pi), n = 0, \pm 1, \pm 2
.\] 
\[
= \ln e^x + iy + i 2 n \pi
.\] 
\[
= z + 2n\pi
.\] 

\subsection*{Branches}
The principal branch is \[
   \log z = \ln r + i \theta \text{ where } r > 0, -\pi < \theta < \pi
.\] 
A branch cut is a portion of a line or curve that is introduced in order to deifne a branch $F$ of a multiple-valued function $f$.

Points on the branch cut for $F$ are singular points of $F$ and any point that is common to all branches of $f$ are called branch points.

\begin{eg}
   \[
      \frac{d}{dz} \log z = \frac{1}{z}, \text{ where } |z | > 0
\]
The branches can be $\alpha < \arg z < \alpha + 2\pi$
\end{eg}



