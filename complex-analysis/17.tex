\section{Example}

We have, 
$$ \int_{{0}}^{{\infty}} {\frac{1}{x^{6} + 1}} \: d{x} $$ 

We consider, 
$$ f(z) = \frac{1}{z^{6} + 1} $$ 

whose isolated singularities are the zeroes of $z^{6} + 1$

If we consider the semicircle in the upper half plane there are three singularities, 
$$ c_0 = e^{i \pi / 6}, c_1 = i , c_2 = e^{i 5\pi / 6}$$ 

So we have, 
$$ \int_{{-R}}^{{R}} {f(x)} \: d{x} + \int_{{C_R}}^{{}} {f(z)} \: d{z} = 2\pi i(B_0 + B_1 + B_2) $$ 

Where $B_k$ is the residue of $f(z)$ at $c_k$

So we have, 
$$ \int_{{-R}}^{{R}} {f(x)} \: d{x} = \frac{2\pi}{3} - \int_{{C_R}}^{{}} {f(z)} \: d{z} $$ 

Now we show that the integral on the right goes to zero when $R \rightarrow \infty$ 

\section{Jordan's Lemma}
\begin{theorem}
   Suppose, 

   (a). $f(z)$ is analytic exterior to $|z| = R_0$ 

   (b). $C_r$ denotes a semicircle $z = Re^{i\theta}(0 \le \theta \le \pi)$ where $R > R_0$ 

   (c). For all points $z$ on $C_R$, there is a positive constant $M_R$ such that, 
   $$ |f(z)| \le M_R \text{ and } \lim_{R \to \infty} M_R = 0$$  

   Then for every positive constant $a$, 
   $$ \lim_{R \to \infty} \int_{C_R} f(z) e^{iaz} dz = 0 $$ 
\end{theorem}
\begin{remark}
   The proof is based on Jordan's Inequality, 
   $$ \int_0^{\pi} e^{-R \sin \theta}d\theta < \frac{\pi}{R} $$ 

   Consider, 
   $$ y = \frac{2\theta}{\pi} \text{ and } y = \sin \theta $$ 
\end{remark}
\begin{proof}
   
   $$ \int_{C_R} f(z) e^{iaz} = \int_0^{\pi} f(Re^{i\theta}) e^{iaRe^{i\theta}}Rie^{i\theta}d\theta$$ 

   Now since $|f(Re^{i\theta})| \le M_R$ and $|e^{iaRe^{i\theta}}| \le e^{-aR \sin \theta}$ 

   it follows that, 
   $$ \bigg | \int_{C_R}f(z) e^{iaz}dz\bigg | \le M_R R\int_0^{\pi} e^{-aR\sin \theta}d\theta < M_R\frac{\pi}{a} $$ 
\end{proof}



\begin{eg}
   
   $$ \int_0^{\infty} \frac{\sin x}{x}dx $$ 

   First let $$f(z) = \frac{e^{iz}}{z}$$. We have a singularity at $z = 0$ so let our contour be such that it jumps at $z = 0$. 

   So we have, 
   
   
   $$ \int_{L_1}f(z) + \int_{L_2}f(z) + \int_{\rho}f(z) + \int_{R} f(z) = \int_C f(z) = 0$$  because there  is no singularity in our contour.
   
   We also have, 
   $$\lim_{R \to \infty}  \int_{C_R} f(z) =  0 $$  from Jordan's lemma 

   $$ \int_{L_1}f(z) + \int_{L_2}f(z) + \int_{\rho}f(z) = 0  $$ 
   Now, 
   $$ \lim_{\rho \to 0} \int_{C_\rho} f(z) = -B_0 \pi i  $$ 

   And the residue is $1$ because $e^{0} = 1$ so, 
   $$ \lim_{\rho \to 0} \int_{C_\rho} f(z) = -\pi i $$ 

   And we also have, 
   $$ \int_{L_1}f(z) + \int_{L_2} f(z) = 2i \int_0^{\infty}\frac{ \sin x }{x} $$ 

   So
$$2i \int_0^{\infty}\frac{ \sin x }{x}  = \pi i$$
$$ \int_0^{\infty}\frac{ \sin x }{x}  = \frac{\pi}{2}$$

   % So the integral evaluates to $\frac{\pi}{2}$
\end{eg}



